\blu{2-15-16: 
Last time we left off with the proof of the Sauer-Shelah lemma. To remind you, we were finding ways to find interesting subsets where matrices behave well. Now recall we had a linear algebraic fact which I owe you; I will prove it in an analytic way. The proof has been moved to Section~\ref{sec:kf}.}

\section{Proof of RIP}

Now we need another geometric lemma for the proof of %\fixme{change name of theorem gen-srank to restricted invertibility principle? fix reference} 
Theorem~\ref{thm:gen-srank}, the restricted invertibility principle. 

\begin{lem}[Step 1] \llabel{lem:step1}
Fix $m, n, r \in \N$. Let $A: \R^m \to \R^n$ be a linear operator with rank$(A) \geq r$. For every $\tau \subseteq \{1, \ldots, m\}$, denote
\[
E_{\tau} = \left(\txtn{span}((Ae_j)_{j \in \tau})\right)^{\perp}.
\] 
Then there exists $\tau \subseteq \{1, \ldots, m\}$ with $|\tau| = r$ such that for all $j \in \tau$, 
\[
\|\txtn{Proj}_{E_{\tau \setminus \{j\}}} Ae_j\|_2 \geq \frac{1}{\sqrt{m}}\left(\sum_{i = 1}^m s_i(A)^2\right)^{1/2}.
\]
\end{lem}
Basically we're taking the projection of the $j^{th}$ column onto the orthogonal completement of the span of the subspace of all columns in the set except for the $j^{th}$ one, and bounding the norm of that by a dimension term and the square root of the sum of the eigenvalues. 
(This is sharp asymptotically, and may in fact even be sharp as written too---I need to check. \fixme{Check this?})

\begin{proof}
For every $\tau \subseteq \{1, \ldots, m\}$, denote
\[
K_{\tau} = \txtn{conv}\left(\{\pm Ae_j\}_{j \in \tau}\right)
\]
The idea is to make the convex hull have big volume. Once we do that, wewill get all these inequalities for free. Let $\tau \subseteq \{1, \ldots, m\}$ be the subset of size $r$ that maximizes $\txtn{vol}_r(K_{\tau})$. We know that $\txtn{vol}_r(K_{\tau}) > 0$. Observe that for any $\beta \subseteq \{1, \ldots, m\}$ of size $r - 1$ and $i \not\in \beta$, we have 
\[
K_{\beta \cup \{i\}} = \txtn{conv}\left(K_{\beta} \cup \{\pm Ae_i \}\right),
\]
which is a double cone. 
%So all you're doing is having found $E_{\beta}$ and a point $Ae_i$, for the convex hull you get a double cone $E_{\beta}, +Ae_i, -Ae_i$ where $E_{\beta}$ is the orthogonal complement of the space spanned by $\beta$. So 

\ig{images/5-1}{.25}

What is the height of this cone? It is $\|\txtn{Proj}_{E_{\beta}} Ae_i\|_2$, as $E_\be$ is the orthogonal complement of the space spanned by $\be$. Therefore, the $r$-dimensional volume is given by 
\[
\txtn{vol}_r(K_{\beta \cup \{i\}}) = 2\cdot \frac{\txtn{vol}_{r -1}(K_{\beta}) \cdot \|\txtn{Proj}_{E_{\beta}} Ae_i\|_2}{r}
\]
Because $|\tau| = r$ is the maximizing subset of $F_{\Omega}$, for any $j \in \tau$ and $i \in \{1, \ldots, m\}$, choosing $\beta = \tau \setminus \{j\}$, we get
\bal
\vol_r(K_{\be\cup \{j\}}) & \ge \vol_r(K_{\be \cup \{i\}})\\
\implies
\|\txtn{Proj}_{E_{\tau \setminus \{j\}}} Ae_j\|_2 &\geq \|\txtn{Proj}_{E_{\tau \setminus \{j\}}} Ae_i\|_2.
\end{align*}
for every $j \in \tau$ and $i \in \{1, \ldots, m\}$. 
Summing,
\[
m\|\txtn{Proj}_{E_{\tau \setminus \{j\}}} Ae_j\|_2^2
\ge \sum_{i = 1}^m \|\txtn{Proj}_{E_{\tau \setminus \{j\}}} Ae_i\|_2^2 = \|\txtn{Proj}_{E_{\tau \setminus \{j\}}} A\|_{S_2}^2.
\]
Then, for all $j \in \tau$, 
\beq{eq:rip-s1-1}
\|\txtn{Proj}_{E_{\tau \setminus \{j\}}} Ae_j\|_2 \geq \frac{1}{\sqrt{m}}\|\txtn{Proj}_{E_{\tau \setminus \{j\}}} A\|_{S_2}
\eeq
Let's denote $P = \txtn{Proj}_{E_{\tau \setminus \{j\}}}$. Note $P$ is an orthogonal projection of rank $r - 1$. Then, 
\beq{eq:rip-s1-2}
\begin{split}
\|PA\|_{S_2}^2 &= \txtn{Tr}((PA)^*(PA)) = \txtn{Tr}(A^*P^*PA) = \txtn{Tr}(A^*PA) = \txtn{Tr}(AA^*P)\\
&= \txtn{Tr}(AA^*) - \txtn{Tr}(AA^*(I - P)) \geq \sum_{i = 1}^m s_i(A)^2 - \sum_{i = 1}^{r - 1} s_i(A)^2 = \sum_{i = r}^m s_i(A)^2
\end{split}
\eeq
using the Ky Fan maximal principle~\ref{lem:kf}, 
since $I - P$ is a projection of rank $m - r + 1$. 
%So the maximum this could be is the tail. %, which is the variational argument we proved earlier (Ky Fan maximal principle).
%And this is what we claimed. 

Putting~\eqref{eq:rip-s1-1} and~\eqref{eq:rip-s1-2} together gives the result.
\end{proof}

\blu{In our proof of the restricted invertibility principle, this is the first step. Before proving it, let me just tell you what the second step looks like. }

\begin{lem}[Step $2$] \llabel{lem:step2}
Let $k, m, n \in \N$, $A: \R^m \to \R^n$, $\rank(A) > k$. Let $\omega \subeq \{1, \ldots, m\}$ with $|\om| = \txtn{rank}(A)$ such that $\{Ae_j\}_{j \in \omega}$ are linearly independent. Denote for every $j \in \Omega$ 
\[
F_j = E_{\omega \setminus \{j\}} = \left(\txtn{span}(Ae_i)_{i \in \omega \setminus \{j\}}\right).
\]
Then there exists $\sigma \subseteq \omega$ with $|\si| =k$ such that 
\[
\|\left(AJ_{\sigma}\right)^{-1}\|_{S_{\infty}} %\leq
\lesssim
 \frac{\sqrt{\txtn{rank}(A)}}{\sqrt{\txtn{rank}(A) - k}} \cdot \max_{j \in \omega} \rc{{\|\txtn{Proj}_{F_j} Ae_j\|}}
\]
%pass to where uniformly big
%first using geometry, find subspace making all the proj big, using singular values. Once you have this lemma, you apply that first, substitute in here.
\end{lem}

Most of the work is in the second step. First we pass to a subset where we have some information about the shortest possible orthogonal project. But Step $1$ saves us by bounding what this can be. Here we use the Grothendieck inequality, Sauer-Shelah, etc. Everything: It's simple, but it kills the restricted invertibility principle. 

%\begin{thm} Step $1$ and Step $2$ imply the Restricted Invertibility Principle.
%\end{thm}
\begin{proof}[Proof of Theorem~\ref{thm:gen-srank} given Step 1 and 2]
Take $A: \R^m \to \R^n$. 
By Step $1$ (Lemma~\ref{lem:step1}), we can find subset $\tau \subseteq \{1, \ldots, m\}$ with $|\tau| = r$ such that for all $j \in \tau$, 
\beq{eq:step1}
\|\txtn{Proj}_{E_{\tau \setminus \{j\}}} Ae_j \|_2 \geq \frac{1}{\sqrt{m}}\left(\sum_{i = r}^m s_i(A)^2\right)^{1/2}.
\eeq
Now we apply Step $2$ (Lemma~\ref{lem:step2}) to $AJ_{\tau}$, using $\omega = \tau$, and find a further subset $\sigma \subseteq \tau$ such that 
\bal
\|\left(AJ_{\sigma}\right)^{-1}\|_{S_{\infty}} 
&\le \min_{k<r<\rank(A)}\sfc{\rank(A)}{\rank(A)-r} \max_{j\in \om}\rc{{\ve{\Proj_{F_j} Ae_j}}}\\
&\leq \min_{k < r < \txtn{rank}(A)} \sqrt{\frac{mr}{(r - k)\sum_{i = r}^m s_i(A)^2}}
\end{align*}
which we get by plugging directly in $r$ for the rank and using Step $1$~\eqref{eq:step1} to get the denominator. 
\end{proof}
%\fixme{? How did it get into the denominator?}
