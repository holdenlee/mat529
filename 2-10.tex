\blu{2-10-16}

We were in the process of proving three or four subset selection principles, which we will somehow use to prove the RIP. 

Now I owe you a proof (just ask me for the linear algebra proof) - I'll show you an analytic proof. 

We proved the little Grothendieck inequality (Theorem~\ref{thm:lgt}), which is part of an amazing area of mathematics with many applications. It's little, but it's also very useful. Just to remind you, we had an linear operator $T: \R^m \to \R^n$. Then for every $x_1, \cdots, x_k \in \R^m$, we get a bounded operator. If you look at the sum of the Euclidean lengths $\left(\sum_{i = 1}^k \|Tx\|_2^2\right)^{1/2} \leq \sqrt{\pi/2}\|T\|_{l_{\infty}^m \to l_2^n} \cdot \max_{1 \leq j \leq m} \left(\sum_{i = 1}^k x_{rj}^2\right)^{1/2}$. This is really the way Grothendieck did it, but the proof we saw is really the original proof, re-organized. For completeness, we'll show the fact that this inequality is sharp (cannot be improved). 

\begin{cor} $\sqrt{\pi/2}$ is the best constant in Theorem~\ref{thm:lgt}. \\
\end{cor}
\begin{proof}
Define $g_1, \cdots, g_k$ be i.i.d standard Gaussians, defined on probability space $(\Omega, P)$. We define $T: L_{\infty}(\Omega, \mathbb{P}) \to \R^k$. Then $Tf = \left(\mathbb{E}(fg_1), \mathbb{E}(fg_2), \cdots, \mathbb{E}(fg_k) \right)$. Choose $X_r = \frac{g_r}{\left(\sum_i^k g_i^2\right)^{1/2}}$. This is nothing more than a vector on the $k$-dimensional unit sphere. So it's a bounded function. We also note that $x_r$ is a function on the measure space $\Omega$.  We can also write
\[
\sum_{r = 1}^k x_r(\omega)^2 = \sum_{r = 1}^k \frac{g_r(\omega)^2}{\sum_{i = 1}^r g_i(\omega)^2} = 1
\]
We can use the Central Limit Theorem to make things precise: $g_r \approx \frac{\epsilon_{r_1} + \cdots + \epsilon_{r_N}}{\sqrt{N}}$ as $N \to \infty$. So all these statements will be asymptotically true. Where does the family of random variables $\{g_r\}$ live in $\Omega$? Well $\Omega = \{\pm 1\}^{NK}$. So $L_{\infty}(\Omega) = \l_{\infty}^{2^{NK}}$, which is some huge dimension, but it's still finite. So $\omega$ will really be a coordinate in $\Omega$. 

Now we show two things; nothing more than computations. 
\begin{enumerate}

\item $\|T\|_{L_{\infty}(\Omega, \textbf{P}) \to l_2^k} = \sqrt{2/\pi}$, 

\item We also show $\sum_{r = 1}^k \|Tx_r\|_2^2 \to^{k \to \infty} 1$. 

\end{enumerate}

First we tackle the first case. We have 
\begin{align}
\begin{split}
\|T\|_{\infty \to 1} &= \text{sup}_{\|f\|_{\infty} \leq 1}\left(\sum_{r = 1}^k \mathbb{E}\left[fg_r\right]^2 \right)^{1/2}
\\
&= \text{sup}_{\|f\|_{\infty} \leq 1} \text{sup}_{\sum_{r = 1}^k \alpha_r^2 = 1} \sum_{r = 1}\alpha_r \mathbb{E}\left[fg_r\right]
\\
&= \text{sup}_{\sum_{r = 1}^k} \text{sup}_{\|f\|_{\infty} \leq 1} \mathbb{E}\left[f\sum_{i = 1}^k \alpha_r g_r \right]
\\
&= \text{sup}_{\sum_{r = 1}^k} \mathbb{E}\left| \sum_{r = 1}^k \alpha_r g_r \right| = \mathbb{E} |g_1| = \sqrt{\frac{2}{\pi}}
\end{split}
\end{align}

as we claimed. Now we tackle the second computation: 
\begin{align}
\begin{split}
\sum_{r = 1}^k \|Tx_r\|_2^2 &= \sum_{r = 1}^k \left(\mathbb{E} \left[\frac{g_r^2}{\left(\sum_{i = 1}^k g_i^2\right)^{1/2}}\right]\right)^2
\\
&= K \left(\mathbb{E}\left[\frac{g_1^2}{\left(\sum_{i = 1}^k g_i^2\right)^{1/2}}\right]\right)
\\
&= K\left(\frac{1}{K}\mathbb{E}\left[\sum_{r = 1}^k \frac{g_r^2}{\left(\sum g_i^2\right)^{1/2}}\right]\right)^2
\\
&= \frac{1}{K} \left(\mathbb{E}\left[\left(\sum_{i = 1}^k g_i^2\right)^{1/2}\right]\right)^2
\end{split}
\end{align}
and you can use Stirling to finish. This is just a $\chi^2$-distribution. 

In this case $\mathbb{E} \frac{g_1g_2}{\left(\sum_i g_i^2\right)^{1/2}} = \mathbb{E} \frac{g_1 (-g_2)}{\left(\sum g_i^2\right)^{1/2}}$. 
Also note that if $(g_1, \cdots, g_k) \in \R^k$ is a standard Gaussian, then $\frac{(g_1, \cdots, g_k)}{\left(\sum_{i = 1}^k g_i^2\right)^{1/2}}$ and $\left(\sum_{i = 1}^k g_i^2)^{1/2}\right)$ are independent. In other words, the length and angle are independent: This is just polar coordinates, you can check this. 
\end{proof}

Now, how does this relate to the Restricted Invertibility Problem? 

\begin{thm} Pietsch Domination Theorem.\llabel{thm:pdt} \\
Fix $m, n \in \mathbb{N}$ and $M  > 0$. Suppose that $T: \R^m \to \R^n$ is a linear operator such that for every $x_1, \cdots, x_k \in \R^m$ have 
\[
\left(\sum_{r = 1}^k \|Tx_r\|_2^2\right)^{1/2} \leq M \text{max}_{1 \leq j \leq m} \left(\sum_{r = 1}^k x_{rj}^2\right)^{1/2}
\]
Then there exist $\mu = (\mu_1, \cdots, \mu_m) \in \R^m$ with $\mu_1 \geq 0$ and $\sum_{i = 1}^m = 1$ such that for every $x \in \R^m$
\[
\|Tx\|_2 \leq M\left(\sum_{i = 1}^M \mu_ix_i^2\right)^{1/2}
\]
It's really an iff: The latter is a stronger statement than the former, and in fact they are equivalent.
You can come out with a probability measure, a way to weight the coordinates, such that the norm of T as an operator as a standard norm from $l_{\infty}$ to $l_2$, is bounded by $M$. 
\end{thm}
\begin{proof}
Define $K \subseteq \R^m$ with 
\[
K = \left\{y \in \R^m: y_i = \sum_{r = 1}^k \|Tx_r\|_2^2 - M^2\sum_{r = 1}^m x_{ri}^2 \text{ for some } k, x_1, \cdots, x_k \in \R^m\right\}
\]
Basically we cleverly select a convex set. Every $n$-tuple of vectors in $\R^m$ gives you a new vector in $\R^m$. Let's check that $K$ is convex. We have to check if two vectors $y, z \in K$ have all points on the line between them in $K$. $y \in K$ means that 
\[
y_i = \sum_{r = 1}^k \|Tx_r\|_2^2 - M^2 \sum_{r = 1}^m x_{ri}^2
\] 
\[
z_i = \sum_{r = 1}^l \|Tw_r\|_2^2 - M^2 \sum_{r = 1}^l w_{ri}^2
\]
for all $i$. So what can you say about the average $\frac{y_i + z_i}{2}$? It comes from $\left(\frac{x_1}{\sqrt{2}}, \cdots, \frac{x_k}{\sqrt{2}}, \frac{w_1}{\sqrt{2}}, \cdots, \frac{w_l}{\sqrt{2}}\right)$. So trivially by design this is a convex set. 

Now, the assumption of the theorem says that 
\[
\left(\sum_{r = 1}^k \|Tx_r\|_2^2\right)^{1/2} \leq M \text{max}_{1 \leq j \leq m} \left(\sum_{r = 1}^k x_{rj}^2\right)^{1/2}
\]
which implies 
\[
\|Tx_r\|_2^2 - M^2 \text{max}_{1 \leq j \leq m} \sum_{r = 1}^m x_{rj}^2 \leq 0
\]
which implies $K \cap (0, \infty)^m = \emptyset$. By the hyperplane separation theorem (for two disjoint convex sets in $\R^m$ with at least one compact, there is a hyperplane between them), there exists $0 \neq \mu = (\mu_1, \cdots, \mu_m) \in \R^m$. We have 
\[
\langle \mu, y \rangle \leq \langle \mu, z\rangle
\]
for all $y \in K$ and $z \in (0, \infty)^m$. By renormalizing, $\sum_{i = 1}^m = 1$. Moreover $\mu$ cannot have any strictly negative coordinate: Otherwise you could take $z$ to have arbitrarily large value at a strictly negative coordinate with zeros everywhere else, implying $\langle u, z \rangle$ is no longer bounded from below, a contradiction. Therefore, $\mu$ is a probability vector and $\langle \mu, z \rangle$ can be arbitrarily small. So for every $y \in K$, $\sum_{i = 1}^m \mu_iy_i \leq 0$. Then $y_i = \|Tx\|_2^2 - M^2x_i \in K$, and if you write this out, $\|Tx\|_2^2 - M^2 \sum_{i = 1}^n \mu_i y_i \leq 0$, which is exactly what we wanted. 
\end{proof}

\begin{lem} \llabel{lem:projbound}
$m, n \in \mathbb{N}$, $\epsilon \in (0, 1)$, $T: \R^n \to \R^m$ a linear operator. Then $\exists \sigma \subset \{1, \cdots, m\}$ with $|\sigma| \geq (1 - \epsilon)m$ such that 
\[
\|\text{Proj}_{\R^{\sigma}} T\|_{S_{\infty}} \leq \sqrt{\frac{\pi}{2\epsilon m}} \|T\|_{l_2^n \to l_1^m}
\] 
We will find ways to restrict a matrix to a big big submatrix, but we won't be able to control its operator norm, but we will be able to control the norm from $l_2^n \to l_1^m$. So then you go to a further subset, which this becomes an operator norm on, which is an improvement which Grothendieck gave us. This is the first very useful tool to start finding big sub matrices. 
\end{lem}
\begin{proof}
We have $T: l_2^n \to l_1^m$, $T^*: l_{\infty}^m \to l_2^n$. Now some abstract nonsense gives us that for Banach spaces, the norm of an operator and its adjoint are equal, i.e. $\|T\|_{l_2^n \to l_1^m}  = \|T^*\|_{l_{\infty}^m \to l_2^n}$. This statement follows from Hahn-Banach theorem (come see me if you haven't seen this before, I'll tell you what book to read). 
From the Little Grothendieck inequality (Theorem~\ref{thm:lgi}), $T^*$ satisfies the assumption of the Pietsch domination theorem with $M = \sqrt{\frac{\pi}{2}} \|T\|_{l_2^n \to l_1^m}$ (we're applying it to $T^*$). So we have probability vector $(\mu_1, \cdots, \mu_m)$ such that for every $y \in \R^m$
\[
\|T^*y\|_2 = M\left(\sum_{i = 1}^m \mu_iy_i^2\right)^{1/2}
\]
with $M = \sqrt{\frac{\pi}{2}} \|T\|_{l_2^n \to l_1^m}$. Define $\sigma = \left\{i \in \{1, \cdots, m\}: \mu_i \leq \frac{1}{m\epsilon}\right\}$, then $|\sigma| \geq (1 - \epsilon)m$ by Markov's inequality. We can also see this by writing
\[
1 = \sum \mu_i = \sum_{i \in \sigma} \mu_i + \sum_{i \not\in \sigma} \mu_i > \sum_{i \in \sigma} \mu_i + \frac{m - |\sigma|}{m\epsilon}
\]
which follows since $\mu_j$ for $j \not\in \sigma$ has $\mu_j > \frac{1}{m\epsilon}$. Continuing, 
\[
\frac{m\epsilon - m + |\sigma|}{m\epsilon} \geq \sum_{i \in \sigma} \mu_i
\]
\[
|\sigma| \geq (m\epsilon)\sum_{i \in \sigma} \mu_i +  m(1 - \epsilon)
\]
Then, since $(m\epsilon)\sum_{i \in \sigma} \mu_i  \geq 0$ since $\mu$ is a probability distribution, we have
\[
|\sigma| \geq m(1 - \epsilon)
\]

Now take $x \in \R^n$ and choose $y \in \R^m$ with $\|y\|_2 = 1$. Then 
\[
\langle y, \text{Proj}_{\R^{\sigma}} Tx \rangle^2 = \|\text{Proj}_{\R^{\sigma}} Tx\|_2^2 \leq \langle T^* \text{Proj}_{\R^{\sigma}} y, x \rangle^2 \leq \|T^*\text{Proj}_{\R^{\sigma}} y\|_2^2 \cdot \|x\|_2^2
\] 
\[
\leq \frac{\pi}{2}\|T\|_{l_2^n \to l_1^m} \left(\sum_{i \in \sigma} \mu_iy_i^2\right) \|x\|_2^2 \leq \frac{\pi}{2} \|T\|_{l_2^n \to l_1^m}^2 \frac{1}{m\epsilon}\|x\|_2^2
\]
by Cauchy-Schwarz. Then, taking square roots gives the desired result.
\end{proof}
In the previous proof, we used a lot of duality to get an interesting subset. 

\begin{rem}
In Lemma~\ref{lem:projbound}, I think that either the constant $\pi/2$ is sharp (no subset are bigger; it could come from the Gaussians), or there is a different constant here. If the constant is $1$, I think you can optimize the previous argument and get the constant to be arbitrarily close to $1$, which would have some nice applications: In other words, getting $\sqrt{\frac{\pi}{2\epsilon m}}$ as close to $1$ as possible would be good. I didn't check before class, but you might want to check if you can carry out this argument using the Gaussian argument we made for the sharpness of $\frac{\pi}{2}$ in Grothendieck's inequality (Theorem~\ref{thm:lgt}). It's also possible that there is a different universal constant. 
\end{rem}

Now we will give another lemma which is very easy and which we will use a lot. 
\begin{lem} Sauer-Shelah. \llabel{lem:saushel} \\
Take integers $m, n \in \mathbb{N}$ and suppose that we have a large set $\Omega \subseteq \{\pm 1\}^n$ with 
\[
|\Omega| > \sum_{k = 0}^{m - 1} {n \choose k}
\]
Then $\exists \sigma \subseteq \{1, \cdots, n\}$ such that with $|\sigma| = m$, if you project onto $\R^{\sigma}$ the set of vectors, you get the entire cube: $\text{Proj}_{\R^{\sigma}}(\Omega) = \{\pm 1\}^{\sigma}$. For every $\epsilon \in \{\pm 1\}^{\sigma}$, there are signs $\delta = (\delta_1, \cdots, \delta_n) \in \Omega$ such that $\delta_j = \epsilon_j$ for $j \in \sigma$.
\end{lem}

Note that Lemma~\ref{lem:saushel} is used in the proof of the Fundamental Theorem of Statistical Learning Theory. 

\begin{proof}
% The strengthening of induction I'm about to use is due to Pajor.
We want to prove by induction on $n$. First denote the shattering set
\[
\txtn{sh}(\Omega) = \{\sigma \subseteq \{1, \cdots, n\}: \txtn{Proj}_{\R^{\sigma}}\Omega = \{\pm 1\}^{\sigma}\}
\]
The claim is that the number of sets shattered by a given set is $|\text{sh}(\Omega)| \geq |\Omega|$. The empty set case is trivial. What happens when $n = 1$? $\Omega \subset \{-1, 1\}$, and thus the set is shattered. Assume that our claim holds for $n$, and now set $\Omega \subseteq \{\pm 1\}^{n + 1} = \{\pm 1\}^n \times \{\pm 1\}$. Define
\[
\Omega_+ = \{\omega \in \{\pm 1\}^n: (\omega, 1) \in \Omega\}
\] 
\[
\Omega_{-} = \{ \omega \in \{\pm 1\}^n: (\omega, -1) \in \Omega\}
\]
Then, letting $\tilde{\Omega}_+ = \{(\omega, 1)\in \{\pm 1\}^{n+1}: \omega \in \Omega_+\}$ and $\tilde{\Omega}_-$ similarly, we have $|\Omega| = |\tilde{\Omega}_+| + |\tilde{\Omega}_-| = |\Omega_+| + |\Omega_-|$. By our inductive step, we have sh$(\Omega_+) \geq |\Omega_+|$ and sh$(\Omega_-) \geq |\Omega_-|$. Note that any subset that shatters $\Omega_+$ also shatters $\Omega$, and likewise for $\Omega_{-}$. Note that if a set $\Omega'$ shatters both of them, we are allowed to add on an extra coordinate to get $\Omega' \times \{\pm 1\}$ which shatters $\Omega$. Therefore, 
\[
\txtn{sh}(\Omega_+) \cup \txtn{sh}(\Omega_-) \cup \left\{\sigma \cup \{n + 1\}: \sigma \in \txtn{sh}(\Omega_+) \cap \txtn{sh}(\Omega_-)\right\} \subseteq \txtn{sh}(\Omega)
\]
where the last union is disjoint since the dimensions are different. Therefore, we can now use this set inclusion to complete the induction using the principle of inclusion-exclusion: 
\bal
|\txtn{sh}(\Omega)| &\geq |\txtn{sh}(\Omega_+) \cup \txtn{sh}(\Omega_-)| + |\txtn{sh}(\Omega_+) \cap \txtn{sh}(\Omega_-)| \hspace{3em} \txtn{  (disjoint sets)}
\\
&= |\txtn{sh}(\Omega_+)| + |\txtn{sh}(\Omega_-)| - |\txtn{sh}(\Omega_+) \cap \txtn{sh}(\Omega_-)|  + |\txtn{sh}(\Omega_+) \cap \txtn{sh}(\Omega_-)| 
\\
&= |\txtn{sh}(\Omega_+)| + |\txtn{sh}(\Omega_-)|
\\
&\geq |\Omega_+| + |\Omega_-| = |\Omega|
\end{align*}
which completes the induction as desired. 
\end{proof}

\begin{cor} If $|\Omega| \geq 2^{n -1}$ then there exists $\sigma \subseteq \{1, \cdots, n\}$ with $|\sigma| \geq \lceil \frac{n + 1}{2} \rceil \geq \frac{n}{2}$ such that $\txtn{Proj}_{\R^{\sigma}} \Omega = \{\pm 1\}^{\sigma}$. If you have half of the points in terms of cardinality, you get half of the dimension: At least.
\end{cor}

We will primarily use the corollary. 


