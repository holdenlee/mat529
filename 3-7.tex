\blu{3-7: Finishing Bourgain. There's a remaining lemma about the Poisson semigroup that we're going to do today (Lemma~\ref{lem:bdt-1}); it's nice, but it's not as nice as what came before.}

%TODO: Put in a summary of the proof. (This is messy right now.)
%Let me try to paraphrase the high level strategy before. What we did last time: We had a $\delta$-net in $X$, a function $f$ going into $Y$. Then we almost extended $f$ to $F$. The idea was to take the derivative of $(P_t * F)(x)$ and hope this is big. It's already bounded by the Lipschitz constant. What does it mean that this derivative is big? It means that the norm in $Y$ is at least a constant times the norm of $x$, for every $x$. 
%\[
%\left\|(P_t * F)'(x)\right\|_Y \ge \|x\|_X
%\]
%If you're invertible; if you have this inequality for a net in the sphere, you have the inequality everywhere. If you know the inequality on the net, you know it globally only if the $\delta$ of the net is smaller than the norm of the operators. So you fix this net: $\mc{F} \subseteq S_X$, a $c/D$-net in $S_X$.  
%Then for all $a \in \mc{F}$
%\[
%\left\|(P_t * F)'(x)(a)\right\|_Y = \left\| \partial_a(P_t * F)(x)\right\|_Y
%\]
%Let $\nu$ be the normalized volume on $\frac{1}{2}B_X$. It's enough to show that the measure 
%\[
%\nu\left(x \in (1/2)B_X: \forall a \in \mc{F}, \|\partial_a(P_t*F)(x)\|_Y \ge \frac{c}{\|x\|_X}\right)
%\]
%This is the sum in the net of this measure. 
%And you can directly write down 
%\[
%1 - \sum_{a \in \mc{F}} \nu(x \in (1/2)B_X: \|\partial_a(P_t*F)(x)\|_Y < \frac{c}{\|x\|_X})
%\]
%Then you want to see how you can make this probability zero. 
%
%Bourgain wants to make a probabilistic argument; he says investigate the evolution of the probability over the semigroup: It depends on time $t$. And there must exist a time for which this is $< 1$. This is a semigroup argument combined with probability, and what you are analyzing as things flow forward in time is how the probability changes. We bound this probability using Markov. Then there's all kinds of lemmas which need to come in (and there's one more which I need to finish today), but then this is essentially Bourgain's theorem. We don't know of another way to prove this today. 

The last remaining lemma to prove is Lemma~\ref{lem:bdt-1}.
%\begin{proof}

Remember that $\frac{1}{\sqrt{n}}\|x\|_2 \leq \|x\|_X \leq \|x\|_2$ for all $x \in X$. 
We need three facts about the Poisson kernel.

%\begin{enumerate}
%\item 
\begin{pr}[Fact 1]\llabel{pr:pois1}
\[
\int_{\R^n \setminus (rB_X)} P_t(x)dx \leq \frac{t\sqrt{n}}{r}.
\]
\end{pr}
\begin{proof}
First note that the Poisson semigroup has the property that \beq{eq:pois-rescale}
P_t(x) = \frac{1}{t^n}P_1(x/t),
\eeq 
i.e., $P_t$ is just a rescaling of $P_1$. So it's enough to prove this for $t = 1$. We have
\begin{align*}
\int_{\|x\|_X \geq r} P_t(x)\dx 
&\leq \int_{\|x\|_2 \geq r} P_t(x) \dx\\
&= \int_{\|x\|_2 \geq r/t} P_1(x)\dx\\
&\quad \text{by change of variables and~\eqref{eq:pois-rescale}}\\
&= C_n S_{n - 1} \int_{r/t}^{\infty} \frac{s^{n - 1}}{(1 + s^2)^{(n + 1)/2}} ds \\
&\quad \text{polar coordinates, where }S_{n-1}=\vol(\bS^{n-1})\\
&\leq C_nS_{n - 1} \int_{r/t}^{\infty} \frac{1}{s^2} ds\\
& = \frac{t}{r}C_nS_{n - 1}\\
&\le \fc{t\sqrt n}r,
\end{align*}
where the last inequality follows from expanding in terms of the Gamma function and using Stirling's formula. 

Above we changed to polar coordinates using
\[
\int_{\ve{x}_2\le R} f(\ve{x})\dx=\int_0^R S_{n-1}\ve{x}^{n-1} f(r)\,dr.
\]
\end{proof}
%\item
\begin{pr}[Fact 2]\label{pr:pois2}
For all $y \in \R^n$, 
$$\int_{\R^n} |P_t(x) - P_t(x + y)| dx \le \frac{\sqrt{n}\|y\|_2}{t}.$$
\end{pr}
\begin{proof}
It's again enough to prove this when $t = 1$. 
\begin{align*}
\int_{\R^n} |P_t(x) - P_t(x + y)| \dx 
&= \int_{\R^n} \left| \int_{0}^1 \langle \nabla P_t(x + sy), y \rangle ds\right| \dx 
\\
&\leq \|y\|_2 \int_{\R^n} \|\nabla P_t(x)\|_2 dx&\text{Cauchy-Schwarz}\\
	&\le 
	\|y\|_2(n + 1)C_n\int_{\R^n} \frac{\|x\|_2}{(1 + \|x\|_2^2)^{\frac{n + 3}{2}}} dx&\text{computing gradient}\\
	&= \|y\|_2 (n + 1)C_n S_{n - 1} \int_{0}^{\infty} \frac{r^n}{(1 + r^2)^{\frac{n + 3}{2}}} \,dr  &\text{polar coordinates}\\
	&\le \sqrt{n}\ve{y}_2.
\end{align*}
The integral and multiplicative constants perfectly cancels out the $n + 1$ and becomes 1 (calculation omitted).
\end{proof}
%\item
\begin{pr}[Fact 3]\label{pr:pois3}
For all $0 < t < \frac{1}{2}$ and $x \in B_X$, we have 
\[
\left\|P_t * F(x) - F(x) \right\|_Y \le \sqrt{n} t\log(3/t).
\]
\end{pr}
\begin{proof}
The LHS equals
\begin{align*}
\left\|\int_{\R^n} \left(F(x - y) - F(x)\right)P_t(y) dy \right\|_Y &\leq 
\ub{\int_{x + 3B_X} \left\| F(x - y) - F(x)\right\|_Y P_t(y) dy}{\stepcounter{equation}\mbox{(\theequation)}} + 
\ub{c\int_{\R^n \setminus (x + 3B_X)} P_t(y) dy}{\stepcounter{equation}\mbox{(\theequation)}}.
\end{align*}
\addtocounter{equation}{-2}\refstepcounter{equation}\label{eq:pois3-1}
\refstepcounter{equation}\label{eq:pois3-2}

Using $\left\|F(x - y) - F(x)\right\|_Y \leq \|F\|_{\text{Lip}}\frac{c}{\rho}\cdot \|y\|_X$ and that $\ve{F}_{\text{Lip}}$ is a constant,
\bal
\eqref{eq:pois3-1} &=  \int_{x + 3B_X} \|y\|_XP_t(y)dy\\
&\le \int_{4\sqrt{n}B_{l_2^n}} \|y\|_2 P_t(y) dy&\text{using }x\in B_X\implies x + 3B_X\subeq 4\sqrt n B_{l_2}\\
&= tC_nS_{n - 1} \int_{0}^{\frac{4\sqrt{n}}{t}} \frac{s^n}{(1 + s^2)^{(n + 1)/2}}\,ds&\text{polar coordinates}\\
&\le t\sqrt{n} \pa{\int_0^{\sqrt{n}} \frac{1}{\sqrt{n}}\,ds + \int_{\sqrt{n}}^{4\sqrt{n}/t} \rc{s}ds}\\
&\le t\sqrt n \pa{1+\ln \pf{4}{t}},
\end{align*}
where we used the fact that $\fc{s^n}{(1+s^2)^{\fc{n+1}{2}}}$ is maximized when $s = \sqrt{n}$, and is always $\leq \rc s$.

For the second term, note that $2B_X\subeq x+3B_X$, so 
\[
\eqref{eq:pois3-2} =c\int_{\R^n\bs 2B_X}P_t(y)\le \fc{ct\sqrt n}{2}
\]
where we used the tail bound of Fact 1 (Pr. \ref{pr:pois2}). Adding these two bounds gives the result.
\end{proof}
%\end{enumerate}

Now we have these three facts, we can finish the proof of the lemma. 

\begin{proof}[Proof of Lemma~\ref{lem:bdt-1}]
\begin{clm}[$P_t*F$ separates points]
Let $\theta = cD\sqrt{n} t\log(3/t)$. 
%All of this is about directional derivatives which are big. So let's see how $P_t$ separates points. Suppose I take points 
Let $w, y \in \frac{1}{2}B_X$ with $\ve{w-y}_X\ge \te$. Then
\[
\left\|P_t * F(w) - P_t*F(y)\right\|_Y \geq 1/D
\]
\end{clm}
We do not claim at all that these numbers are sharp. 
\begin{proof}
We can find a point $p\in \cal N_\de$ which is $\delta$ away from $w$, and $q \in \cal N_{\de}$ which is $\delta$ away from $y$. We also know that $F$ is close to $f$, and we can use Fact 3 (Pr. \ref{pr:pois3}) to bound the error of convolving. 

\ig{images/11-3}{.25}

By the triangle inequality 
\begin{align*}
\left\|P_t * F(w) - P_t*F(y)\right\|_Y 
&\geq \|f(p) - f(q)\| - \|F(p) - f(p)\| - \|F(q) - f(q)\| \\
&\quad - \|F(w) - F(p)\| - \|F(y) - F(q)\| \\
&\quad- \|P_t*F(w) - F(w)\| - \|P_t*F(y) - F(y)\|\\
&\ge \frac{\|p - q\|}{D} - 2n\delta - c\delta - \sqrt{n}t\log(3/t)\\ &\geq \frac{\|y - w\| - 2\delta}{D} - 2n\delta - c\delta - \sqrt{n}t\log(3/t) .
\end{align*}
Taking $$\theta = cD\sqrt{n}t\log(3/t),$$ we find this is at least a constant times $\|y - w\|/D$, we need $\theta = cD\sqrt{n}t\log(3/t)$. Note that $\sqrt{n}t\log(3/t)$ is the largest error term because by the assumptions in the lemma, it is greater than $2n\delta$. 
\end{proof}
Now consider
\[
\left\|P_t * F(z + \theta a) - P_t * F(z)\right\|
\]
By the claim, if $z \in \frac{1}{4}B_X$, then we know 
\begin{align}
\theta/D &\leq \left\|P_t * F(z + \theta a) - P_t* F(z)\right\|\\
&= \int_{0}^{\theta} \left\|\partial_a(P_t*F)(z + sa)\right\|_Y 
\label{eq:bdt-int}
\end{align}
Suppose $\ve{x-y}\le \rc4$ (we will apply~\eqref{eq:bdt-int} to $x-y=z$), $x\in \rc2 B_X$, $y\in \rc8 B_X$. By throwing away part of the integral,
\begin{align}
&\quad \frac{1}{\theta}\int_0^{\te} \int_{\R^n} \left\| \partial_a(P_t * F)(x + sa - y)\right\|_Y P_{Rt}(y) dy\\
&\ge
\frac{1}{\theta} \int_0^{\theta} \int_{(1/8)B_X} \left\|\partial_a(P_t * F)(x - y + sa)\right\|_Y P_{Rt}(y) \,ds \dy\\
& = 
%So if $x$ is in half of the ball and $y$ is $1/8$ of the ball, thi sis fine. Now you can use Fubini to switch the order of integrals: 
 \int_{(1/8)B_X} \left(\frac{1}{\theta}\int_0^{\theta} \left\|\partial_a(P_t * F)(x - y + sa)\right\|_Y ds \right) P_{Rt}(y) \dy&\text{by Fubini}\\
&\ge \frac{1}{D} \int_{(1/8)B_X} P_{Rt}(y)dy &\text{by }\eqref{eq:bdt-int}\\
&\ge \fc cD
\label{eq:bdt-f1}
\end{align}
for some constant $c$.
%which is bounded below. 
%So our statement amounts to saying the 
The calculation above says that the average length in every direction is big, and if you average the average length again, it is still big.
How do we get rid of the additional averaging? We care about 
\begin{align*}
\|\partial_a(P_t * F)\| * P_{Rt}(x) 
&= \int_{\R^n} \|\partial_a(P_t *F)(x - y)\|_Y P_{Rt}(y) dy\\
%Then shifting, we can add $ + sa$ for every $s$, then add a correction term, and now we can integrate over $s \in [0, \theta]$ and average: 
&=
\frac{1}{\theta} \int_0^{\theta} \left(\int_{\R^n} \|\partial_a(P_t *F)(x - y + sa)\|_Y P_{Rt}(y) dy \right)\\
&\quad  - \frac{1}{\theta} \int_0^{\theta} \int_{\R^n} \|\partial_a(P_t *F)(x - y)\|_Y (P_{Rt}(y + sa) - P_{Rt}(y)) dy \\
&> \frac{c}{D} - \frac{c'}{\theta} \int_0^{\theta} \int_{\R^n} | P_{Rt}(y + sa) - P_{Rt}(y)| ds \dy&\text{by \eqref{eq:bdt-f1} and \fixme{what?}}\\
&\geq \frac{c}{D} - \frac{c'}{\theta} \int_0^{\theta} \frac{\sqrt{n}s\|a\|_2}{Rt} ds&\text{by Fact 2 (Pr. \ref{pr:pois2})}
\end{align*}
This error term is
$$
\frac{c'}{\theta} \int_0^{\theta} \frac{\sqrt{n}s\|a\|_2}{Rt} ds
 = \fc{cDn\log \pf{3}{t}}R = O\prc{D}$$
by~\eqref{eq:bdt1-1}, so $\|\partial_a(P_t * F)\| * P_{Rt}(x)=\Om\prc{D}$  (if we chose constants correctly), as desired. \fixme{Factor of $\sqrt n$?}
%\frac{c'\theta n}{2Rt}$, and $\frac{\theta n}{Rt} < \frac{c}{D}$ so we are fine. 
\end{proof}

\fixme{To summarize: You compute the error such that you have distance $\theta$. Then you look at the average derivative on each line like this. Then, the average derivative is roughly $\|\partial_a(P_t * F)\| * P_{Rt}(x)$. So averaging the derivative in a bigger ball is the same up to constants as taking a random starting point and averaging over the ball. What's written in the lemma is that the derivative of a given point in this direction, averaged over a little bit bigger ball, it's not unreasonable to see that at first take starting point and going in direction $\theta$ and averaging in the bigger ball is equivalent to randomizing over starting points. }
%\end{proof}

%We will prove an easy fact next time: If you look at $L_1$, and change the metric to $\sqrt{\|x - y\|_1}$, this is isometric to a subset of $L_2$. %This is what is called $L_1$ is a squared $L_2$ metric. If you take this embedding into $L_2$, let's think how Lipschitz the square-root mapping is. 
%With this embedding, you don't distortion distances by more than a factor $1/\sqrt{\delta}$ on the net, even though the distortion on the whole space is bigger. (This is just a sketch.)
%
%To get better results, you need more exotic spaces. 

%\step{2} Extend $F_1$ to the whole space to $F_2$ such that 
%\begin{enumerate}
%\item
%$\forall x\in \cal N_\de$, $\ve{F_2(x)-f(x)}_Y \lesssim L\de$.
%\item
%$\ve{F_2(x)-F_2(y)}_Y\lesssim L(\ve{x-y}_X + \de)$.
%%not smooth yet. 
%\item
%$\Supp(F_2)\subeq 2B_X$.
%\item
%$F_2$ is smooth.
%%F_1 had no bounded continuity, but is a sum against a partition of unity. 
%%just smooth without any bounds fine.
%%spiky.
%%when I say not smooth, I mean no bounds.
%%I need the norm to be smooth for this.
%\end{enumerate}•%

%Denote $\al(t)=\max\{1-|1-t|,0\}$. %

%\ig{images/9-1}{.25}%

%Let
%\[
%F_2(x)=\al(\ve{x}_X) F_1\pa{x}{\ve{x}_X}.
%\]
%%0 the moment it passes 2.
%$F_2$ still satisfies condition 1. As for condition 2, 
%\bal
%\ve{F_2(x)-F_2(y)}_Y &= \ve{\al (\ve{x}_X)F_1\pf{x}{\ve{x}_X} - \al(\ve{y}_X) F_1\pf{y}{\ve{y}_X}} \\
%&\le |\al(\ve{x})-\al(\ve{y})|\ub{\ve{F_1\pf{x}{\ve{x}_X}}}{\le 2L}+\al(\ve{y})\ve{F_1\pf{x}{\ve{x}_X} - F_1\pf{y}{\ve{y}_X} }\\
%&\le (\ve{x}-\ve{y})2L + \al(\ve{y}) L\pa{
%\ve{\nv{x}-\nv{y}}+4\de 
%} \\
%&\le 2L\ve{x-y}+L\al(\ve{y}) \pa{\ve{x}\ab{\rc{\ve{x}}-\rc{\ve{y}}} + \fc{\ve{x-y}}{\ve{y}} + 4\de}\\
%&\le 2L\ve{x-y} +L\al(\ve{y}) \pa{\fc{\ve{x-y}}{\ve{y}} + \fc{\ve{x-y}}{\ve y}  + 4\de}\\
%&\lesssim L(\ve{x-y}+\de),
%%mult by 4de, use bounded by 1
%\end{align*}
%where in the last step we used $\al(\ve{y})\le \ve{y}$ and $\al(\ve{y})\le 1$. %

%Note $F_2$ is smooth because the sum for $F_1$ was against a partition of unity and $\ved_X$ is smooth, although we don't have uniform bounds on smoothness for $F_2$.
%%F_1 had no bounded continuity, but is a sum against a partition of unity. 
%%just smooth without any bounds fine.
%%spiky.
%%when I say not smooth, I mean no bounds.
%%I need the norm to be smooth for this.%

%%For the next step we need the following. %

%\step{3} We make $F$ smoother by convolving.
%\begin{lem}[Begun, 1999]
%Let $F_2:X\to Y$ satisfy $\ve{F_2(x)-F_2(y)}_Y\le L(\ve{x-y}_X+\de)$. Let $\tau \ge c\de$. Define 
%\[
%F(x) = \rc{\Vol(\tau B_X)}\int_{\tau B_X} F_2(x+y)\,dy.
%\]
%Then 
%\[
%\ve{F}_{\text{Lip}} \le L\pa{1+\fc{\de n}{2\tau}}.
%\]
%\end{lem}
%The lemma proves the almost extension theorem as follows. We passed from $f:\cal N_\de\to Y$ to $F_1$ to $F_2$ to $F$. 
%If $x\in \cal N_\de$, 
%\bal
%\ve{F(x)-f(x)}_Y &=\ve{
%\rc{\Vol(\tau B_X)} \int_{B_X} (F_2(x+y) - f(x))\,dy
%}\\
%&\le \rc{\Vol(\tau B_X)}\int_{\tau B_X}\ve{F_2(x+y)-F_2(x)}_Y + \ub{\ve{F_2(x)-f(x)}_Y}{\de L} \dy\\
%&\le \rc{\Vol(\tau B_X)}\int_{\tau B_X}(L(\ub{\ve{y}_X}{\le\tau}+\de L)) \dy\lesssim L\tau.
%\end{align*}
%Now we prove the lemma. 
%\begin{proof}
%We need to show
%\[
%\ve{F(x)-F(y)}_Y \le L\pa{1+\fc{\de n}{2\tau}} \ve{x-y}_X.
%\]
%\Wog $y=0$, $\Vol(\tau B_X)=1$. Denote 
%\bal
%M&=\tau B_{X}\bs (x+\tau B_X)\\
%M'&=(x+\tau B_X) \bs \tau B_X.
%\end{align*}%

%\ig{images/9-2}{.25}%

%We have
%\bal
%F(0)-F(x) &= \int_M F_z(y)\,dy - \int_{M'} F_z(y)\,dy.
%\end{align*}
%Define $\om(z)$ to be the Euclidean length of the interval $(z+\R x)\cap (\tau B_X)$. By Fubini,
%\[
%\int_{\Proj_{X^{\perp}} (\tau B_X)} \om(z) \,dz = \Vol_n(\tau B_X)=1.
%%intersection of projection.
%\]
%Denote
%\bal
%W&= \set{z\in \tau B_X}{(z+\R x)\cap (\tau B_X)\cap (x+\tau B_X)\ne \phi}\\
%N&= \tau B_X\bs W.
%\end{align*}
%Define $C:M\to M'$ a shift in direction $X$ on every fiber that maps the interval $(z+\R x)\cap M\to (z+\R x)\cap M'$. %

%\ig{images/9-3}{.25}%

%$C$ is a measure preserving transformation with
%\[
%\ve{z-C(z)}_X =\begin{cases}
%\ve{x}_X , &z\le N\\
%\om(z) \fc{\ve{x}_X}{\ve{x}_2},& z\in W\cap M.
%\end{cases}
%\]
%(In the second case we translate by an extra factor $\fc{\om(z)}{\ve{x}_2}$.)
%%(In the second case we add the total length $
%%C maps $M'$ to $M$.
%%do a clever change of variable differently in each fiber.
%Then 
%\bal
%\ve{F(0)-F(x)}_Y &=\ve{\int_M F_2(y)\dy - \int_{M'}F_2(y)\dy}_Y\\
%&= \ve{\int_M(F_2(y) - F_2(C(y)))\dy}_Y\\
%&\le \int_M L(\ve{y-C(y)}_X+\de)\dy\\
%&\le \int_M L (\ve{y-C(y)}_X + \de)\dy\\
%&=L\de \Vol(M) + L \int_M \ve{y-C(y)}_X\dy\\
%\int_M \ve{y-C(y)}_X\dy 
%%orth decomp but not unit vector, integrate the length multiply by norm of direction. jacobian.
%&=\int_{N}\ve{x}_X\dy + \int_{W\cap M}\fc{\om(y)\ve{x}_X}{\ve{x}_2}\dy\\
%&=\ve{x}_X \Vol(N) + \int_{\Proj(W\cap M)} \fc{\om(z) \ve{x}_X}{\ve{x}_2} \ve{x}_2\,dz&\text{orthogonal decomposition}\\
%&=\ve{x}_X \Vol(N) + \Vol(\tau B_X\bs N) \\
%&=\ve{x}_X \Vol(\tau B_X)=\ve{x}_X.
%\end{align*}
%%w as the entire length. What I get is the entire volume. 
%We show $M=\tau B_X\bs (x+\tau B_X) \subeq \tau B_X\bs (1-\fc{\ve{x}}{\tau}) \tau B_X$. Indeed, for $y\in M$,
%\bal
%\ve{y-x}_X&\ge \tau\\
%\ve{y} & \ge \tau - \ve{x} = \pa{1-\fc{\ve{x}}{\tau}}\tau.
%\end{align*}
%\end{proof}
%Then 
%\[
%\Vol(M) \le \Vol(\tau B_X)-\Vol\pa{\pa{1-\fc{\ve{x}}{\tau}}\tau B_X}=1-\pa{1-\fc{\ve{x}}{\tau}} \lesssim \fc{n\ve{x}}{\tau}
%\]
%\end{proof}
%Bourgain did it in a more complicated, analytic way avoiding geometry. Begun notices that careful geometry is sufficient.%
%

%Later we will show this theorem is sharp.%

%%iteration!%

%\section{Proof of Bourgain's discretization theorem}%

%At small distances there is no guarantee on the function $f$. Just taking derivatives is dangerous. It might be true that we can work with the initial function. But the only way Bourgain figured out how to prove the theorem was to make a 1-parameter family of functions.%

%\subsection{The Poisson semigroup}
%\begin{df}
%The \ivocab{Poisson kernel} is $P_t(x):\R^n\to \R$ given by
%\[
%P_t(x)=\fc{C_nt}{(t^2+\ve{x}_2^2)^{\fc{n+1}2}}, \quad C_n=\fc{\Ga\pf{n+1}2}{\pi^{\fc{n+1}2}}.
%\]
%%Convolution becomes product under FT
%\end{df}%

%\begin{pr}[Properties of Poisson kernel]
%\begin{enumerate}
%\item
%For all $t>0$, $\int_{\R^n} P_t(x)\,dx=1$.
%\item
%(Semigroup property) $P_t*P_s=P_{t+s}$. 
%\item
%$\wh{P_t}(x) = e^{-2\pi\ve{x}_2t}$.
%\end{enumerate}•
%\end{pr}%

%\begin{lem}
%Let $F$ be the function obtained from Bourgain's almost extension theorem~\ref{thm:baet}.
%For all $t>0$, $\ve{P_t*F}_{\text{Lip}}\lesssim 1$.
%%P_t is prob measure.
%%Average values of $F$.
%%poU, geometr y of ball, average with decaying weights.
%%all averaging of Lipschitz things.%

%%1+\ep subtlety, ceases to be true, need restriction on $T$. Not relevant. $1+\ep$ version do more carefully.
%\end{lem}
%%Note to get $P_t*F$ we had three averaging arguments: partition of unity, averaging with respect to a ball, and then averaging with decaying weights.
%We have 
%\bal
%P_t*F(x)-P_t*F(y) &= \int_{\R^n} P_t(z)(F(x-z) - F(x-y))\,dz.
%\end{align*}
%Our goal is to show there exists $t_0>0$, $x\in B\in \rc B_X$ such that if we define
%\[
%T=(P_{t_0}*F)'(x):X\to Y,
%\]
%we have $\ve{T}\lesssim 1$. Moreover $\ve{T^{-1}}\lesssim D$.
%$T_y=\lim_{h\to \iy} \fc{P_{t_0}*F(x+hy) - P_{t_0}*F(x)}{h}$.
%%pigeonhole, must exist