\blu{2-8-16}

%\begin{thm}[Spielman-Srivastava]
%For $k<\text{srank}(A)=\pf{\ve{A}_{S_2}^2}{\ve{A}_{S_{\iy}}}^2$, $k=(1-\ep)\text{srank}(A)$. 
%$\exists \si\subeq \{1,\ldots, m\}, |\si|=k$, ...
%%,/\op&2. 
%%constant prop of m/ ... squred. 
%%bourgain-tz up to stble rank.
%%shatten root m.
%\end{thm}

A subsequent theorem gave the same theorem but instead of the stable rank, used something bigger.
\begin{thm}[Marcus, Spielman, Srivastava]\llabel{thm:mss4}
If 
\[
k<\rc4 \pf{\ve{A}_{S_2}}{\ve{A}_{S_4}}^4,
\]
there exists $\si\subeq \{1,\ldots, m\}$, $|\si|=k$ such that
\[
\ve{(AJ_{\si})^{-1}}_{S_{\iy}} \lesssim \fc{\sqrt m}{\ve{A}_{S_2}}.
\]
%stable rank. 
%got strange norms.
\end{thm}
A lot of these norms started popping up in people's results. The correct generalization is the following.
\begin{df}
For $p>2$, define the \ivocab{stable $p$th rank} by \[\text{srank}_p(A)= \pf{\ve{A}_{S_2}}{\ve{A}_{S_p}}^{\fc{2p}{p-2}}.\]
\end{df}
\begin{exr}
Show that if $p\ge q>2$, then
\[
\text{srank}_p(A) \le \text{srank}_q(A).
\]
(Hint: Use H\"older's inequality.)
\end{exr}
\begin{proof}[Proof of Theorem~\ref{thm:gen-srank}]
Note that 
\bal
\ve{A}_{S_2}^2 & =\sum_{j=1}^m s_j(A)^2\\
&=\sum_{j=1}^{r-1} s_j(A)^2 + \sum_{j=r}^m s_j(A)^2\\
&\le (r-1)^{1-\fc 2p} \pa{\sum_{j=1}^{r-1} s_j(A)^p}^{\fc 2p} + \sum_{j=r}^m s_j(A)^2\\
&\le (r-1)^{1-\fc 2p} \ve{A}_{S_p}^2  + \sum_{j=1}^m s_j(A)^2\\
\sum_{j=r}^m s_j(A)^2 & \ge \ve{A}_{S_2}^2 \pa{1-\pf{r-1}{\text{srank}_p(A)}^{1-\fc2p}}\\
\ve{(AJ_\si)^{-1}}&\lesssim\min_{k+1\le r\le \rank(A)} \sfc{mr}{(r-k)\ve{A}_{S_2}^2 \pa{1-\pf{r-1}{\text{srank}_p(A)}^{1-\fc 2p}}}\\
&=\fc{\sqrt m}{\ve{A}_{\iy}}\min_{k+1\le r\le \rank(A)} \sfc{r}{(r-k)\pa{1-\pf{r-1}{\text{srank}_p(A)}^{1-\fc 2p}}}
\end{align*}
\end{proof}
To optimize, fixing the stable rank, differentiate in $r$, and set to 0. All theorems in the literature follow from this theorem; in particular, we get all the bounds we got before. %3, 2.1.
There was nothing special about the number 4 in Theorem~\ref{thm:mss4}; this is about the distribution of the eigenvalues. 

We'll be doing linear algebra. It's mostly mechanical, except we'll need this lemma.
\begin{lem}[Ky Fan maximum principle]\index{Ky Fan maximum principle}
Suppose that $P:\R^m\to \R^m$ is a rank $k$ orthogonal projection. Then
\[
\Tr(A^*AP ) \le \sum_{i=1}^k s_i(A)^2.
\]
\end{lem}
\begin{proof}
\fixme{This proof isn't complete. I will fix it next time.}

We will prove that if $B:\R^m\to \R^m$ is positive semidefinite, then
\[
\Tr(BP)\le \sum_{i=1}^k s_i(B).
\]
To get the lemma, set $B=A^*A$. 

Apply arbitrarily small perturbation $s_i$ so that
\[
s_1(B)>s_2(B)>\cdots > s_m(B).
\]
Let $v_1,\ldots, v_m$ be an orthonormal basis such that $Bv_i=s_i(B) v_i$. Let $u_1,\ldots, u_k$ be an orthonormal basis of $P\R^m$ ordered so that 
\[
\an{Bu_1,u} \ge \an{Bu_2,u_2}\ge \cdots \ge \an{Bu_k,u_k}.
\]
We calculate
\bal
\Tr(BP) & =\sum_{i=1}^k  \an{BPu_i, u_i}\\
&=\sum_{i=1}^k \an{Bu_i,u_i}.
\end{align*}
We will prove by induction on $i$ that 
\[
\an{Bu_i,u_i}\le s_i(B).
\]
For $i=1$, 
\[
s_1(B)=\ve{B}_{S_{\iy}},
\]
and (when $\ve{u_1}=1$)
\[
\an{Bu_1,u_1} \le \ve{B}_{S_{\iy}} = s_1(B).
\]
Suppose we proved the claim for $j-1$. If \fixme{?$ \an{Bu_i,u_i}\le s_j(B)$} $\an{Bu_{j-1},u_{j-1}}\le s_j(B)$ then we're done because $\an{Bu_j,u_j}\le \an{Bu_{j-1},u_{j-1}} \le s_j(B)$. So we may assume that $\an{Bu_{j-1}, u_{j-1}}>s_j(B)$. 

Write the $u$'s in the basis of $v$'s:
\[
u_i = \sum_{l=1}^m c_{il}v_l.
\]
The fact that the $u_i$'s are orthonormal means that the $c_i$'s are probability vectors,
\[
\sum_{l=1}^m c_{il}^2=1.
\]
We have 
\[
\an{Bu_i,u_i} = \sumo lm m c_{il}^2 s_l(B).
\]
If $c_{il}^2>0$ for any $l\ge j$. 
%there can be no weight from $j$ onwards.
%u_j in span of $v_j$ upwards. Need for 1 to $j$. 
%1 to $u_{j-1}$. 
\end{proof}

\subsection{Finding big subsets}
We'll present 4 lemmas for finding big subsets with certain properties. We'll put them together at the end.
\begin{thm}[Little Grothendieck inequality]\llabel{thm:lgi}
Fix $k,m,n\in \N$. Suppose that $T:\R^m\to \R^n$ is a linear operator. Then for every $x_1,\ldots, x_k\in \R^m$,
\[
\sumo rk \ve{Tx_r}_2^2\le \fc{\pi}2\ve{T}_{\ell_{\iy}^m \to \ell_2^n}^2 \sumo rk x_{ri}^2
\]
for some $c\in \{1,\ldots, m\}$ where $x_r=(x_{r1},\ldots, x_{rm})$.
\end{thm}
Later we will show $\fc{\pi}2$ is sharp. 

If we had only 1 vector, what does this say?
\[
\ve{Tx_1}_2\le \sfc{\pi}2\ve{T}_{\ell_{\iy}^m \to \ell_2^n}\ve{X_1}_{\iy}
\]
%The definition of the operator norm is at most ... times the \iy$ norm
We know the inequality for $k=1$ with constant 1, and we get it true for arbitrary many vectors, losing an universal constant ($\pf{\pi}2$). After we see the proof, the example where $\fc{\pi}2$ is attained will be natural.

We give Grothendieck's original proof.

The key claim is the following.
\begin{clm}\llabel{clm:lgi}
\beq{eq:lgi1}
\sumo jm \pa{\sumo rk (T^*Tx_r)_j^2}^{\rc2}
\le \sfc{\pi}2 \ve{T}_{\ell_{\iy}^m \to \ell_2^n}\pa{\sum_{r=1}^k \ve{Tx_r}^2}^{\rc 2}.
\eeq
\end{clm}

\begin{proof}[Proof of Theorem~\ref{thm:lgi}]
Assuming the claim, we prove the theorem.
\bal
\sumo rk \ve{Tx_r}_2^2 & = \sumo rk \an{Tx_r,Tx_r}\\
&=\sumo rk \an{x_r, T^*Tx_r}\\
&=\sumo rk \sumo jm x_{rj}(T^*Tx_r)_j\\
&\le \sumo jm \pa{\sumo rk x_{rj}^2}^{\rc 2} \pa{\sumo rk (T^*Tx_r)_j^2}^{\rc 2}&\text{by Cauchy-Schwarz}\\
&\le \pa{\max_{1\le j\le m} \pa{\sumo rk x_{rj}^2}^{\rc 2}}
\pa{\sumo jm \sumo rk (T^*Tx_r)_j^2}^{\rc 2}\\
&\le \max_{1\le j\le m}\pa{\sumo ik x_{ij}^2}^{\rc 2}\sfc{\pi}2 \ve{T}_{\ell_{\iy}^m \to \ell_2^n} \pa{\sumo rk \ve{Tx_r}_2^2}^{\rc 2}\\
\sumo rk \ve{Tx_r}_2^2 & \le \fc{\pi}2 \ve{T}_{\ell_{\iy}^m \to \ell_2^n}^2 \max_j \sumo rk x_{ij}^2.
%bound by square root of multiple of same term, bootstrap.
\end{align*}
We bounded by a square root of the multiple of the same term, a bootstrapping argument. In the last step, divide and square.
%where we used Cauchy-Schwarz
\end{proof}

\begin{proof}[Proof of Claim~\ref{clm:lgi}]
Let $g_1,\ldots, g_k$ be iid standard Gaussian random variables. For every fixed $j\in \{1,\ldots, m\}$, 
\[
\sum_{r=1}^k g_r (T^* T x_r)_j.
\]
This is a Gaussian random variable with mean 0 and variance  %whatever the $L^2$ norm of these coefficients are
$\sumo rk (T^*Tx_r)_j^2$. Taking the expectation,
\bal
\E\ab{\sumo rk g_r(T^*T x_r)_j}
&= \pa{\sumo rk (T^*T x_r)_j^2}^{\rc 2} \sfc 2{\pi}.
\end{align*}
Sum these over $j$:
\bal
\E \ba{
\sumo jm \ab{T^* (\sumo rk g_r T x_r)_j}
} & = \sfc 2\pi \sumo jm \pa{\sumo rk (T^*Tx_r)_j^2}^{\rc 2}\\
\sumo jm \pa{\sumo rk (T^*Tx_r)_j^2}^{\rc 2}
&= \sfc{\pi}2 \E\ba{\sumo jm \ab{T^* \sumo rk g_r(Tx_r)_j}}.
\end{align*}
Define a random sign vector $z\in \{\pm 1\}^m$ by 
\[
z_j = \sign\pa{\pa{T^*\sumo rk g_r Tx_r}_j}
\]
Then 
\bal
\sumo jm \ab{(T^* \sumo rk g Tx_r)_j} 
&=\an{z, T^* \sumo rk g_r Tx_r}\\
&= \an{Tz,\sumo rk g_r Tx_r}\\
& \le \ve{Tz}_2 \ve{\sumo rk g_r Tx_r}_2\\
%l_\iy to l_2. all most whatever norm of operator is.
&\le \ve{T}_{\ell_{\iy}^m \to \ell_2^n} \ve{\sumo rk g_r Tx_r}_2
%expointed was bounded pointwise pointwise.
\end{align*}
This is a pointwise inequality. Taking expectations,
\bal
\E\ba{
\sum_{j=1}^m \ab{\pa{T^* \sumo rk g_r Tx_r}_j}
} & \le \ve{T}_{\ell_{\iy}^m \to \ell_2^n}\pa{\E \ve{\sumo rk g_r Tx_r}_2^2}^{\rc 2}.
\end{align*}
What is the second moment? Expand:
\bal
\E\ve{\sumo rk g_i Tx_r}_2^2 
&=\sumo rk \ve{Tx_r}_2^2=\E\ba{\sum_{ij} g_i g_j \an{Tx_i,Tx_j}}.
\end{align*}
%bound by L^2 norm above
\end{proof}

Why use the Gaussians? The identity characterizes the Gaussians using rotation invariance. %The expectation of the Gaussian 
%\sumo jm\sumo rk ... \sfc{\pi}2
%up to this piit use gaussians
Using other random variables gives other constants that are not sharp.

There will be lots of geometric lemmas. Some fact about restricting matrices. Another geometric argument to give  a different method for selecting subsets. Combinatorial lemma for selecting subsets. Put together in crazy induction.

From this proof you can reverse engineer vectors that make the inequality sharp. You need to come up with $T$ and the points.

Let $g_1,g_2,\ldots, $ be iid Gaussians on the probability space $(\Om, P)$. Let $T:L_{\iy}(\Om, P)\to \ell_2^k$ be %infinite $l^{\iy}$ space. 
%abstract nonsense: approx
%replace Gaussians with central limit theorem, take $\pm1$ bits.
%never equality for finite. (ratio of gamma functions)
%in limit converges
\[
Tf = (\E[fg_1],\ldots, \E[fg_m]).
\]
let $x_r \in L_{\iy} (\Om, P)$, 
\[
x_r = \fc{g_r}{\pa{\sumo ik g_i^2}^{\rc 2}}.
\]
%no black magic, just understand this.