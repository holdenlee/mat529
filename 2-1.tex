\chapter{The Ribe Program}

\blu{2/1/16}


\section{The Ribe Program}

The main motivation for most of what we will discuss is called the Ribe Program, which is a research program many hundreds of papers large. We will see some snapshots of it, and it all comes from a theorem from $1975$, \textbf{Ribe's rigidity theorem}~\ref{thm:ribe}, which we will state now and prove later in a modern way. This theorem was Martin Ribe's dissertation, which started a whole direction of mathematics, but after he wrote his dissertation he left mathematics. He's apparently a government official in Sweden. The theorem is in the context of Banach spaces; a relation between their linear structure and their structure as metric spaces. Now for some terminology. 

\begin{df}[Banach space]
A \vocab{Banach space} is a complete, normed vector space. Therefore, a Banach space is equipped with a metric which defines vector length and distances between vectors. It is complete, so every Cauchy sequence of converges to a limit defined inside the space. 
\end{df}


\begin{df}
Let $(X,\ved_X), (Y,\ved_Y)$ be Banach spaces. We say that $X$ is (crudely) \ivocab{finitely representable} in $Y$ if there exists some constant $K>0$ such that for every finite-dimensional linear subspace $F\subeq X$, there is a linear operator $S: F \to Y$ such that for every $x\in F$, 
\[
\ve{x}_X\le \ve{Sx}_Y\le K\ve{x}_X.
\]
%every finite dimensional subspace of $X$ is represented in $Y$.
\end{df}
Note $K$ is decided once and for all, before the subspace $F$ is chosen.

(Some authors use ``finitely representable" to mean that this is true for any $K=1+\ep$. We will not follow this terminology.)

Finite representability is important because %$X$ is finitely representable in $Y$  
it allows us to conclude that $X$ has the same finite dimensional linear properties (\ivocab{local properties}) as $Y$. That is, it preserves any invariant involves finitely many vectors, their lengths, etc.

Let's introduce some local properties like type. To motivate the definition, 
%For example, for type $p\ge 1$, take any $y_1,\ldots, y_n\in Y$. 
consider the triangle inequality, which says 
\[
\ve{y_1+\cdots +y_n}_Y\le \ve{y_1}_Y+\cdots +\ve{y_n}_Y.
\]
In what sense can we improve the triangle inequality? In $L^1$ this is the best you can say. In many spaces there are ways to improve it if you think about it correctly.

For any choice $\ep_1,\ldots, \ep_n\in \{\pm 1\}$, 
\[
\ve{\sum_{i=1}^n \ep_i y_i}_Y \le \sum_{i=1}^n \ve{y_i}_Y.
\]
\begin{df}\llabel{df:type}
Say that $X$ has \ivocab{type} $p$ if there exists $T>0$ such that for every $n, y_1,\ldots, y_n\in Y$, 
\[
\EE_{\ep\in \{\pm 1\}^n} \ve{\sum_{i=1}^n \ep_i y_i}_Y\le T\ba{\sum_{i=1}^n \ve{y_j}_Y^p}^{\rc p}.
\]
The $L^p$ norm is always at most the $L^1$ norm; if the lengths are spread out, this is asymptotically much better. Say $Y$ has \ivocab{nontrivial type} if $p>1$.
\end{df}

For example, $L_p(\mu)$ has type $\min(p,2)$.

Later we'll see a version of ``type" for metric spaces. How far is the triangle inequality from being an equality is a common theme in many questions. In the case of normed spaces, this controls a lot of the geometry. Proving a result for $p>1$ is hugely important.

%X innherits the local linear properties of $Y$.

\begin{pr}\llabel{pr:finrep-type}
If $X$ is finitely representable and $Y$ has type $p$ then also $X$ has type $p$.
\end{pr}

\begin{proof}
Let $x_1,\ldots, x_n\in X$. Let $F=\spn\{x_1,\ldots, x_n\}$. Finite representability gives me $S:F\to Y$. Let $y_i=Sx_i$. What can we say about $\sum \ep_iy_i$?
\bal
 \EE_{\ep}\ve{\sum_{i=1}^n \ep_iy_i}_Y &=  \EE_{\ep}\ve{S(\sui \ep_i x_i)}_Y\\
&\ge \EE_{\ep} \ve{\sui \ep_iX_i}_X\\
 \EE_{\ep}\ve{\sum_{i=1}^n \ep_iy_i}_Y &\le T\pa{\sui \ve{Sx_i}^p}^{\rc p}\\
& \le TK\pa{\sui \ve{x_i}^p}^{\rc p}.
%preserves any invariant involves finitely many vectors, their lengths, etc.
\end{align*}
Putting these two inequalities together gives the result.
\end{proof}

\begin{thm}[Kahane's inequality]\index{Kahane's inequality}
For any normed space $Y$ and $q\ge 1$, for all $n$, $y_1,\ldots, y_n\in Y$,
\[
\E\ve{\sum_{i=1}^n \ep_i y_i}\gtrsim_q \pa{
\E\ba{\ve{\sum_{i=1}^n \ep_i y_i}_Y^q}
}^{\rc q}.
\]
Here $\gtrsim_q$ means ``up to a constant"; subscripts say what the constant depends on. The constant here does not depend on the norm $Y$.
\end{thm}
Kahane's Theorem tells us that the LHS of Definition~\ref{df:type} can be replaced by any norm, if we change $\le$ to $\lesssim$. We get that having type $p$ is equivalent to 
\[
\E\ve{\sum_{i=1}^n \ep_i y_i}_Y^p \lesssim T^p \sum_{i=1}^n \ve{y_i}_Y^p.
\]
Recall the \ivocab{parallelogram identity} in a Hilbert space $H$:
\[
\E\ve{\sum_{i=1}^n \ep_i y_i}^2 = \sum_{i=1}^n \ve{y_i}_H^2.
\]
A different way to understand 
%not as good a way
the inequality in the definition of ``type" is: how far is a given norm from being an Euclidean norm? 
The \ivocab{Jordan-von Neumann Theorem} says that if  parallelogram identity holds then it's a Euclidean space. What happes if we turn it in an inequality?
\[
\E\ve{\sum_{i=1}^n \ep_i y_i}_H^2 \gle T\sum_{i=1}^n \ve{y_i}_H^2.
\]
Either inequality \emph{still} characterizes a Euclidean space. 

What happens if we add constants or change the power? We recover the definition for type and cotype (which has the inequality going the other way):
\[
\E\ve{\sum_{i=1}^n \ep_i y_i}_H^q \begin{array}{c}
\gtrsim\\
\lesssim
\end{array} \sum_{i=1}^n \ve{y_i}_H^q.
\]

\begin{df}
Say it has \ivocab{cotype $q$} if
\[
\sum_{i=1}^n \ve{y_i}_Y^q \lesssim C^q \E \ve{\sum_{i=1}^n \ep_i y_i}_Y^q
\]
\end{df}
R. C. James invented the local theory of Banach spaces, the study of geometry that involves properties involving finitely many vectors ($\forall x_1,\ldots, x_n, P(x_1,\ldots, x_n)$ holds). As a counterexample, reflexivity cannot be characterized using finitely many vectors (this is a theorem).

Ribe discovered link between metric and linear spaces.

First, terminology.

\begin{df}
Two Banach spaces are \ivocab{uniformly homeomorphic} if there exists $f:X\to Y$ that is 1-1 and onto and $f,f^{-1}$ are uniformly continuous. 
\end{df}

Without the word ``uniformly", if you think of the spaces as topological spaces, all of them are equivalent. Things become interesting when you quantify! ``Uniformly" means you're controlling the quantity.
\begin{thm}[Kadec]\index{Kadec's Theorem}
Any two infinite-dimensional separable Banach spaces are homeomorphic.
\end{thm}
%They are all the same as Hilbert spaces.
This is a amazing fact: these spaces are all topologically equivalent to Hilbert spaces!

Over time people people found more examples of Banach spaces that are homeomorphic but not uniformly homeomorphic. Ribe's rigidity theorem clarified a big chunk of what was happening.

\index{rigidity theorem}
\begin{thm}[Rigidity Theorem, Martin Ribe (1975)]\llabel{thm:ribe}
%government official in Sweden.
Suppose that $X,Y$ are uniformly homeomorphic Banach spaces.  Then $X$ is finitely representable in $Y$ and $Y$ is finitely representable in $X$.
\end{thm} 
For example, for $L^p$ and $L^q$, for $p\ne q$ it's always that case that one is not finitely representable in the other, and hence by Ribe's Theorem, $L^p,L^q$ are not uniformly homeomorphic.
%averages of sums with random signs
%squaring things
%$\ep$-uncorrelated, very linear
%we will deduce from this general phen that you can't deform $L_p$ into $L_q$ even in a nonlinear category.
%if you're equivalent in a weak category, you're equivalent in a stronger category.
(When I write $L_p$, I mean $L_p(\R)$.)

\begin{thm}
For every $p\ge 1,p\ne2$, $L_p$ and $\ell_p$ are finitely representable in each other, yet not uniformly homeomorphic.
\end{thm}
(Here $\ell_p$ is the sequence space.)

\begin{exr}
Prove the first part of this theorem: $L_p$ is finitely representable in $\ell_p$.
%take a finite basis, approximate
\end{exr}
Hint: approximate using step functions. You'll need to remember some measure theory.

When $p=2$, $L_p,\ell_p$ are separable and isometric.

The theorem in various cases was proved by:
\begin{enumerate}
\item
$p=1$: Enflo
\item
$1<p<2$: Bourgain
\item
$p>2$: Gorelik, applying the Brouwer fixed point theorem (topology)
\end{enumerate}
%can't deform function into sequence space in uniformly continuous way.

Every linear property of a Banach signs which is local (type, cotype, etc.; involving summing, powers, etc.) is preserved under a general nonlinear deformation.

After Ribe's rigidity theorem, people wondered: can we reformulate the local theory of Banach spaces without mentioning anything about the linear structure? 
Ribe's rigidity theorem is more of an existence statement, we can't see anything about an explicit dictionary which maps statements about linear sums into statements about metric spaces. So people started to wonder whether we could reformulate the local theory of Banach spaces, but only looks at distances between pairs instead of summing things up. Local theory is one of the hugest subjects in analysis. If you could actually find a dictionary which takes one linear theorem at a time, and restate it with only distances, there is a huge potential here! Because the definition of type only involves distances between points, we can talk about a metric space's type or cotype. So maybe we can use the intution given by linear arguments, and then state things for metric spaces which often for very different reasons remain true from the linear domain. And then now maybe you can apply these arguments to graphs, or groups! We could be able to prove things about the much more general metric spaces. Thus, we end up applying theorems on linear spaces in situations with \emph{a priori nothing} to do with linear spaces. This is massively powerful.


%uniform homomorphism
%In metric spaces, we are only allowed to discuss distances between points, not linear properties (ex. summing up). Suppose we can reformulate local theory in this way---find a dictionary that reformulates each linear property and theorem about linear properties as properties and theorems involving distances. Then we can state the analogous theorems for metric spaces. In particular, we can discuss when metric spaces have type and cotype. Maybe the theorems remain true---often they do, for different reasons. Now we can apply the theorem to graphs, groups, etc. %Somewhere
%transl concept, theorems, phenomenon fro linear spaces to metric spaces.
%Thus, we end up applying theorems on linear spaces in situations with \emph{a priori nothing} to do with linear spaces. This is massively powerful.
%massively powerful. 
%mysterious dictionary.

There are very crucial entries that are missing in the dictionary. We don't even now how to define many of the properties! %how to reformulate
This program has many interesting proofs. Some of the most interesting conjectures are how to define things!

\begin{cor}\llabel{cor:uh-type}
If $X,Y$ are uniformly homeomorphic and if one of them is of type $p$, then the other does. 
\end{cor}
This follows from Ribe's Theorem~\ref{thm:ribe} and Proposition~\ref{pr:finrep-type}. Can we prove something like this theorem without using Ribe's Theorem~\ref{thm:ribe}?
%going back to the abstract principle?
We want to reformulate the definition of type using only the distance, so this becomes self-evident.

Enflo had an amazing idea. 
Suppose $X$ is a Banach space, $x_1,\ldots, x_n\in X$. The type $p$ inequality  says 
\beq{eq:type-p}
\E\ba{\ve{\sum_{i=1}^n \ep_i x_i}^p} \lesssim_X \sum_{i=1}^n \ve{x_i}^p.
\eeq
Let's rewrite this in a silly way. Define $f:\{\pm 1\}^n\to X$ by
\[
f(\ep_1,\ldots, \ep_n)=\sum_{i=1}^n \ep_i x_i.
\]
Write $\ep=(\ep_1,\ldots, \ep_n)$. Multiplying by $2^n$, we can write the inequality~\eqref{eq:type-p} as
\beq{eq:type-gen}
\E\ba{
\ve{f(\ep) - f(-\ep)}^p
}\lesssim_X
\sum_{i=1}^n \E\ba{ \ve{f(\ep)-f(\ep_1,\ldots, \ep_{i-1}, -\ep_i, \ep_{i+1},\ldots, \ep_n)}^p}.
\eeq
This inequality just involves distances between points $f(\ep)$, so it is the reformulation we seek.

\begin{df}\llabel{df:enflo}
A metric space $(X,d_X)$ has \ivocab{Enflo type $p$} if there exists $T>0$ such that for every $n$ and every $f:\{\pm 1\}^n\to X$,
\[
\E[d_X(f(\ep), f(-\ep))^p]\le T^p \sum_{i=1}^n \E[ d_X(f(\ep), f(\ep_1,\ldots, \ep_{i-1}, -\ep_i, \ep_{i+1},\ldots, \ep_n))^p].
\]
\end{df}

This is bold. It wasn't true before for a general function! The discrete cube (Boolean hypercube) $\{\pm 1\}^n$ is all the $\epsilon$ vectors, of which there are $2^n$. Our function just assigns $2^n$ points arbitrarily. No structure whatsoever. As they are indexed this way, you see nothing. But you're free to label them by vertices of the cube however you want. But there are many labelings! In~\eqref{eq:type-gen}, the points had to be vertices of a cube, but in Definition~\ref{df:enflo}, they are arbitrary. The moment you choose the labelings, you impose a cube structure between the points. Some of them are diagonals of the cube, some of them are edges. $\epsilon$ and $-\epsilon$ are antipodal points. But it's not really a diagonal. They are points on a manifold, and are a function of how you decided to label them. What this sum says is that the sum over all diagonals, the length of the diagonals to the power $p$ is less than the sum over edges to the $p^{th}$ powers (these are the points where one $\epsilon_i$ is different). Thus we can see 
\[
\sum \diag^p\lesssim_X\sum \text{edge}^p.
\]
The total $p$th power of lengths of diagonals is up to a constant, at most the same thing over all edges.
%you should worry about this!
%The discrete cube is all these vectors with $\ep$. The function goes into the Banach space and assigns $2^n$ arbitrary points.
%In~\eqref{eq:type-gen}, the points had to be vertices of a cube, but in Definition~\ref{df:enflo}, they are arbitrary. The moment you choose the labelings, you impose a cube structure between the points. %on the cube.

\ig{images/1-1}{.25}


This is a vast generalization of type; we don't even know a Banach space satisfies this.
%This is a vast generalization
The following is one of my favorite conjectures.
\begin{conj}[Enflo]
If a Banach space has type $p$ then it also has Enflo type $p$.
\end{conj}
This has been open for 40 years. We will prove the following.
\begin{thm}[Bourgain-Milman-Wolfson, Pisier]
If $X$ is a Banach space of type $p>1$ then $X$ also has type $p-\ep$ for every $\ep>0$.
\end{thm}
If you know the type inequality for parallelograms, you get it for arbitrary sets of points, up to $\ep$.
Basically, you're getting arbitrarily close to $p$ instead of getting the exact result. 
We also know that the conjecture stated before is true for a lot of specific Banach spaces, though we do not yet have the general result. 
For instance, this is true for the $L_4$ norm.  Index functions by vertices; some pairs are edges, some are diagonals; then the $L^4$ norm of the diagonals is at most that of the edges.

How do you go from knowing this for a linear function to deducing this for an arbitrary function? 

Once you do this, you have a new entry in the hidden Ribe dictionary. If $X$ and $Y$ are uniformly homeomorphic Banach spaces and $Y$ has Enflo type $p$, then so is $X$. The minute you throw away the linear structure, Corollary~\ref{cor:uh-type} becomes easy. It requires a tiny argument. Now you can take a completely arbitrary function $f: \{\pm 1\}^n \to X$. There exists a homeomorphism $\psi: X \to Y$ such that $\psi, \psi^{-1}$ are uniformly continuous. Now we want to deduce that the same inequality in $Y$ gives the same inequality in $X$.  

\begin{pr}\llabel{pr:uh-enflo}
If $X,Y$ are uniformly homeomorphic Banach spaces and $Y$ has Enflo type $p$, then so does $X$.
\end{pr}

This is an example of making the abstract Ribe theorem explicit.
\begin{lem}[Corson-Klee]\index{Corson-Klee Lemma}\llabel{lem:corson-klee}
If $X,Y$ are Banach spaces and $\psi:X\to Y$ are uniformly continuous, then for every $a>0$ there exists $L(a)$ such that 
\[
\ve{x_1-x_2}_X\ge a \implies \ve{\psi(x_1)-\psi(x_2)}\le L\ve{x_1-x_2}.
%lower bound gives lipschitz
\]
\end{lem}

\begin{proof}[Proof sketch of~\ref{pr:uh-enflo} given Lemma~\ref{lem:corson-klee}]
%Given $f:\{\pm 1\}^n \to X$, 
By definition of uniformly homeomorphic, there exists a homeomorphism $\psi:X\to Y$ such that $\psi,\psi^{-1}$ are uniformly continuous. Lemma~\ref{lem:corson-klee} tells us that $\psi$ perserves distance up to a constant. Dividing so that the smallest nonzero distance you see is at least 1, we get the same inequality in the image and the preimage.
\end{proof}
%Then we know 
%distance preserved under constants.
%can rescale. 
%divide so smallest nonzer distance you see is at least 1. then it's the same inequality in the image and preimge.
%make Ribe's theorem explicit.
%version when I took power 1. Improvement of triangle inequlities
%don't know equivalences of this type in metric.
\begin{proof}[Proof details.]
Let $\ep^i$ denote $\ep$ with the $i$th coordinate flipped. We need to prove 
\[
\E(d_X(f(\ep), f(-\ep))^p) \le T_f^p \sui \E(d_X(f(\ep), f(\ep^i))^p)
\]
Without loss of generality, by scaling $f$ we may assume that all the points $f(\ep)$ are distance at least 1 apart. ($X$ is a Banach space, so distance scales linearly; this doesn't affect whether the inequality holds.)

Let $\psi:X\to Y$ be such that $\psi,\psi^{-1}$ are uniform homeomorphisms. Because $\psi^{-1}$ is uniformly homeomorphic, there is $C$ such that $d_Y(y_1,y_2)\le 1$ implies $d_X(\psi^{-1}(y_1),\psi^{-1}(y_2))< C$. WLOG, by scaling $f$ we may assume that all the points $f(\ep)$ are $\max(1,C)$ apart, so that the points $\psi\circ f(\ep)$ are at least 1 apart.

We know that for any $g:\{\pm 1\}^n\to Y$ that 
\[
\E(d_X(g(\ep), g(-\ep))^p) \le T_g^p \sui \E(d_X(g(\ep), g(\ep^i))^p).
\]
We apply this to $g=\psi\circ f$,
\[
\xymatrix{
& X\ar[dd]^{\psi}\\
\{\pm 1\}^n \ar[ru]^f \ar[rd]_{g=\psi\circ f}& \\
& Y
}
\]
to get
\bal
\E(d_X(f(\ep), f(-\ep))^p) &=\E(d_X(\psi^{-1}\circ g(\ep), \psi^{-1}\circ g(-\ep))^p)\\
&\le L_{\psi^{-1}}(1) \E(d_Y(g(\ep), g(-\ep))^p)\\
&\le L_{\psi^{-1}}(1) T_g^p \sui \E(d_Y(g(\ep), g(\ep^i)))\\
&\le L_{\psi^{-1}}(1)L_{\psi}(1) T_g^p \sui \E(d_X(g(\ep), g(\ep^i))^p)
\end{align*}
as needed.
\end{proof}

The parallelogram inequality for exponent 1 instead of 2 follows from using the triangle inequality on all possible paths for all paths of diagonals. Type $p>1$ is a strengthening of the triangle inequality. For which metric spaces does it hold?

%Fruits of Ribe's theorem.
%Switch gears to metric spaces.
What's an example of a metric space where the inequality doesn't hold with $p>1$? The cube itself (with $L^1$ distance).
\[
n^p\nleq n.
\]
I will prove to you that this is the only obstruction: given a metric space that doesn't contain bi-Lipschitz embeddings of arbitrary large cubes, the inequality holds. 

We know an alternative inequality involving distance equivalent to type; I can prove it. It is, however, not a satisfactory solution to the Ribe program. There are other situations where we have complete success.

We will prove some things, then switch gears, slow down and discuss Grothendieck's inequality and applications. They will come up in the nonlinear theory later.