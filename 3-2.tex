
\blu{3-2: Today we continue with Bourgain's Theorem. Summarizing our progress so far:
\begin{enumerate}
\item
Initially we had a function $f: N_{\delta} \to Y$ and a $\delta$-net $N_{\delta}$ of $B_X$.
\item
From $f$ we got a new function $F: X \to Y$ with $\txtn{supp}(F) \subseteq 3 B_X$, satisfying the following. (None of the constants matter.) 
\begin{enumerate}
\item
$\|F\|_{L_p} \le C$ for some constant.
\item 
For all $x \in N_{\delta}$, $\|F(x) - f(x)\|_Y \leq n\delta$. 
\end{enumerate}
\end{enumerate}
}

Let $P_t: \R^n \to \R$ be the Poisson kernel. What can be said about the convolution $P_t * F$? We know that for all $t$ and for all $x$, $\|(P_t * F)(x)\|_{X \to Y} \le 1$. The goal is to make this function invertible. More precisely, the norm of the inverted operator is not too small. 

We'll ensure this one direction at a time by investigating the directional derivatives. For $g: \R^n \to Y$, the directional derivative is defined by: for all $a \in S_X$, \[\partial_a g(x) = \lim_{t \to 0} \frac{g(x + ta) - g(x)}{t}.\]

There isn't any particular reason why we need to use the Poisson kernel; there are many other kernels which satisfy this. We \emph{definitely} need the semigroup property, in addition to all sorts of decay conditions. 

%There will be many lemmas we need to go through. 
We need a lot of lemmas.

\begin{lem}[Key lemma 1]\llabel{lem:bdt-1}
%Let $c, c', c''$ be universal constants which arise from the proof. 
There are universal constants $C,C',C''$ such that the following hold.
Suppose $t \in (0, \frac{1}{2}]$, $R \in (0, \infty)$. 
Let $\delta \in (0, \frac{1}{100n})$ satisfy 
\begin{align}
\delta &\leq C \frac{t\log(1/t)}{\sqrt{n}} \leq C' \frac{1}{n^{5/2}D^2}
\label{eq:bdt1-1}
\\
\label{eq:bdt1-2}
 C''n^{3/2}D^2\log(1/t) &\leq R \leq \frac{C''}{t\sqrt{n}}.
\end{align}
%
%
%\begin{enumerate}
%
%\item $ \delta \leq C \frac{t\log(1/t)}{\sqrt{n}} \leq c' \frac{1}{n^{5/2}D^2}$
%
%\item $ c''n^{3/2}D^2\log(1/t) \leq R \leq \frac{c''}{t\sqrt{n}}$
%
%\end{enumerate}

Then for all $x \in \frac{1}{2} B_X, a \in S_X$, we have
\[
\|\partial_a(P_t*F)\|_Y * P_{Rt}(x) \ge \frac{1}{D}
\]
\end{lem}
This result says that given a direction $a$ and look at how long the vector $\pl_a (P_t*F)$ is, it is not only large, but large on average. Here the average is over the time period $Rt$. 
%How large on average? 
%?
%You let the Poisson semigroup fight itself (flow in times $t$), you'll be at $Rt$. We want this to be long for every direction, but the average is also a function of $t$ (the measure with respect to averaging). So we have a directional derivative result with weird averaging: We flow an exponetially small amount of time, from this information you want to deduce the situation where we are invertible. 

The first condition says is that $t$ is not too small, but the lower bound is extremely small. The upper bound is $\prc{n}^{k}$ for some $k$ whereas the lower bound is something like $e^{-e^{n}}$, so there's a huge gap. %ridiculously small at $(e^n)^n$. So there's a huge gap.

The second condition says that $\log\prc{t}$ is still exponential. 


This lemma will come up later on and it will be clear. 

From now on, let us assume this key lemma and I will show you how to finish. We will go back and prove it later. 

\begin{lem}[Key lemma $2$]\llabel{lem:bdt-2}
There's a constant $C_1$ such that the following holds (we can take $C_1=8$). 
Let $\mu$ be any Borel probability measure on $S_X$. For every $R, A \in (0, \infty)$, there exists $ \frac{A}{(R + 1)^{m + 1}} \leq t \leq A$ such that 
\[
\int_{S_X} \int_{\R^n} \|\partial_a (P_t * F)(x)\|_Y dx d\mu(a) \leq \int_{S_X} \ub{\int_{\R^n} \|\partial_a(P_{(R + 1)t} * F)\|_Y dx}{(*)} d\mu(a) + \frac{C_1\txtn{vol}(3B_X)}{m}
\]
\end{lem}
Basically, under convolution, we can take the derivative under the integral sign. 
Thus (*) is an average of what we wrote on the left. Averaging only makes things smaller for norms, because norms are convex. Thus the  right integral is less than the left integral, from Jensen's inequality. 

The lemma says that  adding that small factor, we get a bound in the opposite direction. %If the norm is strongly convex, we would actually get that our average is a constant. %?
We will argue that there must be a scale at which Jensen's inequality stabilizes, i.e., $\|\partial_a(P_t*F)\|_Y*P_{Rt}(x)$ stabilizes to a constant.

Note this is really the pigeonhole argument, geometrically.

\begin{proof}
Bourgain uses this technique a lot in his papers: find some point where the inequality is equality, and then leverage that. 

If the result does not hold, then for every $t$ in the range $\frac{t}{(R + 1)^{m + 1}} \leq t \leq A$, the reverse holds. We'll use the inequality at eveyr point in a geometric series. For  all $k \in\{ 0, \ldots, m + 1\}$, 
\[
\int_{S_X} \int_{\R^n} \|\partial_a(P_{A(R + 1)^{k - m - 1}}*F)(x)\|_Ydxd\mu(a) > \int_{S_X} \int_{\R^n} \|\partial_a(P_{A(R + 1)^{k - m}}*F)(x)\|_Y dxd\mu(a) + \frac{C_1\txtn{vol}(3B_X)}{m}.
\]
%We basically are iterating this inequality many times, starting from the lower bound condition stated in the lemma. 
Summing up these inequalities and telescoping
\begin{equation}\label{eq:bdt-poisson1}
\int_{S_X} \int_{\R^n} \|\partial_a(P_{A(R + 1)^{-m - 1}}*F)(x)\|_Y dxd\mu(a) > \int_{S_X} \int_{\R^n} \|\partial_a(P_{A(R + 1)}*F)(x)\|_Y dxd\mu(a) + \frac{8(m + 1)\txtn{vol}(3B_X)}{m}
\eeq
(recall that we've been using the semigroup properties of the Poisson kernel this whole time). Now why is this conclusion absurd? 
Take $C_1$ to be the bound on the Lipschitz constant of $F$. Because $\|\partial_aF\|_Y \leq 8$, we have$\partial_a(P_{A(R+1)^{-m - 1}}*F)(x)\le C_1$. Since partial derivatives commute with the integral sign, we get 
\begin{align}
\int_{S_X}
\int_{\R^n} \|\partial_a(P_{A(R+1)^{-m - 1}}*F)(x)\|_Y \dx\,d\mu(a) &= \int_{S_X}\int_{\R^n} \|(P_{A(R+1)^{-m - 1}}*\partial_aF)(x)\|_Y \dx\,d\mu(a)\\
&\le \int_{S_X}\int_{\R^n} C_1\le C_1\vol(B_X)\label{eq:bdt-poisson2}
\end{align}
because  $F$ is Lipschitz and the Poisson semigroup integrates to unity. 
Together~\eqref{eq:bdt-poisson1} and~\eqref{eq:bdt-poisson2} give a contradiction.
\end{proof}

Now assuming Lemma~\ref{lem:bdt-1} (Key Lemma $1$), let's complete the proof of Bourgain's discretization theorem. Assume from now on $\delta < \left(\frac{1}{cD}\right)^{(CD)^{2n}}$ where $C$ is a large enough constant that we will choose $(C = 500$ or something). 

Let $\mathcal{F} \subseteq S_X$ be a $\frac{1}{C_2D}$-net in $S_X$. Then $|\mathcal{F}| \leq (C_3D)^n$ for some $C_3$. We will apply the Lemma~\ref{lem:bdt-2} (Key Lemma 2) with $\mu$ the uniform measure on $\mathcal{F}$, 
\begin{align*}A &= (1/CD)^{5n}\\
R + 1 &= (CD)^{4n}\\
m &= \lceil (CD)^{n + 1} \rceil.
\end{align*}
 Then there exists $(1/(CD))^{(CD)^{2n}} \leq t \leq (1/(CD))^{5n}$ such that  %Then we can find the $t$ such that this holds. 
\beq{eq:bdt-pf-lem2}
\sum_{a \in \mathcal{F}} \int_{\R^n} \|\partial_a(P_t *F)(x)\|_Y dx \leq \sum_{a \in \mathcal{F}} \int_{\R^n} \|\partial_a(P_{(R + 1)t}*F)(x)\|_Y dx + \frac{8\txtn{vol}(3B_X)}{m} |\mathcal{F}|
\eeq
We check the conditions of Lemma~\ref{lem:bdt-1} (Key Lemma $1$).
\begin{enumerate}
\item
For~\eqref{eq:bdt1-1}, note $t$ is exponentially small, so the RHS inequality is satisfied.
For the LHS inequality, note $\de \le t$ and $\fc{C\ln \prc{t}}{\sqrt n}\le 1$. To see the second inequality, note $\fc{C\ln \prc{t}}{\sqrt n} \ge \fc{C5n \ln (CD)}{\sqrt n}\ge 1$ (for $n$ large enough).
\item 
For~\eqref{eq:bdt1-1}, note the LHS is dominated by $\ln \prc{t} \le (CD)^{2n} \ln(CD)$ which is much less than $R=(CD)^{4n}-1$, and $\rc t$, the dominating term on the RHS, is $\ge (CD)^{5n}$.
%the RHS is doubly exponential.
\end{enumerate}•
%If $\delta$ satisfies $\delta <\prc{CD}^{(CD)^{2n}}=t$, then the conditions of Lemma~\ref{lem:bdt-1} (Key Lemma $1$) hold. The first condition must be bigger than $t$, and the upper bound on $t$ is exponential even. So we're fine. Now we check the bound on $\log(1/t)$, which is at most exponential in $D$. What is $R$? We made that exponential too, so there is room to spare. Finally, $1/t$ exceeds $R$ as well since $1/t$ is $(CD)^{5n}$ while $R$ was $(CD)^{4n}$. So we're good. 

In my paper \fixme{(reference?)}, I wrote what the exact constants are. They're written in the paper, and are not that important. We're choosing our constants big enough so that our inequalities hold true with room to spare. 

Now we can use Key Lemma $1$, which says
\beq{eq:bdt-pf-lem1}
\left(\|\partial_a(P_t*F)\|_Y * P_{Rt}\right)(x)\ge \rc{D}
\eeq
 %Look at the norm of the derivative in direction $a$ $\|\partial_a(P_{(R + 1)t}*F)(x)\|_Y$. Note that 
Using $P_{(R + 1)t} = P_{Rt} * P_t$ (semigroup property) and Jensen's inequality on the norm, which is a convex function, we have 
\beq{eq:bdt-conv}
\|\partial_a(P_{(R + 1)t}*F)(x)\|_Y = \|\left(\partial_a(P_t*F)\right)*P_{Rt}(x)\|_Y \leq \left(\|\partial_a(P_t*F)\|_Y * P_{Rt}\right)(x).
\eeq
Since the norm is a convex function, by Jensen's we get our inequality. %Note that we are using semigroup properties liberally. 

Let 
\[
\psi(x):=\left(\|\partial_a(P_t*F)\|_Y * P_{Rt}\right)(x) - \|\partial_a\left(P_{(R + 1)t}*F\right)(x)\|_Y
\]
From~\eqref{eq:bdt-conv}, $\psi(x)\ge 0$ pointwise. 
%Think of this experssion above as a random variable. It's positive on half the ball. So we can use Markov's inequality. So you get that the volume over all the $x \in \frac{1}{2}B_X$ is 
Using Markov's inequality,
\[
\vol\set{x\in \rc{2}B_X}{\psi(x)>\rc D}\leq D\int_{\R^n} \left(\|\partial_a(P_t*F)\|_Y * P_{Rt}\right)(x) - \|\partial_a\left(P_{(R + 1)t}*F\right)(x)\|_Y ) dx.
%\txt{vol}\left(x \in \frac{1}{2}B_X : \left(\|\partial_a(P_t*F)\|_Y * P_{Rt}\right)(x) - \|\partial_a\left(P_{(R + 1)t}*F\right)(x)\|_Y  > 1/D\right)
\]
%So for every $x$, the points such that the measure is bigger than $1/D$ is bounded by 
%\[
%\leq D\int_{R^n} \left(\|\partial_a(P_t*F)\|_Y * P_{Rt}\right)(x) - \|\partial_a\left(P_{(R + 1)t}*F\right)(x)\|_Y ) dx
%\]
because we can upper bound the integral over the ball by an integral over $\R^n$. 
Note we are using the probabilistic method. %, since we are only proving the existence of such a point.

This inequality was for a fixed $a$. We now use the union bound over $\cal F$, a $\de$-net of $a$'s to get
%We want to use the union bound. Previously we were doing things for fixed $a$, now we want to do things for every $a$. Let's call
%\[
%\psi(x) = \left(\|\partial_a(P_t*F)\|_Y * P_{Rt}\right)(x) - \|\partial_a\left(P_{(R + 1)t}*F\right)(x)\|_Y 
%\]
%Then rewriting
\begin{align}
&\quad \txtn{vol}\set{x \in \frac{1}{2}B_X}{\exists a \in \mathcal{F}, \psi_a(x) > 1/D}\\ &\leq \sum_{a \in \mathcal{F}} \txtn{vol}(x \in \frac{1}{2}B_X: \psi_a(x) > 1/D)\\
&\leq D\sum_{a \in \mathcal{F}} \int_{\R^n}\left(\|\partial_a(P_t*F)\|_Y * P_{Rt})(x) - \|\partial_a(P_{(R + 1)t} * F)(x)\|_Y\right) dx\\
&=D\sum_{a \in \mathcal{F}} \int_{\R^n}\left(\|\partial_a(P_t*F)\|_Y)(x) - \|\partial_a(P_{(R + 1)t} * F)(x)\|_Y\right) dx \label{eq:bdt-convo}\\
&\le \fc{C_1\vol(3B_X)}{m} (C_3D)^n < \vol\pa{\rc2B_X}.\label{eq:bdt-vol}
\end{align}
where~\eqref{eq:bdt-convo} follows because when you convolve something with a probability measure, the integral over $\R^n$ does not change, and~\eqref{eq:bdt-vol} follows from~\eqref{eq:bdt-pf-lem2} and our choice of $m$.

%We know that when you average pointwise, the first term in the subtraction is bigger than $1/D$. 

%Convoling with a probability measure does not change anything, so we can get rid of $P_{Rt}$. Then, 
%\[
%= D\sum_{a \in \mathcal{F}} \int_{\R^n}\left(\|\partial_a(P_t*F)\|_Y)(x) - \|\partial_a(P_{(R + 1)t} * F)(x)\|_Y\right) dx \leq \frac{8\txtn{vol}(3B_X)}{m}(10D)^n < \txtn{vol}(\frac{1}{2}B_X)
%\]
%by our choice of $m$ and since stabilization made this less than one. 

%This is very technical, but we 
We've proved that there must exist a point in half the ball such that for every $a \in \mathcal{F}$, our net, we have
\[
\frac{1}{D} \geq \|\partial_a(P_t *F)\|_{Y} * P_{Rt}(x) - \|\partial_a(P_{(R + 1)t}*F)(x)\|_Y
\]
\fixme{I think we want either this to be $\rc{2D}$, or~\eqref{eq:bdt-pf-lem1} to be $\rc{D/2}$, in order to get the following bound (with $\rc{2D}$ or $\rc{D}$). This involves changing some constants in the proof.}

%So we got that the for the \emph{average}, the $P_{Rt}$ goes away. 
From this we can conclude that for this $x$,
\[
\|\partial_a(P_{(R + 1)t} * F)(x)\|_Y \ge 1/D
\]
You should just look at what the exact constants are in the paper. 

%%NOTE: can include the following once it's fixed

%The crucial point was that we got something to stabilize using the pidgeonhole argument, and you got a difference where one of the terms is pointwise big, which causes the term you're subtracting from it to also be pointwise big. So now we choose $T$ a linear mapping to be
%\[
%T := (P_{(R + 1)t} * F)'(x)
%\]
%where $x$ is the special point from the probabilistic argument. Now we can check what happens. Now we know $\|T\| \leq 8$, and for every $a \in \mathcal{F}$ in the net, $\|Ta\| \ge b/D$ for some constant $b$. 
%The main key here is that we need to show for \textbf{every} point in the net that $\|\partial_a (P_t * F)\|_Y$ is good. You already get that this is big for one particular direction just from $\|\partial_a (P_t * F)\|_Y * P_{Rt}(x) \ge 1/D$. But there are exponentially many points in the net which we need to show this for. 
%
%Take any $z \in S_X$, find $a \in \mathcal{F}$ s.t. $\|a - z\| < \frac{1}{100D}$. Then $\|Tz\| \geq \|Ta\| - \|T(a - z)\| \geq \|Ta\| - \|T\|\cdot\|az\| > \frac{b}{D} - 8\|a - t\| \geq 1/D$.  Then $\|T^{-1}\| \leq D$. \fixme{Fix this explanation a bit more? A bit confusing}

Note this is very much a probabiblistic existence statement or result. %Now you write down what it means. Suppose for contradiction that the probability is not small enough. The reason you succeeded to 
%We bounded the probability using a proof by contradiction. 
Usually we estimate by hand the random variable we want to care about. Here we want to prove an existential bound, so we estimate the probability of the bad case. But we estimate the probability of the bad case using another proof by contradiction.
%, and we assume that it doesn't work, so estimating probability here is also by contradiction! 

It remains to estimate some integrals. 

%\step{2} Extend $F_1$ to the whole space to $F_2$ such that 
%\begin{enumerate}
%\item
%$\forall x\in \cal N_\de$, $\ve{F_2(x)-f(x)}_Y \lesssim L\de$.
%\item
%$\ve{F_2(x)-F_2(y)}_Y\lesssim L(\ve{x-y}_X + \de)$.
%%not smooth yet. 
%\item
%$\Supp(F_2)\subeq 2B_X$.
%\item
%$F_2$ is smooth.
%%F_1 had no bounded continuity, but is a sum against a partition of unity. 
%%just smooth without any bounds fine.
%%spiky.
%%when I say not smooth, I mean no bounds.
%%I need the norm to be smooth for this.
%\end{enumerate}•%

%Denote $\al(t)=\max\{1-|1-t|,0\}$. %

%\ig{images/9-1}{.25}%

%Let
%\[
%F_2(x)=\al(\ve{x}_X) F_1\pa{x}{\ve{x}_X}.
%\]
%%0 the moment it passes 2.
%$F_2$ still satisfies condition 1. As for condition 2, 
%\bal
%\ve{F_2(x)-F_2(y)}_Y &= \ve{\al (\ve{x}_X)F_1\pf{x}{\ve{x}_X} - \al(\ve{y}_X) F_1\pf{y}{\ve{y}_X}} \\
%&\le |\al(\ve{x})-\al(\ve{y})|\ub{\ve{F_1\pf{x}{\ve{x}_X}}}{\le 2L}+\al(\ve{y})\ve{F_1\pf{x}{\ve{x}_X} - F_1\pf{y}{\ve{y}_X} }\\
%&\le (\ve{x}-\ve{y})2L + \al(\ve{y}) L\pa{
%\ve{\nv{x}-\nv{y}}+4\de 
%} \\
%&\le 2L\ve{x-y}+L\al(\ve{y}) \pa{\ve{x}\ab{\rc{\ve{x}}-\rc{\ve{y}}} + \fc{\ve{x-y}}{\ve{y}} + 4\de}\\
%&\le 2L\ve{x-y} +L\al(\ve{y}) \pa{\fc{\ve{x-y}}{\ve{y}} + \fc{\ve{x-y}}{\ve y}  + 4\de}\\
%&\lesssim L(\ve{x-y}+\de),
%%mult by 4de, use bounded by 1
%\end{align*}
%where in the last step we used $\al(\ve{y})\le \ve{y}$ and $\al(\ve{y})\le 1$. %

%Note $F_2$ is smooth because the sum for $F_1$ was against a partition of unity and $\ved_X$ is smooth, although we don't have uniform bounds on smoothness for $F_2$.
%%F_1 had no bounded continuity, but is a sum against a partition of unity. 
%%just smooth without any bounds fine.
%%spiky.
%%when I say not smooth, I mean no bounds.
%%I need the norm to be smooth for this.%

%%For the next step we need the following. %

%\step{3} We make $F$ smoother by convolving.
%\begin{lem}[Begun, 1999]
%Let $F_2:X\to Y$ satisfy $\ve{F_2(x)-F_2(y)}_Y\le L(\ve{x-y}_X+\de)$. Let $\tau \ge c\de$. Define 
%\[
%F(x) = \rc{\Vol(\tau B_X)}\int_{\tau B_X} F_2(x+y)\,dy.
%\]
%Then 
%\[
%\ve{F}_{\text{Lip}} \le L\pa{1+\fc{\de n}{2\tau}}.
%\]
%\end{lem}
%The lemma proves the almost extension theorem as follows. We passed from $f:\cal N_\de\to Y$ to $F_1$ to $F_2$ to $F$. 
%If $x\in \cal N_\de$, 
%\bal
%\ve{F(x)-f(x)}_Y &=\ve{
%\rc{\Vol(\tau B_X)} \int_{B_X} (F_2(x+y) - f(x))\,dy
%}\\
%&\le \rc{\Vol(\tau B_X)}\int_{\tau B_X}\ve{F_2(x+y)-F_2(x)}_Y + \ub{\ve{F_2(x)-f(x)}_Y}{\de L} \dy\\
%&\le \rc{\Vol(\tau B_X)}\int_{\tau B_X}(L(\ub{\ve{y}_X}{\le\tau}+\de L)) \dy\lesssim L\tau.
%\end{align*}
%Now we prove the lemma. 
%\begin{proof}
%We need to show
%\[
%\ve{F(x)-F(y)}_Y \le L\pa{1+\fc{\de n}{2\tau}} \ve{x-y}_X.
%\]
%\Wog $y=0$, $\Vol(\tau B_X)=1$. Denote 
%\bal
%M&=\tau B_{X}\bs (x+\tau B_X)\\
%M'&=(x+\tau B_X) \bs \tau B_X.
%\end{align*}%

%\ig{images/9-2}{.25}%

%We have
%\bal
%F(0)-F(x) &= \int_M F_z(y)\,dy - \int_{M'} F_z(y)\,dy.
%\end{align*}
%Define $\om(z)$ to be the Euclidean length of the interval $(z+\R x)\cap (\tau B_X)$. By Fubini,
%\[
%\int_{\Proj_{X^{\perp}} (\tau B_X)} \om(z) \,dz = \Vol_n(\tau B_X)=1.
%%intersection of projection.
%\]
%Denote
%\bal
%W&= \set{z\in \tau B_X}{(z+\R x)\cap (\tau B_X)\cap (x+\tau B_X)\ne \phi}\\
%N&= \tau B_X\bs W.
%\end{align*}
%Define $C:M\to M'$ a shift in direction $X$ on every fiber that maps the interval $(z+\R x)\cap M\to (z+\R x)\cap M'$. %

%\ig{images/9-3}{.25}%

%$C$ is a measure preserving transformation with
%\[
%\ve{z-C(z)}_X =\begin{cases}
%\ve{x}_X , &z\le N\\
%\om(z) \fc{\ve{x}_X}{\ve{x}_2},& z\in W\cap M.
%\end{cases}
%\]
%(In the second case we translate by an extra factor $\fc{\om(z)}{\ve{x}_2}$.)
%%(In the second case we add the total length $
%%C maps $M'$ to $M$.
%%do a clever change of variable differently in each fiber.
%Then 
%\bal
%\ve{F(0)-F(x)}_Y &=\ve{\int_M F_2(y)\dy - \int_{M'}F_2(y)\dy}_Y\\
%&= \ve{\int_M(F_2(y) - F_2(C(y)))\dy}_Y\\
%&\le \int_M L(\ve{y-C(y)}_X+\de)\dy\\
%&\le \int_M L (\ve{y-C(y)}_X + \de)\dy\\
%&=L\de \Vol(M) + L \int_M \ve{y-C(y)}_X\dy\\
%\int_M \ve{y-C(y)}_X\dy 
%%orth decomp but not unit vector, integrate the length multiply by norm of direction. jacobian.
%&=\int_{N}\ve{x}_X\dy + \int_{W\cap M}\fc{\om(y)\ve{x}_X}{\ve{x}_2}\dy\\
%&=\ve{x}_X \Vol(N) + \int_{\Proj(W\cap M)} \fc{\om(z) \ve{x}_X}{\ve{x}_2} \ve{x}_2\,dz&\text{orthogonal decomposition}\\
%&=\ve{x}_X \Vol(N) + \Vol(\tau B_X\bs N) \\
%&=\ve{x}_X \Vol(\tau B_X)=\ve{x}_X.
%\end{align*}
%%w as the entire length. What I get is the entire volume. 
%We show $M=\tau B_X\bs (x+\tau B_X) \subeq \tau B_X\bs (1-\fc{\ve{x}}{\tau}) \tau B_X$. Indeed, for $y\in M$,
%\bal
%\ve{y-x}_X&\ge \tau\\
%\ve{y} & \ge \tau - \ve{x} = \pa{1-\fc{\ve{x}}{\tau}}\tau.
%\end{align*}
%\end{proof}
%Then 
%\[
%\Vol(M) \le \Vol(\tau B_X)-\Vol\pa{\pa{1-\fc{\ve{x}}{\tau}}\tau B_X}=1-\pa{1-\fc{\ve{x}}{\tau}} \lesssim \fc{n\ve{x}}{\tau}
%\]
%\end{proof}
%Bourgain did it in a more complicated, analytic way avoiding geometry. Begun notices that careful geometry is sufficient.%
%

%Later we will show this theorem is sharp.%

%%iteration!%

%\section{Proof of Bourgain's discretization theorem}%

%At small distances there is no guarantee on the function $f$. Just taking derivatives is dangerous. It might be true that we can work with the initial function. But the only way Bourgain figured out how to prove the theorem was to make a 1-parameter family of functions.%

%\subsection{The Poisson semigroup}
%\begin{df}
%The \ivocab{Poisson kernel} is $P_t(x):\R^n\to \R$ given by
%\[
%P_t(x)=\fc{C_nt}{(t^2+\ve{x}_2^2)^{\fc{n+1}2}}, \quad C_n=\fc{\Ga\pf{n+1}2}{\pi^{\fc{n+1}2}}.
%\]
%%Convolution becomes product under FT
%\end{df}%

%\begin{pr}[Properties of Poisson kernel]
%\begin{enumerate}
%\item
%For all $t>0$, $\int_{\R^n} P_t(x)\,dx=1$.
%\item
%(Semigroup property) $P_t*P_s=P_{t+s}$. 
%\item
%$\wh{P_t}(x) = e^{-2\pi\ve{x}_2t}$.
%\end{enumerate}•
%\end{pr}%

%\begin{lem}
%Let $F$ be the function obtained from Bourgain's almost extension theorem~\ref{thm:baet}.
%For all $t>0$, $\ve{P_t*F}_{\text{Lip}}\lesssim 1$.
%%P_t is prob measure.
%%Average values of $F$.
%%poU, geometr y of ball, average with decaying weights.
%%all averaging of Lipschitz things.%

%%1+\ep subtlety, ceases to be true, need restriction on $T$. Not relevant. $1+\ep$ version do more carefully.
%\end{lem}
%%Note to get $P_t*F$ we had three averaging arguments: partition of unity, averaging with respect to a ball, and then averaging with decaying weights.
%We have 
%\bal
%P_t*F(x)-P_t*F(y) &= \int_{\R^n} P_t(z)(F(x-z) - F(x-y))\,dz.
%\end{align*}
%Our goal is to show there exists $t_0>0$, $x\in B\in \rc B_X$ such that if we define
%\[
%T=(P_{t_0}*F)'(x):X\to Y,
%\]
%we have $\ve{T}\lesssim 1$. Moreover $\ve{T^{-1}}\lesssim D$.
%$T_y=\lim_{h\to \iy} \fc{P_{t_0}*F(x+hy) - P_{t_0}*F(x)}{h}$.
%%pigeonhole, must exist