\def\filepath{templates}
%\def\filepath{C:/Users/holden-lee/Dropbox/Math/templates}

\input{\filepath/packages_book.tex}
\input{\filepath/theorems_with_boxes.tex}
\input{\filepath/macros.tex}
\input{\filepath/formatting.tex}
\input{\filepath/other.tex}
\input{\filepath/theorem_num.tex}

\def\name{Metric embeddings and geometric inequalities}


\pagestyle{fancy}
%\addtolength{\headwidth}{\marginparsep} %these change header-rule width
%\addtolength{\headwidth}{\marginparwidth}
\lhead{MAT529}
\chead{} 
\rhead{Metric embeddings and geometric inequalities} 
\lfoot{} 
\cfoot{\thepage} 
\rfoot{} % !! Remember to change the problem set number
\renewcommand{\headrulewidth}{.3pt} 
%\renewcommand{\footrulewidth}{.3pt}
\setlength\voffset{0in}
%\setlength\textheight{648pt}

\begin{document}
%\input{\filepath/titlepage.tex}
%\maketitle
\title{Metric embeddings and geometric inequalities}
\author{Lectures by Professor Assaf Naor \\ Scribes: Holden Lee and Kiran Vodrahalli}
\maketitle


\tableofcontents

%\startcontents
%\printcontents{ }{-1}{}

\chapter*{Introduction}

%Princeton, Spring 2016. 

The topic of this course is geometric inequalities with applications to metric embeddings; and we are actually going to do things more general than metric embeddings: Metric geometry. Today I will roughly explain what I want to cover and hopefully start proving a first major theorem. 
The strategy for this course is to teach novel work. Sometimes topics will be covered in textbooks, but a lot of these things will be a few weeks old. There are also some things which may not have been even written yet. I want to give you a taste for what's going on in the field. Notes from the course I taught last spring are also available. 

One of my main guidelines in choosing topics will be topics that have many accessible open questions. I will mention open questions as we go along. I'm going to really choose topics that have proofs which are entirely self-contained. I'm trying to assume nothing. My intention is to make it completely clear and there should be no steps you don't understand. 

Now this is a huge area. I will explain some of the background today. I'm going to use proofs of some major theorems as excuses to see the lemmas that go into the proofs. Some of the theorems are very famous and major, and you'er going to see some improvements, but along the way, we will see some lemmas which are \textbf{immensely powerful}. So we will always be proving a concrete theorem. But actually somewhere along the way, there are lemmas which have wide applicability to many many areas. These are excuses to discuss methods, though the theorems are important. 

The course can go in many directions: If some of you have some interests, we can always change the direction of the course, so express your interests as we go along. 

%Notes from Assaf Naor's class ``Geometric inequalities with applications to metric embeddings" at 

%More generally we'll cover metric geometry.

%I'll teach things that just appeared or are still in the works. Most of them aren't in books yet; some of them haven't even be written yet. I'll give a taste of what's going on in the field. I'll choose topics with many open questions associated with them and with self-contained proofs. 

%This is a huge area. I'll use proofs of major theorems as excuses to see the lemmas that go into the proofs. The theorems are famous; in addition to being famous, they use lemmas that are immensely powerful. Instead of presenting the geometric inequalities and then applications afterwards, we'll see the lemmas on the way to proving these big theorems.

%twists in proofs new, useful.
%Express your interest---I'm flexible.

%I taught a course.
%no overlap.

%some commericals on what to prove later, puzzles.

%\setcounter{chapter}{1}
\chapter{Intro}

\blu{2/1/16}


\section{The Ribe Program}
Our main motivation is the \ivocab{Ribe Program}. The program is inspired from a theorem from 1975 called Ribe's Rigidity Theorem~\ref{thm:ribe}. The theorem is about Banach spaces, specifically a relationship between their linear structure and metric structure.

\begin{df}
Let $(X,\ved_X), (Y,\ved_Y)$ by Banach spaces. We say that $X$ is (crudely) \ivocab{finitely representable} in $Y$ if there exists $K>0$ such that for every finite-dimensional linear subspace $F\subeq X$, there is a linear operator $S:F\to Y$ such that for every $x\in F$, 
\[
\ve{x}_X\le \ve{Sx}_Y\le K\ve{x}_X.
\]
%every finite dimensional subspace of $X$ is represented in $Y$.
\end{df}
Note $K$ is decided once and for all, before the subspace $F$ is chosen.

(Some authors use ``finitely representable" to mean that this is true for any $K=1+\ep$. We will not follow this terminology.)

Finite representability is important because %$X$ is finitely representable in $Y$  
it allows us to conclude that $X$ has the same finite dimensional linear properties (\ivocab{local properties}) as $Y$. That is, it preserves any invariant involves finitely many vectors, their lengths, etc.

Let's introduce some local properties like type. To motivate the definition, 
%For example, for type $p\ge 1$, take any $y_1,\ldots, y_n\in Y$. 
consider the triangle inequality, which says 
\[
\ve{y_1+\cdots +y_n}_Y\le \ve{y_1}_Y+\cdots +\ve{y_n}_Y.
\]
In what sense can we improve the triangle inequality? In $L^1$ this is the best you can say. In many spaces there are ways to improve it if you think about it correctly.

For any choice $\ep_1,\ldots, \ep_n\in \{\pm 1\}$, 
\[
\ve{\sum_{i=1}^n \ep_i y_i}_Y \le \sum_{i=1}^n \ve{y_i}_Y.
\]
\begin{df}\llabel{df:type}
Say that $X$ has \ivocab{type} $p$ if there exists $T>0$ such that for every $n, y_1,\ldots, y_n\in Y$, 
\[
\EE_{\ep\in \{\pm 1\}^n} \ve{\sum_{i=1}^n \ep_i y_i}_Y\le T\ba{\sum_{i=1}^n \ve{y_j}_Y^p}^{\rc p}.
\]
The $L^p$ norm is always at most the $L^1$ norm; if the lengths are spread out, this is asymptotically much better. Say $Y$ has \ivocab{nontrivial type} if $p>1$.
\end{df}

For example, $L_p(\mu)$ has type $\min(p,2)$.

Later we'll see a version of ``type" for metric spaces. How far is the triangle inequality from being an equality is a common theme in many questions. In the case of normed spaces, this controls a lot of the geometry. Proving a result for $p>1$ is hugely important.

%X innherits the local linear properties of $Y$.

\begin{pr}\llabel{pr:finrep-type}
If $X$ is finitely representable and $Y$ has type $p$ then also $X$ has type $p$.
\end{pr}

\begin{proof}
Let $x_1,\ldots, x_n\in X$. Let $F=\spn\{x_1,\ldots, x_n\}$. Finite representability gives me $S:F\to Y$. Let $y_i=Sx_i$. What can we say about $\sum \ep_iy_i$?
\bal
 \EE_{\ep}\ve{\sum_{i=1}^n \ep_iy_i}_Y &=  \EE_{\ep}\ve{S(\sui \ep_i x_i)}_Y\\
&\ge \EE_{\ep} \ve{\sui \ep_iX_i}_X\\
 \EE_{\ep}\ve{\sum_{i=1}^n \ep_iy_i}_Y &\le T\pa{\sui \ve{Sx_i}^p}^{\rc p}\\
& \le TK\pa{\sui \ve{x_i}^p}^{\rc p}.
%preserves any invariant involves finitely many vectors, their lengths, etc.
\end{align*}
Putting these two inequalities together gives the result.
\end{proof}

\begin{thm}[Kahane's inequality]\index{Kahane's inequality}
For any normed space $Y$ and $q\ge 1$, for all $n$, $y_1,\ldots, y_n\in Y$,
\[
\E\ve{\sum_{i=1}^n \ep_i y_i}\gtrsim_q \pa{
\E\ba{\ve{\sum_{i=1}^n \ep_i y_i}_Y^q}
}^{\rc q}.
\]
Here $\gtrsim_q$ means ``up to a constant"; subscripts say what the constant depends on. The constant here does not depend on the norm $Y$.
\end{thm}
Kahane's Theorem tells us that the LHS of Definition~\ref{df:type} can be replaced by any norm, if we change $\le$ to $\lesssim$. We get that having type $p$ is equivalent to 
\[
\E\ve{\sum_{i=1}^n \ep_i y_i}_Y^p \lesssim T^p \sum_{i=1}^n \ve{y_i}_Y^p.
\]
Recall the \ivocab{parallelogram identity} in a Hilbert space $H$:
\[
\E\ve{\sum_{i=1}^n \ep_i y_i}^2 = \sum_{i=1}^n \ve{y_i}_H^2.
\]
A different way to understand 
%not as good a way
the inequality in the definition of ``type" is: how far is a given norm from being an Euclidean norm? 
The \ivocab{Jordan-von Neumann Theorem} says that if  parallelogram identity holds then it's a Euclidean space. What happes if we turn it in an inequality?
\[
\E\ve{\sum_{i=1}^n \ep_i y_i}_H^2 \gle T\sum_{i=1}^n \ve{y_i}_H^2.
\]
Either inequality \emph{still} characterizes a Euclidean space. 

What happens if we add constants or change the power? We recover the definition for type and cotype (which has the inequality going the other way):
\[
\E\ve{\sum_{i=1}^n \ep_i y_i}_H^q \begin{array}{c}
\gtrsim\\
\lesssim
\end{array} \sum_{i=1}^n \ve{y_i}_H^q.
\]

\begin{df}
Say it has \ivocab{cotype $q$} if
\[
\sum_{i=1}^n \ve{y_i}_Y^q \lesssim C^q \E \ve{\sum_{i=1}^n \ep_i y_i}_Y^q
\]
\end{df}
R. C. James invented the local theory of Banach spaces, the study of geometry that involves properties involving finitely many vectors ($\forall x_1,\ldots, x_n, P(x_1,\ldots, x_n)$ holds). As a counterexample, reflexivity cannot be characterized using finitely many vectors (this is a theorem).

Ribe discovered link between metric and linear spaces.

First, terminology.

\begin{df}
Two Banach spaces are \ivocab{uniformly homeomorphic} if there exists $f:X\to Y$ that is 1-1 and onto and $f,f^{-1}$ are uniformly continuous. 
\end{df}

Without the word ``uniformly", if you think of the spaces as topological spaces, all of them are equivalent. Things become interesting when you quantify! ``Uniformly" means you're controlling the quantity.
\begin{thm}[Kadec]\index{Kadec's Theorem}
Any two infinite-dimensional separable Banach spaces are homeomorphic.
\end{thm}
%They are all the same as Hilbert spaces.
This is a amazing fact: these spaces are all topologically equivalent to Hilbert spaces!

Over time people people found more examples of Banach spaces that are homeomorphic but not uniformly homeomorphic. Ribe's rigidity theorem clarified a big chunk of what was happening.

\index{rigidity theorem}
\begin{thm}[Rigidity Theorem, Martin Ribe (1975)]\llabel{thm:ribe}
%government official in Sweden.
Suppose that $X,Y$ are uniformly homeomorphic Banach spaces.  Then $X$ is finitely representable in $Y$ and $Y$ is finitely representable in $X$.
\end{thm} 
For example, for $L^p$ and $L^q$, for $p\ne q$ it's always that case that one is not finitely representable in the other, and hence by Ribe's Theorem, $L^p,L^q$ are not uniformly homeomorphic.
%averages of sums with random signs
%squaring things
%$\ep$-uncorrelated, very linear
%we will deduce from this general phen that you can't deform $L_p$ into $L_q$ even in a nonlinear category.
%if you're equivalent in a weak category, you're equivalent in a stronger category.
(When I write $L_p$, I mean $L_p(\R)$.)

\begin{thm}
For every $p\ge 1,p\ne2$, $L_p$ and $\ell_p$ are finitely representable in each other, yet not uniformly homeomorphic.
\end{thm}
(Here $\ell_p$ is the sequence space.) 
Proof of finite representability is a good exercise. You'll need to remember some measure theory.

When $p=2$, $L_p,\ell_p$ are separable and isometric.

The theorem in various cases was proved by:
\begin{enumerate}
\item
$p=1$: Enflo
\item
$1<p<2$: Bourgain
\item
$p>2$: Gorelik, applying the Brouwer fixed point theorem (topology)
\end{enumerate}
%can't deform function into sequence space in uniformly continuous way.

Every linear property of a Banach signs which is local (type, cotype, etc.; involving summing, powers, etc.) is preserved under a general nonlinear deformation.

After the theorem, people wondered: can we reformulate the local theory of Banach spaces without mentioning anything about the linear structure? 
%uniform homomorphism
In metric spaces, we are only allowed to discuss distances between points, not linear properties (ex. summing up). Suppose we can reformulate local theory in this way---find a dictionary that reformulates each linear property and theorem about linear properties as properties and theorems involving distances. Then we can state the analogous theorems for metric spaces. In particular, we can discuss when metric spaces have type and cotype. Maybe the theorems remain true---often they do, for different reasons. Now we can apply the theorem to graphs, groups, etc. %Somewhere
%transl concept, theorems, phenomenon fro linear spaces to metric spaces.
Thus, we end up applying theorems on linear spaces in situations with \emph{a priori nothing} to do with linear spaces. This is massively powerful.
%massively powerful. 
%mysterious dictionary.

There are very crucial entries that are missing in the dictionary. We don't even now how to define many of the properties! %how to reformulate
This program has many interesting proofs. Some of the most interesting conjectures are how to define things!

\begin{cor}\llabel{cor:uh-type}
If $X,Y$ are uniformly homeomorphic and if one of them is of type $p$, then the other does. 
\end{cor}
This follows from Ribe's Theorem~\ref{thm:ribe} and Proposition~\ref{pr:finrep-type}. Can we prove something like this theorem without using Ribe's Theorem~\ref{thm:ribe}?
%going back to the abstract principle?
We want to reformulate the definition of type using only the distance, so this becomes self-evident.

Enflo had an amazing idea. 
Suppose $X$ is a Banach space, $x_1,\ldots, x_n\in X$. The type $p$ inequality  says 
\beq{eq:type-p}
\E\ba{\ve{\sum_{i=1}^n \ep_i x_i}^p} \lesssim_X \sum_{i=1}^n \ve{x_i}^p.
\eeq
Let's rewrite this in a silly way. Define $f:\{\pm 1\}^n\to X$ by
\[
f(\ep_1,\ldots, \ep_n)=\sum_{i=1}^n \ep_i x_i.
\]
Write $\ep=(\ep_1,\ldots, \ep_n)$. Multiplying by $2^n$, we can write the inequality~\eqref{eq:type-p} as
\beq{eq:type-gen}
\E\ba{
\ve{f(\ep) - f(-\ep)}^p
}\lesssim_X
\sum_{i=1}^n \E\ba{ \ve{f(\ep)-f(\ep_1,\ldots, \ep_{i-1}, -\ep_i, \ep_{i+1},\ldots, \ep_n)}^p}.
\eeq
This inequality just involves distances between points $f(\ep)$, so it is the reformulation we seek.

\begin{df}\llabel{df:enflo}
A metric space $(X,d_X)$ has \ivocab{Enflo type $p$} if there exists $T>0$ such that $n$ and every $f:\{\pm 1\}^n\to X$,
\[
\E[d_X(f(\ep), f(-\ep))^p]\le T^p \sum_{i=1}^n \E[ d_X(f(\ep), f(\ep_1,\ldots, \ep_{i-1}, -\ep_i, \ep_{i+1},\ldots, \ep_n))].
\]
\end{df}
This is bold. 
%you should worry about this!
%The discrete cube is all these vectors with $\ep$. The function goes into the Banach space and assigns $2^n$ arbitrary points.
In~\eqref{eq:type-gen}, the points had to be vertices of a cube, but in Definition~\ref{df:enflo}, they are arbitrary. The moment you choose the labelings, you impose a cube structure between the points. %on the cube.

\ig{images/1-1}{.25}

The inequality says:
\[
\sum \diag^p\lesssim_X\sum \text{edge}^p.
\]
The total $p$th power of lengths of diagonals is up to a constant, at most the same thing over all edges.

This is a vast generalization of type; we don't even know a Banach space satisfies this.
%This is a vast generalization
The following is one of my favorite conjectures.
\begin{conj}[Enflo]
If a Banach space has type $p$ then it also has Enflo type $p$.
\end{conj}
This has been open for 40 years. We will prove the following.
\begin{thm}[Bourgain-Milman-Wolfson, Pisier]
If $X$ is a Banach space of type $p>1$ then $X$ also has type $p-\ep$ for every $\ep>0$.
\end{thm}
If you know the type inequality for parallelograms, you get it for arbitrary sets of points, up to $\ep$.

The conjecture is true for many spaces. For example, it's true for $L^4$. Index functions by vertices; some pairs are edges, some are diagonals; then the $L^4$ norm of the diagonals is at most that of the edges.

%know just for linear functions, want to deduce for arbitrary functions.

The moment that I throw away the linear structure, Proposition~\ref{pr:uh-type} becomes easy.
\begin{pr}\llabel{pr:uh-enflo}
If $X,Y$ are uniformly homeomorphic Banach spaces and $Y$ has Enflo type $p$, then so does $X$.
\end{pr}

\begin{lem}[Corson-Klee]\index{Corson-Klee Lemma}\llabel{lem:corson-klee}
If $X,Y$ are Banach spaces and $\psi:X\to Y$ are uniformly continuous, then for every $a>0$ there exists $L(a)$ such that 
\[
\ve{x_1-x_2}_X\ge a \implies \ve{\psi(x_1)-\psi(x_2)}\le L\ve{x_1-x_2}.
%lower bound gives lipschitz
\]
\end{lem}

\begin{proof}[Proof of~\ref{pr:uh-enflo} given Lemma~\ref{lem:corson-klee}]
%Given $f:\{\pm 1\}^n \to X$, 
By definition of uniform homeomorphic, there exists a homeomorphism $\psi:X\to Y$ such that $\psi,\psi^{-1}$ are uniformly continuous. Lemma~\ref{lem:corson-klee} tells us that $\psi$ perserves distance up to a constant. Dividing so that the smallest nonzero distance you see is at least 1, we get the same inequality in the image and the preimage.
\end{proof}
%Then we know 
%distance preserved under constants.
%can rescale. 
%divide so smallest nonzer distance you see is at least 1. then it's the same inequality in the image and preimge.
%make Ribe's theorem explicit.
%version when I took power 1. Improvement of triangle inequlities
%don't know equivalences of this type in metric.

The parallelogram inequality for exponent 1 instead of 2 follows from using the triangle inequality on all possible paths for all paths of diagonals. Type $p>1$ is a strengthening of the triangle inequality. For which metric spaces does it hold?

%Fruits of Ribe's theorem.
%Switch gears to metric spaces.
What's an example of a metric space where the inequality doesn't hold with $p>1$? The cube itself (with $L^1$ distance).
\[
n^p\nleq n.
\]
I will prove to you that this is the only obstruction: given a metric space that doesn't contain bi-Lipschitz embeddings of arbitrary large cubes, the inequality holds. 

We know an alternative inequality involving distance equivalent to type; I can prove it. It is, however, not a satisfactory solution to the Ribe program. There are other situations where we have complete success.

We will prove some things, then switch gears, slow down and discuss Grothendieck's inequality and applications. They will come up in the nonlinear theory later.
\section{Bourgain's Theorem implies Ribe's Theorem}

\blu{2-3-16}

%Last time we stated Ribe's Theorem.

We will use the Corson-Klee Lemma~\ref{lem:corson-klee}. %This is the reason that all the inequalities.

%\fixme{Concrete versions of Ribe's Theorem. Charcterize using finie inequality, trivially preserved under uniform h. Minimum smallest distance, scale it to 1. Then the unif homeo is bi-Lipschitz. write the estimates.}
%The hardest part is to show that the property involving 
%

\begin{proof}[Proof of Lemma~\ref{lem:corson-klee}]
Suppose $x,y\in X$, $\ve{x-y}\ge a$. Break up the line segment from $x,y$ into intervals of length $a$; let $x=x_0,x_1,\ldots, x_k=y$ be the endpoints of those intervals, with
\[
\ve{x_{i+1}-x_i}\le a.
\]
The \vocab{modulus of continuity} is defined as
\[
W_f(t)=\sup \set{\ve{f(u)-f(v)}}{u,v\in X,\ve{u-v}\le t}. 
\]
Uniform continuity says $\lim_{t\to 0}W_f(t)=0$. 
The number of intervals is
\[
k\le \fc{\ve{x-y}}{a}+1\le \fc{2\ve{x-y}}a.
\]
Then
\bal
\ve{f(x)-f(y)} & \le \sum_{i=1}^k \ve{f(x_i)-f(x_{i-1})} \\
&\le KW_f(a) \le \fc{2W_f(a)}{a}\ve{x-y},
\end{align*}
so we can let $L(a) = \fc{2W_f(a)}{a}$.
\end{proof}

\subsection{Bourgain's discretization theorem}

There are 3 known proofs of Ribe's Theorem.
\begin{enumerate}
\item
Ribe's original proof, 1987.
\item
HK, 1990, a genuinely different proof.
\item
Bourgain's proof, a Fourier analytic proof which gives a quantitative version. This is the version we'll prove.
\end{enumerate}
Bourgain uses the Discretization Theorem~\ref{thm:bdt}. There is an amazing open problem in this context.

Saying $\delta$ is big says there is a not-too-fine net, which is enough. Therefore we are interested in lower bounds on $\delta$. 

\begin{df}[Discretization modulus]\index{discretization modulus}
Let $X$ be a finite-dimensional normed space $\dim(X)=n<\iy$. Let the target space $Y$ be an arbitrary Banach space.  Consider the unit ball $B_X$ in $X$. Take %an $\de$-net (the distance between points is at most $\de$) 
a maximal $\de$-net $\cal N_\de$ in $B_X$. %(a maximal $\de$-separated subset). 
Suppose we can embed $\cal N_\de$ into $Y$ via $f:\cal N_\de\to Y$. Suppose we know in $Y$ that
\[
\ve{x-y}\le \ve{f(x)-f(y)}\le D\ve{x-y}.
\]
for all $x, y \in N_{\de}$. 
%also interesting not in context of normed spaces.
(We say that $\cal N_\de$ embeds with distortion $D$ into $Y$.)
\end{df}
You can prove using a nice compactness argument that if this holds for $\de$ is small enough, then the entire space $X$ embeds into $Y$ with rough the same distortion. 
Bourgain's discretization theorem~\ref{thm:bdt} says that you can choose $\de=\de_n$ to be independent of the geometry of $X$ and $Y$ such that if you give a $\delta$-approximation of the unit-ball in the $n$-dimensional norm, you succeed in embedding the whole space. 
%you lose a factor
%if $Y$ is reflexive, you can get this easily; for general $Y$ you need more work.
%$\de$-approximation encodes all the information. Whenever you try . Succeed for the net, succeed for object.

I often use this theorem in this way: I use continuous methods to show embedding $X$ into $Y$ requires big distortion; immediately I get an example with a finite object. Let us now make the notion of distortion more precise. 

\begin{df}[Distortion]
Suppose $(X,d_X),(Y,d_Y)$ are metric spaces $D\ge 1$. We say that $X$ embeds into $Y$ with \ivocab{distortion} $D$ if there exists $f:X\to Y$ and $s>0$ such that for all $x,y\in X$,
\[
Sd_X(x,y) \le d_Y(f(x),f(y)) \le DSd_X(x,y).
\]
\nomenclature{$C_Y(X)$}{infimum of $D$ where $X$ imbeds into $Y$ with distortion $D$}
The infimum over those $D\ge 1$ such that $X$ embeds into $Y$ with distortion is denoted $C_Y(X)$. This is a measure of how far $X$ is being from a subgeometry of $Y$. 
%how far from being subgeometry.
\end{df}

\begin{df}[\vocab{Discretization modulus}] \index{discretization modulus}\label{df:disc-mod}
\nomenclature{$\de_{X\hra Y}(\ep)$}{supremum over all those $\de>0$ such that for every $\de$-net $\cal N_\de$ in $B_X$, $C_Y(\cal N_\de)\ge (1-\ep)C_Y(X)$}
Let be a $n$-dimensional normed space and $Y$ be any Banach space, $\ep\in (0,1)$. Let $\de_{X\hra Y}(\ep)$ be the supremum over all those $\de>0$ such that for every $\de$-net $\cal N_\de$ in $B_X$, 
\[C_Y(\cal N_\de)\ge (1-\ep)C_Y(X).\]

Here $B_X:=\set{x\in X}{\ve{x}\le 1}$.
\end{df}
In other words, the distortion of the $\delta$-net is not much larger than the distortion of the whole space. That is, the discrete $\delta$-ball encodes almost all information about the space when it comes to embedding into $Y$: If you got $C_Y(\cal N_\de)$ to be small, then the distortion of the entire object is not much larger.
%a discretization modulus. $\de$-net encodes all the info when it comes to embedding things into $Y$.
\begin{thm}[Bourgain's discretization theorem]\label{thm:bdt}
For every $n$, $\ep\in (0,1)$, for every $X,Y$, $\dim X=n$, 
\[
\de_{X\hra Y}(\ep)\ge  e^{-\pf{n}{\ep}^{Cn}}.
\]
Moreover for $\de=e^{-(2n)^{Cn}}$, we have $C_Y(X)\le 2C_Y(\cal N_\de)$. %via a linear operator. 
\end{thm}
Thus there is a $\de$ dependending on the dimension such that in any $n$-dimensional norm space, the unit ball it encodes all the information of embedding $X$ into \emph{anything else}. It's only a function of the dimension, not of any of the relevant geometry. 



%lower bound
%$\de$ dependent on dimension, 
The theorem says that if you look at a $\de$-net in the unit ball, it encodes all the information about $X$ when it comes to embedding into everything else. %There is a discrete version , 
The amount you have to discretize is just a function of the dimension, and not of any of the other relevant geometry.
\begin{rem}
The proof is via a linear operator. All the inequality says is that you can find a \emph{function} with the given distortion. The proof will actually give a \emph{linear operator}. 
\end{rem}

The best known upper bound is 
\[
\de_{X\hra Y}\prc2 \lesssim\rc n.
\]
%how mch to discretize to encode all information.
The latest progress was $1987$, there isn't a better bound yet. You have a month to think about it before you get corrupted by Bourgain's proof. 

There is a better bound when the target space is a $L^p$ space.
\begin{thm}[Gladi, Naor, Shechtman]
For every $p\ge 1$, if $\dim X=n$,
\[
\de_{X\hra L_p}(\ep)\gtrsim \fc{\ep^2}{n^{\fc 52}}
\]
\end{thm}
(We still don't know what the right power is.) 
The case $p=1$ is important for applications. There are larger classes for spaces where we can write down axioms for where this holds. 
%Ostrowski and ...
There are crazy Banach spaces which don't belong to this class, so we're not done. %We'll get to this in a month. 
We need more tools to show this: Lipschitz extension theorems, etc.

\subsection{Bourgain's Theorem implies Ribe's Theorem}

With the ``moreover," Bourgain's theorem implies Ribe's Theorem~\ref{thm:ribe}.

\begin{proof}[Proof of Ribe's Theorem~\ref{thm:ribe} from Bourgain's Theorem~\ref{thm:bdt}]
Let $X,Y$ be Banach spaces that are uniformly homeomorphic. By Corson-Klee~\ref{lem:corson-klee}, there exists $f:X\to Y$ such that 
\[
\ve{x-y}\ge 1\implies \ve{x-y}\le \ve{f(x)-f(y)}\le K\ve{x-y}.
\]
%can always normalize in the lower bound
(Apply the Corson-Klee lemma for both $f$ and the inverse.)

In particular, if $R>1$ and $\cal N$ is a 1-net in 
\[
RB_X = \set{x\in X}{\ve{x}\le R},
\]
then $C_Y(\cal N)\le K$.
%define distortion to be scale invariant
Equivalently, for every $\de>0$ every $\de$-net in $B_X$ satisfies $C_Y(\cal N)\le K$. 
If $F\subeq X$ is a finite dimension subspace and $\de = e^{-(2\dim F)^{C\dim F}}$, then by the ``moreover" part of Bourgain's Theorem~\ref{thm:bdt}, there exists a linear operator $T: F\to Y$ such that 
\[
\ve{x-y}\le \ve{Tx-Ty}\le 2K\ve{x-y}
\]
for all $x,y\in F$. This means that $X$ is finitely representable. 
\end{proof}
%computer science care about finite things.
The motivation for this program comes in passing from continuous to discrete. The theory has many applications, e.g. to computer science whcih cares about finite things.
I would like an improvement in Bourgain's Theorem~\ref{thm:bdt}.

First we'll prove a theorem that has nothing to do with Ribe's Theorem. There are lemmas we will be using later. It's an easier theorem. It looks unrelated to metric theory, but the lemmas are relevant. 
%digression to warm up.

%grothendieck inequalities, summing. 

\chapter{Restricted invertibility principle}
\section{Restricted invertibility principle}

\subsection{The first restricted invertibility principles}
%link to nonlinear world will appear later.
%famous, great achievements of analysis in the 1980's.
We take basic facts in linear algebra and make things quantitative. This is the lesson of the course: when you make things quantitative, new mathematics appears.

\begin{pr}
If $A:\R^m\to \R^n$ is a linear operator, then there exists a linear subspace $V\subeq \R^n$ with $\dim(V)=\rank(A)$ such that $A:V\to A(V)$ is invertible. 
\end{pr}
%In modern math we need a quantitative version of this.

What's the quantitative question we want to ask about this? Invertibility just says that an inverse exists. Can we find a large subspace where not only is $A$ invertible, but the inverse has small norm?

We insist that the subspace is a coordinate subspace.  Let $e_1,\ldots, e_m$ be the standard basis of $\R^m, e_j=(0,\ldots, \ub1j,0,\ldots)$. 
The goal is to find a ``large" subset $\si\subeq \{1,\ldots, m\}$ such that  $A$ is invertible on $\R^\si$ where 
\nomenclature{$\R^{\si}$}{$\set{(x_1,\ldots, x_n)\in \R^m}{x_i=0\text{ if }i\nin \si}$}
\[
\R^{\si}:=\set{(x_1,\ldots, x_n)\in \R^m}{x_i=0\text{ if }i\nin \si}
\]
and the norm of $A^{-1}:A(\R^\si)\to \R^{\si}$ is small.

A priori this seems a crazy thing to do; take a small multiple of the identity. But we can find conditions that allow us to achieve this goal.

\begin{thm}[Bourgain-Tzafriri restricted invertibility principle, 1987]\index{restricted invertibility principle}\llabel{thm:btrip}
Let $A:\R^m\to \R^m$ be a linear operator such that 
\[
\ve{Ae_j}_2=1
\]
for every $j\in \{1,\ldots, m\}$. Then there exist $\si\subeq \{1,\ldots, m\}$ such that
\begin{enumerate}
\item
$|\si|\ge \fc{cm}{\ve{A}^2}$, where $\ve{A}$ is the operator norm of $A$.
\item
$A$ is invertible on $\R^{\si}$ and the norm of $A^{-1}:A(\R^{\si})\to \R^{\si}$ is at most $C'$ (i.e., $\ve{AJ_{\si}}_{S^{\iy}}\le C'$, to use the notation introduced below).
\end{enumerate}
Here $c,C'$ are universal constants.
\end{thm}
%No, condition on rank of $A$.
%rank has to be big, proportional. 
%if repeated a lot get huge norm. The fact that norm small. Exercise in linear algebra, says something about rank being big.
%the norm is at most a universal constant.

Suppose the biggest eigenvalue is at most 100. Then you can always find a coordinate subset of proportional size such that on this subset, $A$ is invertible and the inverse has norm bounded by a universal  constant.

All of the proofs use something very amazing.

This proof is from 3 weeks ago. 
This has been reproved many times. I'll state a theorem that gives better bound than the entire history. 
%does this imply that the rank is large?
%$\si$ should depend on the norm of $A$.

This was extended to rectangular matrices. (The extension is nontrivial.)

Given $V\subeq \R^m$ a linear subspace with $\dim V=k$ and $A:V\to \R^m$ a linear operator, the singular values of $A$ 
\[
s_1(A)\ge s_2(A)\ge \cdots \ge s_k(A)
\]
are the eigenvalues of $(A^*A)^{\rc 2}$. We can decompose
\[
A=UDV
\]
where $D$ is a matrix with $s_i(A)$'s on the diagonal, and $U,V$ are unitary.
\begin{df}
\nomenclature{$\ve{A}_{S_p}$}{Schatten-von Neumann $p$-norm}
For $p\ge 1$ the \ivocab{Schatten-von Neumann $p$-norm} of $A$ is 
\bal 
\ve{A}_{S_p}
&:=\pa{\sum_{i=1}^k  s_i(A)^p}^{\rc p}\\
&= (\Tr((A^*A)^{\fc p2}))^{\rc p}\\
&= (\Tr((AA^*)^{\fc p2}))^{\rc p}
\end{align*}
\end{df}
The cases $p=\iy,2$ give the operator and Frobenius norm,
\bal
\ve{A}_{S_{\iy}}&= \text{operator norm}\\
\ve{A}_{S_2}& = \sqrt{\Tr(A^*A)} = \pa{\sum a_{ij}^2}^{\rc 2}.
\end{align*}



\begin{exr}
$\ved_{S_p}$ is a norm on $\cal M_{n\times m}(\R)$. You have to prove that given $A,B$,
\[
(\Tr([(A+B)^*(A+B)]^{\fc p2}))^{\rc p}
\le
%nicer facts of linear algebra.
(\Tr((A^*A)^{\fc p2}))^{\rc p}+(\Tr((B^*B)^{\fc p2}))^{\rc p}.
\]
%if they commute it is trivial
\end{exr}
This requires an idea. Note if $A,B$ commute this is trivial. 
Apparently von Neumann wrote a paper called ``Metric Spaces'' in the $1930$s in which he just proves this inequality and doesn't know what to do with it, so it got forgotten for a while until the $1950$s, when Schatten wrote books on applications. When I was a student in grad school, I was taking a class on random matrices. There was two weeks break, I was certain that it was trivial because the professor had not said it was not, and it completely ruined my break though I came up with a different proof of it. It's short, but not trivial: It's not typical linear algebra!. This is like another triangle inequality, which we may need later on.

%if linear algebra then trivial. Random matrices class. 
%sometimes you find a completely different proof.

Spielman and Srivastava have a beautiful theorem. %People in numerical linear algebra call the 
\begin{df} Stable rank. \\
Let $A:\R^m \to \R^n$. 
The \ivocab{stable rank} is defined as
\nomenclature{$\text{srank}(A)$}{stable rank, $\pf{\ve{A}_{S_2}}{\ve{A}_{S_\iy}}^2$}
\[
\text{srank}(A)=\pf{\ve{A}_{S_2}}{\ve{A}_{S_\iy}}^2.
\]
\end{df}
The numerator is the sum of squares of the singular values, and the denominator is the maximal value. Large stable rank means that many singular values are nonzero, and in fact large on average. Many people wanted to get the size of the subset in the Restricted Invertibility Principle  
to be close to the stable rank.


\begin{thm}[Spielman-Srivastava]\llabel{thm:ss}
For every linear operator $A:\R^m\to \R^n$, $\ep\in (0,1)$, 
%userful, better than BT.
there exists $\si\subeq \{1,\ldots, m\}$ with $|\si|\ge (1-\ep)\text{srank}(A)$ such that
\[
\ve{(AJ_\si)^{-1}}_{S_{\iy}} \lesssim \fc{\sqrt m}{\ep\ve{A}_{S_2}}.
\]
Here, $J_\si$ is the identity function restricted to $\R^{\si}$,  $J:\R^{\si}\hra \R^m$. 
\end{thm}
This is stronger than Bourgain-Tzafriri. In Bourgain-Tzafriri the columns were unit vectors. 
\begin{proof}[Proof of Theorem~\ref{thm:btrip} from Theorem~\ref{thm:ss}]
Let $A$ be as in Theorem~\ref{thm:btrip}. Then $\ve{A}_{S_2}=\sqrt{\Tr(A^*A)}=\sqrt m$ and $\text{srank}(A)=\fc{m}{\ve{A}^2_{S_{\iy}}}$. We obtain the existence of 
\[
|\si|\ge (1-\ep)\fc{m}{\ve{A}^2_{S_{\iy}}}
\]
with $\ve{(AJ_{\si})^{-1}}_{S_{\iy}}\lesssim \fc{\sqrt m}=\rc{\ep}$.
%get all the way to the stable rank.
\end{proof}
This is a sharp dependence on $\ep$.

The proof introduces algebraic rather than analytic methods; it was eye-opening. Marcus even got sets bigger than the stable rank and looked at $pf{\ve{A}_{S_2}}{\ve{A}_{S_4}}^2$, which is much stronger.

\subsection{A general restricted invertibility principle}

I'll show a theorem that implies all these intermediate theorems. We use (classical) analysis and geometry instead of algebra.
What matters is not the ratio of the norms, but the tail of the distribution of $s_1(A)^2,\ldots, s_m(A)^2$.
%Look at the tails of the distributions.
\begin{thm}\llabel{thm:gen-srank}
Let $A:\R^m\to \R^n$ be a linear operator. If $k<\rank(A)$  then there exist $\si\subeq \{1,\ldots, m\}$ with $|\si|=k$ such that 
\[
\ve{(AJ_\si)^{-1}}_{S_{\infty}} \lesssim \min_{k<r\le \rank(A)}\sfc{mr}{(r-k) \sum_{i=r}^m s_i(A)^2}.
\]
\end{thm}
You have to optimize over $r$. You can get the ratio of $L_p$ norms from the tail bounds. This implies all the known theorems in restricted invertibility.
The subset can be as big as you want up to the rank, and we have sharp control in the entire range.
This theorem generalizes Spielman-Srivasta (Theorem~\ref{thm:ss}), which had generalized Bourgain-Tzafriri (Theorem ~\ref{thm:btrip}). 
%Next time we'll show how this implies the other theorems, and then prove the theorem. %: show how the 

\blu{2-8-16}

%\begin{thm}[Spielman-Srivastava]
%For $k<\text{srank}(A)=\pf{\ve{A}_{S_2}^2}{\ve{A}_{S_{\iy}}}^2$, $k=(1-\ep)\text{srank}(A)$. 
%$\exists \si\subeq \{1,\ldots, m\}, |\si|=k$, ...
%%,/\op&2. 
%%constant prop of m/ ... squred. 
%%bourgain-tz up to stble rank.
%%shatten root m.
%\end{thm}

Now we will go backwards a bit, and talk about a less general result. After Theorem~\ref{thm:ss}, 
a subsequent theorem gave the same theorem but instead of the stable rank, used something better.
\begin{thm}[Marcus, Spielman, Srivastava]\llabel{thm:mss4}
If 
\[
k<\rc4 \pf{\ve{A}_{S_2}}{\ve{A}_{S_4}}^4,
\]
there exists $\si\subeq \{1,\ldots, m\}$, $|\si|=k$ such that
\[
\ve{(AJ_{\si})^{-1}}_{S_{\iy}} \lesssim \fc{\sqrt m}{\ve{A}_{S_2}}.
\]
%stable rank. 
%got strange norms.
\end{thm}
A lot of these quotients of norms started popping up in people's results. The correct generalization is the following notion.
\begin{df}
For $p>2$, define the \ivocab{stable $p$th rank} by \[\text{srank}_p(A)= \pf{\ve{A}_{S_2}}{\ve{A}_{S_p}}^{\fc{2p}{p-2}}.\]
\end{df}
\begin{exr}
Show that if $p\ge q>2$, then
\[
\text{srank}_p(A) \le \text{srank}_q(A).
\]
(Hint: Use H\"older's inequality.)
\end{exr}
Now we would like to prove how Theorem~\ref{thm:gen-srank} generalizes the previously listed results: 
\begin{proof}[Proof of Generalizability of Theorem~\ref{thm:gen-srank}]
Using H\"older's inequality with $\fc p2$,
\bal
\ve{A}_{S_2}^2 & =\sum_{j=1}^m s_j(A)^2\\
&=\sum_{j=1}^{r-1} s_j(A)^2 + \sum_{j=r}^m s_j(A)^2\\
&\le (r-1)^{1-\fc 2p} \pa{\sum_{j=1}^{r-1} s_j(A)^p}^{\fc 2p} + \sum_{j=r}^m s_j(A)^2\\
&\le (r-1)^{1-\fc 2p} \ve{A}_{S_p}^2  + \sum_{j=r}^m s_j(A)^2\\
\sum_{j=r}^m s_j(A)^2 & \ge 
\ve{A}_{S_2}^2 \pa{1-(r-1)^{-\fc 2p}\fc{\ve{A}_{S_p}^2}{\ve{A}_{S_{2}}^2}}\\
&=\ve{A}_{S_2}^2 \pa{1-\pf{r-1}{\text{srank}_p(A)}^{1-\fc2p}}
\end{align*}
Now we can plug the previous calculation into Theorem~\ref{thm:gen-srank} to demonstrate the way the new theorem generalizes the previous results: 
\bal
\ve{(AJ_\si)^{-1}}&\lesssim\min_{k+1\le r\le \rank(A)} \sfc{mr}{(r-k)\ve{A}_{S_2}^2 \pa{1-\pf{r-1}{\text{srank}_p(A)}^{1-\fc 2p}}}\\
&=\fc{\sqrt m}{\ve{A}_{\iy}}\min_{k+1\le r\le \rank(A)} \sfc{r}{(r-k)\pa{1-\pf{r-1}{\text{srank}_p(A)}^{1-\fc 2p}}}
\end{align*}
This equation implies all the earlier theorems.
%\fixme{How did we get the last 2 lines?}
\end{proof}
To optimize, fix the stable rank, differentiate in $r$, and set to 0. All theorems in the literature follow from this theorem; in particular, we get all the bounds we got before. %3, 2.1.
There was nothing special about the number 4 in Theorem~\ref{thm:mss4}; this is about the distribution of the eigenvalues. 

\subsection{Ky Fan maximum principle}
We'll be doing linear algebra. It's mostly mechanical, except we'll need this lemma.
\begin{lem}[Ky Fan maximum principle]\index{Ky Fan maximum principle}
Suppose that $P:\R^m\to \R^m$ is a rank $k$ orthogonal projection. Then
\[
\Tr(A^*AP ) \le \sum_{i=1}^k s_i(A)^2.
\]
\end{lem}
\begin{proof}
\fixme{This proof isn't complete. I will fix it next time.}

We will prove that if $B:\R^m\to \R^m$ is positive semidefinite, then
\[
\Tr(BP)\le \sum_{i=1}^k s_i(B).
\]
To get the lemma, set $B=A^*A$. 

Apply arbitrarily small perturbation $s_i$ so that
\[
s_1(B)>s_2(B)>\cdots > s_m(B).
\]
Let $v_1,\ldots, v_m$ be an orthonormal basis such that $Bv_i=s_i(B) v_i$. Let $u_1,\ldots, u_k$ be an orthonormal basis of $P\R^m$ ordered so that 
\[
\an{Bu_1,u} \ge \an{Bu_2,u_2}\ge \cdots \ge \an{Bu_k,u_k}.
\]
We calculate
\bal
\Tr(BP) & =\sum_{i=1}^k  \an{BPu_i, u_i}\\
&=\sum_{i=1}^k \an{Bu_i,u_i}.
\end{align*}
We will prove by induction on $i$ that 
\[
\an{Bu_i,u_i}\le s_i(B).
\]
For $i=1$, 
\[
s_1(B)=\ve{B}_{S_{\iy}},
\]
and (when $\ve{u_1}=1$)
\[
\an{Bu_1,u_1} \le \ve{B}_{S_{\iy}} = s_1(B).
\]
Suppose we proved the claim for $j-1$. If \fixme{?$ \an{Bu_i,u_i}\le s_j(B)$} $\an{Bu_{j-1},u_{j-1}}\le s_j(B)$ then we're done because $\an{Bu_j,u_j}\le \an{Bu_{j-1},u_{j-1}} \le s_j(B)$. So we may assume that $\an{Bu_{j-1}, u_{j-1}}>s_j(B)$. 

Write the $u$'s in the basis of $v$'s:
\[
u_i = \sum_{l=1}^m c_{il}v_l.
\]
The fact that the $u_i$'s are orthonormal means that the $c_i$'s are probability vectors,
\[
\sum_{l=1}^m c_{il}^2=1.
\]
We have 
\[
\an{Bu_i,u_i} = \sumo lm m c_{il}^2 s_l(B).
\]
If $c_{il}^2>0$ for any $l\ge j$. 
%there can be no weight from $j$ onwards.
%u_j in span of $v_j$ upwards. Need for 1 to $j$. 
%1 to $u_{j-1}$. 
\end{proof}

\subsection{Finding big subsets}
We'll present 4 lemmas for finding big subsets with certain properties. We'll put them together at the end.
\begin{thm}[Little Grothendieck inequality]\llabel{thm:lgi}
Fix $k,m,n\in \N$. Suppose that $T:\R^m\to \R^n$ is a linear operator. Then for every $x_1,\ldots, x_k\in \R^m$,
\[
\sumo rk \ve{Tx_r}_2^2\le \fc{\pi}2\ve{T}_{\ell_{\iy}^m \to \ell_2^n}^2 \sumo rk x_{ri}^2
\]
for some $i\in \{1,\ldots, m\}$ where $x_r=(x_{r1},\ldots, x_{rm})$.
\end{thm}
Later we will show $\fc{\pi}2$ is sharp. 

If we had only 1 vector, what does this say?
\[
\ve{Tx_1}_2\le \sfc{\pi}2\ve{T}_{\ell_{\iy}^m \to \ell_2^n}\ve{x_1}_{\iy}
\]
%The definition of the operator norm is at most ... times the \iy$ norm
We know the inequality is true for $k=1$ with constant 1, by definition of the operator norm. The theorem is true for arbitrary many vectors, losing an universal constant ($\fc{\pi}2$). After we see the proof, the example where $\fc{\pi}2$ is attained will be natural.

We give Grothendieck's original proof.

The key claim is the following.
\begin{clm}\llabel{clm:lgi}
\beq{eq:lgi1}
\sumo jm \pa{\sumo rk (T^*Tx_r)_j^2}^{\rc2}
\le \sfc{\pi}2 \ve{T}_{\ell_{\iy}^m \to \ell_2^n}\pa{\sum_{r=1}^k \ve{Tx_r}^2}^{\rc 2}.
\eeq
\end{clm}

\begin{proof}[Proof of Theorem~\ref{thm:lgi}]
Assuming the claim, we prove the theorem.
\bal
\sumo rk \ve{Tx_r}_2^2 & = \sumo rk \an{Tx_r,Tx_r}\\
&=\sumo rk \an{x_r, T^*Tx_r}\\
&=\sumo rk \sumo jm x_{rj}(T^*Tx_r)_j\\
&\le \sumo jm \pa{\sumo rk x_{rj}^2}^{\rc 2} \pa{\sumo rk (T^*Tx_r)_j^2}^{\rc 2}&\text{by Cauchy-Schwarz}\\
&\le \pa{\max_{1\le j\le m} \pa{\sumo rk x_{rj}^2}^{\rc 2}}
\pa{\sumo jm \sumo rk (T^*Tx_r)_j^2}^{\rc 2}\\
&\le \max_{1\le j\le m}\pa{\sumo ik x_{ij}^2}^{\rc 2}\sfc{\pi}2 \ve{T}_{\ell_{\iy}^m \to \ell_2^n} \pa{\sumo rk \ve{Tx_r}_2^2}^{\rc 2}\\
\sumo rk \ve{Tx_r}_2^2 & \le \fc{\pi}2 \ve{T}_{\ell_{\iy}^m \to \ell_2^n}^2 \max_j \sumo rk x_{ij}^2.
%bound by square root of multiple of same term, bootstrap.
\end{align*}
We bounded by a square root of the multiple of the same term, a bootstrapping argument. In the last step, divide and square.
%where we used Cauchy-Schwarz
\end{proof}

\begin{proof}[Proof of Claim~\ref{clm:lgi}]
Let $g_1,\ldots, g_k$ be iid standard Gaussian random variables. For every fixed $j\in \{1,\ldots, m\}$, 
\[
\sum_{r=1}^k g_r (T^* T x_r)_j.
\]
This is a Gaussian random variable with mean 0 and variance  %whatever the $L^2$ norm of these coefficients are
$\sumo rk (T^*Tx_r)_j^2$. Taking the expectation,\footnote{$\sfc{1}{2\pi} \int_{-\iy}^{\iy} |x| e^{-\fc{x^2}2} = -2\sfc{1}{2\pi} [e^{-\fc{x^2}2}]^{\iy}_0= \sfc{2}{\pi}$}
\bal
\E\ab{\sumo rk g_r(T^*T x_r)_j}
&= \pa{\sumo rk (T^*T x_r)_j^2}^{\rc 2} \sfc 2{\pi}.
\end{align*}
Sum these over $j$:
\begin{align}
\E \ba{
\sumo jm \ab{T^* (\sumo rk g_r T x_r)_j}
} & = \sfc 2\pi \sumo jm \pa{\sumo rk (T^*Tx_r)_j^2}^{\rc 2}\nonumber\\
\sumo jm \pa{\sumo rk (T^*Tx_r)_j^2}^{\rc 2}
&= \sfc{\pi}2 \E\ba{\sumo jm \ab{T^* \sumo rk g_r(Tx_r)_j}}.
\llabel{eq:lgi2}
\end{align}
Define a random sign vector $z\in \{\pm 1\}^m$ by 
\[
z_j = \sign\pa{\pa{T^*\sumo rk g_r Tx_r}_j}
\]
Then 
\bal
\sumo jm \ab{(T^* \sumo rk g Tx_r)_j} 
&=\an{z, T^* \sumo rk g_r Tx_r}\\
&= \an{Tz,\sumo rk g_r Tx_r}\\
& \le \ve{Tz}_2 \ve{\sumo rk g_r Tx_r}_2\\
%l_\iy to l_2. all most whatever norm of operator is.
&\le \ve{T}_{\ell_{\iy}^m \to \ell_2^n} \ve{\sumo rk g_r Tx_r}_2
%expointed was bounded pointwise pointwise.
\end{align*}
This is a pointwise inequality. Taking expectations and using Cauchy-Schwarz,
\beq{eq:lgi3}
\E\ba{
\sum_{j=1}^m \ab{\pa{T^* \sumo rk g_r Tx_r}_j}
}  \le \ve{T}_{\ell_{\iy}^m \to \ell_2^n}\pa{\E \ve{\sumo rk g_r Tx_r}_2^2}^{\rc 2}.
\eeq
What is the second moment? Expand:
\beq{eq:lgi4}
\E\ve{\sumo rk g_i Tx_r}_2^2 
=\E\ba{\sum_{ij} g_i g_j \an{Tx_i,Tx_j}}=\sumo rk \ve{Tx_r}_2^2.
\eeq
Chaining together~\eqref{eq:lgi2},~\eqref{eq:lgi3},~\eqref{eq:lgi4} gives the result.
%bound by L^2 norm above
\end{proof}

Why use the Gaussians? The identity characterizes the Gaussians using rotation invariance. %The expectation of the Gaussian 
%\sumo jm\sumo rk ... \sfc{\pi}2
%up to this piit use gaussians
Using other random variables gives other constants that are not sharp.

There will be lots of geometric lemmas:
\begin{itemize}
\item
A fact about restricting matrices. 
\item
Another geometric argument to give a different method for selecting subsets. 
\item 
A combinatorial lemma for selecting subsets.
\end{itemize}
Finally we'll put them together in a crazy induction.

From this proof you can reverse engineer vectors that make the inequality sharp. You need to come up with $T$ and the points.

\begin{ex}
Let $g_1,g_2,\ldots,g_k $ be iid Gaussians on the probability space $(\Om, P)$. Let $T:L_{\iy}(\Om, P)\to \ell_2^k$ be %infinite $l^{\iy}$ space. 
%abstract nonsense: approx
%replace Gaussians with central limit theorem, take $\pm1$ bits.
%never equality for finite. (ratio of gamma functions)
%in limit converges
\[
Tf = (\E[fg_1],\ldots, \E[fg_k]).
\]
Let $x_r \in L_{\iy} (\Om, P)$, 
\[
x_r = \fc{g_r}{\pa{\sumo ik g_i^2}^{\rc 2}}.
\]
%no black magic, just understand this.
\end{ex}
\blu{2-10-16}

We were in the process of proving three or four subset selection principles, which we will somehow use to prove the RIP. 

Now I owe you a proof (just ask me for the linear algebra proof) - I'll show you an analytic proof. 

We proved the little Grothendieck inequality (Theorem~\ref{thm:lgt}), which is part of an amazing area of mathematics with many applications. It's little, but it's also very useful. Just to remind you, we had an linear operator $T: \R^m \to \R^n$. Then for every $x_1, \cdots, x_k \in \R^m$, we get a bounded operator. If you look at the sum of the Euclidean lengths $\left(\sum_{i = 1}^k \|Tx\|_2^2\right)^{1/2} \leq \sqrt{\pi/2}\|T\|_{l_{\infty}^m \to l_2^n} \cdot \max_{1 \leq j \leq m} \left(\sum_{i = 1}^k x_{rj}^2\right)^{1/2}$. This is really the way Grothendieck did it, but the proof we saw is really the original proof, re-organized. For completeness, we'll show the fact that this inequality is sharp (cannot be improved). 

\begin{cor} $\sqrt{\pi/2}$ is the best constant in Theorem~\ref{thm:lgt}. \\
\end{cor}
\begin{proof}
Define $g_1, \cdots, g_k$ be i.i.d standard Gaussians, defined on probability space $(\Omega, P)$. We define $T: L_{\infty}(\Omega, \mathbb{P}) \to \R^k$. Then $Tf = \left(\mathbb{E}(fg_1), \mathbb{E}(fg_2), \cdots, \mathbb{E}(fg_k) \right)$. Choose $X_r = \frac{g_r}{\left(\sum_i^k g_i^2\right)^{1/2}}$. This is nothing more than a vector on the $k$-dimensional unit sphere. So it's a bounded function. We also note that $x_r$ is a function on the measure space $\Omega$.  We can also write
\[
\sum_{r = 1}^k x_r(\omega)^2 = \sum_{r = 1}^k \frac{g_r(\omega)^2}{\sum_{i = 1}^r g_i(\omega)^2} = 1
\]
We can use the Central Limit Theorem to make things precise: $g_r \approx \frac{\epsilon_{r_1} + \cdots + \epsilon_{r_N}}{\sqrt{N}}$ as $N \to \infty$. So all these statements will be asymptotically true. Where does the family of random variables $\{g_r\}$ live in $\Omega$? Well $\Omega = \{\pm 1\}^{NK}$. So $L_{\infty}(\Omega) = \l_{\infty}^{2^{NK}}$, which is some huge dimension, but it's still finite. So $\omega$ will really be a coordinate in $\Omega$. 

Now we show two things; nothing more than computations. 
\begin{enumerate}

\item $\|T\|_{L_{\infty}(\Omega, \textbf{P}) \to l_2^k} = \sqrt{2/\pi}$, 

\item We also show $\sum_{r = 1}^k \|Tx_r\|_2^2 \to^{k \to \infty} 1$. 

\end{enumerate}

First we tackle the first case. We have 
\begin{align}
\begin{split}
\|T\|_{\infty \to 1} &= \text{sup}_{\|f\|_{\infty} \leq 1}\left(\sum_{r = 1}^k \mathbb{E}\left[fg_r\right]^2 \right)^{1/2}
\\
&= \text{sup}_{\|f\|_{\infty} \leq 1} \text{sup}_{\sum_{r = 1}^k \alpha_r^2 = 1} \sum_{r = 1}\alpha_r \mathbb{E}\left[fg_r\right]
\\
&= \text{sup}_{\sum_{r = 1}^k} \text{sup}_{\|f\|_{\infty} \leq 1} \mathbb{E}\left[f\sum_{i = 1}^k \alpha_r g_r \right]
\\
&= \text{sup}_{\sum_{r = 1}^k} \mathbb{E}\left| \sum_{r = 1}^k \alpha_r g_r \right| = \mathbb{E} |g_1| = \sqrt{\frac{2}{\pi}}
\end{split}
\end{align}

as we claimed. Now we tackle the second computation: 
\begin{align}
\begin{split}
\sum_{r = 1}^k \|Tx_r\|_2^2 &= \sum_{r = 1}^k \left(\mathbb{E} \left[\frac{g_r^2}{\left(\sum_{i = 1}^k g_i^2\right)^{1/2}}\right]\right)^2
\\
&= K \left(\mathbb{E}\left[\frac{g_1^2}{\left(\sum_{i = 1}^k g_i^2\right)^{1/2}}\right]\right)
\\
&= K\left(\frac{1}{K}\mathbb{E}\left[\sum_{r = 1}^k \frac{g_r^2}{\left(\sum g_i^2\right)^{1/2}}\right]\right)^2
\\
&= \frac{1}{K} \left(\mathbb{E}\left[\left(\sum_{i = 1}^k g_i^2\right)^{1/2}\right]\right)^2
\end{split}
\end{align}
and you can use Stirling to finish. This is just a $\chi^2$-distribution. 

In this case $\mathbb{E} \frac{g_1g_2}{\left(\sum_i g_i^2\right)^{1/2}} = \mathbb{E} \frac{g_1 (-g_2)}{\left(\sum g_i^2\right)^{1/2}}$. 
Also note that if $(g_1, \cdots, g_k) \in \R^k$ is a standard Gaussian, then $\frac{(g_1, \cdots, g_k)}{\left(\sum_{i = 1}^k g_i^2\right)^{1/2}}$ and $\left(\sum_{i = 1}^k g_i^2)^{1/2}\right)$ are independent. In other words, the length and angle are independent: This is just polar coordinates, you can check this. 
\end{proof}

Now, how does this relate to the Restricted Invertibility Problem? 

\begin{thm} Pietsch Domination Theorem.\llabel{thm:pdt} \\
Fix $m, n \in \mathbb{N}$ and $M  > 0$. Suppose that $T: \R^m \to \R^n$ is a linear operator such that for every $x_1, \cdots, x_k \in \R^m$ have 
\[
\left(\sum_{r = 1}^k \|Tx_r\|_2^2\right)^{1/2} \leq M \text{max}_{1 \leq j \leq m} \left(\sum_{r = 1}^k x_{rj}^2\right)^{1/2}
\]
Then there exist $\mu = (\mu_1, \cdots, \mu_m) \in \R^m$ with $\mu_1 \geq 0$ and $\sum_{i = 1}^m = 1$ such that for every $x \in \R^m$
\[
\|Tx\|_2 \leq M\left(\sum_{i = 1}^M \mu_ix_i^2\right)^{1/2}
\]
It's really an iff: The latter is a stronger statement than the former, and in fact they are equivalent.
You can come out with a probability measure, a way to weight the coordinates, such that the norm of T as an operator as a standard norm from $l_{\infty}$ to $l_2$, is bounded by $M$. 
\end{thm}
\begin{proof}
Define $K \subseteq \R^m$ with 
\[
K = \left\{y \in \R^m: y_i = \sum_{r = 1}^k \|Tx_r\|_2^2 - M^2\sum_{r = 1}^m x_{ri}^2 \text{ for some } k, x_1, \cdots, x_k \in \R^m\right\}
\]
Basically we cleverly select a convex set. Every $n$-tuple of vectors in $\R^m$ gives you a new vector in $\R^m$. Let's check that $K$ is convex. We have to check if two vectors $y, z \in K$ have all points on the line between them in $K$. $y \in K$ means that 
\[
y_i = \sum_{r = 1}^k \|Tx_r\|_2^2 - M^2 \sum_{r = 1}^m x_{ri}^2
\] 
\[
z_i = \sum_{r = 1}^l \|Tw_r\|_2^2 - M^2 \sum_{r = 1}^l w_{ri}^2
\]
for all $i$. So what can you say about the average $\frac{y_i + z_i}{2}$? It comes from $\left(\frac{x_1}{\sqrt{2}}, \cdots, \frac{x_k}{\sqrt{2}}, \frac{w_1}{\sqrt{2}}, \cdots, \frac{w_l}{\sqrt{2}}\right)$. So trivially by design this is a convex set. 

Now, the assumption of the theorem says that 
\[
\left(\sum_{r = 1}^k \|Tx_r\|_2^2\right)^{1/2} \leq M \text{max}_{1 \leq j \leq m} \left(\sum_{r = 1}^k x_{rj}^2\right)^{1/2}
\]
which implies 
\[
\|Tx_r\|_2^2 - M^2 \text{max}_{1 \leq j \leq m} \sum_{r = 1}^m x_{rj}^2 \leq 0
\]
which implies $K \cap (0, \infty)^m = \emptyset$. By the hyperplane separation theorem (for two disjoint convex sets in $\R^m$ with at least one compact, there is a hyperplane between them), there exists $0 \neq \mu = (\mu_1, \cdots, \mu_m) \in \R^m$. We have 
\[
\langle \mu, y \rangle \leq \langle \mu, z\rangle
\]
for all $y \in K$ and $z \in (0, \infty)^m$. By renormalizing, $\sum_{i = 1}^m = 1$. Moreover $\mu$ cannot have any strictly negative coordinate: Otherwise you could take $z$ to have arbitrarily large value at a strictly negative coordinate with zeros everywhere else, implying $\langle u, z \rangle$ is no longer bounded from below, a contradiction. Therefore, $\mu$ is a probability vector and $\langle \mu, z \rangle$ can be arbitrarily small. So for every $y \in K$, $\sum_{i = 1}^m \mu_iy_i \leq 0$. Then $y_i = \|Tx\|_2^2 - M^2x_i \in K$, and if you write this out, $\|Tx\|_2^2 - M^2 \sum_{i = 1}^n \mu_i y_i \leq 0$, which is exactly what we wanted. 
\end{proof}

\begin{lem} \llabel{lem:projbound}
$m, n \in \mathbb{N}$, $\epsilon \in (0, 1)$, $T: \R^n \to \R^m$ a linear operator. Then $\exists \sigma \subset \{1, \cdots, m\}$ with $|\sigma| \geq (1 - \epsilon)m$ such that 
\[
\|\text{Proj}_{\R^{\sigma}} T\|_{S_{\infty}} \leq \sqrt{\frac{\pi}{2\epsilon m}} \|T\|_{l_2^n \to l_1^m}
\] 
We will find ways to restrict a matrix to a big big submatrix, but we won't be able to control its operator norm, but we will be able to control the norm from $l_2^n \to l_1^m$. So then you go to a further subset, which this becomes an operator norm on, which is an improvement which Grothendieck gave us. This is the first very useful tool to start finding big sub matrices. 
\end{lem}
\begin{proof}
We have $T: l_2^n \to l_1^m$, $T^*: l_{\infty}^m \to l_2^n$. Now some abstract nonsense gives us that for Banach spaces, the norm of an operator and its adjoint are equal, i.e. $\|T\|_{l_2^n \to l_1^m}  = \|T^*\|_{l_{\infty}^m \to l_2^n}$. This statement follows from Hahn-Banach theorem (come see me if you haven't seen this before, I'll tell you what book to read). 
From the Little Grothendieck inequality (Theorem~\ref{thm:lgi}), $T^*$ satisfies the assumption of the Pietsch domination theorem with $M = \sqrt{\frac{\pi}{2}} \|T\|_{l_2^n \to l_1^m}$ (we're applying it to $T^*$). So we have probability vector $(\mu_1, \cdots, \mu_m)$ such that for every $y \in \R^m$
\[
\|T^*y\|_2 = M\left(\sum_{i = 1}^m \mu_iy_i^2\right)^{1/2}
\]
with $M = \sqrt{\frac{\pi}{2}} \|T\|_{l_2^n \to l_1^m}$. Define $\sigma = \left\{i \in \{1, \cdots, m\}: \mu_i \leq \frac{1}{m\epsilon}\right\}$, then $|\sigma| \geq (1 - \epsilon)m$ by Markov's inequality. We can also see this by writing
\[
1 = \sum \mu_i = \sum_{i \in \sigma} \mu_i + \sum_{i \not\in \sigma} \mu_i > \sum_{i \in \sigma} \mu_i + \frac{m - |\sigma|}{m\epsilon}
\]
which follows since $\mu_j$ for $j \not\in \sigma$ has $\mu_j > \frac{1}{m\epsilon}$. Continuing, 
\[
\frac{m\epsilon - m + |\sigma|}{m\epsilon} \geq \sum_{i \in \sigma} \mu_i
\]
\[
|\sigma| \geq (m\epsilon)\sum_{i \in \sigma} \mu_i +  m(1 - \epsilon)
\]
Then, since $(m\epsilon)\sum_{i \in \sigma} \mu_i  \geq 0$ since $\mu$ is a probability distribution, we have
\[
|\sigma| \geq m(1 - \epsilon)
\]

Now take $x \in \R^n$ and choose $y \in \R^m$ with $\|y\|_2 = 1$. Then 
\[
\langle y, \text{Proj}_{\R^{\sigma}} Tx \rangle^2 = \|\text{Proj}_{\R^{\sigma}} Tx\|_2^2 \leq \langle T^* \text{Proj}_{\R^{\sigma}} y, x \rangle^2 \leq \|T^*\text{Proj}_{\R^{\sigma}} y\|_2^2 \cdot \|x\|_2^2
\] 
\[
\leq \frac{\pi}{2}\|T\|_{l_2^n \to l_1^m} \left(\sum_{i \in \sigma} \mu_iy_i^2\right) \|x\|_2^2 \leq \frac{\pi}{2} \|T\|_{l_2^n \to l_1^m}^2 \frac{1}{m\epsilon}\|x\|_2^2
\]
by Cauchy-Schwarz. Then, taking square roots gives the desired result.
\end{proof}
In the previous proof, we used a lot of duality to get an interesting subset. 

\begin{rem}
In Lemma~\ref{lem:projbound}, I think that either the constant $\pi/2$ is sharp (no subset are bigger; it could come from the Gaussians), or there is a different constant here. If the constant is $1$, I think you can optimize the previous argument and get the constant to be arbitrarily close to $1$, which would have some nice applications: In other words, getting $\sqrt{\frac{\pi}{2\epsilon m}}$ as close to $1$ as possible would be good. I didn't check before class, but you might want to check if you can carry out this argument using the Gaussian argument we made for the sharpness of $\frac{\pi}{2}$ in Grothendieck's inequality (Theorem~\ref{thm:lgt}). It's also possible that there is a different universal constant. 
\end{rem}

Now we will give another lemma which is very easy and which we will use a lot. 
\begin{lem} Sauer-Shelah. \llabel{lem:saushel} \\
Take integers $m, n \in \mathbb{N}$ and suppose that we have a large set $\Omega \subseteq \{\pm 1\}^n$ with 
\[
|\Omega| > \sum_{k = 0}^{n - 1} {n \choose k}
\]
Then $\exists \sigma \subseteq \{1, \cdots, n\}$ such that with $|\sigma| = m$, if you project onto $\R^{\sigma}$ the set of vectors, you get the entire cube: $\text{Proj}_{\R^{\sigma}}(\Omega) = \{\pm 1\}^{\sigma}$. For every $\epsilon \in \{\pm 1\}^{\sigma}$, there are signs $\delta = (\delta_1, \cdots, \delta_n) \in \Omega$ such that $\delta_j = \epsilon_j$ for $j \in \sigma$.
\end{lem}

Note that Lemma~\ref{lem:saushel} is used in the proof of the Fundamental Theorem of Statistical Learning Theory. 

\begin{proof}
% The strengthening of induction I'm about to use is due to Pajor.
We want to prove by induction on $n$. First denote the shattering set
\[
\txtn{sh}(\Omega) = \{\sigma \subseteq \{1, \cdots, n\}: \txtn{Proj}_{\R^{\sigma}}\Omega = \{\pm 1\}^{\sigma}\}
\]
The claim is that the number of sets shattered by a given set is $|\text{sh}(\Omega)| \geq |\Omega|$. The empty set case is trivial. What happens when $n = 1$? $\Omega \subset \{-1, 1\}$, and thus the set is shattered. Assume that our claim holds for $n$, and now set $\Omega \subseteq \{\pm 1\}^{n + 1} = \{\pm 1\}^n \times \{\pm 1\}$. Define
\[
\Omega_+ = \{\omega \in \{\pm 1\}^n: (\omega, 1) \in \Omega\}
\] 
\[
\Omega_{-} = \{ \omega \in \{\pm 1\}^n: (\omega, -1) \in \Omega\}
\]
Then, letting $\tilde{\Omega}_+ = \{(\omega, 1)\in \{\pm 1\}^{n+1}: \omega \in \Omega_+\}$ and $\tilde{\Omega}_-$ similarly, we have $|\Omega| = |\tilde{\Omega}_+| + |\tilde{\Omega}_-| = |\Omega_+| + |\Omega_-|$. By our inductive step, we have sh$(\Omega_+) \geq |\Omega_+|$ and sh$(\Omega_-) \geq |\Omega_-|$. Note that any subset that shatters $\Omega_+$ also shatters $\Omega$, and likewise for $\Omega_{-}$. Note that if a set $\Omega'$ shatters both of them, we are allowed to add on an extra coordinate to get $\Omega' \times \{\pm 1\}$ which shatters $\Omega$. Therefore, 
\[
\txtn{sh}(\Omega_+) \cup \txtn{sh}(\Omega_-) \cup \left\{\sigma \cup \{n + 1\}: \sigma \in \txtn{sh}(\Omega_+) \cap \txtn{sh}(\Omega_-)\right\} \subseteq \txtn{sh}(\Omega)
\]
where the last union is disjoint since the dimensions are different. Therefore, we can now use this set inclusion to complete the induction using the principle of inclusion-exclusion: 
\bal
|\txtn{sh}(\Omega)| &\geq |\txtn{sh}(\Omega_+) \cup \txtn{sh}(\Omega_-)| + |\txtn{sh}(\Omega_+) \cap \txtn{sh}(\Omega_-)| \hspace{3em} \txtn{  (disjoint sets)}
\\
&= |\txtn{sh}(\Omega_+)| + |\txtn{sh}(\Omega_-)| - |\txtn{sh}(\Omega_+) \cap \txtn{sh}(\Omega_-)|  + |\txtn{sh}(\Omega_+) \cap \txtn{sh}(\Omega_-)| 
\\
&= |\txtn{sh}(\Omega_+)| + |\txtn{sh}(\Omega_-)|
\\
&\geq |\Omega_+| + |\Omega_-| = |\Omega|
\end{align*}
which completes the induction as desired. 
\end{proof}

\begin{cor} If $|\Omega| \geq 2^{n -1}$ then there exists $\sigma \subseteq \{1, \cdots, n\}$ with $|\sigma| \geq \lceil \frac{n + 1}{2} \rceil \geq \frac{n}{2}$ such that $\txtn{Proj}_{\R^{\sigma}} \Omega = \{\pm 1\}^{\sigma}$. 
\end{cor}



\blu{2-15-16: 
Last time we left off with the proof of the Sauer-Shelah lemma. To remind you, we were finding ways to find interesting subsets where matrices behave well. Now recall we had a linear algebraic fact which I owe you; I will prove it in an analytic way. The proof has been moved to Section~\ref{sec:kf}.}

\section{Proof of RIP}
\subsection{Step 1}
Now we need another geometric lemma for the proof of %\fixme{change name of theorem gen-srank to restricted invertibility principle? fix reference} 
Theorem~\ref{thm:gen-srank}, the restricted invertibility principle. 

\begin{lem}[Step 1] \llabel{lem:step1}
Fix $m, n, r \in \N$. Let $A: \R^m \to \R^n$ be a linear operator with rank$(A) \geq r$. For every $\tau \subseteq \{1, \ldots, m\}$, denote
\[
E_{\tau} = \left(\txtn{span}((Ae_j)_{j \in \tau})\right)^{\perp}.
\] 
Then there exists $\tau \subseteq \{1, \ldots, m\}$ with $|\tau| = r$ such that for all $j \in \tau$, 
\[
\|\txtn{Proj}_{E_{\tau \setminus \{j\}}} Ae_j\|_2 \geq \frac{1}{\sqrt{m}}\left(\sum_{i = 1}^m s_i(A)^2\right)^{1/2}.
\]
\end{lem}
Basically we're taking the projection of the $j^{th}$ column onto the orthogonal completement of the span of the subspace of all columns in the set except for the $j^{th}$ one, and bounding the norm of that by a dimension term and the square root of the sum of the eigenvalues. 
(This is sharp asymptotically, and may in fact even be sharp as written too---I need to check. \fixme{Check this?})

\begin{proof}
For every $\tau \subseteq \{1, \ldots, m\}$, denote
\[
K_{\tau} = \txtn{conv}\left(\{\pm Ae_j\}_{j \in \tau}\right)
\]
The idea is to make the convex hull have big volume. Once we do that, wewill get all these inequalities for free. Let $\tau \subseteq \{1, \ldots, m\}$ be the subset of size $r$ that maximizes $\txtn{vol}_r(K_{\tau})$. We know that $\txtn{vol}_r(K_{\tau}) > 0$. Observe that for any $\beta \subseteq \{1, \ldots, m\}$ of size $r - 1$ and $i \not\in \beta$, we have 
\[
K_{\beta \cup \{i\}} = \txtn{conv}\left(K_{\beta} \cup \{\pm Ae_i \}\right),
\]
which is a double cone. 
%So all you're doing is having found $E_{\beta}$ and a point $Ae_i$, for the convex hull you get a double cone $E_{\beta}, +Ae_i, -Ae_i$ where $E_{\beta}$ is the orthogonal complement of the space spanned by $\beta$. So 

\ig{images/5-1}{.25}

What is the height of this cone? It is $\|\txtn{Proj}_{E_{\beta}} Ae_i\|_2$, as $E_\be$ is the orthogonal complement of the space spanned by $\be$. Therefore, the $r$-dimensional volume is given by 
\[
\txtn{vol}_r(K_{\beta \cup \{i\}}) = 2\cdot \frac{\txtn{vol}_{r -1}(K_{\beta}) \cdot \|\txtn{Proj}_{E_{\beta}} Ae_i\|_2}{r}
\]
Because $|\tau| = r$ is the maximizing subset of $F_{\Omega}$, for any $j \in \tau$ and $i \in \{1, \ldots, m\}$, choosing $\beta = \tau \setminus \{j\}$, we get
\bal
\vol_r(K_{\be\cup \{j\}}) & \ge \vol_r(K_{\be \cup \{i\}})\\
\implies
\|\txtn{Proj}_{E_{\tau \setminus \{j\}}} Ae_j\|_2 &\geq \|\txtn{Proj}_{E_{\tau \setminus \{j\}}} Ae_i\|_2.
\end{align*}
for every $j \in \tau$ and $i \in \{1, \ldots, m\}$. 
Summing,
\[
m\|\txtn{Proj}_{E_{\tau \setminus \{j\}}} Ae_j\|_2^2
\ge \sum_{i = 1}^m \|\txtn{Proj}_{E_{\tau \setminus \{j\}}} Ae_i\|_2^2 = \|\txtn{Proj}_{E_{\tau \setminus \{j\}}} A\|_{S_2}^2.
\]
Then, for all $j \in \tau$, 
\beq{eq:rip-s1-1}
\|\txtn{Proj}_{E_{\tau \setminus \{j\}}} Ae_j\|_2 \geq \frac{1}{\sqrt{m}}\|\txtn{Proj}_{E_{\tau \setminus \{j\}}} A\|_{S_2}
\eeq
Let's denote $P = \txtn{Proj}_{E_{\tau \setminus \{j\}}}$. Note $P$ is an orthogonal projection of rank $r - 1$. Then, 
\beq{eq:rip-s1-2}
\begin{split}
\|PA\|_{S_2}^2 &= \txtn{Tr}((PA)^*(PA)) = \txtn{Tr}(A^*P^*PA) = \txtn{Tr}(A^*PA) = \txtn{Tr}(AA^*P)\\
&= \txtn{Tr}(AA^*) - \txtn{Tr}(AA^*(I - P)) \geq \sum_{i = 1}^m s_i(A)^2 - \sum_{i = 1}^{r - 1} s_i(A)^2 = \sum_{i = r}^m s_i(A)^2
\end{split}
\eeq
using the Ky Fan maximal principle~\ref{lem:kf}, 
since $I - P$ is a projection of rank $m - r + 1$. 
%So the maximum this could be is the tail. %, which is the variational argument we proved earlier (Ky Fan maximal principle).
%And this is what we claimed. 

Putting~\eqref{eq:rip-s1-1} and~\eqref{eq:rip-s1-2} together gives the result.
\end{proof}

\subsection{Step 2}

\blu{In our proof of the restricted invertibility principle, this is the first step. Before proving it, let me just tell you what the second step looks like. }

\begin{lem}[Step $2$] \llabel{lem:step2}
Let $k, m, n \in \N$, $A: \R^m \to \R^n$, $\rank(A) > k$. Let $\omega \subeq \{1, \ldots, m\}$ with $|\om| = \txtn{rank}(A)$ such that $\{Ae_j\}_{j \in \omega}$ are linearly independent. Denote for every $j \in \Omega$ 
\[
F_j = E_{\omega \setminus \{j\}} = \left(\txtn{span}(Ae_i)_{i \in \omega \setminus \{j\}}\right).
\]
Then there exists $\sigma \subseteq \omega$ with $|\si| =k$ such that 
\[
\|\left(AJ_{\sigma}\right)^{-1}\|_{S_{\infty}} %\leq
\lesssim
 \frac{\sqrt{\txtn{rank}(A)}}{\sqrt{\txtn{rank}(A) - k}} \cdot \max_{j \in \omega} \rc{{\|\txtn{Proj}_{F_j} Ae_j\|}}
\]
%pass to where uniformly big
%first using geometry, find subspace making all the proj big, using singular values. Once you have this lemma, you apply that first, substitute in here.
\end{lem}

Most of the work is in the second step. First we pass to a subset where we have some information about the shortest possible orthogonal project. But Step $1$ saves us by bounding what this can be. Here we use the Grothendieck inequality, Sauer-Shelah, etc. Everything: It's simple, but it kills the restricted invertibility principle. 

%\begin{thm} Step $1$ and Step $2$ imply the Restricted Invertibility Principle.
%\end{thm}
\begin{proof}[Proof of Theorem~\ref{thm:gen-srank} given Step 1 and 2]
Take $A: \R^m \to \R^n$. 
By Step $1$ (Lemma~\ref{lem:step1}), we can find subset $\tau \subseteq \{1, \ldots, m\}$ with $|\tau| = r$ such that for all $j \in \tau$, 
\beq{eq:step1}
\|\txtn{Proj}_{E_{\tau \setminus \{j\}}} Ae_j \|_2 \geq \frac{1}{\sqrt{m}}\left(\sum_{i = r}^m s_i(A)^2\right)^{1/2}.
\eeq
Now we apply Step $2$ (Lemma~\ref{lem:step2}) to $AJ_{\tau}$, using $\omega = \tau$, and find a further subset $\sigma \subseteq \tau$ such that 
\bal
\|\left(AJ_{\sigma}\right)^{-1}\|_{S_{\infty}} 
&\le \min_{k<r<\rank(A)}\sfc{\rank(A)}{\rank(A)-r} \max_{j\in \om}\rc{{\ve{\Proj_{F_j} Ae_j}}}\\
&\leq \min_{k < r < \txtn{rank}(A)} \sqrt{\frac{mr}{(r - k)\sum_{i = r}^m s_i(A)^2}}
\end{align*}
which we get by plugging directly in $r$ for the rank and using Step $1$~\eqref{eq:step1} to get the denominator. 
\end{proof}

%INSERT_HERE

%\appendix
%\input{distribution_chapters/a.tex}

%\bibliographystyle{plain}
%\bibliography{\filepath/refs}

\printnomenclature
\printindex
\end{document}