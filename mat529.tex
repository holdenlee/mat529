\def\filepath{templates}
%\def\filepath{C:/Users/holden-lee/Dropbox/Math/templates}

\input{\filepath/packages_book.tex}
\input{\filepath/theorems_with_boxes.tex}
\input{\filepath/macros.tex}
%\documentclass[10pt]{article}
%\usepackage{amsmath,amsfonts,amssymb,amsthm}
\usepackage[section]{placeins}
\usepackage[labelfont=bf]{caption}
%\usepackage[usenames,dvipsnames]{color} % Required for specifying custom colors and referring to colors by name
%\usepackage[pdftex]{hyperref} % For hyperlinks in the PDF
%\hypersetup{
%  colorlinks=true,
%  linkcolor=MyBlue, 
%  citecolor=MyRed,
%  urlcolor= MyBlue
%}
\usepackage{aliascnt}
%\usepackage[nameinlink, capitalize, noabbrev]{cleveref}   
%\usepackage{graphicx}
%\graphicspath{ {images/} }

\usepackage{algorithm}
\usepackage{algpseudocode}

%\usepackage{geometry}
%\geometry{verbose,tmargin=2cm,bmargin=2.5cm,lmargin=2.5cm,rmargin=2.5cm}

%\makeatletter

%\newtheorem{theorem}{Theorem}[section]
%
%\newtheorem{lemma}[theorem]{Lemma}
%\newtheorem{corollary}[theorem]{Corollary}
%\newtheorem{claim}[theorem]{Claim}
%
%
%\theoremstyle{definition}
%\newtheorem{definition}[theorem]{Definition}
%\theoremstyle{definition}
%\newtheorem{example}[theorem]{Example}
%\theoremstyle{definition}
%\newtheorem{remark}[theorem]{Remark}
%\theoremstyle{definition}
%\newtheorem{conjecture}[theorem]{Conjecture}
%
%\makeatother


%%% fill in details here
%\def \lecturenum  {1}
%\def \lecturedate {October 23; Nov 10, 2015}
%\def \scribe      {Kiran Vodrahalli}
%%%

%\definecolor{MyRed}{rgb}{0.99, 0.0, 0.0} 
%\definecolor{MyGreen}{rgb}{0.0,0.4,0.0} 
%\definecolor{MyBlue}{rgb}{0.0, 0.0, 0.6}
%
%\newcommand{\mc}[1]
%{\mathcal{#1}}

%\newcommand{\txt}[1]
%{\textnormal{#1}}

\newcommand{\prob}[2]
{\textbf{P}_{#1}\{{#2}\}}

\newcommand{\Ek}[2]
{\textbf{E}_{#1}[{#2}]}

\newcommand{\mi}[1]
{\txt{min}_{#1}}

%\newcommand{\ma}[1]
%{\txt{max}_{#1}}

\newcommand{\errs}[1]
{\txt{err}_{\mc{S}}(#1)}

\newcommand{\errd}[1]
{\txt{err}_{\mc{D}}(#1)}


\newcommand{\vcdim}[1]
{\textbf{VC-dim}\left(#1\right)}

\newcommand{\nlg}[1]
{\txt{ln}\left(#1\right)}

\newcommand{\vv}[1]
{\textbf{#1}}

\newcommand{\circled}[1]{(#1)}

%\usepackage{tikz}
%\newcommand*\circled[1]{\tikz[baseline=(char.base)]{
%		\node[shape=circle,draw,inner sep=2pt] (char) {#1};}}
\input{\filepath/formatting.tex}
\input{\filepath/other.tex}
\input{\filepath/theorem_num.tex}

\def\name{Metric embeddings and geometric inequalities}


\pagestyle{fancy}
%\addtolength{\headwidth}{\marginparsep} %these change header-rule width
%\addtolength{\headwidth}{\marginparwidth}
\lhead{MAT529}
\chead{} 
\rhead{Metric embeddings and geometric inequalities} 
\lfoot{} 
\cfoot{\thepage} 
\rfoot{} % !! Remember to change the problem set number
\renewcommand{\headrulewidth}{.3pt} 
%\renewcommand{\footrulewidth}{.3pt}
\setlength\voffset{0in}
%\setlength\textheight{648pt}

\begin{document}
%\input{\filepath/titlepage.tex}
%\maketitle
\title{Metric embeddings and geometric inequalities}
\author{Lectures by Professor Assaf Naor \\ Scribes: Holden Lee and Kiran Vodrahalli}
\maketitle


\tableofcontents

%\startcontents
%\printcontents{ }{-1}{}

\chapter*{Introduction}

%Princeton, Spring 2016. 

The topic of this course is geometric inequalities with applications to metric embeddings; and we are actually going to do things more general than metric embeddings: Metric geometry. Today I will roughly explain what I want to cover and hopefully start proving a first major theorem. 
The strategy for this course is to teach novel work. Sometimes topics will be covered in textbooks, but a lot of these things will be a few weeks old. There are also some things which may not have been even written yet. I want to give you a taste for what's going on in the field. Notes from the course I taught last spring are also available.

One of my main guidelines in choosing topics will be topics that have many accessible open questions. I will mention open questions as we go along. I'm going to really choose topics that have proofs which are entirely self-contained. I'm trying to assume nothing. My intention is to make it completely clear and there should be no steps you don't understand. 

Now this is a huge area. I will explain some of the background today. I'm going to use proofs of some major theorems as excuses to see the lemmas that go into the proofs. Some of the theorems are very famous and major, and you're going to see some improvements, but along the way, we will see some lemmas which are \emph{immensely powerful}. So we will always be proving a concrete theorem. But actually somewhere along the way, there are lemmas which have wide applicability to many many areas. These are excuses to discuss methods, though the theorems are important. 

The course can go in many directions: If some of you have some interests, we can always change the direction of the course, so express your interests as we go along. 

%Notes from Assaf Naor's class ``Geometric inequalities with applications to metric embeddings" at 

%More generally we'll cover metric geometry.

%I'll teach things that just appeared or are still in the works. Most of them aren't in books yet; some of them haven't even be written yet. I'll give a taste of what's going on in the field. I'll choose topics with many open questions associated with them and with self-contained proofs. 

%This is a huge area. I'll use proofs of major theorems as excuses to see the lemmas that go into the proofs. The theorems are famous; in addition to being famous, they use lemmas that are immensely powerful. Instead of presenting the geometric inequalities and then applications afterwards, we'll see the lemmas on the way to proving these big theorems.

%twists in proofs new, useful.
%Express your interest---I'm flexible.

%I taught a course.
%no overlap.

%some commericals on what to prove later, puzzles.

%\setcounter{chapter}{1}
\chapter{Intro}

\blu{2/1/16}


\section{The Ribe Program}
Our main motivation is the \ivocab{Ribe Program}. The program is inspired from a theorem from 1975 called Ribe's Rigidity Theorem~\ref{thm:ribe}. The theorem is about Banach spaces, specifically a relationship between their linear structure and metric structure.

\begin{df}
Let $(X,\ved_X), (Y,\ved_Y)$ by Banach spaces. We say that $X$ is (crudely) \ivocab{finitely representable} in $Y$ if there exists $K>0$ such that for every finite-dimensional linear subspace $F\subeq X$, there is a linear operator $S:F\to Y$ such that for every $x\in F$, 
\[
\ve{x}_X\le \ve{Sx}_Y\le K\ve{x}_X.
\]
%every finite dimensional subspace of $X$ is represented in $Y$.
\end{df}
Note $K$ is decided once and for all, before the subspace $F$ is chosen.

(Some authors use ``finitely representable" to mean that this is true for any $K=1+\ep$. We will not follow this terminology.)

Finite representability is important because %$X$ is finitely representable in $Y$  
it allows us to conclude that $X$ has the same finite dimensional linear properties (\ivocab{local properties}) as $Y$. That is, it preserves any invariant involves finitely many vectors, their lengths, etc.

Let's introduce some local properties like type. To motivate the definition, 
%For example, for type $p\ge 1$, take any $y_1,\ldots, y_n\in Y$. 
consider the triangle inequality, which says 
\[
\ve{y_1+\cdots +y_n}_Y\le \ve{y_1}_Y+\cdots +\ve{y_n}_Y.
\]
In what sense can we improve the triangle inequality? In $L^1$ this is the best you can say. In many spaces there are ways to improve it if you think about it correctly.

For any choice $\ep_1,\ldots, \ep_n\in \{\pm 1\}$, 
\[
\ve{\sum_{i=1}^n \ep_i y_i}_Y \le \sum_{i=1}^n \ve{y_i}_Y.
\]
\begin{df}\llabel{df:type}
Say that $X$ has \ivocab{type} $p$ if there exists $T>0$ such that for every $n, y_1,\ldots, y_n\in Y$, 
\[
\EE_{\ep\in \{\pm 1\}^n} \ve{\sum_{i=1}^n \ep_i y_i}_Y\le T\ba{\sum_{i=1}^n \ve{y_j}_Y^p}^{\rc p}.
\]
The $L^p$ norm is always at most the $L^1$ norm; if the lengths are spread out, this is asymptotically much better. Say $Y$ has \ivocab{nontrivial type} if $p>1$.
\end{df}

For example, $L_p(\mu)$ has type $\min(p,2)$.

Later we'll see a version of ``type" for metric spaces. How far is the triangle inequality from being an equality is a common theme in many questions. In the case of normed spaces, this controls a lot of the geometry. Proving a result for $p>1$ is hugely important.

%X innherits the local linear properties of $Y$.

\begin{pr}\llabel{pr:finrep-type}
If $X$ is finitely representable and $Y$ has type $p$ then also $X$ has type $p$.
\end{pr}

\begin{proof}
Let $x_1,\ldots, x_n\in X$. Let $F=\spn\{x_1,\ldots, x_n\}$. Finite representability gives me $S:F\to Y$. Let $y_i=Sx_i$. What can we say about $\sum \ep_iy_i$?
\bal
 \EE_{\ep}\ve{\sum_{i=1}^n \ep_iy_i}_Y &=  \EE_{\ep}\ve{S(\sui \ep_i x_i)}_Y\\
&\ge \EE_{\ep} \ve{\sui \ep_iX_i}_X\\
 \EE_{\ep}\ve{\sum_{i=1}^n \ep_iy_i}_Y &\le T\pa{\sui \ve{Sx_i}^p}^{\rc p}\\
& \le TK\pa{\sui \ve{x_i}^p}^{\rc p}.
%preserves any invariant involves finitely many vectors, their lengths, etc.
\end{align*}
Putting these two inequalities together gives the result.
\end{proof}

\begin{thm}[Kahane's inequality]\index{Kahane's inequality}
For any normed space $Y$ and $q\ge 1$, for all $n$, $y_1,\ldots, y_n\in Y$,
\[
\E\ve{\sum_{i=1}^n \ep_i y_i}\gtrsim_q \pa{
\E\ba{\ve{\sum_{i=1}^n \ep_i y_i}_Y^q}
}^{\rc q}.
\]
Here $\gtrsim_q$ means ``up to a constant"; subscripts say what the constant depends on. The constant here does not depend on the norm $Y$.
\end{thm}
Kahane's Theorem tells us that the LHS of Definition~\ref{df:type} can be replaced by any norm, if we change $\le$ to $\lesssim$. We get that having type $p$ is equivalent to 
\[
\E\ve{\sum_{i=1}^n \ep_i y_i}_Y^p \lesssim T^p \sum_{i=1}^n \ve{y_i}_Y^p.
\]
Recall the \ivocab{parallelogram identity} in a Hilbert space $H$:
\[
\E\ve{\sum_{i=1}^n \ep_i y_i}^2 = \sum_{i=1}^n \ve{y_i}_H^2.
\]
A different way to understand 
%not as good a way
the inequality in the definition of ``type" is: how far is a given norm from being an Euclidean norm? 
The \ivocab{Jordan-von Neumann Theorem} says that if  parallelogram identity holds then it's a Euclidean space. What happes if we turn it in an inequality?
\[
\E\ve{\sum_{i=1}^n \ep_i y_i}_H^2 \gle T\sum_{i=1}^n \ve{y_i}_H^2.
\]
Either inequality \emph{still} characterizes a Euclidean space. 

What happens if we add constants or change the power? We recover the definition for type and cotype (which has the inequality going the other way):
\[
\E\ve{\sum_{i=1}^n \ep_i y_i}_H^q \begin{array}{c}
\gtrsim\\
\lesssim
\end{array} \sum_{i=1}^n \ve{y_i}_H^q.
\]

\begin{df}
Say it has \ivocab{cotype $q$} if
\[
\sum_{i=1}^n \ve{y_i}_Y^q \lesssim C^q \E \ve{\sum_{i=1}^n \ep_i y_i}_Y^q
\]
\end{df}
R. C. James invented the local theory of Banach spaces, the study of geometry that involves properties involving finitely many vectors ($\forall x_1,\ldots, x_n, P(x_1,\ldots, x_n)$ holds). As a counterexample, reflexivity cannot be characterized using finitely many vectors (this is a theorem).

Ribe discovered link between metric and linear spaces.

First, terminology.

\begin{df}
Two Banach spaces are \ivocab{uniformly homeomorphic} if there exists $f:X\to Y$ that is 1-1 and onto and $f,f^{-1}$ are uniformly continuous. 
\end{df}

Without the word ``uniformly", if you think of the spaces as topological spaces, all of them are equivalent. Things become interesting when you quantify! ``Uniformly" means you're controlling the quantity.
\begin{thm}[Kadec]\index{Kadec's Theorem}
Any two infinite-dimensional separable Banach spaces are homeomorphic.
\end{thm}
%They are all the same as Hilbert spaces.
This is a amazing fact: these spaces are all topologically equivalent to Hilbert spaces!

Over time people people found more examples of Banach spaces that are homeomorphic but not uniformly homeomorphic. Ribe's rigidity theorem clarified a big chunk of what was happening.

\index{rigidity theorem}
\begin{thm}[Rigidity Theorem, Martin Ribe (1975)]\llabel{thm:ribe}
%government official in Sweden.
Suppose that $X,Y$ are uniformly homeomorphic Banach spaces.  Then $X$ is finitely representable in $Y$ and $Y$ is finitely representable in $X$.
\end{thm} 
For example, for $L^p$ and $L^q$, for $p\ne q$ it's always that case that one is not finitely representable in the other, and hence by Ribe's Theorem, $L^p,L^q$ are not uniformly homeomorphic.
%averages of sums with random signs
%squaring things
%$\ep$-uncorrelated, very linear
%we will deduce from this general phen that you can't deform $L_p$ into $L_q$ even in a nonlinear category.
%if you're equivalent in a weak category, you're equivalent in a stronger category.
(When I write $L_p$, I mean $L_p(\R)$.)

\begin{thm}
For every $p\ge 1,p\ne2$, $L_p$ and $\ell_p$ are finitely representable in each other, yet not uniformly homeomorphic.
\end{thm}
(Here $\ell_p$ is the sequence space.) 
Proof of finite representability is a good exercise. You'll need to remember some measure theory.

When $p=2$, $L_p,\ell_p$ are separable and isometric.

The theorem in various cases was proved by:
\begin{enumerate}
\item
$p=1$: Enflo
\item
$1<p<2$: Bourgain
\item
$p>2$: Gorelik, applying the Brouwer fixed point theorem (topology)
\end{enumerate}
%can't deform function into sequence space in uniformly continuous way.

Every linear property of a Banach signs which is local (type, cotype, etc.; involving summing, powers, etc.) is preserved under a general nonlinear deformation.

After the theorem, people wondered: can we reformulate the local theory of Banach spaces without mentioning anything about the linear structure? 
%uniform homomorphism
In metric spaces, we are only allowed to discuss distances between points, not linear properties (ex. summing up). Suppose we can reformulate local theory in this way---find a dictionary that reformulates each linear property and theorem about linear properties as properties and theorems involving distances. Then we can state the analogous theorems for metric spaces. In particular, we can discuss when metric spaces have type and cotype. Maybe the theorems remain true---often they do, for different reasons. Now we can apply the theorem to graphs, groups, etc. %Somewhere
%transl concept, theorems, phenomenon fro linear spaces to metric spaces.
Thus, we end up applying theorems on linear spaces in situations with \emph{a priori nothing} to do with linear spaces. This is massively powerful.
%massively powerful. 
%mysterious dictionary.

There are very crucial entries that are missing in the dictionary. We don't even now how to define many of the properties! %how to reformulate
This program has many interesting proofs. Some of the most interesting conjectures are how to define things!

\begin{cor}\llabel{cor:uh-type}
If $X,Y$ are uniformly homeomorphic and if one of them is of type $p$, then the other does. 
\end{cor}
This follows from Ribe's Theorem~\ref{thm:ribe} and Proposition~\ref{pr:finrep-type}. Can we prove something like this theorem without using Ribe's Theorem~\ref{thm:ribe}?
%going back to the abstract principle?
We want to reformulate the definition of type using only the distance, so this becomes self-evident.

Enflo had an amazing idea. 
Suppose $X$ is a Banach space, $x_1,\ldots, x_n\in X$. The type $p$ inequality  says 
\beq{eq:type-p}
\E\ba{\ve{\sum_{i=1}^n \ep_i x_i}^p} \lesssim_X \sum_{i=1}^n \ve{x_i}^p.
\eeq
Let's rewrite this in a silly way. Define $f:\{\pm 1\}^n\to X$ by
\[
f(\ep_1,\ldots, \ep_n)=\sum_{i=1}^n \ep_i x_i.
\]
Write $\ep=(\ep_1,\ldots, \ep_n)$. Multiplying by $2^n$, we can write the inequality~\eqref{eq:type-p} as
\beq{eq:type-gen}
\E\ba{
\ve{f(\ep) - f(-\ep)}^p
}\lesssim_X
\sum_{i=1}^n \E\ba{ \ve{f(\ep)-f(\ep_1,\ldots, \ep_{i-1}, -\ep_i, \ep_{i+1},\ldots, \ep_n)}^p}.
\eeq
This inequality just involves distances between points $f(\ep)$, so it is the reformulation we seek.

\begin{df}\llabel{df:enflo}
A metric space $(X,d_X)$ has \ivocab{Enflo type $p$} if there exists $T>0$ such that $n$ and every $f:\{\pm 1\}^n\to X$,
\[
\E[d_X(f(\ep), f(-\ep))^p]\le T^p \sum_{i=1}^n \E[ d_X(f(\ep), f(\ep_1,\ldots, \ep_{i-1}, -\ep_i, \ep_{i+1},\ldots, \ep_n))].
\]
\end{df}
This is bold. 
%you should worry about this!
%The discrete cube is all these vectors with $\ep$. The function goes into the Banach space and assigns $2^n$ arbitrary points.
In~\eqref{eq:type-gen}, the points had to be vertices of a cube, but in Definition~\ref{df:enflo}, they are arbitrary. The moment you choose the labelings, you impose a cube structure between the points. %on the cube.

\ig{images/1-1}{.25}

The inequality says:
\[
\sum \diag^p\lesssim_X\sum \text{edge}^p.
\]
The total $p$th power of lengths of diagonals is up to a constant, at most the same thing over all edges.

This is a vast generalization of type; we don't even know a Banach space satisfies this.
%This is a vast generalization
The following is one of my favorite conjectures.
\begin{conj}[Enflo]
If a Banach space has type $p$ then it also has Enflo type $p$.
\end{conj}
This has been open for 40 years. We will prove the following.
\begin{thm}[Bourgain-Milman-Wolfson, Pisier]
If $X$ is a Banach space of type $p>1$ then $X$ also has type $p-\ep$ for every $\ep>0$.
\end{thm}
If you know the type inequality for parallelograms, you get it for arbitrary sets of points, up to $\ep$.

The conjecture is true for many spaces. For example, it's true for $L^4$. Index functions by vertices; some pairs are edges, some are diagonals; then the $L^4$ norm of the diagonals is at most that of the edges.

%know just for linear functions, want to deduce for arbitrary functions.

The moment that I throw away the linear structure, Proposition~\ref{pr:uh-type} becomes easy.
\begin{pr}\llabel{pr:uh-enflo}
If $X,Y$ are uniformly homeomorphic Banach spaces and $Y$ has Enflo type $p$, then so does $X$.
\end{pr}

\begin{lem}[Corson-Klee]\index{Corson-Klee Lemma}\llabel{lem:corson-klee}
If $X,Y$ are Banach spaces and $\psi:X\to Y$ are uniformly continuous, then for every $a>0$ there exists $L(a)$ such that 
\[
\ve{x_1-x_2}_X\ge a \implies \ve{\psi(x_1)-\psi(x_2)}\le L\ve{x_1-x_2}.
%lower bound gives lipschitz
\]
\end{lem}

\begin{proof}[Proof of~\ref{pr:uh-enflo} given Lemma~\ref{lem:corson-klee}]
%Given $f:\{\pm 1\}^n \to X$, 
By definition of uniform homeomorphic, there exists a homeomorphism $\psi:X\to Y$ such that $\psi,\psi^{-1}$ are uniformly continuous. Lemma~\ref{lem:corson-klee} tells us that $\psi$ perserves distance up to a constant. Dividing so that the smallest nonzero distance you see is at least 1, we get the same inequality in the image and the preimage.
\end{proof}
%Then we know 
%distance preserved under constants.
%can rescale. 
%divide so smallest nonzer distance you see is at least 1. then it's the same inequality in the image and preimge.
%make Ribe's theorem explicit.
%version when I took power 1. Improvement of triangle inequlities
%don't know equivalences of this type in metric.

The parallelogram inequality for exponent 1 instead of 2 follows from using the triangle inequality on all possible paths for all paths of diagonals. Type $p>1$ is a strengthening of the triangle inequality. For which metric spaces does it hold?

%Fruits of Ribe's theorem.
%Switch gears to metric spaces.
What's an example of a metric space where the inequality doesn't hold with $p>1$? The cube itself (with $L^1$ distance).
\[
n^p\nleq n.
\]
I will prove to you that this is the only obstruction: given a metric space that doesn't contain bi-Lipschitz embeddings of arbitrary large cubes, the inequality holds. 

We know an alternative inequality involving distance equivalent to type; I can prove it. It is, however, not a satisfactory solution to the Ribe program. There are other situations where we have complete success.

We will prove some things, then switch gears, slow down and discuss Grothendieck's inequality and applications. They will come up in the nonlinear theory later.
\section{Bourgain's Theorem implies Ribe's Theorem}

\blu{2-3-16}

%Last time we stated Ribe's Theorem.

We will use the Corson-Klee Lemma~\ref{lem:corson-klee}. %This is the reason that all the inequalities.

%\fixme{Concrete versions of Ribe's Theorem. Charcterize using finie inequality, trivially preserved under uniform h. Minimum smallest distance, scale it to 1. Then the unif homeo is bi-Lipschitz. write the estimates.}
%The hardest part is to show that the property involving 
%

\begin{proof}[Proof of Lemma~\ref{lem:corson-klee}]
Suppose $x,y\in X$, $\ve{x-y}\ge a$. Break up the line segment from $x,y$ into intervals of length $a$; let $x=x_0,x_1,\ldots, x_k=y$ be the endpoints of those intervals, with
\[
\ve{x_{i+1}-x_i}\le a.
\]
The \vocab{modulus of continuity} is defined as
\[
W_f(t)=\sup \set{\ve{f(u)-f(v)}}{u,v\in X,\ve{u-v}\le t}. 
\]
Uniform continuity says $\lim_{t\to 0}W_f(t)=0$. 
The number of intervals is
\[
k\le \fc{\ve{x-y}}{a}+1\le \fc{2\ve{x-y}}a.
\]
Then
\bal
\ve{f(x)-f(y)} & \le \sum_{i=1}^k \ve{f(x_i)-f(x_{i-1})} \\
&\le KW_f(a) \le \fc{2W_f(a)}{a}\ve{x-y},
\end{align*}
so we can let $L(a) = \fc{2W_f(a)}{a}$.
\end{proof}

\subsection{Bourgain's discretization theorem}

There are 3 known proofs of Ribe's Theorem.
\begin{enumerate}
\item
Ribe's original proof, 1987.
\item
HK, 1990, a genuinely different proof.
\item
Bourgain's proof, a Fourier analytic proof which gives a quantitative version. This is the version we'll prove.
\end{enumerate}
Bourgain uses the Discretization Theorem~\ref{thm:bdt}. There is an amazing open problem in this context.

Saying $\delta$ is big says there is a not-too-fine net, which is enough. Therefore we are interested in lower bounds on $\delta$. 

\begin{df}[Discretization modulus]\index{discretization modulus}
Let $X$ be a finite-dimensional normed space $\dim(X)=n<\iy$. Let the target space $Y$ be an arbitrary Banach space.  Consider the unit ball $B_X$ in $X$. Take %an $\de$-net (the distance between points is at most $\de$) 
a maximal $\de$-net $\cal N_\de$ in $B_X$. %(a maximal $\de$-separated subset). 
Suppose we can embed $\cal N_\de$ into $Y$ via $f:\cal N_\de\to Y$. Suppose we know in $Y$ that
\[
\ve{x-y}\le \ve{f(x)-f(y)}\le D\ve{x-y}.
\]
for all $x, y \in N_{\de}$. 
%also interesting not in context of normed spaces.
(We say that $\cal N_\de$ embeds with distortion $D$ into $Y$.)
\end{df}
You can prove using a nice compactness argument that if this holds for $\de$ is small enough, then the entire space $X$ embeds into $Y$ with rough the same distortion. 
Bourgain's discretization theorem~\ref{thm:bdt} says that you can choose $\de=\de_n$ to be independent of the geometry of $X$ and $Y$ such that if you give a $\delta$-approximation of the unit-ball in the $n$-dimensional norm, you succeed in embedding the whole space. 
%you lose a factor
%if $Y$ is reflexive, you can get this easily; for general $Y$ you need more work.
%$\de$-approximation encodes all the information. Whenever you try . Succeed for the net, succeed for object.

I often use this theorem in this way: I use continuous methods to show embedding $X$ into $Y$ requires big distortion; immediately I get an example with a finite object. Let us now make the notion of distortion more precise. 

\begin{df}[Distortion]
Suppose $(X,d_X),(Y,d_Y)$ are metric spaces $D\ge 1$. We say that $X$ embeds into $Y$ with \ivocab{distortion} $D$ if there exists $f:X\to Y$ and $s>0$ such that for all $x,y\in X$,
\[
Sd_X(x,y) \le d_Y(f(x),f(y)) \le DSd_X(x,y).
\]
\nomenclature{$C_Y(X)$}{infimum of $D$ where $X$ imbeds into $Y$ with distortion $D$}
The infimum over those $D\ge 1$ such that $X$ embeds into $Y$ with distortion is denoted $C_Y(X)$. This is a measure of how far $X$ is being from a subgeometry of $Y$. 
%how far from being subgeometry.
\end{df}

\begin{df}[\vocab{Discretization modulus}] \index{discretization modulus}\label{df:disc-mod}
\nomenclature{$\de_{X\hra Y}(\ep)$}{supremum over all those $\de>0$ such that for every $\de$-net $\cal N_\de$ in $B_X$, $C_Y(\cal N_\de)\ge (1-\ep)C_Y(X)$}
Let be a $n$-dimensional normed space and $Y$ be any Banach space, $\ep\in (0,1)$. Let $\de_{X\hra Y}(\ep)$ be the supremum over all those $\de>0$ such that for every $\de$-net $\cal N_\de$ in $B_X$, 
\[C_Y(\cal N_\de)\ge (1-\ep)C_Y(X).\]

Here $B_X:=\set{x\in X}{\ve{x}\le 1}$.
\end{df}
In other words, the distortion of the $\delta$-net is not much larger than the distortion of the whole space. That is, the discrete $\delta$-ball encodes almost all information about the space when it comes to embedding into $Y$: If you got $C_Y(\cal N_\de)$ to be small, then the distortion of the entire object is not much larger.
%a discretization modulus. $\de$-net encodes all the info when it comes to embedding things into $Y$.
\begin{thm}[Bourgain's discretization theorem]\label{thm:bdt}
For every $n$, $\ep\in (0,1)$, for every $X,Y$, $\dim X=n$, 
\[
\de_{X\hra Y}(\ep)\ge  e^{-\pf{n}{\ep}^{Cn}}.
\]
Moreover for $\de=e^{-(2n)^{Cn}}$, we have $C_Y(X)\le 2C_Y(\cal N_\de)$. %via a linear operator. 
\end{thm}
Thus there is a $\de$ dependending on the dimension such that in any $n$-dimensional norm space, the unit ball it encodes all the information of embedding $X$ into \emph{anything else}. It's only a function of the dimension, not of any of the relevant geometry. 



%lower bound
%$\de$ dependent on dimension, 
The theorem says that if you look at a $\de$-net in the unit ball, it encodes all the information about $X$ when it comes to embedding into everything else. %There is a discrete version , 
The amount you have to discretize is just a function of the dimension, and not of any of the other relevant geometry.
\begin{rem}
The proof is via a linear operator. All the inequality says is that you can find a \emph{function} with the given distortion. The proof will actually give a \emph{linear operator}. 
\end{rem}

The best known upper bound is 
\[
\de_{X\hra Y}\prc2 \lesssim\rc n.
\]
%how mch to discretize to encode all information.
The latest progress was $1987$, there isn't a better bound yet. You have a month to think about it before you get corrupted by Bourgain's proof. 

There is a better bound when the target space is a $L^p$ space.
\begin{thm}[Gladi, Naor, Shechtman]
For every $p\ge 1$, if $\dim X=n$,
\[
\de_{X\hra L_p}(\ep)\gtrsim \fc{\ep^2}{n^{\fc 52}}
\]
\end{thm}
(We still don't know what the right power is.) 
The case $p=1$ is important for applications. There are larger classes for spaces where we can write down axioms for where this holds. 
%Ostrowski and ...
There are crazy Banach spaces which don't belong to this class, so we're not done. %We'll get to this in a month. 
We need more tools to show this: Lipschitz extension theorems, etc.

\subsection{Bourgain's Theorem implies Ribe's Theorem}

With the ``moreover," Bourgain's theorem implies Ribe's Theorem~\ref{thm:ribe}.

\begin{proof}[Proof of Ribe's Theorem~\ref{thm:ribe} from Bourgain's Theorem~\ref{thm:bdt}]
Let $X,Y$ be Banach spaces that are uniformly homeomorphic. By Corson-Klee~\ref{lem:corson-klee}, there exists $f:X\to Y$ such that 
\[
\ve{x-y}\ge 1\implies \ve{x-y}\le \ve{f(x)-f(y)}\le K\ve{x-y}.
\]
%can always normalize in the lower bound
(Apply the Corson-Klee lemma for both $f$ and the inverse.)

In particular, if $R>1$ and $\cal N$ is a 1-net in 
\[
RB_X = \set{x\in X}{\ve{x}\le R},
\]
then $C_Y(\cal N)\le K$.
%define distortion to be scale invariant
Equivalently, for every $\de>0$ every $\de$-net in $B_X$ satisfies $C_Y(\cal N)\le K$. 
If $F\subeq X$ is a finite dimension subspace and $\de = e^{-(2\dim F)^{C\dim F}}$, then by the ``moreover" part of Bourgain's Theorem~\ref{thm:bdt}, there exists a linear operator $T: F\to Y$ such that 
\[
\ve{x-y}\le \ve{Tx-Ty}\le 2K\ve{x-y}
\]
for all $x,y\in F$. This means that $X$ is finitely representable. 
\end{proof}
%computer science care about finite things.
The motivation for this program comes in passing from continuous to discrete. The theory has many applications, e.g. to computer science whcih cares about finite things.
I would like an improvement in Bourgain's Theorem~\ref{thm:bdt}.

First we'll prove a theorem that has nothing to do with Ribe's Theorem. There are lemmas we will be using later. It's an easier theorem. It looks unrelated to metric theory, but the lemmas are relevant. 
%digression to warm up.

%grothendieck inequalities, summing. 

\chapter{Restricted invertibility principle}
\section{Restricted invertibility principle}

\subsection{The first restricted invertibility principles}
%link to nonlinear world will appear later.
%famous, great achievements of analysis in the 1980's.
We take basic facts in linear algebra and make things quantitative. This is the lesson of the course: when you make things quantitative, new mathematics appears.

\begin{pr}
If $A:\R^m\to \R^n$ is a linear operator, then there exists a linear subspace $V\subeq \R^n$ with $\dim(V)=\rank(A)$ such that $A:V\to A(V)$ is invertible. 
\end{pr}
%In modern math we need a quantitative version of this.

What's the quantitative question we want to ask about this? Invertibility just says that an inverse exists. Can we find a large subspace where not only is $A$ invertible, but the inverse has small norm?

We insist that the subspace is a coordinate subspace.  Let $e_1,\ldots, e_m$ be the standard basis of $\R^m, e_j=(0,\ldots, \ub1j,0,\ldots)$. 
The goal is to find a ``large" subset $\si\subeq \{1,\ldots, m\}$ such that  $A$ is invertible on $\R^\si$ where 
\nomenclature{$\R^{\si}$}{$\set{(x_1,\ldots, x_n)\in \R^m}{x_i=0\text{ if }i\nin \si}$}
\[
\R^{\si}:=\set{(x_1,\ldots, x_n)\in \R^m}{x_i=0\text{ if }i\nin \si}
\]
and the norm of $A^{-1}:A(\R^\si)\to \R^{\si}$ is small.

A priori this seems a crazy thing to do; take a small multiple of the identity. But we can find conditions that allow us to achieve this goal.

\begin{thm}[Bourgain-Tzafriri restricted invertibility principle, 1987]\index{restricted invertibility principle}\llabel{thm:btrip}
Let $A:\R^m\to \R^m$ be a linear operator such that 
\[
\ve{Ae_j}_2=1
\]
for every $j\in \{1,\ldots, m\}$. Then there exist $\si\subeq \{1,\ldots, m\}$ such that
\begin{enumerate}
\item
$|\si|\ge \fc{cm}{\ve{A}^2}$, where $\ve{A}$ is the operator norm of $A$.
\item
$A$ is invertible on $\R^{\si}$ and the norm of $A^{-1}:A(\R^{\si})\to \R^{\si}$ is at most $C'$ (i.e., $\ve{AJ_{\si}}_{S^{\iy}}\le C'$, to use the notation introduced below).
\end{enumerate}
Here $c,C'$ are universal constants.
\end{thm}
%No, condition on rank of $A$.
%rank has to be big, proportional. 
%if repeated a lot get huge norm. The fact that norm small. Exercise in linear algebra, says something about rank being big.
%the norm is at most a universal constant.

Suppose the biggest eigenvalue is at most 100. Then you can always find a coordinate subset of proportional size such that on this subset, $A$ is invertible and the inverse has norm bounded by a universal  constant.

All of the proofs use something very amazing.

This proof is from 3 weeks ago. 
This has been reproved many times. I'll state a theorem that gives better bound than the entire history. 
%does this imply that the rank is large?
%$\si$ should depend on the norm of $A$.

This was extended to rectangular matrices. (The extension is nontrivial.)

Given $V\subeq \R^m$ a linear subspace with $\dim V=k$ and $A:V\to \R^m$ a linear operator, the singular values of $A$ 
\[
s_1(A)\ge s_2(A)\ge \cdots \ge s_k(A)
\]
are the eigenvalues of $(A^*A)^{\rc 2}$. We can decompose
\[
A=UDV
\]
where $D$ is a matrix with $s_i(A)$'s on the diagonal, and $U,V$ are unitary.
\begin{df}
\nomenclature{$\ve{A}_{S_p}$}{Schatten-von Neumann $p$-norm}
For $p\ge 1$ the \ivocab{Schatten-von Neumann $p$-norm} of $A$ is 
\bal 
\ve{A}_{S_p}
&:=\pa{\sum_{i=1}^k  s_i(A)^p}^{\rc p}\\
&= (\Tr((A^*A)^{\fc p2}))^{\rc p}\\
&= (\Tr((AA^*)^{\fc p2}))^{\rc p}
\end{align*}
\end{df}
The cases $p=\iy,2$ give the operator and Frobenius norm,
\bal
\ve{A}_{S_{\iy}}&= \text{operator norm}\\
\ve{A}_{S_2}& = \sqrt{\Tr(A^*A)} = \pa{\sum a_{ij}^2}^{\rc 2}.
\end{align*}



\begin{exr}
$\ved_{S_p}$ is a norm on $\cal M_{n\times m}(\R)$. You have to prove that given $A,B$,
\[
(\Tr([(A+B)^*(A+B)]^{\fc p2}))^{\rc p}
\le
%nicer facts of linear algebra.
(\Tr((A^*A)^{\fc p2}))^{\rc p}+(\Tr((B^*B)^{\fc p2}))^{\rc p}.
\]
%if they commute it is trivial
\end{exr}
This requires an idea. Note if $A,B$ commute this is trivial. 
Apparently von Neumann wrote a paper called ``Metric Spaces'' in the $1930$s in which he just proves this inequality and doesn't know what to do with it, so it got forgotten for a while until the $1950$s, when Schatten wrote books on applications. When I was a student in grad school, I was taking a class on random matrices. There was two weeks break, I was certain that it was trivial because the professor had not said it was not, and it completely ruined my break though I came up with a different proof of it. It's short, but not trivial: It's not typical linear algebra!. This is like another triangle inequality, which we may need later on.

%if linear algebra then trivial. Random matrices class. 
%sometimes you find a completely different proof.

Spielman and Srivastava have a beautiful theorem. %People in numerical linear algebra call the 
\begin{df} Stable rank. \\
Let $A:\R^m \to \R^n$. 
The \ivocab{stable rank} is defined as
\nomenclature{$\text{srank}(A)$}{stable rank, $\pf{\ve{A}_{S_2}}{\ve{A}_{S_\iy}}^2$}
\[
\text{srank}(A)=\pf{\ve{A}_{S_2}}{\ve{A}_{S_\iy}}^2.
\]
\end{df}
The numerator is the sum of squares of the singular values, and the denominator is the maximal value. Large stable rank means that many singular values are nonzero, and in fact large on average. Many people wanted to get the size of the subset in the Restricted Invertibility Principle  
to be close to the stable rank.


\begin{thm}[Spielman-Srivastava]\llabel{thm:ss}
For every linear operator $A:\R^m\to \R^n$, $\ep\in (0,1)$, 
%userful, better than BT.
there exists $\si\subeq \{1,\ldots, m\}$ with $|\si|\ge (1-\ep)\text{srank}(A)$ such that
\[
\ve{(AJ_\si)^{-1}}_{S_{\iy}} \lesssim \fc{\sqrt m}{\ep\ve{A}_{S_2}}.
\]
Here, $J_\si$ is the identity function restricted to $\R^{\si}$,  $J:\R^{\si}\hra \R^m$. 
\end{thm}
This is stronger than Bourgain-Tzafriri. In Bourgain-Tzafriri the columns were unit vectors. 
\begin{proof}[Proof of Theorem~\ref{thm:btrip} from Theorem~\ref{thm:ss}]
Let $A$ be as in Theorem~\ref{thm:btrip}. Then $\ve{A}_{S_2}=\sqrt{\Tr(A^*A)}=\sqrt m$ and $\text{srank}(A)=\fc{m}{\ve{A}^2_{S_{\iy}}}$. We obtain the existence of 
\[
|\si|\ge (1-\ep)\fc{m}{\ve{A}^2_{S_{\iy}}}
\]
with $\ve{(AJ_{\si})^{-1}}_{S_{\iy}}\lesssim \fc{\sqrt m}=\rc{\ep}$.
%get all the way to the stable rank.
\end{proof}
This is a sharp dependence on $\ep$.

The proof introduces algebraic rather than analytic methods; it was eye-opening. Marcus even got sets bigger than the stable rank and looked at $pf{\ve{A}_{S_2}}{\ve{A}_{S_4}}^2$, which is much stronger.

\subsection{A general restricted invertibility principle}

I'll show a theorem that implies all these intermediate theorems. We use (classical) analysis and geometry instead of algebra.
What matters is not the ratio of the norms, but the tail of the distribution of $s_1(A)^2,\ldots, s_m(A)^2$.
%Look at the tails of the distributions.
\begin{thm}\llabel{thm:gen-srank}
Let $A:\R^m\to \R^n$ be a linear operator. If $k<\rank(A)$  then there exist $\si\subeq \{1,\ldots, m\}$ with $|\si|=k$ such that 
\[
\ve{(AJ_\si)^{-1}}_{S_{\infty}} \lesssim \min_{k<r\le \rank(A)}\sfc{mr}{(r-k) \sum_{i=r}^m s_i(A)^2}.
\]
\end{thm}
You have to optimize over $r$. You can get the ratio of $L_p$ norms from the tail bounds. This implies all the known theorems in restricted invertibility.
The subset can be as big as you want up to the rank, and we have sharp control in the entire range.
This theorem generalizes Spielman-Srivasta (Theorem~\ref{thm:ss}), which had generalized Bourgain-Tzafriri (Theorem ~\ref{thm:btrip}). 
%Next time we'll show how this implies the other theorems, and then prove the theorem. %: show how the 

\blu{2-8-16}

%\begin{thm}[Spielman-Srivastava]
%For $k<\text{srank}(A)=\pf{\ve{A}_{S_2}^2}{\ve{A}_{S_{\iy}}}^2$, $k=(1-\ep)\text{srank}(A)$. 
%$\exists \si\subeq \{1,\ldots, m\}, |\si|=k$, ...
%%,/\op&2. 
%%constant prop of m/ ... squred. 
%%bourgain-tz up to stble rank.
%%shatten root m.
%\end{thm}

Now we will go backwards a bit, and talk about a less general result. After Theorem~\ref{thm:ss}, 
a subsequent theorem gave the same theorem but instead of the stable rank, used something better.
\begin{thm}[Marcus, Spielman, Srivastava]\llabel{thm:mss4}
If 
\[
k<\rc4 \pf{\ve{A}_{S_2}}{\ve{A}_{S_4}}^4,
\]
there exists $\si\subeq \{1,\ldots, m\}$, $|\si|=k$ such that
\[
\ve{(AJ_{\si})^{-1}}_{S_{\iy}} \lesssim \fc{\sqrt m}{\ve{A}_{S_2}}.
\]
%stable rank. 
%got strange norms.
\end{thm}
A lot of these quotients of norms started popping up in people's results. The correct generalization is the following notion.
\begin{df}
For $p>2$, define the \ivocab{stable $p$th rank} by \[\text{srank}_p(A)= \pf{\ve{A}_{S_2}}{\ve{A}_{S_p}}^{\fc{2p}{p-2}}.\]
\end{df}
\begin{exr}
Show that if $p\ge q>2$, then
\[
\text{srank}_p(A) \le \text{srank}_q(A).
\]
(Hint: Use H\"older's inequality.)
\end{exr}
Now we would like to prove how Theorem~\ref{thm:gen-srank} generalizes the previously listed results: 
\begin{proof}[Proof of Generalizability of Theorem~\ref{thm:gen-srank}]
Using H\"older's inequality with $\fc p2$,
\bal
\ve{A}_{S_2}^2 & =\sum_{j=1}^m s_j(A)^2\\
&=\sum_{j=1}^{r-1} s_j(A)^2 + \sum_{j=r}^m s_j(A)^2\\
&\le (r-1)^{1-\fc 2p} \pa{\sum_{j=1}^{r-1} s_j(A)^p}^{\fc 2p} + \sum_{j=r}^m s_j(A)^2\\
&\le (r-1)^{1-\fc 2p} \ve{A}_{S_p}^2  + \sum_{j=r}^m s_j(A)^2\\
\sum_{j=r}^m s_j(A)^2 & \ge 
\ve{A}_{S_2}^2 \pa{1-(r-1)^{-\fc 2p}\fc{\ve{A}_{S_p}^2}{\ve{A}_{S_{2}}^2}}\\
&=\ve{A}_{S_2}^2 \pa{1-\pf{r-1}{\text{srank}_p(A)}^{1-\fc2p}}
\end{align*}
Now we can plug the previous calculation into Theorem~\ref{thm:gen-srank} to demonstrate the way the new theorem generalizes the previous results: 
\bal
\ve{(AJ_\si)^{-1}}&\lesssim\min_{k+1\le r\le \rank(A)} \sfc{mr}{(r-k)\ve{A}_{S_2}^2 \pa{1-\pf{r-1}{\text{srank}_p(A)}^{1-\fc 2p}}}\\
&=\fc{\sqrt m}{\ve{A}_{\iy}}\min_{k+1\le r\le \rank(A)} \sfc{r}{(r-k)\pa{1-\pf{r-1}{\text{srank}_p(A)}^{1-\fc 2p}}}
\end{align*}
This equation implies all the earlier theorems.
%\fixme{How did we get the last 2 lines?}
\end{proof}
To optimize, fix the stable rank, differentiate in $r$, and set to 0. All theorems in the literature follow from this theorem; in particular, we get all the bounds we got before. %3, 2.1.
There was nothing special about the number 4 in Theorem~\ref{thm:mss4}; this is about the distribution of the eigenvalues. 

\subsection{Ky Fan maximum principle}
We'll be doing linear algebra. It's mostly mechanical, except we'll need this lemma.
\begin{lem}[Ky Fan maximum principle]\index{Ky Fan maximum principle}
Suppose that $P:\R^m\to \R^m$ is a rank $k$ orthogonal projection. Then
\[
\Tr(A^*AP ) \le \sum_{i=1}^k s_i(A)^2.
\]
\end{lem}
\begin{proof}
\fixme{This proof isn't complete. I will fix it next time.}

We will prove that if $B:\R^m\to \R^m$ is positive semidefinite, then
\[
\Tr(BP)\le \sum_{i=1}^k s_i(B).
\]
To get the lemma, set $B=A^*A$. 

Apply arbitrarily small perturbation $s_i$ so that
\[
s_1(B)>s_2(B)>\cdots > s_m(B).
\]
Let $v_1,\ldots, v_m$ be an orthonormal basis such that $Bv_i=s_i(B) v_i$. Let $u_1,\ldots, u_k$ be an orthonormal basis of $P\R^m$ ordered so that 
\[
\an{Bu_1,u} \ge \an{Bu_2,u_2}\ge \cdots \ge \an{Bu_k,u_k}.
\]
We calculate
\bal
\Tr(BP) & =\sum_{i=1}^k  \an{BPu_i, u_i}\\
&=\sum_{i=1}^k \an{Bu_i,u_i}.
\end{align*}
We will prove by induction on $i$ that 
\[
\an{Bu_i,u_i}\le s_i(B).
\]
For $i=1$, 
\[
s_1(B)=\ve{B}_{S_{\iy}},
\]
and (when $\ve{u_1}=1$)
\[
\an{Bu_1,u_1} \le \ve{B}_{S_{\iy}} = s_1(B).
\]
Suppose we proved the claim for $j-1$. If \fixme{?$ \an{Bu_i,u_i}\le s_j(B)$} $\an{Bu_{j-1},u_{j-1}}\le s_j(B)$ then we're done because $\an{Bu_j,u_j}\le \an{Bu_{j-1},u_{j-1}} \le s_j(B)$. So we may assume that $\an{Bu_{j-1}, u_{j-1}}>s_j(B)$. 

Write the $u$'s in the basis of $v$'s:
\[
u_i = \sum_{l=1}^m c_{il}v_l.
\]
The fact that the $u_i$'s are orthonormal means that the $c_i$'s are probability vectors,
\[
\sum_{l=1}^m c_{il}^2=1.
\]
We have 
\[
\an{Bu_i,u_i} = \sumo lm m c_{il}^2 s_l(B).
\]
If $c_{il}^2>0$ for any $l\ge j$. 
%there can be no weight from $j$ onwards.
%u_j in span of $v_j$ upwards. Need for 1 to $j$. 
%1 to $u_{j-1}$. 
\end{proof}

\subsection{Finding big subsets}
We'll present 4 lemmas for finding big subsets with certain properties. We'll put them together at the end.
\begin{thm}[Little Grothendieck inequality]\llabel{thm:lgi}
Fix $k,m,n\in \N$. Suppose that $T:\R^m\to \R^n$ is a linear operator. Then for every $x_1,\ldots, x_k\in \R^m$,
\[
\sumo rk \ve{Tx_r}_2^2\le \fc{\pi}2\ve{T}_{\ell_{\iy}^m \to \ell_2^n}^2 \sumo rk x_{ri}^2
\]
for some $i\in \{1,\ldots, m\}$ where $x_r=(x_{r1},\ldots, x_{rm})$.
\end{thm}
Later we will show $\fc{\pi}2$ is sharp. 

If we had only 1 vector, what does this say?
\[
\ve{Tx_1}_2\le \sfc{\pi}2\ve{T}_{\ell_{\iy}^m \to \ell_2^n}\ve{x_1}_{\iy}
\]
%The definition of the operator norm is at most ... times the \iy$ norm
We know the inequality is true for $k=1$ with constant 1, by definition of the operator norm. The theorem is true for arbitrary many vectors, losing an universal constant ($\fc{\pi}2$). After we see the proof, the example where $\fc{\pi}2$ is attained will be natural.

We give Grothendieck's original proof.

The key claim is the following.
\begin{clm}\llabel{clm:lgi}
\beq{eq:lgi1}
\sumo jm \pa{\sumo rk (T^*Tx_r)_j^2}^{\rc2}
\le \sfc{\pi}2 \ve{T}_{\ell_{\iy}^m \to \ell_2^n}\pa{\sum_{r=1}^k \ve{Tx_r}^2}^{\rc 2}.
\eeq
\end{clm}

\begin{proof}[Proof of Theorem~\ref{thm:lgi}]
Assuming the claim, we prove the theorem.
\bal
\sumo rk \ve{Tx_r}_2^2 & = \sumo rk \an{Tx_r,Tx_r}\\
&=\sumo rk \an{x_r, T^*Tx_r}\\
&=\sumo rk \sumo jm x_{rj}(T^*Tx_r)_j\\
&\le \sumo jm \pa{\sumo rk x_{rj}^2}^{\rc 2} \pa{\sumo rk (T^*Tx_r)_j^2}^{\rc 2}&\text{by Cauchy-Schwarz}\\
&\le \pa{\max_{1\le j\le m} \pa{\sumo rk x_{rj}^2}^{\rc 2}}
\pa{\sumo jm \sumo rk (T^*Tx_r)_j^2}^{\rc 2}\\
&\le \max_{1\le j\le m}\pa{\sumo ik x_{ij}^2}^{\rc 2}\sfc{\pi}2 \ve{T}_{\ell_{\iy}^m \to \ell_2^n} \pa{\sumo rk \ve{Tx_r}_2^2}^{\rc 2}\\
\sumo rk \ve{Tx_r}_2^2 & \le \fc{\pi}2 \ve{T}_{\ell_{\iy}^m \to \ell_2^n}^2 \max_j \sumo rk x_{ij}^2.
%bound by square root of multiple of same term, bootstrap.
\end{align*}
We bounded by a square root of the multiple of the same term, a bootstrapping argument. In the last step, divide and square.
%where we used Cauchy-Schwarz
\end{proof}

\begin{proof}[Proof of Claim~\ref{clm:lgi}]
Let $g_1,\ldots, g_k$ be iid standard Gaussian random variables. For every fixed $j\in \{1,\ldots, m\}$, 
\[
\sum_{r=1}^k g_r (T^* T x_r)_j.
\]
This is a Gaussian random variable with mean 0 and variance  %whatever the $L^2$ norm of these coefficients are
$\sumo rk (T^*Tx_r)_j^2$. Taking the expectation,\footnote{$\sfc{1}{2\pi} \int_{-\iy}^{\iy} |x| e^{-\fc{x^2}2} = -2\sfc{1}{2\pi} [e^{-\fc{x^2}2}]^{\iy}_0= \sfc{2}{\pi}$}
\bal
\E\ab{\sumo rk g_r(T^*T x_r)_j}
&= \pa{\sumo rk (T^*T x_r)_j^2}^{\rc 2} \sfc 2{\pi}.
\end{align*}
Sum these over $j$:
\begin{align}
\E \ba{
\sumo jm \ab{T^* (\sumo rk g_r T x_r)_j}
} & = \sfc 2\pi \sumo jm \pa{\sumo rk (T^*Tx_r)_j^2}^{\rc 2}\nonumber\\
\sumo jm \pa{\sumo rk (T^*Tx_r)_j^2}^{\rc 2}
&= \sfc{\pi}2 \E\ba{\sumo jm \ab{T^* \sumo rk g_r(Tx_r)_j}}.
\llabel{eq:lgi2}
\end{align}
Define a random sign vector $z\in \{\pm 1\}^m$ by 
\[
z_j = \sign\pa{\pa{T^*\sumo rk g_r Tx_r}_j}
\]
Then 
\bal
\sumo jm \ab{(T^* \sumo rk g Tx_r)_j} 
&=\an{z, T^* \sumo rk g_r Tx_r}\\
&= \an{Tz,\sumo rk g_r Tx_r}\\
& \le \ve{Tz}_2 \ve{\sumo rk g_r Tx_r}_2\\
%l_\iy to l_2. all most whatever norm of operator is.
&\le \ve{T}_{\ell_{\iy}^m \to \ell_2^n} \ve{\sumo rk g_r Tx_r}_2
%expointed was bounded pointwise pointwise.
\end{align*}
This is a pointwise inequality. Taking expectations and using Cauchy-Schwarz,
\beq{eq:lgi3}
\E\ba{
\sum_{j=1}^m \ab{\pa{T^* \sumo rk g_r Tx_r}_j}
}  \le \ve{T}_{\ell_{\iy}^m \to \ell_2^n}\pa{\E \ve{\sumo rk g_r Tx_r}_2^2}^{\rc 2}.
\eeq
What is the second moment? Expand:
\beq{eq:lgi4}
\E\ve{\sumo rk g_i Tx_r}_2^2 
=\E\ba{\sum_{ij} g_i g_j \an{Tx_i,Tx_j}}=\sumo rk \ve{Tx_r}_2^2.
\eeq
Chaining together~\eqref{eq:lgi2},~\eqref{eq:lgi3},~\eqref{eq:lgi4} gives the result.
%bound by L^2 norm above
\end{proof}

Why use the Gaussians? The identity characterizes the Gaussians using rotation invariance. %The expectation of the Gaussian 
%\sumo jm\sumo rk ... \sfc{\pi}2
%up to this piit use gaussians
Using other random variables gives other constants that are not sharp.

There will be lots of geometric lemmas:
\begin{itemize}
\item
A fact about restricting matrices. 
\item
Another geometric argument to give a different method for selecting subsets. 
\item 
A combinatorial lemma for selecting subsets.
\end{itemize}
Finally we'll put them together in a crazy induction.

From this proof you can reverse engineer vectors that make the inequality sharp. You need to come up with $T$ and the points.

\begin{ex}
Let $g_1,g_2,\ldots,g_k $ be iid Gaussians on the probability space $(\Om, P)$. Let $T:L_{\iy}(\Om, P)\to \ell_2^k$ be %infinite $l^{\iy}$ space. 
%abstract nonsense: approx
%replace Gaussians with central limit theorem, take $\pm1$ bits.
%never equality for finite. (ratio of gamma functions)
%in limit converges
\[
Tf = (\E[fg_1],\ldots, \E[fg_k]).
\]
Let $x_r \in L_{\iy} (\Om, P)$, 
\[
x_r = \fc{g_r}{\pa{\sumo ik g_i^2}^{\rc 2}}.
\]
%no black magic, just understand this.
\end{ex}
\blu{2-10-16}

We were in the process of proving three or four subset selection principles, which we will somehow use to prove the RIP. 

Now I owe you a proof (just ask me for the linear algebra proof) - I'll show you an analytic proof. 

We proved the little Grothendieck inequality (Theorem~\ref{thm:lgt}), which is part of an amazing area of mathematics with many applications. It's little, but it's also very useful. Just to remind you, we had an linear operator $T: \R^m \to \R^n$. Then for every $x_1, \cdots, x_k \in \R^m$, we get a bounded operator. If you look at the sum of the Euclidean lengths $\left(\sum_{i = 1}^k \|Tx\|_2^2\right)^{1/2} \leq \sqrt{\pi/2}\|T\|_{l_{\infty}^m \to l_2^n} \cdot \max_{1 \leq j \leq m} \left(\sum_{i = 1}^k x_{rj}^2\right)^{1/2}$. This is really the way Grothendieck did it, but the proof we saw is really the original proof, re-organized. For completeness, we'll show the fact that this inequality is sharp (cannot be improved). 

\begin{cor} $\sqrt{\pi/2}$ is the best constant in Theorem~\ref{thm:lgt}. \\
\end{cor}
\begin{proof}
Define $g_1, \cdots, g_k$ be i.i.d standard Gaussians, defined on probability space $(\Omega, P)$. We define $T: L_{\infty}(\Omega, \mathbb{P}) \to \R^k$. Then $Tf = \left(\mathbb{E}(fg_1), \mathbb{E}(fg_2), \cdots, \mathbb{E}(fg_k) \right)$. Choose $X_r = \frac{g_r}{\left(\sum_i^k g_i^2\right)^{1/2}}$. This is nothing more than a vector on the $k$-dimensional unit sphere. So it's a bounded function. We also note that $x_r$ is a function on the measure space $\Omega$.  We can also write
\[
\sum_{r = 1}^k x_r(\omega)^2 = \sum_{r = 1}^k \frac{g_r(\omega)^2}{\sum_{i = 1}^r g_i(\omega)^2} = 1
\]
We can use the Central Limit Theorem to make things precise: $g_r \approx \frac{\epsilon_{r_1} + \cdots + \epsilon_{r_N}}{\sqrt{N}}$ as $N \to \infty$. So all these statements will be asymptotically true. Where does the family of random variables $\{g_r\}$ live in $\Omega$? Well $\Omega = \{\pm 1\}^{NK}$. So $L_{\infty}(\Omega) = \l_{\infty}^{2^{NK}}$, which is some huge dimension, but it's still finite. So $\omega$ will really be a coordinate in $\Omega$. 

Now we show two things; nothing more than computations. 
\begin{enumerate}

\item $\|T\|_{L_{\infty}(\Omega, \textbf{P}) \to l_2^k} = \sqrt{2/\pi}$, 

\item We also show $\sum_{r = 1}^k \|Tx_r\|_2^2 \to^{k \to \infty} 1$. 

\end{enumerate}

First we tackle the first case. We have 
\begin{align}
\begin{split}
\|T\|_{\infty \to 1} &= \text{sup}_{\|f\|_{\infty} \leq 1}\left(\sum_{r = 1}^k \mathbb{E}\left[fg_r\right]^2 \right)^{1/2}
\\
&= \text{sup}_{\|f\|_{\infty} \leq 1} \text{sup}_{\sum_{r = 1}^k \alpha_r^2 = 1} \sum_{r = 1}\alpha_r \mathbb{E}\left[fg_r\right]
\\
&= \text{sup}_{\sum_{r = 1}^k} \text{sup}_{\|f\|_{\infty} \leq 1} \mathbb{E}\left[f\sum_{i = 1}^k \alpha_r g_r \right]
\\
&= \text{sup}_{\sum_{r = 1}^k} \mathbb{E}\left| \sum_{r = 1}^k \alpha_r g_r \right| = \mathbb{E} |g_1| = \sqrt{\frac{2}{\pi}}
\end{split}
\end{align}

as we claimed. Now we tackle the second computation: 
\begin{align}
\begin{split}
\sum_{r = 1}^k \|Tx_r\|_2^2 &= \sum_{r = 1}^k \left(\mathbb{E} \left[\frac{g_r^2}{\left(\sum_{i = 1}^k g_i^2\right)^{1/2}}\right]\right)^2
\\
&= K \left(\mathbb{E}\left[\frac{g_1^2}{\left(\sum_{i = 1}^k g_i^2\right)^{1/2}}\right]\right)
\\
&= K\left(\frac{1}{K}\mathbb{E}\left[\sum_{r = 1}^k \frac{g_r^2}{\left(\sum g_i^2\right)^{1/2}}\right]\right)^2
\\
&= \frac{1}{K} \left(\mathbb{E}\left[\left(\sum_{i = 1}^k g_i^2\right)^{1/2}\right]\right)^2
\end{split}
\end{align}
and you can use Stirling to finish. This is just a $\chi^2$-distribution. 

In this case $\mathbb{E} \frac{g_1g_2}{\left(\sum_i g_i^2\right)^{1/2}} = \mathbb{E} \frac{g_1 (-g_2)}{\left(\sum g_i^2\right)^{1/2}}$. 
Also note that if $(g_1, \cdots, g_k) \in \R^k$ is a standard Gaussian, then $\frac{(g_1, \cdots, g_k)}{\left(\sum_{i = 1}^k g_i^2\right)^{1/2}}$ and $\left(\sum_{i = 1}^k g_i^2)^{1/2}\right)$ are independent. In other words, the length and angle are independent: This is just polar coordinates, you can check this. 
\end{proof}

Now, how does this relate to the Restricted Invertibility Problem? 

\begin{thm} Pietsch Domination Theorem.\llabel{thm:pdt} \\
Fix $m, n \in \mathbb{N}$ and $M  > 0$. Suppose that $T: \R^m \to \R^n$ is a linear operator such that for every $x_1, \cdots, x_k \in \R^m$ have 
\[
\left(\sum_{r = 1}^k \|Tx_r\|_2^2\right)^{1/2} \leq M \text{max}_{1 \leq j \leq m} \left(\sum_{r = 1}^k x_{rj}^2\right)^{1/2}
\]
Then there exist $\mu = (\mu_1, \cdots, \mu_m) \in \R^m$ with $\mu_1 \geq 0$ and $\sum_{i = 1}^m = 1$ such that for every $x \in \R^m$
\[
\|Tx\|_2 \leq M\left(\sum_{i = 1}^M \mu_ix_i^2\right)^{1/2}
\]
It's really an iff: The latter is a stronger statement than the former, and in fact they are equivalent.
You can come out with a probability measure, a way to weight the coordinates, such that the norm of T as an operator as a standard norm from $l_{\infty}$ to $l_2$, is bounded by $M$. 
\end{thm}
\begin{proof}
Define $K \subseteq \R^m$ with 
\[
K = \left\{y \in \R^m: y_i = \sum_{r = 1}^k \|Tx_r\|_2^2 - M^2\sum_{r = 1}^m x_{ri}^2 \text{ for some } k, x_1, \cdots, x_k \in \R^m\right\}
\]
Basically we cleverly select a convex set. Every $n$-tuple of vectors in $\R^m$ gives you a new vector in $\R^m$. Let's check that $K$ is convex. We have to check if two vectors $y, z \in K$ have all points on the line between them in $K$. $y \in K$ means that 
\[
y_i = \sum_{r = 1}^k \|Tx_r\|_2^2 - M^2 \sum_{r = 1}^m x_{ri}^2
\] 
\[
z_i = \sum_{r = 1}^l \|Tw_r\|_2^2 - M^2 \sum_{r = 1}^l w_{ri}^2
\]
for all $i$. So what can you say about the average $\frac{y_i + z_i}{2}$? It comes from $\left(\frac{x_1}{\sqrt{2}}, \cdots, \frac{x_k}{\sqrt{2}}, \frac{w_1}{\sqrt{2}}, \cdots, \frac{w_l}{\sqrt{2}}\right)$. So trivially by design this is a convex set. 

Now, the assumption of the theorem says that 
\[
\left(\sum_{r = 1}^k \|Tx_r\|_2^2\right)^{1/2} \leq M \text{max}_{1 \leq j \leq m} \left(\sum_{r = 1}^k x_{rj}^2\right)^{1/2}
\]
which implies 
\[
\|Tx_r\|_2^2 - M^2 \text{max}_{1 \leq j \leq m} \sum_{r = 1}^m x_{rj}^2 \leq 0
\]
which implies $K \cap (0, \infty)^m = \emptyset$. By the hyperplane separation theorem (for two disjoint convex sets in $\R^m$ with at least one compact, there is a hyperplane between them), there exists $0 \neq \mu = (\mu_1, \cdots, \mu_m) \in \R^m$. We have 
\[
\langle \mu, y \rangle \leq \langle \mu, z\rangle
\]
for all $y \in K$ and $z \in (0, \infty)^m$. By renormalizing, $\sum_{i = 1}^m = 1$. Moreover $\mu$ cannot have any strictly negative coordinate: Otherwise you could take $z$ to have arbitrarily large value at a strictly negative coordinate with zeros everywhere else, implying $\langle u, z \rangle$ is no longer bounded from below, a contradiction. Therefore, $\mu$ is a probability vector and $\langle \mu, z \rangle$ can be arbitrarily small. So for every $y \in K$, $\sum_{i = 1}^m \mu_iy_i \leq 0$. Then $y_i = \|Tx\|_2^2 - M^2x_i \in K$, and if you write this out, $\|Tx\|_2^2 - M^2 \sum_{i = 1}^n \mu_i y_i \leq 0$, which is exactly what we wanted. 
\end{proof}

\begin{lem} \llabel{lem:projbound}
$m, n \in \mathbb{N}$, $\epsilon \in (0, 1)$, $T: \R^n \to \R^m$ a linear operator. Then $\exists \sigma \subset \{1, \cdots, m\}$ with $|\sigma| \geq (1 - \epsilon)m$ such that 
\[
\|\text{Proj}_{\R^{\sigma}} T\|_{S_{\infty}} \leq \sqrt{\frac{\pi}{2\epsilon m}} \|T\|_{l_2^n \to l_1^m}
\] 
We will find ways to restrict a matrix to a big big submatrix, but we won't be able to control its operator norm, but we will be able to control the norm from $l_2^n \to l_1^m$. So then you go to a further subset, which this becomes an operator norm on, which is an improvement which Grothendieck gave us. This is the first very useful tool to start finding big sub matrices. 
\end{lem}
\begin{proof}
We have $T: l_2^n \to l_1^m$, $T^*: l_{\infty}^m \to l_2^n$. Now some abstract nonsense gives us that for Banach spaces, the norm of an operator and its adjoint are equal, i.e. $\|T\|_{l_2^n \to l_1^m}  = \|T^*\|_{l_{\infty}^m \to l_2^n}$. This statement follows from Hahn-Banach theorem (come see me if you haven't seen this before, I'll tell you what book to read). 
From the Little Grothendieck inequality (Theorem~\ref{thm:lgi}), $T^*$ satisfies the assumption of the Pietsch domination theorem with $M = \sqrt{\frac{\pi}{2}} \|T\|_{l_2^n \to l_1^m}$ (we're applying it to $T^*$). So we have probability vector $(\mu_1, \cdots, \mu_m)$ such that for every $y \in \R^m$
\[
\|T^*y\|_2 = M\left(\sum_{i = 1}^m \mu_iy_i^2\right)^{1/2}
\]
with $M = \sqrt{\frac{\pi}{2}} \|T\|_{l_2^n \to l_1^m}$. Define $\sigma = \left\{i \in \{1, \cdots, m\}: \mu_i \leq \frac{1}{m\epsilon}\right\}$, then $|\sigma| \geq (1 - \epsilon)m$ by Markov's inequality. We can also see this by writing
\[
1 = \sum \mu_i = \sum_{i \in \sigma} \mu_i + \sum_{i \not\in \sigma} \mu_i > \sum_{i \in \sigma} \mu_i + \frac{m - |\sigma|}{m\epsilon}
\]
which follows since $\mu_j$ for $j \not\in \sigma$ has $\mu_j > \frac{1}{m\epsilon}$. Continuing, 
\[
\frac{m\epsilon - m + |\sigma|}{m\epsilon} \geq \sum_{i \in \sigma} \mu_i
\]
\[
|\sigma| \geq (m\epsilon)\sum_{i \in \sigma} \mu_i +  m(1 - \epsilon)
\]
Then, since $(m\epsilon)\sum_{i \in \sigma} \mu_i  \geq 0$ since $\mu$ is a probability distribution, we have
\[
|\sigma| \geq m(1 - \epsilon)
\]

Now take $x \in \R^n$ and choose $y \in \R^m$ with $\|y\|_2 = 1$. Then 
\[
\langle y, \text{Proj}_{\R^{\sigma}} Tx \rangle^2 = \|\text{Proj}_{\R^{\sigma}} Tx\|_2^2 \leq \langle T^* \text{Proj}_{\R^{\sigma}} y, x \rangle^2 \leq \|T^*\text{Proj}_{\R^{\sigma}} y\|_2^2 \cdot \|x\|_2^2
\] 
\[
\leq \frac{\pi}{2}\|T\|_{l_2^n \to l_1^m} \left(\sum_{i \in \sigma} \mu_iy_i^2\right) \|x\|_2^2 \leq \frac{\pi}{2} \|T\|_{l_2^n \to l_1^m}^2 \frac{1}{m\epsilon}\|x\|_2^2
\]
by Cauchy-Schwarz. Then, taking square roots gives the desired result.
\end{proof}
In the previous proof, we used a lot of duality to get an interesting subset. 

\begin{rem}
In Lemma~\ref{lem:projbound}, I think that either the constant $\pi/2$ is sharp (no subset are bigger; it could come from the Gaussians), or there is a different constant here. If the constant is $1$, I think you can optimize the previous argument and get the constant to be arbitrarily close to $1$, which would have some nice applications: In other words, getting $\sqrt{\frac{\pi}{2\epsilon m}}$ as close to $1$ as possible would be good. I didn't check before class, but you might want to check if you can carry out this argument using the Gaussian argument we made for the sharpness of $\frac{\pi}{2}$ in Grothendieck's inequality (Theorem~\ref{thm:lgt}). It's also possible that there is a different universal constant. 
\end{rem}

Now we will give another lemma which is very easy and which we will use a lot. 
\begin{lem} Sauer-Shelah. \llabel{lem:saushel} \\
Take integers $m, n \in \mathbb{N}$ and suppose that we have a large set $\Omega \subseteq \{\pm 1\}^n$ with 
\[
|\Omega| > \sum_{k = 0}^{n - 1} {n \choose k}
\]
Then $\exists \sigma \subseteq \{1, \cdots, n\}$ such that with $|\sigma| = m$, if you project onto $\R^{\sigma}$ the set of vectors, you get the entire cube: $\text{Proj}_{\R^{\sigma}}(\Omega) = \{\pm 1\}^{\sigma}$. For every $\epsilon \in \{\pm 1\}^{\sigma}$, there are signs $\delta = (\delta_1, \cdots, \delta_n) \in \Omega$ such that $\delta_j = \epsilon_j$ for $j \in \sigma$.
\end{lem}

Note that Lemma~\ref{lem:saushel} is used in the proof of the Fundamental Theorem of Statistical Learning Theory. 

\begin{proof}
% The strengthening of induction I'm about to use is due to Pajor.
We want to prove by induction on $n$. First denote the shattering set
\[
\txtn{sh}(\Omega) = \{\sigma \subseteq \{1, \cdots, n\}: \txtn{Proj}_{\R^{\sigma}}\Omega = \{\pm 1\}^{\sigma}\}
\]
The claim is that the number of sets shattered by a given set is $|\text{sh}(\Omega)| \geq |\Omega|$. The empty set case is trivial. What happens when $n = 1$? $\Omega \subset \{-1, 1\}$, and thus the set is shattered. Assume that our claim holds for $n$, and now set $\Omega \subseteq \{\pm 1\}^{n + 1} = \{\pm 1\}^n \times \{\pm 1\}$. Define
\[
\Omega_+ = \{\omega \in \{\pm 1\}^n: (\omega, 1) \in \Omega\}
\] 
\[
\Omega_{-} = \{ \omega \in \{\pm 1\}^n: (\omega, -1) \in \Omega\}
\]
Then, letting $\tilde{\Omega}_+ = \{(\omega, 1)\in \{\pm 1\}^{n+1}: \omega \in \Omega_+\}$ and $\tilde{\Omega}_-$ similarly, we have $|\Omega| = |\tilde{\Omega}_+| + |\tilde{\Omega}_-| = |\Omega_+| + |\Omega_-|$. By our inductive step, we have sh$(\Omega_+) \geq |\Omega_+|$ and sh$(\Omega_-) \geq |\Omega_-|$. Note that any subset that shatters $\Omega_+$ also shatters $\Omega$, and likewise for $\Omega_{-}$. Note that if a set $\Omega'$ shatters both of them, we are allowed to add on an extra coordinate to get $\Omega' \times \{\pm 1\}$ which shatters $\Omega$. Therefore, 
\[
\txtn{sh}(\Omega_+) \cup \txtn{sh}(\Omega_-) \cup \left\{\sigma \cup \{n + 1\}: \sigma \in \txtn{sh}(\Omega_+) \cap \txtn{sh}(\Omega_-)\right\} \subseteq \txtn{sh}(\Omega)
\]
where the last union is disjoint since the dimensions are different. Therefore, we can now use this set inclusion to complete the induction using the principle of inclusion-exclusion: 
\bal
|\txtn{sh}(\Omega)| &\geq |\txtn{sh}(\Omega_+) \cup \txtn{sh}(\Omega_-)| + |\txtn{sh}(\Omega_+) \cap \txtn{sh}(\Omega_-)| \hspace{3em} \txtn{  (disjoint sets)}
\\
&= |\txtn{sh}(\Omega_+)| + |\txtn{sh}(\Omega_-)| - |\txtn{sh}(\Omega_+) \cap \txtn{sh}(\Omega_-)|  + |\txtn{sh}(\Omega_+) \cap \txtn{sh}(\Omega_-)| 
\\
&= |\txtn{sh}(\Omega_+)| + |\txtn{sh}(\Omega_-)|
\\
&\geq |\Omega_+| + |\Omega_-| = |\Omega|
\end{align*}
which completes the induction as desired. 
\end{proof}

\begin{cor} If $|\Omega| \geq 2^{n -1}$ then there exists $\sigma \subseteq \{1, \cdots, n\}$ with $|\sigma| \geq \lceil \frac{n + 1}{2} \rceil \geq \frac{n}{2}$ such that $\txtn{Proj}_{\R^{\sigma}} \Omega = \{\pm 1\}^{\sigma}$. 
\end{cor}



\blu{2-15-16: 
Last time we left off with the proof of the Sauer-Shelah lemma. To remind you, we were finding ways to find interesting subsets where matrices behave well. Now recall we had a linear algebraic fact which I owe you; I will prove it in an analytic way. The proof has been moved to Section~\ref{sec:kf}.}

\section{Proof of RIP}
\subsection{Step 1}
Now we need another geometric lemma for the proof of %\fixme{change name of theorem gen-srank to restricted invertibility principle? fix reference} 
Theorem~\ref{thm:gen-srank}, the restricted invertibility principle. 

\begin{lem}[Step 1] \llabel{lem:step1}
Fix $m, n, r \in \N$. Let $A: \R^m \to \R^n$ be a linear operator with rank$(A) \geq r$. For every $\tau \subseteq \{1, \ldots, m\}$, denote
\[
E_{\tau} = \left(\txtn{span}((Ae_j)_{j \in \tau})\right)^{\perp}.
\] 
Then there exists $\tau \subseteq \{1, \ldots, m\}$ with $|\tau| = r$ such that for all $j \in \tau$, 
\[
\|\txtn{Proj}_{E_{\tau \setminus \{j\}}} Ae_j\|_2 \geq \frac{1}{\sqrt{m}}\left(\sum_{i = 1}^m s_i(A)^2\right)^{1/2}.
\]
\end{lem}
Basically we're taking the projection of the $j^{th}$ column onto the orthogonal completement of the span of the subspace of all columns in the set except for the $j^{th}$ one, and bounding the norm of that by a dimension term and the square root of the sum of the eigenvalues. 
(This is sharp asymptotically, and may in fact even be sharp as written too---I need to check. \fixme{Check this?})

\begin{proof}
For every $\tau \subseteq \{1, \ldots, m\}$, denote
\[
K_{\tau} = \txtn{conv}\left(\{\pm Ae_j\}_{j \in \tau}\right)
\]
The idea is to make the convex hull have big volume. Once we do that, wewill get all these inequalities for free. Let $\tau \subseteq \{1, \ldots, m\}$ be the subset of size $r$ that maximizes $\txtn{vol}_r(K_{\tau})$. We know that $\txtn{vol}_r(K_{\tau}) > 0$. Observe that for any $\beta \subseteq \{1, \ldots, m\}$ of size $r - 1$ and $i \not\in \beta$, we have 
\[
K_{\beta \cup \{i\}} = \txtn{conv}\left(K_{\beta} \cup \{\pm Ae_i \}\right),
\]
which is a double cone. 
%So all you're doing is having found $E_{\beta}$ and a point $Ae_i$, for the convex hull you get a double cone $E_{\beta}, +Ae_i, -Ae_i$ where $E_{\beta}$ is the orthogonal complement of the space spanned by $\beta$. So 

\ig{images/5-1}{.25}

What is the height of this cone? It is $\|\txtn{Proj}_{E_{\beta}} Ae_i\|_2$, as $E_\be$ is the orthogonal complement of the space spanned by $\be$. Therefore, the $r$-dimensional volume is given by 
\[
\txtn{vol}_r(K_{\beta \cup \{i\}}) = 2\cdot \frac{\txtn{vol}_{r -1}(K_{\beta}) \cdot \|\txtn{Proj}_{E_{\beta}} Ae_i\|_2}{r}
\]
Because $|\tau| = r$ is the maximizing subset of $F_{\Omega}$, for any $j \in \tau$ and $i \in \{1, \ldots, m\}$, choosing $\beta = \tau \setminus \{j\}$, we get
\bal
\vol_r(K_{\be\cup \{j\}}) & \ge \vol_r(K_{\be \cup \{i\}})\\
\implies
\|\txtn{Proj}_{E_{\tau \setminus \{j\}}} Ae_j\|_2 &\geq \|\txtn{Proj}_{E_{\tau \setminus \{j\}}} Ae_i\|_2.
\end{align*}
for every $j \in \tau$ and $i \in \{1, \ldots, m\}$. 
Summing,
\[
m\|\txtn{Proj}_{E_{\tau \setminus \{j\}}} Ae_j\|_2^2
\ge \sum_{i = 1}^m \|\txtn{Proj}_{E_{\tau \setminus \{j\}}} Ae_i\|_2^2 = \|\txtn{Proj}_{E_{\tau \setminus \{j\}}} A\|_{S_2}^2.
\]
Then, for all $j \in \tau$, 
\beq{eq:rip-s1-1}
\|\txtn{Proj}_{E_{\tau \setminus \{j\}}} Ae_j\|_2 \geq \frac{1}{\sqrt{m}}\|\txtn{Proj}_{E_{\tau \setminus \{j\}}} A\|_{S_2}
\eeq
Let's denote $P = \txtn{Proj}_{E_{\tau \setminus \{j\}}}$. Note $P$ is an orthogonal projection of rank $r - 1$. Then, 
\beq{eq:rip-s1-2}
\begin{split}
\|PA\|_{S_2}^2 &= \txtn{Tr}((PA)^*(PA)) = \txtn{Tr}(A^*P^*PA) = \txtn{Tr}(A^*PA) = \txtn{Tr}(AA^*P)\\
&= \txtn{Tr}(AA^*) - \txtn{Tr}(AA^*(I - P)) \geq \sum_{i = 1}^m s_i(A)^2 - \sum_{i = 1}^{r - 1} s_i(A)^2 = \sum_{i = r}^m s_i(A)^2
\end{split}
\eeq
using the Ky Fan maximal principle~\ref{lem:kf}, 
since $I - P$ is a projection of rank $m - r + 1$. 
%So the maximum this could be is the tail. %, which is the variational argument we proved earlier (Ky Fan maximal principle).
%And this is what we claimed. 

Putting~\eqref{eq:rip-s1-1} and~\eqref{eq:rip-s1-2} together gives the result.
\end{proof}

\subsection{Step 2}

\blu{In our proof of the restricted invertibility principle, this is the first step. Before proving it, let me just tell you what the second step looks like. }

\begin{lem}[Step $2$] \llabel{lem:step2}
Let $k, m, n \in \N$, $A: \R^m \to \R^n$, $\rank(A) > k$. Let $\omega \subeq \{1, \ldots, m\}$ with $|\om| = \txtn{rank}(A)$ such that $\{Ae_j\}_{j \in \omega}$ are linearly independent. Denote for every $j \in \Omega$ 
\[
F_j = E_{\omega \setminus \{j\}} = \left(\txtn{span}(Ae_i)_{i \in \omega \setminus \{j\}}\right).
\]
Then there exists $\sigma \subseteq \omega$ with $|\si| =k$ such that 
\[
\|\left(AJ_{\sigma}\right)^{-1}\|_{S_{\infty}} %\leq
\lesssim
 \frac{\sqrt{\txtn{rank}(A)}}{\sqrt{\txtn{rank}(A) - k}} \cdot \max_{j \in \omega} \rc{{\|\txtn{Proj}_{F_j} Ae_j\|}}
\]
%pass to where uniformly big
%first using geometry, find subspace making all the proj big, using singular values. Once you have this lemma, you apply that first, substitute in here.
\end{lem}

Most of the work is in the second step. First we pass to a subset where we have some information about the shortest possible orthogonal project. But Step $1$ saves us by bounding what this can be. Here we use the Grothendieck inequality, Sauer-Shelah, etc. Everything: It's simple, but it kills the restricted invertibility principle. 

%\begin{thm} Step $1$ and Step $2$ imply the Restricted Invertibility Principle.
%\end{thm}
\begin{proof}[Proof of Theorem~\ref{thm:gen-srank} given Step 1 and 2]
Take $A: \R^m \to \R^n$. 
By Step $1$ (Lemma~\ref{lem:step1}), we can find subset $\tau \subseteq \{1, \ldots, m\}$ with $|\tau| = r$ such that for all $j \in \tau$, 
\beq{eq:step1}
\|\txtn{Proj}_{E_{\tau \setminus \{j\}}} Ae_j \|_2 \geq \frac{1}{\sqrt{m}}\left(\sum_{i = r}^m s_i(A)^2\right)^{1/2}.
\eeq
Now we apply Step $2$ (Lemma~\ref{lem:step2}) to $AJ_{\tau}$, using $\omega = \tau$, and find a further subset $\sigma \subseteq \tau$ such that 
\bal
\|\left(AJ_{\sigma}\right)^{-1}\|_{S_{\infty}} 
&\le \min_{k<r<\rank(A)}\sfc{\rank(A)}{\rank(A)-r} \max_{j\in \om}\rc{{\ve{\Proj_{F_j} Ae_j}}}\\
&\leq \min_{k < r < \txtn{rank}(A)} \sqrt{\frac{mr}{(r - k)\sum_{i = r}^m s_i(A)^2}}
\end{align*}
which we get by plugging directly in $r$ for the rank and using Step $1$~\eqref{eq:step1} to get the denominator. 
\end{proof}

\blu{2-17}

Now we prove Step 2 (Lemma~\ref{lem:step2}). Note we can assume $\om=\{1,\ldots, m\}$ and that the rank is $m$.

First we need some lemmas.
\begin{lem}\llabel{lem:rip-step2-1}
Let $A:\R^m\to \R^n$ be such that $\{Ae_j\}_{j=1}^m$ are linearly independent, and $\si\subeq \{1,\ldots, m\}$, $t\in \N$. Then there exists $\tau\subeq \si$ with 
\[
|\tau|\ge \pa{1-\rc{2^t}}|\si|,
\]
such that denoting $\te = \tau\cup (\{1,\ldots, m\}\bs \si)$, $M=\max_{j \in \omega} \rc{{\|\txtn{Proj}_{F_j} Ae_j\|}}$, for all $\si\in \R^{\te}$,
\[
\sum_{i\in \tau}|a_i|\le 2^{\fc t2} M \sqrt{|\si|} \ve{\sum_{i\in \te} a_iAe_i}_2.
\]
%controlled by $L^2
%ganos pulus
\end{lem}

\ig{images/6-1}{.25}

This is proved by a nice inductive argument.
%This sets us up for the second iteration in the proof.
\begin{lem}\llabel{lem:rip-step2-2}
Let $m,n,t\in \N$ and $\be\subeq \{1,\ldots, m\}$. Let $A:\R^m\to \R^n$ be a linear operator such that $\{Ae_j\}_{j=1}^m$ are linearly independent. Then there exist two subsets
$
\si \subeq \tau \subeq \be
$
such that
$|\tau|\ge \pa{1-\rc{2^t}}|\be|$,  
$|\tau\bs \si| \le \fc{|\be|}{4}$,
and if we denote $\te=\tau\cup (\{1,\ldots, m\}\bs \be)$, $M=\max_{j \in \omega} \rc{{\|\txtn{Proj}_{F_j} Ae_j\|}}$, then 
%Restrict $A$ to the subset $\te$
\[
\ve{\Proj_{\R^{\si}}(AJ_{\te})^{-1}}_{S_{\iy}}\lesssim 2^{\fc t2}M.
\]
\end{lem}
%restrict with big set, compose with orthogonal projection to $\si$. Get operator norm bound

\ig{images/6-2}{.25}

\begin{proof}[Proof of Lemma~\ref{lem:rip-step2-2} from Lemma~\ref{lem:rip-step2-1}]
Apply Lemma~\ref{lem:rip-step2-1} with $\si=\be$. We find $\tau\subeq \be$ with $|\tau|\ge \pa{1-\rc{2^t}}|\be|$ such that 
\[
\sum_{i\in \tau} |a_i|\le 2^{\fc t2} M\sqrt{|\be|}\ve{\sum_{i\in \te} a_iAe_i}_2.
\]
Rewriting gives that %for all $a\in \R^{\si}$,
\bal
\forall a\in \R^{\si},\qquad
\ve{\Proj_{\R^{\tau}}a} &\le 2^{\fc t2} M \sqrt{|\be|} \ve{AJ_{\te}a}_2\\
\implies \ve{\Proj_{\R^{\tau}} (AJ_\te)^{-1}}_{\ell_2^{\te}\to \ell_1^{\tau}}&\lesssim 2^{\fc t2} M \sqrt{|\be|}.
\end{align*}
%projection times b... is exactly operator norm.
%extctly what came up in LGI in PDT. 
%applied to vector at most $L^2$ of vector
Denote $\ep=\fc{|\be|}{4|\tau|}$. By Lemma \fixme{which?}, there exists $\si\subeq \tau$, $|\si|\ge (1-\ep)|\tau|$ such that 
\bal
\ve{\Proj_{\R^{\si}} (AJ_{\te})^{-1}}_{S^{\iy}}=
\ve{\Proj_{\R^{\si}}\Proj_{\R^{\tau}} (AJ_\te)^{-1}}_{S^{\iy}}
&\lesssim \sfc{\pi}{2\ep |\tau|} 2^{\fc t2} M\sqrt{|\be|}\\
&\lesssim 2^{\fc t2}M.
\end{align*}
%sauer shelah, many good signs
%first lemma is $L^1$.
%pass to signs? get second lemma.
%Show lemma 2 and finish, or lemma 1.
\end{proof}

\begin{proof}[Proof of Lemma \ref{lem:step2} (Step 2)]
%volume, grothendieck
%only remaining is combinatorics (SS).
%inductive construction.
Fix an integer $r$ such that $\rc{2^{2r+1}}\le 1-\fc km\le \rc{2^{2r}}$. 
Proceed inductively as follows. First set
\bal
\tau_0&=\{1,\ldots, m\}\\
\si_0&=\phi.
\end{align*}
Suppose $u\in \{0,\ldots, r+1\}$ and we constructed $\si_k,\tau_k\subeq \{1,\ldots, m\}$ such that if we denote $\be_u =\tau_u \bs \si_u$, $\te_u = \tau_u \cup (\{1,\ldots, m\}\bs \be_{u-1})$, then
\begin{enumerate}
\item
$\si_u\subeq \tau_u \subeq \be_{u-1}$
\item
$|\tau_u|\ge \pa{1-\rc{2^{2r-u+4}}}|\be_{u-1}|$
\item
$|\be_u|\le \rc 4 |\be_{u-1}|$
\item
$\ve{\Proj_{\R^{\si_u}}(AJ_{\te_u})^{-1}}_{S_{\iy}} \lesssim 2^{r-\fc u2}M$.
\end{enumerate}

\ig{images/6-3}{.25}

Let $H=2r-u+4$. For instance, $|\tau_1|\ge \pa{1-\rc{2^{2r+3}}}|\be_0|$.
%the projection onto $\R^{\si}$. is $\fc p2$. constant don't care about.
What is the new $\be$? %$\tau_u0\si_u$.
%we can chop off a piece such that 
%projection onto $\R_{\si_2}$.

For the inductive step, apply Lemma~\ref{lem:rip-step2-2} on $\be_{u-1}$ with $t=2r-u+4$ to get $\si_u\subeq \tau_u\subeq \be_{u-1}$ such that $|\tau_u|\ge \pa{1-\rc{2^{2r-u+4}}}|\be_{u-1}|$, 
$
|\tau_u\bs \si_u|\le \fc{|\be_{u-1}|}{4}
$
%use lemma
%add omre and more pieces.
%set up so can get what want, ...
%\te decreasing
\beq{eq:rip-s2-1}
\ve{\Proj_{\R^{\si_u}}(AJ_{\te_u})^{-1}}\lesssim 
2^{r-u/2}M.
\eeq

We know $|\be_{u-1}|\le \fc{m}{4^{u-1}}$, 
\begin{align*}
\be_{u-1}&=\be_u \sqcup \si_u \sqcup (\be_{u-1}\bs \tau_u)\\
|\be_{u-1}|&=|\be_u| + |\si_u|  +(|\be_{u-1}| - |\tau_u|)\\
|\si_u|&=|\be_{u-1}|-|\be_u| -(|\be_{u-1}| - |\tau_u|)\\
&\ge |\be_{u-1}|-|\be_u|-\fc{|\be_{u-1}|}{2^{2r-u+4}}\\
&\ge |\be_{u-1}|-|\be_u| - \fc{m}{2^{2r+u+2}}.
\end{align*}
Our choice for the invertible subset is
\[
\si = \bigsqcup_{u=1}^{r+1}\si_u.
\]
Telescoping gives
\bal
|\si|&=\sumo u{r+1} |\si_u| \ge |\be_0|-|\be_{r+1}| - \fc{m}{2^{2r+2}}\sumo u{\iy} \rc{2^u} \\
&\ge m-\fc{m}{4^{r+1}}-\fc{m}{2^{2r+2}}\\
&=m\pa{1-\rc{2^{2r+1}}}\ge m\fc{k}m=k.
\end{align*}
%bookkeeping
%new universe in $\tau_u$. Next bit is inside
Observe that  $\si\subeq \bigcap_{u=1}^{r+1} \te_u$ and  for every $u$, 
%??? Where do we use this?
\bal
\si_u,\ldots, \si_{r+1}&\subeq \tau_u\\ 
\si_1,\ldots, \si_{u-1} &\subeq \{1,\ldots, m\} \bs \be_{u-1}
\end{align*}
%union is $\te_u$.
%more confusing on the board.

This allows us to use the conclusion for all the $\si_u$'s at once. 

For $a\in \R^{\si}$, $J_{\si}a\subeq J_{\te_u} \R^{\te_u}$, 
\[
\Proj_{\R^{\si_u}}(AJ_{\te_u})^{-1} (AJ_\si) a = \Proj_{\R^{\si_u}}J_\si a.
\] 
%a sucset of $J_\si$. Apply $A$ to it.
%R^\si$ inside $\R^{\si_u}$, so honest inverse. 
Then, breaking $J_\si a$ into orthogonal components,
\bal
\ve{J_\si a}_2^2 &=\sumo  u{r+1} \ve{\Proj_{\R^{\si_u}}J_{\si}a}_2^2\\
%blocks. l^2 norm is sum over blocks
&=\sum_{u=1}^{r+1}\ve{\Proj_{\R^{\si_u}}(AJ_{\si_u})^{-1}(AJ_\si)a}_2^2 \\
%??? Where did this line come from?
&\lesssim \sum_{u=1}^{r+1} 2^{2r-u} M^2 \ve{AJ_\si a}_2^2&\text{by~\eqref{eq:rip-s2-1}}\\
&\lesssim 2^{2r} M^2 \ve{AJ_\si a}_2^2\\
&\le \fc{M^2}{1-\fc km}\ve{AJ_\si a}_2^2\\
\ve{J_\si a}_2 &\le \sfc{m}{m-k} M \ve{AJ_\si a}_2
\end{align*}
Since this is true for all $a\in \R^{\si}$,
\bal
\implies \ve{(AJ_\si)^{-1}}_{S_{\iy}} &\lesssim \sfc{m}{m-k} M.
%use lemma as black box.
\end{align*}
%in the end it's combinatorics.
\end{proof}
It remains to prove Lemma~\ref{lem:rip-step2-1}.

%greedily instead of inductively?
%improve over original
%subset of proportional size, first bite gets the constrant fraction. To get to $1-\ep$. Each time, get half of what remains. it would be nice. get very close to rank. 
%aggressive
%initial proof they do it once.

%SS. first time go to half. 
%to get more, look at half of what remains. Hopefully we can continue. My hunch  a different way: skip over use of SS.

%not true that SS take half of what remains
%give bound from $L^1$ to $L^2$. Signs 
%grothendieck further subset
%correct wrong norms to operator norms.

%Another aspect: this bound $\lesssim \sfc{m}{m-k}$ is existential statement. 
%Hinted that Lemma 1 uses SS. 
%Cloud of points. Some subset of coordinates such that when you project it you see the full cube. I'm quite certain that it's NP-hard. (How is the set given to you?) Sets we care about have  simple description.
%Formulate correct algorithmic Sauer-Shelah lemma.
%subset of half the diension.
%formulate the question correctly. Some sort of description.
%formulate question.
%good chance for the particular sets we use, there is poly time algorithm.
%intersection of cube with ellipsoid. In this case make algorithmic.

How can we make this theorem algorithmic?

The way the Pietsch Domination Theorem~\ref{thm:pdt} worked was by duality. 
%pitch.
%I'm not worried. It worked by duality. 
We look at a certain explicitly defined convex set. We found a separating hyperplane which must be a probability measure. Then we had a probabilistic construction. This part is fine.



The bottleneck for making this an algorithm (I do believe this will become an algorithm) consists of 2 parts:
\begin{enumerate}
\item
Sauer-Shelah lemma~\ref{lem:saushel}: We have some cloud of points in the boolean cube, and we know there is some large subset of coordinates (half of them) such that when you project to it you see the full cube.  I'm quite certain that it's NP-hard in general. (Work is necessary to formulate the correct algorithmic Sauer-Shelah lemma. How is the set given to you?) 

We only need to answer the algorithmic question tailored to our sets, which have a simple description: the intersection of an ellipsoid with a cube. There is a good chance that there is a polynomial time algorithm in this case. This question has other applications as well (perhaps one could generalize to the intersection of the cube with other convex bodies).
%question tailored to our sets (if the intersection of an ellipsoid (or other convex bodies) with the boolean cube has $>2^{n-1}$ points, find a large shattering set in polynomial time; this would have other applications). 
\footnote{There could be an algorithm out there, but I couldn't find one in the literature.}
%Formulate correct algorithmic Sauer-Shelah lemma.
%subset of half the diension.
%formulate the question correctly. Some sort of description.
%formulate question.
%good chance for the particular sets we use, there is poly time algorithm.
%intersection of cube with ellipsoid. In this case make algorithmic.
\item
The second bottleneck is finding a subset with maximal volume. 

It's another place where we chose a subset, the subset that maximizes the volume out of all subsets of a given size (Lemma~\ref{lem:step1}). Specifically, we want the set of columns of a matrix that maximizes the volume of the convex hull.
Computing the volume of the convex hull requires some thought.
%subset max vol among all subsets of various size. look at one which maximizes volume. 
Also, there are $\binom nr$ subsets of size $r$; if $r=n/2$ there are exponentially many. We need a way to find subsets with maximum volume fast. There might be a replacement algorithm which approximately maximizes this volume.
\end{enumerate}
% going through Sauer-Shelah, formulating the algorithmic 
%Bottleneck for making this algorithm. 
%tailored for our proof... intersection of cubes with ellipsoids, $>2^{n-1}$, polytime find subset. 
%cubes intersected with other convex bodies?
%I don't know the algorithmic literature.
%I searched and looked at the literature.= and asked 
%If you think about it, look at it.

%??? Where should figure 6-4 be?
\blu{2-22: We are at the final lemma, Lemma~\ref{lem:rip-step2-1} in the Restricted Invertiblity Theorem.}

%Let me just remind you were we were. We were at the final lemma in the Restricted Invertiblity Theorem. We have a linear map $A: \R^m \to \R^n$, and have $\{Ae_j\}_{j = 1}^m$ linearly independent, and denote 
%\[
%F_j = \left(\txtn{span}(\{Ae_j\}_{i \neq j})\right)^{\perp}
%\]
%and then denote $M = \max_{i \leq j \leq m} \frac{1}{\|\txtn{Proj}_{F_1}Ae_j\|}$. 
%
%We have reduced everything to the following lemma: 
%
%\begin{lem} \llabel{lem:SS-induct} (Final lemma). \\
%Suppose $\sigma \subseteq \{1, \cdots, m\}$ with $t \geq 0$ an integer. Then there exists $\tau \subseteq \sigma$ with $|\tau| \geq (1 - \frac{1}{2^t})|\sigma|$ such that if we denote $\theta = \tau \cup \left(\{1, \cdots, m\} \setminus \sigma\right)$, then 
%\[
%\sum_{i \in \tau} |a_i| \le 2^{t/2} M \sqrt{|\sigma|} \|\sum_{i \in \theta}a_iAe_i\|_2 
%\]
%\end{lem}

The proof of this will be an inductive application of the Sauer-Shelah lemma. A very important idea comes from Giannopoulos. If you naively try to use Sauer-Shelah, it won't work out. We will give a stronger statement of the previous lemma which we can prove by induction. 

\begin{lem} \llabel{lem:SS-induct-stronger} (Stronger version of Final lemma). \\
Take $m, n \in \mathbb{N}$, $A: \R^m \to \R^n$ a linear operator such that $\{Ae_j\}_{j = 1}^m$ are linearly independent. Suppose that $k \geq 0$ is an integer and $\sigma \subseteq \{1, \cdots, m\}$. Then there exists $\tau \subseteq \sigma$ with $|\tau| \geq (1 - \frac{1}{2^k})|\sigma|$ such that for every $\theta \supseteq \tau$ for all $a \in \R^m$ we have
\[
 \sum_{i \in \tau} |a_i| \leq M\sqrt{|\sigma|}\left(\sum_{r = 1}^k 2^{r/2}\right)\left\|\sum_{i \in \theta} a_i Ae_i\right\|_2 + (2^k - 1)\sum_{i \in \theta \cap (\sigma \setminus \tau)} |a_i|       (*)
\]
\end{lem}

Our first lemma (all we need to complete Restricted Invertibility Principle) is the case where $\theta = \tau \cup \{1, \cdots, m\}\setminus \sigma$, $t = k$. 

We prove the stronger version via induction on $k$. 
\begin{proof}
As $k$ becomes bigger, we're eating more and more out from the set $\sigma$. So we're going to use Sauer-Shelah, taking half of $\sigma$, and then a bit more and a bit more. 
For $k = 0$, this is vacuous, since we take $\tau$ to be an empty set. Now via induction assume for $k$ that we found already $\tau \subseteq \sigma$, with $|\tau| \geq (1 - \frac{1}{2^k})|\sigma|$ and satisfies $(*)$ for every $\tau \subseteq \theta$. If $\sigma = \tau$ already, then $\tau$ satisfies for $k + 1$ as well, since WLOG $|\sigma \setminus \tau| > 0$. Now define $v_j$ is the projection 
\[
v_j = \frac{\Proj_{F_j} Ae_j}{\left\|\Proj_{F_j} Ae_j\right\|_2^2}
\]
Then $\langle v_i, Ae_j \rangle = \delta_{ij}$, by definition since we're looking at a dual basis for the $Ae_j$s. 

Now we want to user Sauer-Shelah so we're going to define a certain subset of the cube. Define 
\[
\Omega = \{\epsilon \in \{\pm 1\}^{\sigma \setminus \tau}: \left\|\sum_{i \in \sigma \setminus \tau} \epsilon_i v_i \right\|_2 \leq M\sqrt{2|\sigma \setminus \tau|}\}\]

So this is really an ellipsoid intersected with the cube, since the $v_i$s are not orthogonal.  Then we have 
\[
M^2 |\sigma \setminus \tau| \geq \sum_{i \in \sigma \setminus \tau} \frac{1}{\|\txtn{Proj}_{F_j}Ae_j\|^2} = \sum_{j \in \sigma \setminus \tau} \|v_j\|_2^2
\]
\[
= \frac{1}{2^{|\sigma \setminus \tau|}} \sum_{\epsilon \in \{\pm 1\}^{\sigma \setminus \tau}} \| \sum_{j \in \sigma \setminus \tau} \epsilon_j v_j \|_2^2
\]
where the last step is true for any vectors (sum the squares and the pairwise correlations disappear). 

Now we're using Markov's inequality to get 
\[
\geq \frac{1}{2^{|\sigma \setminus \tau|}} \left(2^{|\sigma \setminus \tau|} - |\Omega| \right)M^2 2|\sigma \setminus \tau|
\]
which gives 
\[
|\Omega| > 2^{|\sigma \setminus \tau| - 1}
\]

Then by Sauer-Shelah lemma, there exists $\beta \subseteq \sigma \setminus \tau$ such that 
\[
\Proj_{\R^{\beta}} \Omega = \{\pm 1 \}^{\beta}
\]
and 
\[
|\beta| \geq \frac{1}{2}|\sigma \setminus \tau|
\]

Now define $\tau^* = \tau \cup \beta$. We will show that $\tau^*$ satisfies the inductive hypothesis with $k + 1$. Each time we find a certain set of coordinates to add to what we have before. $|\tau^*|$ is the correct size because 
\[
|\tau^*| = |\tau| + |\beta| \geq |\tau| + \frac{|\sigma| - |\tau|}{2} = \frac{|\tau| + |\sigma|}{2} \geq \left(1 - \frac{1}{2^{k + 1}}\right)|\sigma|
\]
where we used that $|\tau| \geq \left(1 - \frac{1}{2^{k}}\right)|\sigma|$. So at least $\tau^*$ is the right size. 

Now, suppose $\theta \supseteq \tau^*$. For every $a \in \R^m$, we claim there exists some $\epsilon \in \Omega$ such that $\forall j \in \beta$ such that $\epsilon_j = \txtn{sign}(a_j)$. For any $\beta$, we can find some vector in the cube that has the sign pattern of our given vector $a$. What does being in $\Omega$ mean? It means that at least the dual basis is small there. $\epsilon \in \Omega$ says that 
\[
\|\sum_{i \in \sigma \setminus \tau} \epsilon_i v_i\|_2 \leq M \sqrt{2|\sigma \setminus \tau|} \leq \frac{M\sqrt{2|\sigma|}}{2^{k/2}}
\]
That was how we chose our ellipsoid. So we know a bound for just $\tau$ already, now let's do it with the addition of $\beta$. Well, 
\[
\sum_{i \in \beta} |a_i| = \langle \sum_{i \in \beta} a_iAe_i, \sum_{i \in \sigma \setminus \tau} \epsilon v_i \rangle
\]
which is precisely because the $v_i$'s were a dual basis and the dot products will be one. We only know the $\epsilon_i$ are the signs when you're inside $\beta$.  This equals 
\[
= \langle \sum_{i \in \theta} a_iAe_i, \sum_{i \in \sigma \setminus \tau} \epsilon v_i \rangle - \sum_{i \in (\theta \setminus \beta)\cap (\sigma \setminus \tau)} \epsilon_ia_i
\]
Note that $(\theta \setminus \beta)\cap (\sigma \setminus \tau) = \theta \cap (\sigma \setminus \tau^*)$. We can't control the signs $\epsilon_i$ any more. Then, we get applying Sauer-Shelah
\[
\leq \left\| \sum_{i \in \theta} a_i Ae_i\right\|_2 \cdot \left\| \sum_{i \in \sigma \setminus \tau} \epsilon v_i \right\|_2 + \sum_{i \in \theta \cap (\sigma \setminus \tau^*)} |a_i|
\]
\[
\leq \left\| \sum_{i \in \theta} a_iAe_i \right\|_2 \cdot \frac{M\sqrt{2|\sigma|}}{2^{k/2}} + \sum_{i \in \theta \cap (\sigma \setminus \tau^*)} |a_i|
\]
since Sauer-Shelah told us nothing about the signs of $\epsilon_i$, so we just take the worst possible thing. 

Then 
\[
\sum_{i \in \beta} |a_i| \leq \frac{M\sqrt{2|\sigma|}}{2^{k/2}}\left\|\sum_{i \in \theta} a_iAe_i\right\|_2 + \sum_{i \in \theta \setminus (\sigma \setminus \tau^*)} |a_i|
\]
Using the inductive step, 
\bal
\sum_{i \in \tau^*} |a_i| &= \sum_{i \in \tau} |a_i| + \sum_{i \in \beta} |a_i| 
\\
&\leq M\sqrt{|\sigma|} \alpha_k \left\| \sum_{i \in \theta} a_iAe_i\right\|_2 + \left(2^k - 1\right)\sum_{i \in \theta \cap (\sigma \setminus \tau)}|a_i| + \sum_{i \in \beta} |a_i|
\\
&= \alpha_k \sqrt{|\sigma|}\left\| \sum_{i \in \theta} a_iAe_i\right\|_2 + \left(2^k - 1\right)\sum_{i \in \theta \cap (\sigma \setminus \tau^*)} |a_i| + 2^k \sum_{i \in \beta} |a_i|
\end{align*}
In the last step, we throw $\beta$ away, but with weight $2^k - 1$. 
And now use what we got before for the bound on $\sum_{i \in \beta}|a_i|$ and plug it in to get
\[
\leq \left(\alpha_k + 2^{(k + 1)/2}\right)\sqrt{|\sigma|}\left\|\sum_{i \in \theta} a_iAe_i\right\|_2 + \left(2^{k + 1} - 1\right) \sum_{i \in \theta \cap (\sigma \setminus \tau^*)} |a_i|
\]
which is exactly the inductive hypothesis. I looked through the original Gianpopoulous paper, and it was clear he tried out many many things to find which inductive hypothesis makes everything go through cleanly. You want to bound an $l_1$ sum from above, so you want to use duality, and then use Sauer-Shelah to get signs such that the norm of the dual-basis is small. 
\end{proof}

\begin{rem} Algorithmic Sauer-Shelah. \\
Now regarding the use of Sauer-Shelah, we can see that we are only using it for intersecting cubes with ellipsoid. So regarding what I said last time, what we need is an algorithm for finding these intersections. The reason these ellipsoids are big is because we are actually multiplying by $\sqrt{2}$ times the expectation. So this algorithm is probably do-able. Maybe afterwards, you could ask for higher dimensional shapes. I've seen some references that worked for Sauer-Shelah when sets were of a special form, namely of size $o(n)$. This is something more geometric. I don't think there's literature about Sauer-Shelah for intersection of surfaces with small degree. This is a tiny motivation to do it, but it's still interesting independently. 
\end{rem}


\section{Bourgain's Discretization Theorem}

We will prove Bourgain's Discretization Theorem. This will take maybe two weeks, and has many interesting ingredients along the way. By doing this, we will also prove Ribe's theorem, which is what we stated at the beginning. 

Let's remind ourselves of the definition: 

\begin{df} Discretization Modulus. \\
$(X, \|\cdot\|_X), (Y, \|\cdot\|_Y)$ are Banach spaces. Let $\epsilon \in (0, 1)$. Then $\delta_{X \hookrightarrow Y}(\epsilon)$ is the supremum over $\delta > 0$ such that for every $\delta$-net $N_{\delta}$ of the unit ball of $X$, the distortion 
\[
C_Y(X) \leq \frac{C_Y(N_{\delta})}{1 - \epsilon}
\]
$C_Y$ is smallest bi-Lipschitz distortion by which you can embed $X$ into $Y$. There are ideas required to get $1 - \epsilon$. I'll decide by next time if I want to do the full thing or just do a constant. Think of $\epsilon = 1/2$. What we're saying is that if we succeed in embedding a $\delta$-net into $Y$, then we succeeded in the full space with a distortion twice as much. A priori it's not even clear that there exists such a $\delta$. There's a nontrivial compactness argument to prove you can (Lesbegue density points). But we will just prove bounds on it assuming it exists. 
\end{df}

Now, Bourgain's discretization theorem says 
\begin{thm} Bourgain's discretization theorem. \\
If dim$(X) = n$, dim$(Y) = \infty$, then 
\[
\delta_{X \hra Y}(\epsilon) \geq e^{-\left(\frac{n}{\epsilon}\right)^{C*n}}
\]
for $C$ a universal constant. 
\end{thm}
The way to read this is: If $\delta$ is bigger than this number which is only dependent on $n$, then given any Banach spaces with dimension $n$, there are mappings with this granularity for any such spaces. 

\begin{rem}
It doesn't matter what mapping you're actually using, the proof will give a linear mapping and we won't end up needing them. Assuming linear map in the definition is not necessary. 
Rademacher's theorem says that for any mapping from $\R^n \to \R^n$ is differentiable almost everywhere for bi-Lipschitz derivative, and you can extend this to $n = \infty$, but you need some additional properties: You need to be embedding into the dual space, and the limit needs to be in the weak$-*$ topology. That derivative is almost everywhere, but you can definitely have a sequence of norm$-1$ vectors that tend to $0$. A weak $*$ limit of a function can degenerately become $0$, the upper bound is not the issue. But you can prove that this doesn't happen almost everywhere. You can look at the Principle of Local Reflexivity, which says $Y^{**}$ and $Y$ are not the same for infinite dimensions. The double dual of all sequences which tend to $0$ is $L_{\infty}$, a bigger space, but the difference between these is never appearing in finite dimensional phenomena. 
\end{rem}

From now on, $B_X = \left\{x \in X: \|X\|_X \leq 1 \right\}$, the ball. $S_X = \partial B = \left\{x \in X: \|X\|_X = 1\right\}$, the boundary. 

Later on we will be differentiating things without thinking about it, so I just want to prove to you first that $X \to \|X\|_X$ is smooth on $X \setminus \{0\}$. 

\begin{lem} For all $\delta \in (0, 1)$ there exists some $\delta$-net of $S_X$ with $|N_{\delta}| \leq \left(1 + \frac{2}{\delta}\right)^n$. 
\end{lem}
\begin{proof}
Let $N_{\delta} \subseteq S_X$ be maximal with respect to inclusion such that $\|x - y\|_X > \delta$ for every distinct $x, y \in N_{\delta}$. We want it to be both separated and $\delta$-dense. For every $z \in S_{X}$, if $z \in N_{\delta}$, $\{z\} \cup N_{\delta}$ implies there exists $x \in N_{\delta}$, $\|x - y\|_X \leq \delta$. The balls $\{x + \frac{\delta}{2}B_X\}_{x \in M_{\delta}}$ are pairwise disjoint. Moreover, the balls are all contained in $1 + \frac{\delta}{2}B_X$. And then we get volume $(\frac{\delta}{2})^n\txtn{vol}(B_X)$, and we can get
\[
\txtn{vol}((1 + \frac{\delta}{2})B_X) \geq \sum_{X \in M_{\delta}} \txtn{vol}(x + \frac{\delta}{2}B_X)
\]
\[
\left(1 + \frac{\delta}{2}\right)^n\txtn{vol}(B_X) = |N_{\delta}|\left(\frac{\delta}{2}\right)^n\txtn{vol}(B_X)
\]
If you ask what the smallest size of a $\delta$-net is, there are bounds, but they are not sharp. There is a lot of literature about the relations between these things, we just need an upper bound. 

Now say you have your convex body, and you find your $\delta$-net $N_{\delta}$  of $S_X$ with $|N_{\delta}| = N \leq (1 + \frac{2}{\delta})^n$ which is finite. Then for every $x \in N_{\delta}$, choose any $z^* \in X^*$ unit vector on the sphere ($\langle z^*, z \rangle = 1$) and $\|z^*\|_{X^*} = 1$ by Hahn-Banach. It normalizes the net-point. What would be a good approximation? Let $k$ be an integer such that $N^{1/(2k)} \leq 1 + \delta$. Then define 
\[
\|x\| := \left(\sum_{z \in N_{\delta}} \langle z^*, x \rangle^{2k}\right)^{1/(2k)}
\]
Each term, separately $|\langle z^*, x \rangle| \leq \|x\|_X$, thus we know $(1 - \delta)\|x\|_X \leq \|x\| \leq N^{1/2k}\|x\|_X \leq (1 + \delta)\|x\|_X$. If $x \in S_X$, then choose $z \in N_{\delta}$ such that $\|x - z\|_X \leq \delta$, and thus $1 - \langle z^*, x \rangle = \langle z^*, z - x \rangle \leq \|z - x\| \leq \delta$, and $\langle z^*, x \rangle \geq 1 - \delta$. Thus any norm is up to $1 + \delta$ some really nice smooth norm. $\delta$ was arbitrary, if you prove for Bourgain, you prove it for any norm, and now without loss of generality I can differentiate. 
\end{proof}

Let me just explain the strategy of how we will prove Bourgain's discretization theorem. We're given a $\delta$-net $N_{\delta} \subseteq B_X$ with $\delta \leq e^{-(n/\epsilon)^{Cn}}$, and we know $\exists $ $f: N_{\delta} \to Y$ such that $\frac{1}{D}\|x - y\|_X \leq \|f(x) - f(y)\|_Y \leq \|x - y\|_X$, which means you can embed with distortion $D$. Our goal: If $\delta < e^{-(D/\epsilon)^{Cn}}$, then there exists a linear operator $T: X \to Y$ such that $\|T\|\cdot\|T^{-1}\| \leq 1 + 20\epsilon$. 

We will need a little background in convex geometry. We're going to find the correct coorindate system (John ellipsoid), which will give us a dot product structure and a natural Laplacian. A priori $f$ is defined on the net. We're going to find that it extends to the whole space in a nice way (Bourgain's extension theorem) which doesn't coincide with the function on the net, but is not too far away from it. Then we will solve the Laplace equation. We will start at initial condition, and then evolve $f$ according to the Poisson semigroup. This extended function is going to be smooth the minute you flow a little bit away from your discrete function. And you can differentiate it! We will prove that there's a point where the derivative satisfies what we want: The point cannot not exist (pigeonhole style argument), but we won't be able to pinpoint where the derivative behaves nicely. And this will come from estimates of the Poisson kernel, and we will jump into the Fourier analysis. 


\blu{2-24}

\section{Bourgain's almost extension theorem}
%Let $\de\in (0,\rc2)$, $X$ an $n$-dimensional normed space, $\cal N_\de$ a $\de$-net, $B_X$ the unit ball, $\dim X=n$. Let $Y$ be another Banach space with $\dim Y=\iy$. Suppose that there exists $f:\cal N_\de\to Y$ such that 
%\[
%\rc{D} \ve{x-y}_X\le \ve{f(x)-f(y)}_Y \le \ve{x-y}_X.
%\]
%Our goal is to show that if $\de\le e^{-D^{Cn}}$ then this implies that there exists $T:X\to Y$ linear operator invertible $\ve{T}\ve{T^{-1}}\lesssim D$. 
%
%Without loss of generality we can assume $D\le n$. 
\begin{thm}[Bourgain's almost extension theorem] 
\index{Bourgain's almost extension theorem}
\label{thm:baet}
Let $X$ be a $n$-dimensional normed space, $Y$ a Banach space, $\cal N_\de\subeq S_X$ a $\de$-net of $S_X$, $\tau\ge C\de$. Suppose $f:\cal N_\de\to Y$ is $L$-Lipschitz. Then there exists $F:X\to Y$ such that 
\begin{enumerate}
\item $\ve{F}_{\text{Lip}} \lesssim (1+\fc{\de n}{\tau}) L$.
\item $\ve{F(x)-f(x)}_Y\le \tau L$ for all $x\in \cal N_\de$. 
\item $\Supp(F) \subeq (2+\tau)B_X$.
\item $F$ is smooth.
%3, 4 added: 2-29 (useful to have compact support)
\end{enumerate} % all these constants are small.
Parts 3 and 4 will come ``for free."
%if $\de$ small, intuitively close. 
%biLip on net small, not biLip but separates points far apart, Lip when choose parameters. Smooth out, can differentiate. Derivative is a linear operator.
%The deriv is linear operator, norm will be bounded.
%That will be our $T$. Choose parameter so that universal constant. 
%inverse of $T$ bounded
%only think know separate far 
%definitely not invertible
%there is an existence statement.
%nice geometric fact
%if you nail values on the net and extend even thogh there may be flat regions they can not not flat everywhere.
%insist that take exactly the values. In that setting there is a huge amount of literature. If you insist on keeping value, the Lipshitz consant must grow. 
\end{thm}
\subsection{Lipschitz extension problem}

\begin{thm}[Johnson-Lindenstrauss-Schechtman, 1986]
Let $X$ be a $n$-dimensional normed space $A\subeq X$, $Y$ be a Banach space, and $f:A\to Y$ be $L$-Lipschitz. There exists $F:X\to Y$ such that $F|_A=f$ and $\ve{F}_{\text{Lip}}\lesssim nL$.
\end{thm}
We know a lower bound of $\sqrt n$; losing $\sqrt n$ is sometimes needed. (The lower bound for nets on the whole space is $\sqrt[4]{n}$.) A big open problem is what the true bound is. 

This was done a year before Bourgain.Why didn't he use this theorem? This theorem is not sufficient because the Lipschitz constant grows with $n$.

%We want to extend it so the norm of the derivative is free, and we only fight against hte inverse. 
We want to show that $\ve{T}\ve{T^{-1}}\lesssim D$. We can't bound the norm of $T^{-1}$ with anything less than the distortion $D$, so to prove Bourgain embedding we can't lose anything in the Lipschitz constant of $T$; the Lipschitz constant can't go to $\iy$ as $n\to \iy$. 
 %going to $\iy$, the only thing we can say about the derivative is that it is at most that. Anything going to $\iy$ destroys the argument. %$\ve{T^{-1}}$ 
%The $T^{-1}$ cannot be better than $\rc{D}$. 

Bourgain had the idea of relaxing the requirement that the new function be strictly an extension (i.e., agree with the original function where it is defined). What's extremely important  is that the new function be Lipschitz with a constant independent of $n$.
 
We need $\ve{F}_{\text{Lip}}\lesssim L$. 
%lower bounds show go to $\iy$.
Let's normalize so $L=1$. 

When the parameter is $\tau=n\de$, we want $\ve{f(x)-F(x)}\lesssim n\de$. Note $\de$ is small (less than the inverse of any polynomial), so losing $n$ is nothing.\footnote{If people get $\de$ to be polynomial, then we'll have to start caring about $n$.}
%change substantial
%strategy extend and work with derivatives.
%huge slopes, areas where derivative huge.

How sharp is Theorem~\ref{thm:baet}? Given a 1-Lipschitz function on a $\de$-net, if we want to almost embed it without losing anything, how close can close can we guarantee it to  be from the original function
\begin{thm}
There exists a $n$-dimensional normed space $X$, Banach space $Y$, $\de>0$, $\cal N_\de\subeq S_X$ a $\de$-net, 1-Lipschitz function $f:\cal N_\de\to Y$ such that if $f:X\to Y$, $\ve{F}_{\text{Lip}}\lesssim 1$ then there exists $x\in \cal N_\de$ such that 
\[
\ve{F(x)-f(x)}\gtrsim \fc{n}{e^{c\sqrt{\ln n}}}.
\]
\end{thm}
Thus, what Bourgain proved is essentially sharp.
This is a fun construction with Grothendieck's inequality.

%strategy, choose coordinate system

Our strategy is as follows. Consider $P_t*F$, where 
\[
P_t(x)=\fc{C_nt}{(t^2+\ve{x}_2^2)^{\fc{n+1}2}}, \quad C_n=\fc{\Ga\pf{n+1}2}{\pi^{\fc{n+1}2}},
\]
%diffble
and $P_t$ is the \ivocab{Poisson kernel}. 
%if start with Lip, average out is still Lip
Let $(TF)(x)=(P_t*F)'(x)$; $T$ is linear and $\ve{T}\lesssim 1$. As $t\to 0$, this becomes closer to $F$. We hope when $t\to 0$ that it is invertible. This is true. We give a proof by contradiction (we won't actually find what $t$ is) using the pigeonhole principle. %it will be a smoothing argument, but with the nice twist. 2 things: extension which can hve flat regions. Poisson semigroup is morally averaging over ball of radius $t$. Suppose $F'=0$. If $>N\de$, then have many pairs which are different. Averaging feels like it's correct this. But it could have cancellations. Argue that things cannot cancel, remain invertible. Always an argument required.

The Poisson kernel depends on an Euclidean norm in it. 
In order for the argument to work, we have to choose the right Euclidean norm.
%The whole thing will work if you choose the Euclidean norm coorectly. %, an orthonormal 

\begin{thm}[John's theorem]
Let $X$ be a $n$-dimensional normed space, identified with $\R^n$. Then there exists a linear operator $T:\R^n\to \R^n$ such that $\rc{\sqrt{n}}\ve{Tx}_2\le \ve{x}_X \le \ve{Tx}_2$ for all $x\in X$.
\end{thm}
John's Theorem says we can always find $T$ which sandwiches the $X$-norm up to a factor of $\sqrt n$. 
%We will choose  the coordinates such that $\ve{Tx}_2\le \ve{x}_2$. %$T^*T$. 
%%A different way to write the theorem is
%Consider $\cal E=\set{x}{\ve{Tx}_2\le 1}$. If you are in the ellipsoid, then you are in the unit ball. If $\ve{Tx}_2\le 1$, then $\ve{x}\le 1$. 
``Everything is a Hilbert space up to a $\sqrt n$ factor."

%Can always 
%This shows why we can always assume $D\le \sqrt n$.
%$n$-dimensional subspaces really Euclidean. %REC
\begin{proof}
\Wog{} $D\le n$. 
%Need $Y$ to be dimension at least $n$. Otherwise $X$ doesn't embed into $Y$. In any infinite-de
%REC, Goretski. 
%you can always embed $X$ into an arbitrary $n$-dimensional subspace in $Y$ losing $\sqrt n\sqrt n=n$. Actually you can lose just $\sqrt n$, by a theorem of Goretzky. 
%$n$-dimensional subspaces. 
Let $\cal E$ be an ellipsoid of maximal volume contained in $B_X$. 
\[
\max\set{\vol(SB_{\ell_2^n})}{S\in M_n(\R), SB_{\ell_2^n} \subeq B_X}.
%maximizer is uniquely defined canonical shape. 
\]
$\cal E$ exists by compactness. 
Take $T=S^{-1}$. 

The goal is to show that $\sqrt n \cal E\supeq B_X$. We can choose the coordinate system such that $\cal E=B_{\ell_2^n}=B_2^n$.

Suppose by way of contradiction that $\sqrt nB_2^n\nsupeq B_X$. Then there exists $x\in B_X$ such that $\ve{x}_X\le 1$ yet $\ve{x}_2>\sqrt n$. %$x\nin \sqrt{n}B_2^n$ iff 

\ig{images/8-2}{.25}

Denote $d=\ve{x}_2, y=\fc xd$. Then $\ve{y}_2=1$ and $dy \in B_X$. 

%by apply rotation, fix structure
By applying a rotation, we can assume WLOG $y=e_1$. 

Claim: Define for $a>1,b<1$ %mult first by $a$.
\[
\cal E_{a,b} :=\set{\sumo in t_ie_i}{\pf{t_1}{a}^2 + \sum_{i=2}^n \pf{t_i}b^2 \le 1}.
\]
%come up with a,b
Stretching by $a$ and squeezing by $b$ can make the ellipse  grow and stay inside the body, contradicting that it is the minimizer.
%times det
More precisely, we show that there exists $a>1$ and $b<1$ such that $\cal E_{a,b}\subeq B_X$ and $\Vol(\cal E_{a,b}) > \Vol(B_2^n)$. This is equivalent to $ab^{n-1}>1$.

We need a lemma with some trigonometry.
\begin{lem}[2-dimensional lemma]
Suppose $a>1$, $b\in (0,1)$, and $\fc{a^2}{d^2}+b^2\pa{1-\rc{d^2}} \le 1$. Then $\cal E_{a,b}\subeq B_X$.
\end{lem}
%de_1,-de_1.
\begin{proof}
Let $b=\sfc{d^2-a^2}{d^2-1}$. Then $\psi(a)=ab^{n-1}=a\pf{d^2-a^2}{d^2-1}^{\fc{n-1}2}$, $\psi(1) = 1$. It is enough to show $\psi'(1)>0$. Now 
\[
\psi'(a) =\pf{d^2-a^2}{d^2-1}^{\fc{n-1}2-1} \fc{d^2-na^2}{d^2-1},
\]
which is indeed $>0$ for $d>\sqrt n$. 
Note this is really a 2-dimensional argument; there is 1 special direction. We shrunk the $y$ direction and expanded the $x$ direction. 

%Now we have a rhombus. %ellipse is in rhombas. 
%In $\R^2$ map the ellipse $\pf{t_1}{a}^2+\pf{t_2}{b}^2\le 1$ 
%%not the whole rhombus is in the body
%%if ellipse in rhombus, then also in body. 
It's enough to show the new ellipse $\cal E_{a,b}$ is in the rhombus in the picture. Calculations are left to the reader. %because $d<1$.
%just remember the picture. 
%map the ellipse $\le 1$ to ...
%\fixme{PICTURE}.

\ig{images/8-3}{.25}
\end{proof}
\end{proof}

\subsection{Proof}

\begin{proof}[Proof of Theorem~\ref{thm:baet}]
By translating $f$ (so that it is 0 at some point), \wog we can assume that for all $x\in \cal N_\de$, $\ve{f(x)}_Y\le 2L$. 

\step{1} (Rough (i.e., noncontinuous) extension $F_1$ on $S_X$.) We show that there exists $F_1:S_X\to Y$ such that for all $x\in \cal N_\de\to Y$ such that for all $x\in \cal N_\de$,
\begin{enumerate}
\item
$\ve{F_1(x)-f(x)}_Y\le 2L\de$.
\item
$\forall x,y\in S_X$, $\ve{F_1(x)-F_1(y)}\le L(\ve{x-y}_X+4\de)$.
\end{enumerate}•


This is a partition of unity argument. %Consider the open cover 
Write $\cal N_{\de}=\{X_1,\ldots, X_N\}$. Consider
\[
\{(x_p+2\de B_X)\cap S_X\}_{p=1}^N,
\]
which is an open cover of $S_X$.

Let $\{\phi_p:S_X\to [0,1]\}_{p=1}^N$ be a partition of unity subordinted to this open cover. This means that
\begin{enumerate}
\item
$\Supp \phi_p\subeq x_p + 2\de B_X$,
\item
$\sum_{p=1}^{N}\phi_p(x)=1$ for all $x\in S_X$.
\end{enumerate}
For all $x\in S_X$, define $F_1(x) = \sum_{p=1}^N\phi_p(x) f(x_p)\in Y$. Then as $F_1$ is a weighted sum of $f(x_p)$'s, $\ve{F_1}_{\iy}\le 2L$. 
%convex combination moves along with $x$.
If $x\in  \cal N_\de$, because $\phi_p(x)$ is 0 when $|x-x_p|>2\de$,
\[
\ve{F_1(x)-f(x)}_Y = \ve{\sum_{p:\ve{x-x_p}_X\le 2\de} \phi_p(x) (f(x_p) - f(x))} \le \sum_{\ve{x-x_p}\le 2\de} \phi_p(x)L\ve{x-x_p}_X\le 2L \de.
\]
For $x,y\in S_X$,
\bal
\ve{F_1(x)-F_1(y)}&=\ve{\sumr{\ve{x-x_p}\le 2\de}{\ve{y-x_q}\le 2\de} (f(x_p)-f(x_q))\phi_p(x)\phi_q(x)}\\
& \le
\sumr{\ve{x-x_p}\le 2\de}{\ve{y-x_q}\le 2\de} L\ve{x_p-x_q} \phi_p(x)\phi_p(y)\\
%init fun defined on sphere, 2\de.
%lost contin, alm lip + additive term how to fix
%homog
%extend, conv
&\le L(\ve{x-y}+4\de).
%\ve{x_p-x_q} \le \ve{x_p-x}+\ve{x-y}+\ve{y-x_q}.
\end{align*}

\blu{2-29: We continue proving Bourgain's almost extension theorem.}

\step{2} Extend $F_1$ to $F_2$ on the whole space such that 
\begin{enumerate}
\item
$\forall x\in \cal N_\de$, $\ve{F_2(x)-f(x)}_Y \lesssim L\de$.
\item
$\ve{F_2(x)-F_2(y)}_Y\lesssim L(\ve{x-y}_X + \de)$.
%not smooth yet. 
\item
$\Supp(F_2)\subeq 2B_X$.
\item
$F_2$ is smooth.
%F_1 had no bounded continuity, but is a sum against a partition of unity. 
%just smooth without any bounds fine.
%spiky.
%when I say not smooth, I mean no bounds.
%I need the norm to be smooth for this.
\end{enumerate}•

Denote $\al(t)=\max\{1-|1-t|,0\}$. 

\ig{images/9-1}{.25}

Let
\[
F_2(x)=\al(\ve{x}_X) F_1\pa{x}{\ve{x}_X}.
\]
%0 the moment it passes 2.
$F_2$ still satisfies condition 1. As for condition 2, 
\bal
\ve{F_2(x)-F_2(y)}_Y &= \ve{\al (\ve{x}_X)F_1\pf{x}{\ve{x}_X} - \al(\ve{y}_X) F_1\pf{y}{\ve{y}_X}} \\
&\le |\al(\ve{x})-\al(\ve{y})|\ub{\ve{F_1\pf{x}{\ve{x}_X}}}{\le 2L}+\al(\ve{y})\ve{F_1\pf{x}{\ve{x}_X} - F_1\pf{y}{\ve{y}_X} }\\
&\le (\ve{x}-\ve{y})2L + \al(\ve{y}) L\pa{
\ve{\nv{x}-\nv{y}}+4\de  
} \\
&\le 2L\ve{x-y}+L\al(\ve{y}) \pa{\ve{x}\ab{\rc{\ve{x}}-\rc{\ve{y}}} + \fc{\ve{x-y}}{\ve{y}} + 4\de}\\
&\le 2L\ve{x-y} +L\al(\ve{y}) \pa{\fc{\ve{x-y}}{\ve{y}} + \fc{\ve{x-y}}{\ve y}  + 4\de}\\
&\lesssim L(\ve{x-y}+\de),
%mult by 4de, use bounded by 1
\end{align*}
where in the last step we used $\al(\ve{y})\le \ve{y}$ and $\al(\ve{y})\le 1$. 

Note $F_2$ is smooth because the sum for $F_1$ was against a partition of unity and $\ved_X$ is smooth, although we don't have uniform bounds on smoothness for $F_2$.
%F_1 had no bounded continuity, but is a sum against a partition of unity. 
%just smooth without any bounds fine.
%spiky.
%when I say not smooth, I mean no bounds.
%I need the norm to be smooth for this.

%For the next step we need the following. 

\step{3} We make $F$ smoother by convolving.
\begin{lem}[Begun, 1999]
Let $F_2:X\to Y$ satisfy $\ve{F_2(x)-F_2(y)}_Y\le L(\ve{x-y}_X+\de)$. Let $\tau \ge c\de$. Define 
\[
F(x) = \rc{\Vol(\tau B_X)}\int_{\tau B_X} F_2(x+y)\,dy.
\]
Then 
\[
\ve{F}_{\text{Lip}} \le L\pa{1+\fc{\de n}{2\tau}}.
\]
\end{lem}
The lemma proves the almost extension theorem as follows. We passed from $f:\cal N_\de\to Y$ to $F_1$ to $F_2$ to $F$. 
If $x\in \cal N_\de$, 
\bal
\ve{F(x)-f(x)}_Y &=\ve{
\rc{\Vol(\tau B_X)} \int_{B_X} (F_2(x+y) - f(x))\,dy
}\\
&\le \rc{\Vol(\tau B_X)}\int_{\tau B_X}\ve{F_2(x+y)-F_2(x)}_Y + \ub{\ve{F_2(x)-f(x)}_Y}{\de L} \dy\\
&\le \rc{\Vol(\tau B_X)}\int_{\tau B_X}(L(\ub{\ve{y}_X}{\le\tau}+\de L)) \dy\lesssim L\tau.
\end{align*}
Now we prove the lemma. 
\begin{proof}
We need to show
\[
\ve{F(x)-F(y)}_Y \le L\pa{1+\fc{\de n}{2\tau}} \ve{x-y}_X.
\]
\Wog $y=0$, $\Vol(\tau B_X)=1$. Denote 
\bal
M&=\tau B_{X}\bs (x+\tau B_X)\\
M'&=(x+\tau B_X) \bs \tau B_X.
\end{align*}

\ig{images/9-2}{.25}

We have
\bal
F(0)-F(x) &= \int_M F_z(y)\,dy - \int_{M'} F_z(y)\,dy.
\end{align*}
Define $\om(z)$ to be the Euclidean length of the interval $(z+\R x)\cap (\tau B_X)$. By Fubini,
\[
\int_{\Proj_{X^{\perp}} (\tau B_X)} \om(z) \,dz = \Vol_n(\tau B_X)=1.
%intersection of projection.
\]
Denote
\bal
W&= \set{z\in \tau B_X}{(z+\R x)\cap (\tau B_X)\cap (x+\tau B_X)\ne \phi}\\
N&= \tau B_X\bs W.
\end{align*}
Define $C:M\to M'$ a shift in direction $X$ on every fiber that maps the interval $(z+\R x)\cap M\to (z+\R x)\cap M'$. 

\ig{images/9-3}{.25}

$C$ is a measure preserving transformation with
\[
\ve{z-C(z)}_X =\begin{cases}
\ve{x}_X , &z\le N\\
\om(z) \fc{\ve{x}_X}{\ve{x}_2},& z\in W\cap M.
\end{cases}
\]
(In the second case we translate by an extra factor $\fc{\om(z)}{\ve{x}_2}$.)
%(In the second case we add the total length $
%C maps $M'$ to $M$.
%do a clever change of variable differently in each fiber.
Then 
\bal
\ve{F(0)-F(x)}_Y &=\ve{\int_M F_2(y)\dy - \int_{M'}F_2(y)\dy}_Y\\
&= \ve{\int_M(F_2(y) - F_2(C(y)))\dy}_Y\\
&\le \int_M L(\ve{y-C(y)}_X+\de)\dy\\
&\le \int_M L (\ve{y-C(y)}_X + \de)\dy\\
&=L\de \Vol(M) + L \int_M \ve{y-C(y)}_X\dy\\
\int_M \ve{y-C(y)}_X\dy 
%orth decomp but not unit vector, integrate the length multiply by norm of direction. jacobian.
&=\int_{N}\ve{x}_X\dy + \int_{W\cap M}\fc{\om(y)\ve{x}_X}{\ve{x}_2}\dy\\
&=\ve{x}_X \Vol(N) + \int_{\Proj(W\cap M)} \fc{\om(z) \ve{x}_X}{\ve{x}_2} \ve{x}_2\,dz&\text{orthogonal decomposition}\\
&=\ve{x}_X \Vol(N) + \Vol(\tau B_X\bs N) \\
&=\ve{x}_X \Vol(\tau B_X)=\ve{x}_X.
\end{align*}
%w as the entire length. What I get is the entire volume. 
We show $M=\tau B_X\bs (x+\tau B_X) \subeq \tau B_X\bs (1-\fc{\ve{x}}{\tau}) \tau B_X$. Indeed, for $y\in M$,
\bal
\ve{y-x}_X&\ge \tau\\
\ve{y} & \ge \tau - \ve{x} = \pa{1-\fc{\ve{x}}{\tau}}\tau.
\end{align*}
\end{proof}
Then 
\[
\Vol(M) \le \Vol(\tau B_X)-\Vol\pa{\pa{1-\fc{\ve{x}}{\tau}}\tau B_X}=1-\pa{1-\fc{\ve{x}}{\tau}} \lesssim \fc{n\ve{x}}{\tau}
\]
\end{proof}
Bourgain did it in a more complicated, analytic way avoiding geometry. Begun notices that careful geometry is sufficient.


Later we will show this theorem is sharp.

%iteration!

\section{Proof of Bourgain's discretization theorem}

At small distances there is no guarantee on the function $f$. Just taking derivatives is dangerous. It might be true that we can work with the initial function. But the only way Bourgain figured out how to prove the theorem was to make a 1-parameter family of functions.

\subsection{The Poisson semigroup}
\begin{df}
The \ivocab{Poisson kernel} is $P_t(x):\R^n\to \R$ given by
\[
P_t(x)=\fc{C_nt}{(t^2+\ve{x}_2^2)^{\fc{n+1}2}}, \quad C_n=\fc{\Ga\pf{n+1}2}{\pi^{\fc{n+1}2}}.
\]
%Convolution becomes product under FT
\end{df}

\begin{pr}[Properties of Poisson kernel]
\begin{enumerate}
\item
For all $t>0$, $\int_{\R^n} P_t(x)\,dx=1$.
\item
(Semigroup property) $P_t*P_s=P_{t+s}$. 
\item
$\wh{P_t}(x) = e^{-2\pi\ve{x}_2t}$.
\end{enumerate}•
\end{pr}

\begin{lem}
Let $F$ be the function obtained from Bourgain's almost extension theorem~\ref{thm:baet}.
For all $t>0$, $\ve{P_t*F}_{\text{Lip}}\lesssim 1$.
%P_t is prob measure.
%Average values of $F$.
%poU, geometr y of ball, average with decaying weights.
%all averaging of Lipschitz things.

%1+\ep subtlety, ceases to be true, need restriction on $T$. Not relevant. $1+\ep$ version do more carefully.
\end{lem}
%Note to get $P_t*F$ we had three averaging arguments: partition of unity, averaging with respect to a ball, and then averaging with decaying weights.
\begin{proof}
We have 
\bal
P_t*F(x)-P_t*F(y) &= \int_{\R^n} P_t(z)(F(x-z) - F(x-y))\,dz.
\end{align*}
Now use the fact that $F$ is Lipschitz.
\end{proof}
Our goal is to show there exists $t_0>0$, $x\in B$ %\in \rc{} B_X$ 
such that if we define
\beq{eq:bdt-T}
T=(P_{t_0}*F)'(x):X\to Y,
\eeq
(i.e., $Ta=\pl_a(P_{t_0}*F)(x)$).
We showed $\ve{T}\lesssim 1$. It remains to show $\ve{T^{-1}}\lesssim D$. Then $T$ has distortion at most $O(D)$, and hence $T$ gives the desired extension in Bourgain's discretizaton theorem~\ref{thm:bdt}.
%$T_y=\lim_{h\to \iy} \fc{P_{t_0}*F(x+hy) - P_{t_0}*F(x)}{h}$.
%pigeonhole, must exist

\blu{3-2: Today we continue with Bourgain's Theorem. Summarizing our progress so far:
\begin{enumerate}
\item
Initially we had a function $f: \cal N_{\de} \to Y$ and a $\delta$-net $\cal N_{\de}$ of $B_X$.
\item
From $f$ we got a new function $F: X \to Y$ with $\txtn{supp}(F) \subseteq 3 B_X$, satisfying the following. (None of the constants matter.) 
\begin{enumerate}
\item
$\|F\|_{L_p} \le C$ for some constant.
\item 
For all $x \in \cal N_{\de}$, $\|F(x) - f(x)\|_Y \leq n\delta$. 
\end{enumerate}
\end{enumerate}
}

Let $P_t: \R^n \to \R$ be the Poisson kernel. What can be said about the convolution $P_t * F$? We know that for all $t$ and for all $x$, $\|(P_t * F)(x)\|_{X \to Y} \le 1$. The goal is to make this function invertible. More precisely, the norm of the inverted operator is not too small. 

We'll ensure this one direction at a time by investigating the directional derivatives. For $g: \R^n \to Y$, the directional derivative is defined by: for all $a \in S_X$, \[\partial_a g(x) = \lim_{t \to 0} \frac{g(x + ta) - g(x)}{t}.\]

There isn't any particular reason why we need to use the Poisson kernel; there are many other kernels which satisfy this. We \emph{definitely} need the semigroup property, in addition to all sorts of decay conditions. 

%There will be many lemmas we need to go through. 
We need a lot of lemmas.

%\begin{lem} Inequality lemma. \\
%Suppose $t, R, \delta > 0$ satisfy: 
%\begin{enumerate}
%
%\item $\delta \leq c \frac{t\log(3/t)}{\sqrt{n}} \leq c' \frac{1}{n^{5/2}D^2}$
%
%\item $c'' n^{3/2}D^2 \log(3/t) \leq R \leq \frac{c''}{t\sqrt{n}}$
%
%\end{enumerate}
%Then for every $x \in \frac{1}{2}B_X$ and $a \in \delta x$ a direction, (note $\partial_a$ is a directional derivative) 
%\[
%\left(\left\|\partial_a(P_t*F)\right\|_Y * P_{Rt}\right)(x) \ge c''''/D
%\]
%\end{lem}

\begin{lem}[Key lemma 1]\llabel{lem:bdt-1}
%Let $c, c', c''$ be universal constants which arise from the proof. 
There are universal constants $C,C',C''$ such that the following hold.
Suppose $t \in (0, \frac{1}{2}]$, $R \in (0, \infty),\delta \in (0, \frac{1}{100n})$ satisfy 
\begin{align}
\delta &\leq C \frac{t\log(3/t)}{\sqrt{n}} \leq C' \frac{1}{n^{5/2}D^2}
\label{eq:bdt1-1}
\\
\label{eq:bdt1-2}
 C''n^{3/2}D^2\log(3/t) &\leq R \leq \frac{C''}{t\sqrt{n}}.
\end{align}
Then for all $x \in \frac{1}{2} B_X, a \in S_X$, we have
\[
(\|\partial_a(P_t*F)\|_Y * P_{Rt})(x) \ge \frac{C'''}{D}
\]
\end{lem}
This result says that given a direction $a$ and look at how long the vector $\pl_a (P_t*F)$ is, it is not only large, but large on average. Here the average is over the time period $Rt$. 

\fixme{Note that I haven't been very careful with constants; there are some points where the constant is missing/needs to be adjusted.}
%How large on average? 
%?
%You let the Poisson semigroup fight itself (flow in times $t$), you'll be at $Rt$. We want this to be long for every direction, but the average is also a function of $t$ (the measure with respect to averaging). So we have a directional derivative result with weird averaging: We flow an exponetially small amount of time, from this information you want to deduce the situation where we are invertible. 

The first condition says is that $t$ is not too small, but the lower bound is extremely small. The upper bound is $\prc{n}^{k}$ for some $k$ whereas the lower bound is something like $e^{-e^{n}}$, so there's a huge gap. %ridiculously small at $(e^n)^n$. So there's a huge gap.

The second condition says that $\log\prc{t}$ is still exponential. 


This lemma will come up later on and it will be clear. 

From now on, let us assume this key lemma and I will show you how to finish. We will go back and prove it later. 

\begin{lem}[Key lemma $2$]\llabel{lem:bdt-2}
There's a constant $C_1$ such that the following holds (we can take $C_1=8$). 
Let $\mu$ be any Borel probability measure on $S_X$. For every $R, A \in (0, \infty)$, there exists $ \frac{A}{(R + 1)^{m + 1}} \leq t \leq A$ such that 
\[
\int_{S_X} \int_{\R^n} \|\partial_a (P_t * F)(x)\|_Y dx d\mu(a) \leq \int_{S_X} \ub{\int_{\R^n} \|\partial_a(P_{(R + 1)t} * F)\|_Y dx}{(*)} d\mu(a) + \frac{C_1\txtn{vol}(3B_X)}{m}
\]
\end{lem}
Basically, under convolution, we can take the derivative under the integral sign. 
Thus (*) is an average of what we wrote on the left. Averaging only makes things smaller for norms, because norms are convex. Thus the  right integral is less than the left integral, from Jensen's inequality. 

The lemma says that  adding that small factor, we get a bound in the opposite direction. %If the norm is strongly convex, we would actually get that our average is a constant. %?
We will argue that there must be a scale at which Jensen's inequality stabilizes, i.e., $\|\partial_a(P_t*F)\|_Y*P_{Rt}(x)$ stabilizes to a constant.

Note this is really the pigeonhole argument, geometrically.

\begin{proof}
Bourgain uses this technique a lot in his papers: find some point where the inequality is equality, and then leverage that. 

If the result does not hold, then for every $t$ in the range $\frac{t}{(R + 1)^{m + 1}} \leq t \leq A$, the reverse holds. We'll use the inequality at eveyr point in a geometric series. For  all $k \in\{ 0, \ldots, m + 1\}$, 
\[
\int_{S_X} \int_{\R^n} \|\partial_a(P_{A(R + 1)^{k - m - 1}}*F)(x)\|_Ydxd\mu(a) > \int_{S_X} \int_{\R^n} \|\partial_a(P_{A(R + 1)^{k - m}}*F)(x)\|_Y dxd\mu(a) + \frac{C_1\txtn{vol}(3B_X)}{m}.
\]
%We basically are iterating this inequality many times, starting from the lower bound condition stated in the lemma. 
Summing up these inequalities and telescoping
\begin{equation}\label{eq:bdt-poisson1}
\int_{S_X} \int_{\R^n} \|\partial_a(P_{A(R + 1)^{-m - 1}}*F)(x)\|_Y dxd\mu(a) > \int_{S_X} \int_{\R^n} \|\partial_a(P_{A(R + 1)}*F)(x)\|_Y dxd\mu(a) + \frac{8(m + 1)\txtn{vol}(3B_X)}{m}
\eeq
(recall that we've been using the semigroup properties of the Poisson kernel this whole time). Now why is this conclusion absurd? 
Take $C_1$ to be the bound on the Lipschitz constant of $F$. Because $\|\partial_aF\|_Y \leq 8$, we have$\partial_a(P_{A(R+1)^{-m - 1}}*F)(x)\le C_1$. Since partial derivatives commute with the integral sign, we get 
\begin{align}
\int_{S_X}
\int_{\R^n} \|\partial_a(P_{A(R+1)^{-m - 1}}*F)(x)\|_Y \dx\,d\mu(a) &= \int_{S_X}\int_{\R^n} \|(P_{A(R+1)^{-m - 1}}*\partial_aF)(x)\|_Y \dx\,d\mu(a)\\
&\le \int_{S_X}\int_{\R^n} C_1\le C_1\vol(B_X)\label{eq:bdt-poisson2}
\end{align}
because  $F$ is Lipschitz and the Poisson semigroup integrates to unity. 
Together~\eqref{eq:bdt-poisson1} and~\eqref{eq:bdt-poisson2} give a contradiction.
\end{proof}

Now assuming Lemma~\ref{lem:bdt-1} (Key Lemma $1$), let's complete the proof of Bourgain's discretization theorem. Assume from now on $\delta < \left(\frac{1}{cD}\right)^{(CD)^{2n}}$ where $C$ is a large enough constant that we will choose $(C = 500$ or something). 

\begin{proof}[Proof of Bourgain's Discretization Theorem~\ref{thm:bdt}]
Let $\mathcal{F} \subseteq S_X$ be a $\frac{1}{C_2D}$-net in $S_X$. Then $|\mathcal{F}| \leq (C_3D)^n$ for some $C_3$. We will apply the Lemma~\ref{lem:bdt-2} (Key Lemma 2) with $\mu$ the uniform measure on $\mathcal{F}$, 
\begin{align*}A &= (1/CD)^{5n}\\
R + 1 &= (CD)^{4n}\\
m &= \lceil (CD)^{n + 1} \rceil.
\end{align*}
 Then there exists $(1/(CD))^{(CD)^{2n}} \leq t \leq (1/(CD))^{5n}$ such that  %Then we can find the $t$ such that this holds. 
\beq{eq:bdt-pf-lem2}
\sum_{a \in \mathcal{F}} \int_{\R^n} \|\partial_a(P_t *F)(x)\|_Y dx \leq \sum_{a \in \mathcal{F}} \int_{\R^n} \|\partial_a(P_{(R + 1)t}*F)(x)\|_Y dx + \frac{8\txtn{vol}(3B_X)}{m} |\mathcal{F}|
\eeq
We check the conditions of Lemma~\ref{lem:bdt-1} (Key Lemma $1$).
\begin{enumerate}
\item
For~\eqref{eq:bdt1-1}, note $t$ is exponentially small, so the RHS inequality is satisfied.
For the LHS inequality, note $\de \le t$ and $\fc{C\ln \prc{t}}{\sqrt n}\le 1$. To see the second inequality, note $\fc{C\ln \prc{t}}{\sqrt n} \ge \fc{C5n \ln (CD)}{\sqrt n}\ge 1$ (for $n$ large enough).
\item 
For~\eqref{eq:bdt1-1}, note the LHS is dominated by $\ln \prc{t} \le (CD)^{2n} \ln(CD)$ which is much less than $R=(CD)^{4n}-1$, and $\rc t$, the dominating term on the RHS, is $\ge (CD)^{5n}$.
%the RHS is doubly exponential.
\end{enumerate}•
%If $\delta$ satisfies $\delta <\prc{CD}^{(CD)^{2n}}=t$, then the conditions of Lemma~\ref{lem:bdt-1} (Key Lemma $1$) hold. The first condition must be bigger than $t$, and the upper bound on $t$ is exponential even. So we're fine. Now we check the bound on $\log(1/t)$, which is at most exponential in $D$. What is $R$? We made that exponential too, so there is room to spare. Finally, $1/t$ exceeds $R$ as well since $1/t$ is $(CD)^{5n}$ while $R$ was $(CD)^{4n}$. So we're good. 

In my paper \fixme{(reference?)}, I wrote what the exact constants are. They're written in the paper, and are not that important. We're choosing our constants big enough so that our inequalities hold true with room to spare. 

Now we can use Key Lemma $1$, which says
\beq{eq:bdt-pf-lem1}
\left(\|\partial_a(P_t*F)\|_Y * P_{Rt}\right)(x)\ge \rc{D}
\eeq
 %Look at the norm of the derivative in direction $a$ $\|\partial_a(P_{(R + 1)t}*F)(x)\|_Y$. Note that 
Using $P_{(R + 1)t} = P_{Rt} * P_t$ (semigroup property) and Jensen's inequality on the norm, which is a convex function, we have 
\beq{eq:bdt-conv}
\|\partial_a(P_{(R + 1)t}*F)(x)\|_Y = \|\left(\partial_a(P_t*F)\right)*P_{Rt}(x)\|_Y \leq \left(\|\partial_a(P_t*F)\|_Y * P_{Rt}\right)(x).
\eeq
Since the norm is a convex function, by Jensen's we get our inequality. %Note that we are using semigroup properties liberally. 

Let 
\[
\psi(x):=\left(\|\partial_a(P_t*F)\|_Y * P_{Rt}\right)(x) - \|\partial_a\left(P_{(R + 1)t}*F\right)(x)\|_Y
\]
From~\eqref{eq:bdt-conv}, $\psi(x)\ge 0$ pointwise. 
%Think of this experssion above as a random variable. It's positive on half the ball. So we can use Markov's inequality. So you get that the volume over all the $x \in \frac{1}{2}B_X$ is 
Using Markov's inequality,
\[
\vol\set{x\in \rc{2}B_X}{\psi(x)>\rc D}\leq D\int_{\R^n} \left(\|\partial_a(P_t*F)\|_Y * P_{Rt}\right)(x) - \|\partial_a\left(P_{(R + 1)t}*F\right)(x)\|_Y ) dx.
%\txt{vol}\left(x \in \frac{1}{2}B_X : \left(\|\partial_a(P_t*F)\|_Y * P_{Rt}\right)(x) - \|\partial_a\left(P_{(R + 1)t}*F\right)(x)\|_Y  > 1/D\right)
\]
%So for every $x$, the points such that the measure is bigger than $1/D$ is bounded by 
%\[
%\leq D\int_{R^n} \left(\|\partial_a(P_t*F)\|_Y * P_{Rt}\right)(x) - \|\partial_a\left(P_{(R + 1)t}*F\right)(x)\|_Y ) dx
%\]
because we can upper bound the integral over the ball by an integral over $\R^n$. 
Note we are using the probabilistic method. %, since we are only proving the existence of such a point.

This inequality was for a fixed $a$. We now use the union bound over $\cal F$, a $\de$-net of $a$'s to get
%We want to use the union bound. Previously we were doing things for fixed $a$, now we want to do things for every $a$. Let's call
%\[
%\psi(x) = \left(\|\partial_a(P_t*F)\|_Y * P_{Rt}\right)(x) - \|\partial_a\left(P_{(R + 1)t}*F\right)(x)\|_Y 
%\]
%Then rewriting
\begin{align}
&\quad \txtn{vol}\set{x \in \frac{1}{2}B_X}{\exists a \in \mathcal{F}, \psi_a(x) > 1/D}\\ &\leq \sum_{a \in \mathcal{F}} \txtn{vol}(x \in \frac{1}{2}B_X: \psi_a(x) > 1/D)\\
&\leq D\sum_{a \in \mathcal{F}} \int_{\R^n}\left(\|\partial_a(P_t*F)\|_Y * P_{Rt})(x) - \|\partial_a(P_{(R + 1)t} * F)(x)\|_Y\right) dx\\
&=D\sum_{a \in \mathcal{F}} \int_{\R^n}\left(\|\partial_a(P_t*F)\|_Y)(x) - \|\partial_a(P_{(R + 1)t} * F)(x)\|_Y\right) dx \label{eq:bdt-convo}\\
&\le \fc{C_1\vol(3B_X)}{m} (C_3D)^n < \vol\pa{\rc2B_X}.\label{eq:bdt-vol}
\end{align}
where~\eqref{eq:bdt-convo} follows because when you convolve something with a probability measure, the integral over $\R^n$ does not change, and~\eqref{eq:bdt-vol} follows from~\eqref{eq:bdt-pf-lem2} and our choice of $m$.

%We know that when you average pointwise, the first term in the subtraction is bigger than $1/D$. 

%Convoling with a probability measure does not change anything, so we can get rid of $P_{Rt}$. Then, 
%\[
%= D\sum_{a \in \mathcal{F}} \int_{\R^n}\left(\|\partial_a(P_t*F)\|_Y)(x) - \|\partial_a(P_{(R + 1)t} * F)(x)\|_Y\right) dx \leq \frac{8\txtn{vol}(3B_X)}{m}(10D)^n < \txtn{vol}(\frac{1}{2}B_X)
%\]
%by our choice of $m$ and since stabilization made this less than one. 

%This is very technical, but we 
We've proved that there must exist a point in half the ball such that for every $a \in \mathcal{F}$, our net, we have
\[
\frac{1}{D} \geq \|\partial_a(P_t *F)\|_{Y} * P_{Rt}(x) - \|\partial_a(P_{(R + 1)t}*F)(x)\|_Y
\]
\fixme{I think we want either this to be $\rc{2D}$, or~\eqref{eq:bdt-pf-lem1} to be $\rc{D/2}$, in order to get the following bound (with $\rc{2D}$ or $\rc{D}$). This involves changing some constants in the proof.}

%So we got that the for the \emph{average}, the $P_{Rt}$ goes away. 
From this we can conclude that for this $x$,
\[
\ve{Ta}_Y=\|\partial_a(P_{(R + 1)t} * F)(x)\|_Y \ge 1/D
\]
where $a\in S_X$ we let $t_0=(R+1)t$. This shows $\ve{T^{-1}}\le D$.
\end{proof}
For the exact constants, see the paper \fixme{(reference)}.
%You should just look at what the exact constants are in the paper. 

%%NOTE: can include the following once it's fixed

%The crucial point was that we got something to stabilize using the pidgeonhole argument, and you got a difference where one of the terms is pointwise big, which causes the term you're subtracting from it to also be pointwise big. So now we choose $T$ a linear mapping to be
%\[
%T := (P_{(R + 1)t} * F)'(x)
%\]
%where $x$ is the special point from the probabilistic argument. Now we can check what happens. Now we know $\|T\| \leq 8$, and for every $a \in \mathcal{F}$ in the net, $\|Ta\| \ge b/D$ for some constant $b$. 
%The main key here is that we need to show for \textbf{every} point in the net that $\|\partial_a (P_t * F)\|_Y$ is good. You already get that this is big for one particular direction just from $\|\partial_a (P_t * F)\|_Y * P_{Rt}(x) \ge 1/D$. But there are exponentially many points in the net which we need to show this for. 
%
%Take any $z \in S_X$, find $a \in \mathcal{F}$ s.t. $\|a - z\| < \frac{1}{100D}$. Then $\|Tz\| \geq \|Ta\| - \|T(a - z)\| \geq \|Ta\| - \|T\|\cdot\|az\| > \frac{b}{D} - 8\|a - t\| \geq 1/D$.  Then $\|T^{-1}\| \leq D$. \fixme{Fix this explanation a bit more? A bit confusing}

Note this is very much a probabiblistic existence statement or result. %Now you write down what it means. Suppose for contradiction that the probability is not small enough. The reason you succeeded to 
%We bounded the probability using a proof by contradiction. 
Usually we estimate by hand the random variable we want to care about. Here we want to prove an existential bound, so we estimate the probability of the bad case. But we estimate the probability of the bad case using another proof by contradiction.
%, and we assume that it doesn't work, so estimating probability here is also by contradiction! 

It remains to estimate some integrals. 

%\step{2} Extend $F_1$ to the whole space to $F_2$ such that 
%\begin{enumerate}
%\item
%$\forall x\in \cal N_\de$, $\ve{F_2(x)-f(x)}_Y \lesssim L\de$.
%\item
%$\ve{F_2(x)-F_2(y)}_Y\lesssim L(\ve{x-y}_X + \de)$.
%%not smooth yet. 
%\item
%$\Supp(F_2)\subeq 2B_X$.
%\item
%$F_2$ is smooth.
%%F_1 had no bounded continuity, but is a sum against a partition of unity. 
%%just smooth without any bounds fine.
%%spiky.
%%when I say not smooth, I mean no bounds.
%%I need the norm to be smooth for this.
%\end{enumerate}•%

%Denote $\al(t)=\max\{1-|1-t|,0\}$. %

%\ig{images/9-1}{.25}%

%Let
%\[
%F_2(x)=\al(\ve{x}_X) F_1\pa{x}{\ve{x}_X}.
%\]
%%0 the moment it passes 2.
%$F_2$ still satisfies condition 1. As for condition 2, 
%\bal
%\ve{F_2(x)-F_2(y)}_Y &= \ve{\al (\ve{x}_X)F_1\pf{x}{\ve{x}_X} - \al(\ve{y}_X) F_1\pf{y}{\ve{y}_X}} \\
%&\le |\al(\ve{x})-\al(\ve{y})|\ub{\ve{F_1\pf{x}{\ve{x}_X}}}{\le 2L}+\al(\ve{y})\ve{F_1\pf{x}{\ve{x}_X} - F_1\pf{y}{\ve{y}_X} }\\
%&\le (\ve{x}-\ve{y})2L + \al(\ve{y}) L\pa{
%\ve{\nv{x}-\nv{y}}+4\de 
%} \\
%&\le 2L\ve{x-y}+L\al(\ve{y}) \pa{\ve{x}\ab{\rc{\ve{x}}-\rc{\ve{y}}} + \fc{\ve{x-y}}{\ve{y}} + 4\de}\\
%&\le 2L\ve{x-y} +L\al(\ve{y}) \pa{\fc{\ve{x-y}}{\ve{y}} + \fc{\ve{x-y}}{\ve y}  + 4\de}\\
%&\lesssim L(\ve{x-y}+\de),
%%mult by 4de, use bounded by 1
%\end{align*}
%where in the last step we used $\al(\ve{y})\le \ve{y}$ and $\al(\ve{y})\le 1$. %

%Note $F_2$ is smooth because the sum for $F_1$ was against a partition of unity and $\ved_X$ is smooth, although we don't have uniform bounds on smoothness for $F_2$.
%%F_1 had no bounded continuity, but is a sum against a partition of unity. 
%%just smooth without any bounds fine.
%%spiky.
%%when I say not smooth, I mean no bounds.
%%I need the norm to be smooth for this.%

%%For the next step we need the following. %

%\step{3} We make $F$ smoother by convolving.
%\begin{lem}[Begun, 1999]
%Let $F_2:X\to Y$ satisfy $\ve{F_2(x)-F_2(y)}_Y\le L(\ve{x-y}_X+\de)$. Let $\tau \ge c\de$. Define 
%\[
%F(x) = \rc{\Vol(\tau B_X)}\int_{\tau B_X} F_2(x+y)\,dy.
%\]
%Then 
%\[
%\ve{F}_{\text{Lip}} \le L\pa{1+\fc{\de n}{2\tau}}.
%\]
%\end{lem}
%The lemma proves the almost extension theorem as follows. We passed from $f:\cal N_\de\to Y$ to $F_1$ to $F_2$ to $F$. 
%If $x\in \cal N_\de$, 
%\bal
%\ve{F(x)-f(x)}_Y &=\ve{
%\rc{\Vol(\tau B_X)} \int_{B_X} (F_2(x+y) - f(x))\,dy
%}\\
%&\le \rc{\Vol(\tau B_X)}\int_{\tau B_X}\ve{F_2(x+y)-F_2(x)}_Y + \ub{\ve{F_2(x)-f(x)}_Y}{\de L} \dy\\
%&\le \rc{\Vol(\tau B_X)}\int_{\tau B_X}(L(\ub{\ve{y}_X}{\le\tau}+\de L)) \dy\lesssim L\tau.
%\end{align*}
%Now we prove the lemma. 
%\begin{proof}
%We need to show
%\[
%\ve{F(x)-F(y)}_Y \le L\pa{1+\fc{\de n}{2\tau}} \ve{x-y}_X.
%\]
%\Wog $y=0$, $\Vol(\tau B_X)=1$. Denote 
%\bal
%M&=\tau B_{X}\bs (x+\tau B_X)\\
%M'&=(x+\tau B_X) \bs \tau B_X.
%\end{align*}%

%\ig{images/9-2}{.25}%

%We have
%\bal
%F(0)-F(x) &= \int_M F_z(y)\,dy - \int_{M'} F_z(y)\,dy.
%\end{align*}
%Define $\om(z)$ to be the Euclidean length of the interval $(z+\R x)\cap (\tau B_X)$. By Fubini,
%\[
%\int_{\Proj_{X^{\perp}} (\tau B_X)} \om(z) \,dz = \Vol_n(\tau B_X)=1.
%%intersection of projection.
%\]
%Denote
%\bal
%W&= \set{z\in \tau B_X}{(z+\R x)\cap (\tau B_X)\cap (x+\tau B_X)\ne \phi}\\
%N&= \tau B_X\bs W.
%\end{align*}
%Define $C:M\to M'$ a shift in direction $X$ on every fiber that maps the interval $(z+\R x)\cap M\to (z+\R x)\cap M'$. %

%\ig{images/9-3}{.25}%

%$C$ is a measure preserving transformation with
%\[
%\ve{z-C(z)}_X =\begin{cases}
%\ve{x}_X , &z\le N\\
%\om(z) \fc{\ve{x}_X}{\ve{x}_2},& z\in W\cap M.
%\end{cases}
%\]
%(In the second case we translate by an extra factor $\fc{\om(z)}{\ve{x}_2}$.)
%%(In the second case we add the total length $
%%C maps $M'$ to $M$.
%%do a clever change of variable differently in each fiber.
%Then 
%\bal
%\ve{F(0)-F(x)}_Y &=\ve{\int_M F_2(y)\dy - \int_{M'}F_2(y)\dy}_Y\\
%&= \ve{\int_M(F_2(y) - F_2(C(y)))\dy}_Y\\
%&\le \int_M L(\ve{y-C(y)}_X+\de)\dy\\
%&\le \int_M L (\ve{y-C(y)}_X + \de)\dy\\
%&=L\de \Vol(M) + L \int_M \ve{y-C(y)}_X\dy\\
%\int_M \ve{y-C(y)}_X\dy 
%%orth decomp but not unit vector, integrate the length multiply by norm of direction. jacobian.
%&=\int_{N}\ve{x}_X\dy + \int_{W\cap M}\fc{\om(y)\ve{x}_X}{\ve{x}_2}\dy\\
%&=\ve{x}_X \Vol(N) + \int_{\Proj(W\cap M)} \fc{\om(z) \ve{x}_X}{\ve{x}_2} \ve{x}_2\,dz&\text{orthogonal decomposition}\\
%&=\ve{x}_X \Vol(N) + \Vol(\tau B_X\bs N) \\
%&=\ve{x}_X \Vol(\tau B_X)=\ve{x}_X.
%\end{align*}
%%w as the entire length. What I get is the entire volume. 
%We show $M=\tau B_X\bs (x+\tau B_X) \subeq \tau B_X\bs (1-\fc{\ve{x}}{\tau}) \tau B_X$. Indeed, for $y\in M$,
%\bal
%\ve{y-x}_X&\ge \tau\\
%\ve{y} & \ge \tau - \ve{x} = \pa{1-\fc{\ve{x}}{\tau}}\tau.
%\end{align*}
%\end{proof}
%Then 
%\[
%\Vol(M) \le \Vol(\tau B_X)-\Vol\pa{\pa{1-\fc{\ve{x}}{\tau}}\tau B_X}=1-\pa{1-\fc{\ve{x}}{\tau}} \lesssim \fc{n\ve{x}}{\tau}
%\]
%\end{proof}
%Bourgain did it in a more complicated, analytic way avoiding geometry. Begun notices that careful geometry is sufficient.%
%

%Later we will show this theorem is sharp.%

%%iteration!%

%\section{Proof of Bourgain's discretization theorem}%

%At small distances there is no guarantee on the function $f$. Just taking derivatives is dangerous. It might be true that we can work with the initial function. But the only way Bourgain figured out how to prove the theorem was to make a 1-parameter family of functions.%

%\subsection{The Poisson semigroup}
%\begin{df}
%The \ivocab{Poisson kernel} is $P_t(x):\R^n\to \R$ given by
%\[
%P_t(x)=\fc{C_nt}{(t^2+\ve{x}_2^2)^{\fc{n+1}2}}, \quad C_n=\fc{\Ga\pf{n+1}2}{\pi^{\fc{n+1}2}}.
%\]
%%Convolution becomes product under FT
%\end{df}%

%\begin{pr}[Properties of Poisson kernel]
%\begin{enumerate}
%\item
%For all $t>0$, $\int_{\R^n} P_t(x)\,dx=1$.
%\item
%(Semigroup property) $P_t*P_s=P_{t+s}$. 
%\item
%$\wh{P_t}(x) = e^{-2\pi\ve{x}_2t}$.
%\end{enumerate}•
%\end{pr}%

%\begin{lem}
%Let $F$ be the function obtained from Bourgain's almost extension theorem~\ref{thm:baet}.
%For all $t>0$, $\ve{P_t*F}_{\text{Lip}}\lesssim 1$.
%%P_t is prob measure.
%%Average values of $F$.
%%poU, geometr y of ball, average with decaying weights.
%%all averaging of Lipschitz things.%

%%1+\ep subtlety, ceases to be true, need restriction on $T$. Not relevant. $1+\ep$ version do more carefully.
%\end{lem}
%%Note to get $P_t*F$ we had three averaging arguments: partition of unity, averaging with respect to a ball, and then averaging with decaying weights.
%We have 
%\bal
%P_t*F(x)-P_t*F(y) &= \int_{\R^n} P_t(z)(F(x-z) - F(x-y))\,dz.
%\end{align*}
%Our goal is to show there exists $t_0>0$, $x\in B\in \rc B_X$ such that if we define
%\[
%T=(P_{t_0}*F)'(x):X\to Y,
%\]
%we have $\ve{T}\lesssim 1$. Moreover $\ve{T^{-1}}\lesssim D$.
%$T_y=\lim_{h\to \iy} \fc{P_{t_0}*F(x+hy) - P_{t_0}*F(x)}{h}$.
%%pigeonhole, must exist
\blu{3-7: Finishing Bourgain. There's a remaining lemma about the Poisson semigroup that we're going to do today (Lemma~\ref{lem:bdt-1}); it's nice, but it's not as nice as what came before.}

%TODO: Put in a summary of the proof. (This is messy right now.)
%Let me try to paraphrase the high level strategy before. What we did last time: We had a $\delta$-net in $X$, a function $f$ going into $Y$. Then we almost extended $f$ to $F$. The idea was to take the derivative of $(P_t * F)(x)$ and hope this is big. It's already bounded by the Lipschitz constant. What does it mean that this derivative is big? It means that the norm in $Y$ is at least a constant times the norm of $x$, for every $x$. 
%\[
%\left\|(P_t * F)'(x)\right\|_Y \ge \|x\|_X
%\]
%If you're invertible; if you have this inequality for a net in the sphere, you have the inequality everywhere. If you know the inequality on the net, you know it globally only if the $\delta$ of the net is smaller than the norm of the operators. So you fix this net: $\mc{F} \subseteq S_X$, a $c/D$-net in $S_X$.  
%Then for all $a \in \mc{F}$
%\[
%\left\|(P_t * F)'(x)(a)\right\|_Y = \left\| \partial_a(P_t * F)(x)\right\|_Y
%\]
%Let $\nu$ be the normalized volume on $\frac{1}{2}B_X$. It's enough to show that the measure 
%\[
%\nu\left(x \in (1/2)B_X: \forall a \in \mc{F}, \|\partial_a(P_t*F)(x)\|_Y \ge \frac{c}{\|x\|_X}\right)
%\]
%This is the sum in the net of this measure. 
%And you can directly write down 
%\[
%1 - \sum_{a \in \mc{F}} \nu(x \in (1/2)B_X: \|\partial_a(P_t*F)(x)\|_Y < \frac{c}{\|x\|_X})
%\]
%Then you want to see how you can make this probability zero. 
%
%Bourgain wants to make a probabilistic argument; he says investigate the evolution of the probability over the semigroup: It depends on time $t$. And there must exist a time for which this is $< 1$. This is a semigroup argument combined with probability, and what you are analyzing as things flow forward in time is how the probability changes. We bound this probability using Markov. Then there's all kinds of lemmas which need to come in (and there's one more which I need to finish today), but then this is essentially Bourgain's theorem. We don't know of another way to prove this today. 

The last remaining lemma to prove is Lemma~\ref{lem:bdt-1}.
%\begin{proof}

Remember that $\frac{1}{\sqrt{n}}\|x\|_2 \leq \|x\|_X \leq \|x\|_2$ for all $x \in X$. 
We need three facts about the Poisson kernel.

%\begin{enumerate}
%\item 
\begin{pr}[Fact 1]\llabel{pr:pois1}
\[
\int_{\R^n \setminus (rB_X)} P_t(x)dx \leq \frac{t\sqrt{n}}{r}.
\]
\end{pr}
\begin{proof}
First note that the Poisson semigroup has the property that \beq{eq:pois-rescale}
P_t(x) = \frac{1}{t^n}P_1(x/t),
\eeq 
i.e., $P_t$ is just a rescaling of $P_1$. So it's enough to prove this for $t = 1$. We have
\begin{align*}
\int_{\|x\|_X \geq r} P_t(x)\dx 
&\leq \int_{\|x\|_2 \geq r} P_t(x) \dx\\
&= \int_{\|x\|_2 \geq r/t} P_1(x)\dx\\
&\quad \text{by change of variables and~\eqref{eq:pois-rescale}}\\
&= C_n S_{n - 1} \int_{r/t}^{\infty} \frac{s^{n - 1}}{(1 + s^2)^{(n + 1)/2}} ds \\
&\quad \text{polar coordinates, where }S_{n-1}=\vol(\bS^{n-1})\\
&\leq C_nS_{n - 1} \int_{r/t}^{\infty} \frac{1}{s^2} ds\\
& = \frac{t}{r}C_nS_{n - 1}\\
&\le \fc{t\sqrt n}r,
\end{align*}
where the last inequality follows from expanding in terms of the Gamma function and using Stirling's formula. 

Above we changed to polar coordinates using
\[
\int_{\ve{x}_2\le R} f(\ve{x})\dx=\int_0^R S_{n-1}\ve{x}^{n-1} f(r)\,dr.
\]
\end{proof}
%\item
\begin{pr}[Fact 2]\label{pr:pois2}
For all $y \in \R^n$, 
$$\int_{\R^n} |P_t(x) - P_t(x + y)| dx \le \frac{\sqrt{n}\|y\|_2}{t}.$$
\end{pr}
\begin{proof}
It's again enough to prove this when $t = 1$. 
\begin{align*}
\int_{\R^n} |P_t(x) - P_t(x + y)| \dx 
&= \int_{\R^n} \left| \int_{0}^1 \langle \nabla P_t(x + sy), y \rangle ds\right| \dx 
\\
&\leq \|y\|_2 \int_{\R^n} \|\nabla P_t(x)\|_2 dx&\text{Cauchy-Schwarz}\\
	&\le 
	\|y\|_2(n + 1)C_n\int_{\R^n} \frac{\|x\|_2}{(1 + \|x\|_2^2)^{\frac{n + 3}{2}}} dx&\text{computing gradient}\\
	&= \|y\|_2 (n + 1)C_n S_{n - 1} \int_{0}^{\infty} \frac{r^n}{(1 + r^2)^{\frac{n + 3}{2}}} \,dr  &\text{polar coordinates}\\
	&\le \sqrt{n}\ve{y}_2.
\end{align*}
The integral and multiplicative constants perfectly cancels out the $n + 1$ and becomes 1 (calculation omitted).
\end{proof}
%\item
\begin{pr}[Fact 3]\label{pr:pois3}
For all $0 < t < \frac{1}{2}$ and $x \in B_X$, we have 
\[
\left\|P_t * F(x) - F(x) \right\|_Y \le \sqrt{n} t\log(3/t).
\]
\end{pr}
\begin{proof}
The LHS equals
\begin{align*}
\left\|\int_{\R^n} \left(F(x - y) - F(x)\right)P_t(y) dy \right\|_Y &\leq 
\ub{\int_{x + 3B_X} \left\| F(x - y) - F(x)\right\|_Y P_t(y) dy}{\stepcounter{equation}\mbox{(\theequation)}} + 
\ub{c\int_{\R^n \setminus (x + 3B_X)} P_t(y) dy}{\stepcounter{equation}\mbox{(\theequation)}}.
\end{align*}
\addtocounter{equation}{-2}\refstepcounter{equation}\label{eq:pois3-1}
\refstepcounter{equation}\label{eq:pois3-2}

Using $\left\|F(x - y) - F(x)\right\|_Y \leq \|F\|_{\text{Lip}}\frac{c}{\rho}\cdot \|y\|_X$ and that $\ve{F}_{\text{Lip}}$ is a constant,
\bal
\eqref{eq:pois3-1} &=  \int_{x + 3B_X} \|y\|_XP_t(y)dy\\
&\le \int_{4\sqrt{n}B_{l_2^n}} \|y\|_2 P_t(y) dy&\text{using }x\in B_X\implies x + 3B_X\subeq 4\sqrt n B_{l_2}\\
&= tC_nS_{n - 1} \int_{0}^{\frac{4\sqrt{n}}{t}} \frac{s^n}{(1 + s^2)^{(n + 1)/2}}\,ds&\text{polar coordinates}\\
&\le t\sqrt{n} \pa{\int_0^{\sqrt{n}} \frac{1}{\sqrt{n}}\,ds + \int_{\sqrt{n}}^{4\sqrt{n}/t} \rc{s}ds}\\
&\le t\sqrt n \pa{1+\ln \pf{4}{t}},
\end{align*}
where we used the fact that $\fc{s^n}{(1+s^2)^{\fc{n+1}{2}}}$ is maximized when $s = \sqrt{n}$, and is always $\leq \rc s$.

For the second term, note that $2B_X\subeq x+3B_X$, so 
\[
\eqref{eq:pois3-2} =c\int_{\R^n\bs 2B_X}P_t(y)\le \fc{ct\sqrt n}{2}
\]
where we used the tail bound of Fact 1 (Pr. \ref{pr:pois2}). Adding these two bounds gives the result.
\end{proof}
%\end{enumerate}

Now we have these three facts, we can finish the proof of the lemma. 

\begin{proof}[Proof of Lemma~\ref{lem:bdt-1}]
\begin{clm}[$P_t*F$ separates points]
Let $\theta = cD\sqrt{n} t\log(3/t)$. 
%All of this is about directional derivatives which are big. So let's see how $P_t$ separates points. Suppose I take points 
Let $w, y \in \frac{1}{2}B_X$ with $\ve{w-y}_X\ge \te$. Then
\[
\left\|P_t * F(w) - P_t*F(y)\right\|_Y \geq 1/D
\]
\end{clm}
We do not claim at all that these numbers are sharp. 
\begin{proof}
We can find a point $p\in \cal N_\de$ which is $\delta$ away from $w$, and $q \in \cal N_{\de}$ which is $\delta$ away from $y$. We also know that $F$ is close to $f$, and we can use Fact 3 (Pr. \ref{pr:pois3}) to bound the error of convolving. 

\ig{images/11-3}{.25}

By the triangle inequality 
\begin{align*}
\left\|P_t * F(w) - P_t*F(y)\right\|_Y 
&\geq \|f(p) - f(q)\| - \|F(p) - f(p)\| - \|F(q) - f(q)\| \\
&\quad - \|F(w) - F(p)\| - \|F(y) - F(q)\| \\
&\quad- \|P_t*F(w) - F(w)\| - \|P_t*F(y) - F(y)\|\\
&\ge \frac{\|p - q\|}{D} - 2n\delta - c\delta - \sqrt{n}t\log(3/t)\\ &\geq \frac{\|y - w\| - 2\delta}{D} - 2n\delta - c\delta - \sqrt{n}t\log(3/t) .
\end{align*}
Taking $$\theta = cD\sqrt{n}t\log(3/t),$$ we find this is at least a constant times $\|y - w\|/D$, we need $\theta = cD\sqrt{n}t\log(3/t)$. Note that $\sqrt{n}t\log(3/t)$ is the largest error term because by the assumptions in the lemma, it is greater than $2n\delta$. 
\end{proof}
Now consider
\[
\left\|P_t * F(z + \theta a) - P_t * F(z)\right\|
\]
By the claim, if $z \in \frac{1}{4}B_X$, then we know 
\begin{align}
\theta/D &\leq \left\|P_t * F(z + \theta a) - P_t* F(z)\right\|\\
&= \int_{0}^{\theta} \left\|\partial_a(P_t*F)(z + sa)\right\|_Y 
\label{eq:bdt-int}
\end{align}
Suppose $\ve{x-y}\le \rc4$ (we will apply~\eqref{eq:bdt-int} to $x-y=z$), $x\in \rc2 B_X$, $y\in \rc8 B_X$. By throwing away part of the integral,
\begin{align}
&\quad \frac{1}{\theta}\int_0^{\te} \int_{\R^n} \left\| \partial_a(P_t * F)(x + sa - y)\right\|_Y P_{Rt}(y) dy\\
&\ge
\frac{1}{\theta} \int_0^{\theta} \int_{(1/8)B_X} \left\|\partial_a(P_t * F)(x - y + sa)\right\|_Y P_{Rt}(y) \,ds \dy\\
& = 
%So if $x$ is in half of the ball and $y$ is $1/8$ of the ball, thi sis fine. Now you can use Fubini to switch the order of integrals: 
 \int_{(1/8)B_X} \left(\frac{1}{\theta}\int_0^{\theta} \left\|\partial_a(P_t * F)(x - y + sa)\right\|_Y ds \right) P_{Rt}(y) \dy&\text{by Fubini}\\
&\ge \frac{1}{D} \int_{(1/8)B_X} P_{Rt}(y)dy &\text{by }\eqref{eq:bdt-int}\\
&\ge \fc cD
\label{eq:bdt-f1}
\end{align}
for some constant $c$.
%which is bounded below. 
%So our statement amounts to saying the 
The calculation above says that the average length in every direction is big, and if you average the average length again, it is still big.
How do we get rid of the additional averaging? We care about 
\begin{align*}
\|\partial_a(P_t * F)\| * P_{Rt}(x) 
&= \int_{\R^n} \|\partial_a(P_t *F)(x - y)\|_Y P_{Rt}(y) dy\\
%Then shifting, we can add $ + sa$ for every $s$, then add a correction term, and now we can integrate over $s \in [0, \theta]$ and average: 
&=
\frac{1}{\theta} \int_0^{\theta} \left(\int_{\R^n} \|\partial_a(P_t *F)(x - y + sa)\|_Y P_{Rt}(y) dy \right)\\
&\quad  - \frac{1}{\theta} \int_0^{\theta} \int_{\R^n} \|\partial_a(P_t *F)(x - y)\|_Y (P_{Rt}(y + sa) - P_{Rt}(y)) dy \\
&> \frac{c}{D} - \frac{c'}{\theta} \int_0^{\theta} \int_{\R^n} | P_{Rt}(y + sa) - P_{Rt}(y)| ds \dy&\text{by \eqref{eq:bdt-f1} and \fixme{what?}}\\
&\geq \frac{c}{D} - \frac{c'}{\theta} \int_0^{\theta} \frac{\sqrt{n}s\|a\|_2}{Rt} ds&\text{by Fact 2 (Pr. \ref{pr:pois2})}
\end{align*}
This error term is
$$
\frac{c'}{\theta} \int_0^{\theta} \frac{\sqrt{n}s\|a\|_2}{Rt} ds
 = \fc{cDn\log \pf{3}{t}}R = O\prc{D}$$
by~\eqref{eq:bdt1-1}, so $\|\partial_a(P_t * F)\| * P_{Rt}(x)=\Om\prc{D}$  (if we chose constants correctly), as desired. \fixme{Factor of $\sqrt n$?}
%\frac{c'\theta n}{2Rt}$, and $\frac{\theta n}{Rt} < \frac{c}{D}$ so we are fine. 
\end{proof}

\fixme{To summarize: You compute the error such that you have distance $\theta$. Then you look at the average derivative on each line like this. Then, the average derivative is roughly $\|\partial_a(P_t * F)\| * P_{Rt}(x)$. So averaging the derivative in a bigger ball is the same up to constants as taking a random starting point and averaging over the ball. What's written in the lemma is that the derivative of a given point in this direction, averaged over a little bit bigger ball, it's not unreasonable to see that at first take starting point and going in direction $\theta$ and averaging in the bigger ball is equivalent to randomizing over starting points. }
%\end{proof}

%We will prove an easy fact next time: If you look at $L_1$, and change the metric to $\sqrt{\|x - y\|_1}$, this is isometric to a subset of $L_2$. %This is what is called $L_1$ is a squared $L_2$ metric. If you take this embedding into $L_2$, let's think how Lipschitz the square-root mapping is. 
%With this embedding, you don't distortion distances by more than a factor $1/\sqrt{\delta}$ on the net, even though the distortion on the whole space is bigger. (This is just a sketch.)
%
%To get better results, you need more exotic spaces. 

%\step{2} Extend $F_1$ to the whole space to $F_2$ such that 
%\begin{enumerate}
%\item
%$\forall x\in \cal N_\de$, $\ve{F_2(x)-f(x)}_Y \lesssim L\de$.
%\item
%$\ve{F_2(x)-F_2(y)}_Y\lesssim L(\ve{x-y}_X + \de)$.
%%not smooth yet. 
%\item
%$\Supp(F_2)\subeq 2B_X$.
%\item
%$F_2$ is smooth.
%%F_1 had no bounded continuity, but is a sum against a partition of unity. 
%%just smooth without any bounds fine.
%%spiky.
%%when I say not smooth, I mean no bounds.
%%I need the norm to be smooth for this.
%\end{enumerate}•%

%Denote $\al(t)=\max\{1-|1-t|,0\}$. %

%\ig{images/9-1}{.25}%

%Let
%\[
%F_2(x)=\al(\ve{x}_X) F_1\pa{x}{\ve{x}_X}.
%\]
%%0 the moment it passes 2.
%$F_2$ still satisfies condition 1. As for condition 2, 
%\bal
%\ve{F_2(x)-F_2(y)}_Y &= \ve{\al (\ve{x}_X)F_1\pf{x}{\ve{x}_X} - \al(\ve{y}_X) F_1\pf{y}{\ve{y}_X}} \\
%&\le |\al(\ve{x})-\al(\ve{y})|\ub{\ve{F_1\pf{x}{\ve{x}_X}}}{\le 2L}+\al(\ve{y})\ve{F_1\pf{x}{\ve{x}_X} - F_1\pf{y}{\ve{y}_X} }\\
%&\le (\ve{x}-\ve{y})2L + \al(\ve{y}) L\pa{
%\ve{\nv{x}-\nv{y}}+4\de 
%} \\
%&\le 2L\ve{x-y}+L\al(\ve{y}) \pa{\ve{x}\ab{\rc{\ve{x}}-\rc{\ve{y}}} + \fc{\ve{x-y}}{\ve{y}} + 4\de}\\
%&\le 2L\ve{x-y} +L\al(\ve{y}) \pa{\fc{\ve{x-y}}{\ve{y}} + \fc{\ve{x-y}}{\ve y}  + 4\de}\\
%&\lesssim L(\ve{x-y}+\de),
%%mult by 4de, use bounded by 1
%\end{align*}
%where in the last step we used $\al(\ve{y})\le \ve{y}$ and $\al(\ve{y})\le 1$. %

%Note $F_2$ is smooth because the sum for $F_1$ was against a partition of unity and $\ved_X$ is smooth, although we don't have uniform bounds on smoothness for $F_2$.
%%F_1 had no bounded continuity, but is a sum against a partition of unity. 
%%just smooth without any bounds fine.
%%spiky.
%%when I say not smooth, I mean no bounds.
%%I need the norm to be smooth for this.%

%%For the next step we need the following. %

%\step{3} We make $F$ smoother by convolving.
%\begin{lem}[Begun, 1999]
%Let $F_2:X\to Y$ satisfy $\ve{F_2(x)-F_2(y)}_Y\le L(\ve{x-y}_X+\de)$. Let $\tau \ge c\de$. Define 
%\[
%F(x) = \rc{\Vol(\tau B_X)}\int_{\tau B_X} F_2(x+y)\,dy.
%\]
%Then 
%\[
%\ve{F}_{\text{Lip}} \le L\pa{1+\fc{\de n}{2\tau}}.
%\]
%\end{lem}
%The lemma proves the almost extension theorem as follows. We passed from $f:\cal N_\de\to Y$ to $F_1$ to $F_2$ to $F$. 
%If $x\in \cal N_\de$, 
%\bal
%\ve{F(x)-f(x)}_Y &=\ve{
%\rc{\Vol(\tau B_X)} \int_{B_X} (F_2(x+y) - f(x))\,dy
%}\\
%&\le \rc{\Vol(\tau B_X)}\int_{\tau B_X}\ve{F_2(x+y)-F_2(x)}_Y + \ub{\ve{F_2(x)-f(x)}_Y}{\de L} \dy\\
%&\le \rc{\Vol(\tau B_X)}\int_{\tau B_X}(L(\ub{\ve{y}_X}{\le\tau}+\de L)) \dy\lesssim L\tau.
%\end{align*}
%Now we prove the lemma. 
%\begin{proof}
%We need to show
%\[
%\ve{F(x)-F(y)}_Y \le L\pa{1+\fc{\de n}{2\tau}} \ve{x-y}_X.
%\]
%\Wog $y=0$, $\Vol(\tau B_X)=1$. Denote 
%\bal
%M&=\tau B_{X}\bs (x+\tau B_X)\\
%M'&=(x+\tau B_X) \bs \tau B_X.
%\end{align*}%

%\ig{images/9-2}{.25}%

%We have
%\bal
%F(0)-F(x) &= \int_M F_z(y)\,dy - \int_{M'} F_z(y)\,dy.
%\end{align*}
%Define $\om(z)$ to be the Euclidean length of the interval $(z+\R x)\cap (\tau B_X)$. By Fubini,
%\[
%\int_{\Proj_{X^{\perp}} (\tau B_X)} \om(z) \,dz = \Vol_n(\tau B_X)=1.
%%intersection of projection.
%\]
%Denote
%\bal
%W&= \set{z\in \tau B_X}{(z+\R x)\cap (\tau B_X)\cap (x+\tau B_X)\ne \phi}\\
%N&= \tau B_X\bs W.
%\end{align*}
%Define $C:M\to M'$ a shift in direction $X$ on every fiber that maps the interval $(z+\R x)\cap M\to (z+\R x)\cap M'$. %

%\ig{images/9-3}{.25}%

%$C$ is a measure preserving transformation with
%\[
%\ve{z-C(z)}_X =\begin{cases}
%\ve{x}_X , &z\le N\\
%\om(z) \fc{\ve{x}_X}{\ve{x}_2},& z\in W\cap M.
%\end{cases}
%\]
%(In the second case we translate by an extra factor $\fc{\om(z)}{\ve{x}_2}$.)
%%(In the second case we add the total length $
%%C maps $M'$ to $M$.
%%do a clever change of variable differently in each fiber.
%Then 
%\bal
%\ve{F(0)-F(x)}_Y &=\ve{\int_M F_2(y)\dy - \int_{M'}F_2(y)\dy}_Y\\
%&= \ve{\int_M(F_2(y) - F_2(C(y)))\dy}_Y\\
%&\le \int_M L(\ve{y-C(y)}_X+\de)\dy\\
%&\le \int_M L (\ve{y-C(y)}_X + \de)\dy\\
%&=L\de \Vol(M) + L \int_M \ve{y-C(y)}_X\dy\\
%\int_M \ve{y-C(y)}_X\dy 
%%orth decomp but not unit vector, integrate the length multiply by norm of direction. jacobian.
%&=\int_{N}\ve{x}_X\dy + \int_{W\cap M}\fc{\om(y)\ve{x}_X}{\ve{x}_2}\dy\\
%&=\ve{x}_X \Vol(N) + \int_{\Proj(W\cap M)} \fc{\om(z) \ve{x}_X}{\ve{x}_2} \ve{x}_2\,dz&\text{orthogonal decomposition}\\
%&=\ve{x}_X \Vol(N) + \Vol(\tau B_X\bs N) \\
%&=\ve{x}_X \Vol(\tau B_X)=\ve{x}_X.
%\end{align*}
%%w as the entire length. What I get is the entire volume. 
%We show $M=\tau B_X\bs (x+\tau B_X) \subeq \tau B_X\bs (1-\fc{\ve{x}}{\tau}) \tau B_X$. Indeed, for $y\in M$,
%\bal
%\ve{y-x}_X&\ge \tau\\
%\ve{y} & \ge \tau - \ve{x} = \pa{1-\fc{\ve{x}}{\tau}}\tau.
%\end{align*}
%\end{proof}
%Then 
%\[
%\Vol(M) \le \Vol(\tau B_X)-\Vol\pa{\pa{1-\fc{\ve{x}}{\tau}}\tau B_X}=1-\pa{1-\fc{\ve{x}}{\tau}} \lesssim \fc{n\ve{x}}{\tau}
%\]
%\end{proof}
%Bourgain did it in a more complicated, analytic way avoiding geometry. Begun notices that careful geometry is sufficient.%
%

%Later we will show this theorem is sharp.%

%%iteration!%

%\section{Proof of Bourgain's discretization theorem}%

%At small distances there is no guarantee on the function $f$. Just taking derivatives is dangerous. It might be true that we can work with the initial function. But the only way Bourgain figured out how to prove the theorem was to make a 1-parameter family of functions.%

%\subsection{The Poisson semigroup}
%\begin{df}
%The \ivocab{Poisson kernel} is $P_t(x):\R^n\to \R$ given by
%\[
%P_t(x)=\fc{C_nt}{(t^2+\ve{x}_2^2)^{\fc{n+1}2}}, \quad C_n=\fc{\Ga\pf{n+1}2}{\pi^{\fc{n+1}2}}.
%\]
%%Convolution becomes product under FT
%\end{df}%

%\begin{pr}[Properties of Poisson kernel]
%\begin{enumerate}
%\item
%For all $t>0$, $\int_{\R^n} P_t(x)\,dx=1$.
%\item
%(Semigroup property) $P_t*P_s=P_{t+s}$. 
%\item
%$\wh{P_t}(x) = e^{-2\pi\ve{x}_2t}$.
%\end{enumerate}•
%\end{pr}%

%\begin{lem}
%Let $F$ be the function obtained from Bourgain's almost extension theorem~\ref{thm:baet}.
%For all $t>0$, $\ve{P_t*F}_{\text{Lip}}\lesssim 1$.
%%P_t is prob measure.
%%Average values of $F$.
%%poU, geometr y of ball, average with decaying weights.
%%all averaging of Lipschitz things.%

%%1+\ep subtlety, ceases to be true, need restriction on $T$. Not relevant. $1+\ep$ version do more carefully.
%\end{lem}
%%Note to get $P_t*F$ we had three averaging arguments: partition of unity, averaging with respect to a ball, and then averaging with decaying weights.
%We have 
%\bal
%P_t*F(x)-P_t*F(y) &= \int_{\R^n} P_t(z)(F(x-z) - F(x-y))\,dz.
%\end{align*}
%Our goal is to show there exists $t_0>0$, $x\in B\in \rc B_X$ such that if we define
%\[
%T=(P_{t_0}*F)'(x):X\to Y,
%\]
%we have $\ve{T}\lesssim 1$. Moreover $\ve{T^{-1}}\lesssim D$.
%$T_y=\lim_{h\to \iy} \fc{P_{t_0}*F(x+hy) - P_{t_0}*F(x)}{h}$.
%%pigeonhole, must exist
\blu{3-21: A better bound in Bourgain Discretization}

%First I will go back to where we left off last time. When do you need a discretization bound $\de$  going to zero? 

\section{Upper bound on $\de_{X\to Y}$}

We give an example where we need a discretization bound $\de$ going to 0.

%I will explain this point again. There were two ingredients that I stated. My goal from last class is to show something very weak: 
\begin{thm}\label{thm:1/n}
There exist Banach spaces $X, Y$ with dim $X = n$ such that if $\delta < 1$ and $\cal N_\de$ is a $\delta$-net in the unit ball and
%$C_Y(\cal N_\de)$ is the distortion of the embedding into Y such that 
$C_Y(\cal N_\de) \gtrsim C_Y(X)$, then $\delta \lesssim \frac{1}{n}$. 
\end{thm}
Together with the lower bound we proved, this gives
\[
\frac{1}{\rho n^{Cn}} < \delta_{X \to Y}(1/2) < \rc n
\]
The lower bound  would be most interesting to improve. 
%We are mostly interested in changing the lower bound.
 
The example will be $X = \ell_1^n, Y = \ell_2$. %And the two ingredients we proved were first that $(L_1, \sqrt{\|x - y\|_1})$ is a metric space to a subset of $L_2$.

We need two ingredients.

Firstly, we show that $(L_1,\sqrt{\ve{x-y}_1})$ is isometric to a subset of $L_2$. More generally, we prove the following.
\begin{lem}
Given a measure space $(\Omega, \mu)$, 
$(L_1(\mu), \sqrt{\|x - y\|_1})$ is isometric to a subset of $L_2(\Omega \times \R, \mu \times \lambda)$, where $\lambda$ is the Lesbegue measure of $\R$. 
\end{lem}
Note all separable Hilbert spaces are the same, $L_2(\Omega \times \R, \mu \times \lambda)$ is the same whenever it is separable.
%All Hilbert spaces are separable. 
\begin{proof}
First we define $T: L_1(\mu) \to L_2(\mu \times \lambda)$ as follows. For $f \in L_1(\mu)$, let
\[
Tf(w, x) = 
\begin{cases}
1 & 0 \leq f(w) \leq x \\
-1 & x \leq f(x) \leq 0 \\
0 & \txtn{otherwise}
\end{cases}
\]

\ig{images/12-1}{.25}

Consider two functions $f_1, f_2 \in L_1(\mu)$. The function $|Tf_1 - Tf_2|$ is the indicator of the area between the graph of $f_1$ and the graph of $f_2$. 

\ig{images/12-2}{.25}

Thus the $L_2$ norm is, letting  $\one$ be the signed indicator,
\begin{align*}
\|Tf_1 - Tf_2\|_{L_2(\mu \times \lambda)} &= \sqrt{\int_{\Omega}\int_{\R}\left(\one_{f_1(w), f_2(w)}\left(x\right)^2\right)\dx\,d\mu}\\
&=\sqrt{\int_{\Omega} | f_1(w) - f_2(w) | \,d\mu(w)}
\end{align*}
\end{proof}
More generally, if $q \leq p < \infty$, then $(L_q, \|x - y\|_q^{p/q})$ is isometric to a subset of $L_p$. We may prove this later if we need it. 

The second ingredient is due to Enflo .

\begin{thm}[Enflo, $1969$]
The best possible embedding of the hypercube into Hilbert space has distortion $\sqrt{n}$: 
\[
C_2(\{0, 1\}^n, \| \cdot \|_1) = \sqrt{n}
\]
\end{thm}
Note that we don't just calculate $C_2$ up to a constants here; we know it exactly. This is very rare. 
\begin{proof}
We need to show an upper bound and a lower bound. 
\begin{enumerate}
\item
To show the upper bound, we show the identity mapping $\{0, 1\}^n \to \ell_2^n$  has distortion $\sqrt n$. %What can we say about their $\ell_2$ distance? 
We have
\[\|x - y\|_2 = \left(\sum_{i = 1}^n |x_i - y_i|^2 \right)^{1/2} = \left(\sum_{i = 1}^n |x_i - y_i| \right)^{1/2} = \sqrt{\|x - y\|_1},\] 
since the values of $x, y$ are only $0, 1$. Then
\[
\frac{1}{\sqrt{n}} \|x - y\|_1 \leq \|x - y\|_2 = \sqrt{\|x - y\|_1} \leq \|x - y\|_1
\]
Thus $C_2(\{0, 1\}^n) \leq \sqrt{n}$. 

(So this theorem tells us that if you want to represent the boolean hypercube as a Euclidean geometry, nothing does better than the identity mapping.)
\item
For the lower bound $C_2(\{0,1\}^n,\ved_1)\ge\sqrt n$, we use the following.
\begin{lem}[Enflo's inequality]\label{lem:enflo-ineq}
We claim that for every $f: \{0, 1\}^n \to \ell_2$, $$\sum_{x \in \mathbb{F}_2^n}\|f(x) - f(x + e)\|_2^2 \leq \sum_{j = 1}^n \sum_{x \in \mathbb{F}_2^n} \|f(x + e_j) - f(x)\|_2^2,$$ where $e = (1, \ldots, 1)$ and $e_i$ are standard basis vectors of $\R^n$. 

Here, addition is over $\mathbb{F}_2^n$. 
\end{lem}
This is specific to $\ell_2$. See the next subsection for the proof.

In other words, Hilbert space is of Enflo-type $2$ (Definition~\ref{df:enflo}): the sum of the squares of the lengths of the diagonals is at most the sum of the squares of the lengths of the edges. 

Suppose $$\frac{1}{D}\|x - y\|_1 \leq \|f(x) - f(y)\|_2 \leq \|x - y\|_1.$$ Our goal is to show that $D \geq \sqrt{n}$. 
Letting $y=x+e$,
\[\|f(x + e) - f(x)\|_2 \geq \frac{1}{D}\|(x + e) - x\|_1 = \frac{n}{D}.\] 
Plugging into Enflo's inequality~\ref{lem:enflo-ineq}, we have, since there are $2^n$ points in the space,
\begin{align*}
2^n(n/D)^2 \leq n\cdot 2^n \cdot 1^2\
\implies D^2 \geq n
\end{align*}
as desired.
\end{enumerate}
\end{proof}

\begin{proof}[Proof of Theorem~\ref{thm:1/n}]
Let $\cal N_\de$ be a $\delta$-net in $B_{\ell_1^n}$. Let $T: \ell_1^n \to L_2$ satisfy $\|T(x) - T(y)\|_2 = \sqrt{\|x - y\|_1}$. $T$ restricted to $\cal N_\de$ has distortion $\le\sqrt{\frac{2}{\delta}}$ because for $x, y \in \cal N_\de$ with $x \neq y$, 
we have
\[
\frac{1}{\sqrt{2}}\|x - y\|_1 \leq \sqrt{\|x - y\|_1} \leq \frac{1}{\sqrt{\delta}} \|x - y\|_1.
\]
However, the distortion of $\ell_1^n$ in $\ell_2$ is $\sqrt{n}$. The condition $C_Y(\cal N_\de)\gtrsim C_Y(X)$ means that for some constant $K$,  $$\sqrt{\frac{2}{\delta}} \geq K\sqrt{n} \implies \delta \leq \frac{2}{K^2n}.$$
\end{proof}


\subsection{Fourier analysis on the Boolean cube and Enflo's inequality}
%We want to show $C_2(\{0, 1\}^n, \|\cdot\|_1) \leq \sqrt{n}$. 
%(this is addition in boolean hypercube). 

We can prove this by induction on $n$ (exercise). Instead, I will show a proof that is Fourier analytic and which generalizes to other situations.


\begin{df}[Walsh function]
Let $A \subseteq \{1, \ldots, n\}$. Define the Walsh function 
\[W_A: \mathbb{F}_2^n \to \{\pm 1\}\] by
\[W_A(x) = (-1)^{j \in A}.\]
\end{df}

\begin{pr}[Orthonormality]
For $A,B \subseteq \{1, \ldots, n\}$. 
\[\EE_{x \in \mathbb{F}_2^n}W_A(x)W_B(x) = \delta_{AB}.\]
Thus, $\{W_A\}_{A \subseteq \{1, \ldots, n\}}$ is an orthonormal basis of $L_2(\mathbb{F}_2^n)$.
\end{pr}

\begin{proof} 
Without loss of generality $j\in A\bs B$. Then
$$\EE{(-1)^{\sum_{j \in A} x_j + \sum_{j \in B} x_j}} = 0.$$ 
\end{proof}

\begin{cor}
Let $f: \mathbb{F}_2^n \to X$  where $X$ is a Banach space. Then
\[
f = \sum_{A \subseteq \{1, \cdots, n\}} \hat{f}(a)W_A
\]
where $\hat{f}(A) = \frac{1}{2^n}\sum_{x \in \mathbb{F}_2^n} f(x)(-1)^{\sum_{j \in A}x_j}$.
\end{cor}
\begin{proof}
How do we prove two vectors are the same in a Banach space? It is enough to show that any composition with a linear functional gives the same value; thus it suffices to prove the claim for $X=\R$. The case $X=\R$ holds by orthonormality.
\end{proof}
%This does not hold only for reals, it applies to vectors too though the intuition we gave is for the real line. 

\begin{proof}[Proof of Lemma~\ref{lem:enflo-ineq}]
It is sufficient to prove the claim for $X=\R$; we get the inequality for $\ell_2$ by summing coordinates. (Note this is a luxury specific to $\ell_2$.)

The summand of the LHS of the inequality is
\begin{align*}
f(x) - f(x + e) &= \sum_{A \subseteq \{1, \ldots, n\}} \hat{f}(A)\left(W_A(x) - W_A(x + e)\right)\\
&=\sum_{A \subseteq \{1, \ldots, n\}} \hat{f}(A)W_A(x)\left(1 - (-1)^{|A|}\right) \\
&= \sum_{A \subseteq \{1, \ldots, n\}, |A| \text{ odd}} 2\hat{f}(A)W_A(x)
\end{align*}
The summand of the RHS of the inequality is
\begin{align*}
-f(x + e_j) + f(x) &= \sum_{A \subseteq \{1, \ldots, n\}} \hat{f}(A)(W_A(x) - W_A(x + e_j))\\
& = \sum_{A \subseteq \{1, \ldots, n\}, j \in A} 2\hat{f}(A)W_A(x).
\end{align*}
Summing gives
\begin{align*}
\sum_{x \in \mathbb{F}_2^n}\left(f(x) - f(x + e)\right)^2 &= 2^n \left\|\sum_{|A| \txtn{ odd}} 2\hat{f}(A)W_A\right\|_{L_2(\mathbb{F}_2^n)}^2 \\
&= 2^n \sum_{|A| \txtn{ odd}} 4(\hat{f}(A))^2\\
\sum_{j = 1}^n \sum_{x \in \mathbb{F}_2^n} \left(f(x + e_j) - f(x)\right)^2 &= \sum_{j = 1}^n 2^n \sum_{A: j \in A} 4\left(\hat{f}(A)\right)^2\\
&= 2^n \sum_A \sum_{j \in A} 4\hat{f}(A)^2 \\
&= 2^n\sum_{A} 4|A|\hat{f}(A)^2
\end{align*}
%We put the left hand side and right hand side formulas together, and 
From this we see that the claim is equivalent to
\[
\sum_{A \subseteq \{1, \ldots, n\}, |A| \txtn{ odd}} \hat{f}(A)^2 \leq \sum_{A \subseteq \{1, \ldots, n\}}|A|\hat{f}(A)^2
\]
which is trivially true. 
\end{proof}
This inequality invites improvements, as this is a large gap. The Fourier analytic proof made this gap clear.

%This is what we called earlier in the semester 


\section{Improvement for $L_p$}


\begin{thm} \llabel{thm:bourgain-lp}
Suppose $p \geq 1$ and $X$ is an $n$-dimensional normed space. Then 
\[
\delta_{X \hookrightarrow L_p}(\ep) \gtrsim \frac{\epsilon^2}{n^{5/2}}
\]
\end{thm}

In the world of embeddings into $L_p$, we are in the right ballpark for the value of $\de$---we know it is a power of $n$, if not exactly $1/n$. (It is an open question what the right power is.) This is a special case of a more general theorem, which I won't state. This proof doesn't use many properties of $L_p$. 

\begin{proof}
This follows from Theorem~\ref{thm:bourgain-lp2} and Theorem~\ref{thm:kol-rep}.
%The proof of this comes from the next theorem. 
\end{proof}

We will prove the theorem for $\ep=\rc2$. The proof needs to be modified for the $\ep$-version.

\begin{thm}  \llabel{thm:bourgain-lp2}
Let $X, Y$ be Banach spaces, with $\dim X = n < \infty$. Suppose that there exists a universal constant $c$ such that $\delta \leq \frac{c\ep^2}{n^{5/2}}$, and that $\cal N_\de$ is a $\delta$-net of $B_X$. \footnote{Let me briefly say something about vector-valued $L_p$ spaces. Whenever you write $L_p(\mu, Z)$, this equals all functions $f: \Omega \to Z$ such that $\left(\int \|f(\omega)\|_Z^p d\mu(\omega)\right)^{1/p} < \infty$ (our norm is bounded).}
Then there exists a separable probability space $(\Omega, \mu)$\footnote{it will in the end be a uniform measure on half the ball}, a finite dimensional subspace $Z \subseteq Y$ and a linear operator $T: X \to L_{\infty}(\mu, Z)$ such that for every $x \in X$ with $\|x\|_X = 1$, 
\[
\frac{1 - \ep}{D} \leq \int_{\Omega} \|Tx(\omega)\|_Y d\mu(\omega) \leq \txtn{esssup}_{\omega \in \Omega} \|(Tx)\omega\|_Y \leq 1 + \epsilon
\]

\end{thm}

Note that the inequality can also be written as  $$\frac{1 - \ep}{D}  \|Tx\|_{L_1(\mu, Z)} \leq \|Tx\|_{L_{\infty}(\mu, Z)}.$$ 

Now what happens when $Y = L_p([0, 1])$? We have
\[
L_p(\mu, Z) \subseteq L_p(\mu, Y) = L_p(\mu, L_p) = L_p(\mu \times \lambda)
\]
A different way to think of this is that a function $f:X\to L_{\iy}(\mu, L_p([0,1]))$ can be thought of as a function of $x\in X$ and $\om\in [0,1]$, $f(\omega, x): \Omega \times [0, 1] \to \R$. 

This is a very classical fact in measure theory: 
\begin{thm}[Kolmogorov's representation theorem]\label{thm:kol-rep}
Any separable $L_p(\mu)$ space is isometric to a subset of $L_p$.
\end{thm}
This is an immediate corollary of the fact that if I give you a separable probability space, there is a separable transformation of the probability space $(\Omega, \mu) \cong ([0, 1], \lambda)$ where $\lambda$ is Lebesgue measure. So there is a measure preserving isomorphism if the probability measure is atom-free.
In general, the possible cases are
\begin{itemize}
\item
$(\Omega, \mu) \cong ([0,1],\la)$ if it is atom free.
\item
$(\Omega, \mu) \cong [0,1]\times \{1,\ldots,n\}$ if there are finitely many ($n$) atoms
\item
$(\Omega, \mu) \cong [0,1]\times \N$ if there are countably many atoms,
\item
$(\Omega, \mu) \cong \{1,\ldots, n\}$ or $\N$ if it's completely atomic.
\end{itemize}•
See Halmos's book for deails. This is just called Kolmogorov's representation theorem. 
%Because of this fact, we don't really care what $\mu$ was. 

Now for this to be an improvement of Bourgain's discretization theorem for $L_p$, we need that if $Z\sub Y$ is a finite dimensional subspace and $W\sub L_{\infty}(\mu, Z)$ is a finite dimensional subspace such that on $W$, the $L_{\infty}$ and $L_1$ norms are equivalent (a very strong restriction), then $W$ embeds in $Y$. 
That's the only property of $L_p$ we're using. 

%Consider $W = TX \subseteq L_{\infty}(\mu, Y)$, a finite dimensional subspace of bounded functions. But on this finite dimensional subspace, it just happens to be the case that the $L_{\infty}$ norm is the space as the $L_1$ norm, up to a constant. % The expectation is the same as the max, up to constants. That's a big restriction on the functions. 
%If this is true of any subspace $W$ such subspace already embeds back into $Y$, then you're finished. Our theorem gives an $n^{power}$ solution. We might ask whether this is true in general. However, this is not true: Not every $Y$ has this property. 

However, not every $Y$ has this property.

\begin{thm}[Ostrovskii and Randrianantoanina]
Not every $Y$ has this property. 
\end{thm}
You can construct spaces of functions taking values in a finite dimensional subspace of it which cannot be embedded back into $Y$. Nevertheless, you have to work to find bad counterexamples. 
This is not published yet. 

\blu{Next class we'll prove Theorem~\ref{thm:bourgain-lp2}, which implies our first theorem. What is the advantage of formulating the theorem in this way? We can use Bourgain's almost-extention theorem along the ball. The function is already Lipschitz, so we can already differentitate it. The measure itself will be the unit ball. We'll look at the derivative of the extended function in direction $\omega$ at point $x$. That is what our $T$ will be. We will look in all possible directions. Then the lower bound we have here says that the derivative of the function is only invertible on average, not at every point in the sphere. That tends to be big on average. We will work for that, but you see the advantage: In this formulation, we're going to get an average lower-bound, not an always lower-bound, and in this way we will be able to shave off two exponents. But this is not for Banach spaces in general. The advantage we have here is that $L_p$ of $L_p$ is $L_p$. }



%\step{2} Extend $F_1$ to the whole space to $F_2$ such that 
%\begin{enumerate}
%\item
%$\forall x\in \cal N_\de$, $\ve{F_2(x)-f(x)}_Y \lesssim L\de$.
%\item
%$\ve{F_2(x)-F_2(y)}_Y\lesssim L(\ve{x-y}_X + \de)$.
%%not smooth yet. 
%\item
%$\Supp(F_2)\subeq 2B_X$.
%\item
%$F_2$ is smooth.
%%F_1 had no bounded continuity, but is a sum against a partition of unity. 
%%just smooth without any bounds fine.
%%spiky.
%%when I say not smooth, I mean no bounds.
%%I need the norm to be smooth for this.
%\end{enumerate}•%

%Denote $\al(t)=\max\{1-|1-t|,0\}$. %

%\ig{images/9-1}{.25}%

%Let
%\[
%F_2(x)=\al(\ve{x}_X) F_1\pa{x}{\ve{x}_X}.
%\]
%%0 the moment it passes 2.
%$F_2$ still satisfies condition 1. As for condition 2, 
%\bal
%\ve{F_2(x)-F_2(y)}_Y &= \ve{\al (\ve{x}_X)F_1\pf{x}{\ve{x}_X} - \al(\ve{y}_X) F_1\pf{y}{\ve{y}_X}} \\
%&\le |\al(\ve{x})-\al(\ve{y})|\ub{\ve{F_1\pf{x}{\ve{x}_X}}}{\le 2L}+\al(\ve{y})\ve{F_1\pf{x}{\ve{x}_X} - F_1\pf{y}{\ve{y}_X} }\\
%&\le (\ve{x}-\ve{y})2L + \al(\ve{y}) L\pa{
%\ve{\nv{x}-\nv{y}}+4\de 
%} \\
%&\le 2L\ve{x-y}+L\al(\ve{y}) \pa{\ve{x}\ab{\rc{\ve{x}}-\rc{\ve{y}}} + \fc{\ve{x-y}}{\ve{y}} + 4\de}\\
%&\le 2L\ve{x-y} +L\al(\ve{y}) \pa{\fc{\ve{x-y}}{\ve{y}} + \fc{\ve{x-y}}{\ve y}  + 4\de}\\
%&\lesssim L(\ve{x-y}+\de),
%%mult by 4de, use bounded by 1
%\end{align*}
%where in the last step we used $\al(\ve{y})\le \ve{y}$ and $\al(\ve{y})\le 1$. %

%Note $F_2$ is smooth because the sum for $F_1$ was against a partition of unity and $\ved_X$ is smooth, although we don't have uniform bounds on smoothness for $F_2$.
%%F_1 had no bounded continuity, but is a sum against a partition of unity. 
%%just smooth without any bounds fine.
%%spiky.
%%when I say not smooth, I mean no bounds.
%%I need the norm to be smooth for this.%

%%For the next step we need the following. %

%\step{3} We make $F$ smoother by convolving.
%\begin{lem}[Begun, 1999]
%Let $F_2:X\to Y$ satisfy $\ve{F_2(x)-F_2(y)}_Y\le L(\ve{x-y}_X+\de)$. Let $\tau \ge c\de$. Define 
%\[
%F(x) = \rc{\Vol(\tau B_X)}\int_{\tau B_X} F_2(x+y)\,dy.
%\]
%Then 
%\[
%\ve{F}_{\text{Lip}} \le L\pa{1+\fc{\de n}{2\tau}}.
%\]
%\end{lem}
%The lemma proves the almost extension theorem as follows. We passed from $f:\cal N_\de\to Y$ to $F_1$ to $F_2$ to $F$. 
%If $x\in \cal N_\de$, 
%\bal
%\ve{F(x)-f(x)}_Y &=\ve{
%\rc{\Vol(\tau B_X)} \int_{B_X} (F_2(x+y) - f(x))\,dy
%}\\
%&\le \rc{\Vol(\tau B_X)}\int_{\tau B_X}\ve{F_2(x+y)-F_2(x)}_Y + \ub{\ve{F_2(x)-f(x)}_Y}{\de L} \dy\\
%&\le \rc{\Vol(\tau B_X)}\int_{\tau B_X}(L(\ub{\ve{y}_X}{\le\tau}+\de L)) \dy\lesssim L\tau.
%\end{align*}
%Now we prove the lemma. 
%\begin{proof}
%We need to show
%\[
%\ve{F(x)-F(y)}_Y \le L\pa{1+\fc{\de n}{2\tau}} \ve{x-y}_X.
%\]
%\Wog $y=0$, $\Vol(\tau B_X)=1$. Denote 
%\bal
%M&=\tau B_{X}\bs (x+\tau B_X)\\
%M'&=(x+\tau B_X) \bs \tau B_X.
%\end{align*}%

%\ig{images/9-2}{.25}%

%We have
%\bal
%F(0)-F(x) &= \int_M F_z(y)\,dy - \int_{M'} F_z(y)\,dy.
%\end{align*}
%Define $\om(z)$ to be the Euclidean length of the interval $(z+\R x)\cap (\tau B_X)$. By Fubini,
%\[
%\int_{\Proj_{X^{\perp}} (\tau B_X)} \om(z) \,dz = \Vol_n(\tau B_X)=1.
%%intersection of projection.
%\]
%Denote
%\bal
%W&= \set{z\in \tau B_X}{(z+\R x)\cap (\tau B_X)\cap (x+\tau B_X)\ne \phi}\\
%N&= \tau B_X\bs W.
%\end{align*}
%Define $C:M\to M'$ a shift in direction $X$ on every fiber that maps the interval $(z+\R x)\cap M\to (z+\R x)\cap M'$. %

%\ig{images/9-3}{.25}%

%$C$ is a measure preserving transformation with
%\[
%\ve{z-C(z)}_X =\begin{cases}
%\ve{x}_X , &z\le N\\
%\om(z) \fc{\ve{x}_X}{\ve{x}_2},& z\in W\cap M.
%\end{cases}
%\]
%(In the second case we translate by an extra factor $\fc{\om(z)}{\ve{x}_2}$.)
%%(In the second case we add the total length $
%%C maps $M'$ to $M$.
%%do a clever change of variable differently in each fiber.
%Then 
%\bal
%\ve{F(0)-F(x)}_Y &=\ve{\int_M F_2(y)\dy - \int_{M'}F_2(y)\dy}_Y\\
%&= \ve{\int_M(F_2(y) - F_2(C(y)))\dy}_Y\\
%&\le \int_M L(\ve{y-C(y)}_X+\de)\dy\\
%&\le \int_M L (\ve{y-C(y)}_X + \de)\dy\\
%&=L\de \Vol(M) + L \int_M \ve{y-C(y)}_X\dy\\
%\int_M \ve{y-C(y)}_X\dy 
%%orth decomp but not unit vector, integrate the length multiply by norm of direction. jacobian.
%&=\int_{N}\ve{x}_X\dy + \int_{W\cap M}\fc{\om(y)\ve{x}_X}{\ve{x}_2}\dy\\
%&=\ve{x}_X \Vol(N) + \int_{\Proj(W\cap M)} \fc{\om(z) \ve{x}_X}{\ve{x}_2} \ve{x}_2\,dz&\text{orthogonal decomposition}\\
%&=\ve{x}_X \Vol(N) + \Vol(\tau B_X\bs N) \\
%&=\ve{x}_X \Vol(\tau B_X)=\ve{x}_X.
%\end{align*}
%%w as the entire length. What I get is the entire volume. 
%We show $M=\tau B_X\bs (x+\tau B_X) \subeq \tau B_X\bs (1-\fc{\ve{x}}{\tau}) \tau B_X$. Indeed, for $y\in M$,
%\bal
%\ve{y-x}_X&\ge \tau\\
%\ve{y} & \ge \tau - \ve{x} = \pa{1-\fc{\ve{x}}{\tau}}\tau.
%\end{align*}
%\end{proof}
%Then 
%\[
%\Vol(M) \le \Vol(\tau B_X)-\Vol\pa{\pa{1-\fc{\ve{x}}{\tau}}\tau B_X}=1-\pa{1-\fc{\ve{x}}{\tau}} \lesssim \fc{n\ve{x}}{\tau}
%\]
%\end{proof}
%Bourgain did it in a more complicated, analytic way avoiding geometry. Begun notices that careful geometry is sufficient.%
%

%Later we will show this theorem is sharp.%

%%iteration!%

%\section{Proof of Bourgain's discretization theorem}%

%At small distances there is no guarantee on the function $f$. Just taking derivatives is dangerous. It might be true that we can work with the initial function. But the only way Bourgain figured out how to prove the theorem was to make a 1-parameter family of functions.%

%\subsection{The Poisson semigroup}
%\begin{df}
%The \ivocab{Poisson kernel} is $P_t(x):\R^n\to \R$ given by
%\[
%P_t(x)=\fc{C_nt}{(t^2+\ve{x}_2^2)^{\fc{n+1}2}}, \quad C_n=\fc{\Ga\pf{n+1}2}{\pi^{\fc{n+1}2}}.
%\]
%%Convolution becomes product under FT
%\end{df}%

%\begin{pr}[Properties of Poisson kernel]
%\begin{enumerate}
%\item
%For all $t>0$, $\int_{\R^n} P_t(x)\,dx=1$.
%\item
%(Semigroup property) $P_t*P_s=P_{t+s}$. 
%\item
%$\wh{P_t}(x) = e^{-2\pi\ve{x}_2t}$.
%\end{enumerate}•
%\end{pr}%

%\begin{lem}
%Let $F$ be the function obtained from Bourgain's almost extension theorem~\ref{thm:baet}.
%For all $t>0$, $\ve{P_t*F}_{\text{Lip}}\lesssim 1$.
%%P_t is prob measure.
%%Average values of $F$.
%%poU, geometr y of ball, average with decaying weights.
%%all averaging of Lipschitz things.%

%%1+\ep subtlety, ceases to be true, need restriction on $T$. Not relevant. $1+\ep$ version do more carefully.
%\end{lem}
%%Note to get $P_t*F$ we had three averaging arguments: partition of unity, averaging with respect to a ball, and then averaging with decaying weights.
%We have 
%\bal
%P_t*F(x)-P_t*F(y) &= \int_{\R^n} P_t(z)(F(x-z) - F(x-y))\,dz.
%\end{align*}
%Our goal is to show there exists $t_0>0$, $x\in B\in \rc B_X$ such that if we define
%\[
%T=(P_{t_0}*F)'(x):X\to Y,
%\]
%we have $\ve{T}\lesssim 1$. Moreover $\ve{T^{-1}}\lesssim D$.
%$T_y=\lim_{h\to \iy} \fc{P_{t_0}*F(x+hy) - P_{t_0}*F(x)}{h}$.
%%pigeonhole, must exist

\blu{3/23: We will prove the improvement of Bourgain's Discretization Theorem.}

\blu{3/28: Finishing up Bourgain's Theorem}

Previously we had $X, Y$ Banach spaces with dim$(X) = n$. Let $N_{\delta} \subseteq B_X$ be a $\delta$-net, $\delta \leq c/n^2D$, $f: N_{\delta} \to Y$. 
We have 
\[
\frac{1}{D}\|x - y\| \leq \|f(x) - f(y)\| \leq \|x - y\|
\]
WLOG $Y = \txtn{span}(f(N_{\delta}))$. 
The almost extension theorem gave us a smooth map from $F:X \to Y$, $\|F\|_{Lip} \lesssim 1$ and for all $x \in N_{\delta}$, $\|F(x) - f(x)\| \lesssim n\delta$. 
Letting $\mu$ be the normalized Lebesgue measure on $1/2 B_X$, define $T:X \to L_{\infty}(\mu, Y)$ by
\[
(Ty)(x) = F'(x)y
\]
for all $y \in X$. 

The goal was to prove 
\[
\|Ty\|_{L_1(\mu, Y)} \gtrsim \frac{1}{D}\|y\|
\]
and the approach was to show for $y \in X$ the average $\frac{1}{\txtn{vol}(1/2 B_X)}\int_{1/2 B_X}\|F'(x)y\| dy$ is big. 

Fix a linear isometric embedding $J: X \to l_{\infty}$., we proved that there exists $G: Y \to l_{\infty}$ such that $\forall x \in N_{\delta}, G(f(x)) = J(x)$. We have $\|G\|_{Lip} \leq D$. Fix $H: Y \to l_{\infty}$ where $H$ is smooth, $\|H\|_{Lip} \leq D$, and $\forall x \in F(B_X)$, $\|H(x) - G(x)\| \leq nD\delta$. 

Define $S: L_1(\mu, Y) \to l_{\infty}$ by
\[
Sh = \int_{1/2 B_X}H'(F(x))(h(x)) d\mu(x)
\]
for all $h \in L_1(\mu, Y)$. Note that $\|S\|_{L_1(\mu, Y) \to l_{\infty}} \leq D$. 
By the chain rule, $STy = \int_{1/2 B_X}(H \circ F)'(x)y d\mu(x)$. 
We checked for every $x \in B_X$, $\|H(F(x)) - Jx\|_{l_{\infty}]} \lesssim nD\delta$. 

Suppose you are on a point on the net. Then you know $H$ is very close to $G$. $F$ was $n\delta$ close to $f$, and $H$ is $D$-Lipschitz, which is how we get $n\delta D$ (take a net point, find the closest point on the net, and use the fact that the functions are $D$-Lipschitz).

This is where we stopped. Now how do we finish? 

We need a small geometric lemma: 

\begin{lem} Geometric Lemma.  \label{lem:geolem}\\
$U, V$ are Banach spaces, dim$(U) = n$. Let $U = (\R^n, \|\cdot\|_U)$. 
Suppose that $g: B_U \to V$ is continuous on $B_U$ and smooth on the interior. 
Then 
\[
\left\|\frac{1}{\txtn{vol}(B_U)}\int_{B_U} g'(u) du\right\|_{U \to V} \leq n\|g\|_{L_{\infty}(S_X)}
\]
\end{lem}
where the inside of the LHS norm should be thought of as an operator. Let $R$ be the normalized integral operator. Then 
\[
Rx = \frac{1}{\txtn{vol}(B_U)}\int_{B_U} g'(u)x du
\]

The way to think of using this is if you have an $L_{\infty}$ bound, you get an $L_{1}$ bound. 
Assuming this lemma, let's apply it to $g = H\circ F - J$. We get
\bal
\left\|\int_{1/2 B_X}\left((H\circ F)' - J\right) d\mu(x)\right\|_{X \to l_{\infty}} &\lesssim n^2D\delta
\\
\left\|\int_{1/2 B_X}(H\circ F)'(x)d\mu(x) - J\right\|_{X \to l_{\infty}} &= \|ST - J\|_{X \to l_{\infty}} \lesssim n^2D\delta
\end{align*}
Now we want to bound from below $\|Ty\|_{L_1(\mu, Y)}$. We have 
\bal
\|Ty\|_{L_1(\mu, Y)} &\geq \frac{\|STy\|_{l_{\infty}}}{\|S\|_{L_1(\mu, Y) \to l_{\infty}}}
\\
&\geq \frac{1}{D}\|STy\|_{l_{\infty}} \geq \frac{1}{D}\left(\|Jy\| - \|ST - J\|_{X \to l_{\infty}}\|y\|\right)
\\
&= \frac{1}{D}\left(\|y\| - (n^2D\delta\|y\|)\right)
\end{align*}
There is a bit of  magic in the punchline. We want to bound the operator below, and we understand it as an average of derivatives. We succeeded to show the function itself is small, and there is our geometric lemma which gives a bound on the derivative if you know a bound on the function. 

Now let's prove the lemma.
\begin{proof}[Proof of Lemma~\ref{lem:geolem}]
Fix a direction $y \in \R^n$ and normalize so that $\|y\|_2 = 1$. For every $u \in \txtn{Proj}_{y^{\perp}}(B_U)$, let $a_U \leq b_U \in \R$ be such that $u + \R y \cap B_U = u + [a_U, b_U]y$ (basically, this is the intersection of the projection line with the ball). 

Using Fubini, 
\bal
\left\|\frac{1}{\txtn{vol}(B_U)}\int_{B_U} g'(u) du\right\|_V &= \left\|\frac{1}{\txtn{vol}(B_U)} \int_{\txtn{Proj}_{y^{\perp}}(B_U)} \int_{a_U}^{b_U} \frac{d}{ds}g(u + sy)ds du\right\|_V
\\
&= \left\|\frac{1}{\txtn{vol}(B_U)} \int_{\txtn{Proj}_{y^{\perp}}(B_U)} \left(g(u + b_Uy) - g(u + a_Uy)\right) du \right\|_V
\\
&\leq \frac{1}{\txtn{vol}_n(B_U)}\cdot 2\txtn{vol}_{n-1}(\txtn{Proj}_{y^{\perp}}(B_U)) \|g\|_{L_{\infty}(S_{X}, V)}
\end{align*}
We need to show that 
\[
\frac{2\txtn{vol}_{n -1}\left(\txtn{Proj}_{y^{\perp}}(B_U)\right)}{\txtn{vol}_n(B_U)} \leq n\|y\|_U
\]
The convex hull of $y$ over the projection is the cone over the projection. 
\[
\txtn{vol}_n(\txtn{conv}\left(\frac{y}{\|y\|_U} \cup \txtn{Proj}_{y^{\perp}}(B_U)\right)) = \frac{1}{n\|y\|_U}\cdot \txtn{vol}_{n-1}\left(\txtn{Proj}_{y^{\perp}}(B_U)\right)
\]
Letting $K = \txtn{conv}\left(\left\{\pm \frac{y}{\|y\|_U}\right\} \cap \txtn{Proj}_{y^{\perp}}(B_U)\right)$, this is the same as saying that $\txtn{vol}_n(K) \leq \txtn{vol}_n(B_U)$. Note that $K$ is the double cone, that is where the factor of $2$ got absorbed. This is an application of Fubini's theorem. 

Geometrically, the idea is that when we look on the projective lines for whatever part of the cone is not in our ball set, we will be able to fit the outside inside the set. 
In formulas, $c_U$ is the largest multiple of $u$ which is inside the boundary of the cone. $c_U \geq 1$. 
\[
K = \bigcup_{u \in \txtn{Proj}_{y^{\perp}}(B_U)} \left(u + \left[ -\frac{c_U - 1}{c_U\|y\|_U}, +\frac{c_U - 1}{c_U\|y\|_U} \right]\right)
\]
We also have 
\[
\frac{1}{c_U}\left(c_Uu + a_{c_Uu}y\right) \pm (1 - \frac{1}{c_U})\frac{y}{\|y\|_U} \in B_U
\]
and we get that $K \subset B_U$ by Fubini. 
\end{proof}

Thus, we've completed the proof of Bourgain's theorem for the semester. The big remaining question is can we do something like this in the general case? 

\section{Kirszbraun's Extension Theorem}
Last time we proved nonlinear Hahn-Banach theorem. I want to prove one more Lipschitz Extension theorem, which I can do in ten minutes which we will need later in the semester. 

\begin{thm} Kirszbraun's extension theorem (1934). \label{thm:kirszbraun}
Let $H_1, H_2$ be Hilbert spaces and $A \subseteq H_1$. Let $f: A \to H_2$ Lipschitz. Then, there exists a function $F: H_1 \to H_2$ that extends $f$ and has the same Lipschitz constant.
\end{thm}
We did this for real valued functions and $l_{\infty}$ functions. This version is non-trivial, and relates to many open problems. 
\begin{proof}
There is an equivalent geometric formulation.
Let $H_1, H_2$ be Hilbert spaces $\{x_i\}_{i \in I} \subseteq H_1$ and $\{y_i\}_{i \in I} \subseteq H_2$, $\{r_i\}_{i \in I} \subseteq (0, \infty)$. Suppose that $\forall i, j \in I$, $\|y_i - y_j\|_{H_2} \leq \|x_i - y_i\|_{H_1}$. If 
\[
\bigcap_{i \in I} B_{H_1}(x_i, r_i) \neq \emptyset
\]
then 
\[
\bigcap_{i \in I} B_{H_1}(y_i, r_i) \neq \emptyset
\]
as well.

Intuitively, this says the configuration of points in $H_2$ are a squeezed version of the $H_1$ points. Then, we're just saying something obvious. If there's some point that intersects all balls in $H_1$, then using the same radii in the squeezed version will also be nonempty.
Our geometric formulation will imply extension. We have $f: A \to H_2$, WLOG $\|f\|_{Lip} = 1$. For all $a, b \in A$, $\|f(a) - f(b)\|_{H_2} \leq \|a - b\|_{H_1}$. Fix any $x \in H_1 \setminus A$. What can we say about the intersection of the following balls?: $\bigcap_{a \in A} B_{H_1}(a, \|a - x\|_{H_1})$. Well by design it is not empty since $x$ is in this set. So the conclusion from the geometric formulation says 
\[
\bigcap_{a \in A} B_{H_2}\left(f(a), \|a - x\|_{H_1}\right) \neq \emptyset
\]
So take some $y$ in this set. Then $\|y - f(a)\|_{H_2} \leq \|x - a\|_{H_2}$ $\forall a \in A$. 
Define $F(x) = y$. Then we can just do the one-more-point argument with Zorn's lemma to finish. 

Let us now prove the geometric formulation. It is enough to prove the geometric formulation when $|I| < \infty$ and $H_1, H_2$ are finite dimensional. 
To show all the balls intersect, it is enough to show that finitely many of them intersect (this is the finite intersection property: balls are weakly compact). 
Now the minute $I$ is finite, we can write $I = \{1, \cdots, n\}$, $H_1' = \txtn{span}\{x_1, \cdots, x_n\}, H_2' = \txtn{span}\{y_1, \cdots, y_n\}$. This reduces everything to finite dimensions. 
We have a nice argument using Hahn-Banach.
\fixme{FINISH THIS}
\end{proof}

\begin{rem}
In other norms, the geometric formulation in the previous proof is just not true. This effectively characterizes Hilbert spaces.
Related is the Kneser-Poulsen conjecture, which effectively says the same thing in terms of volumes:
\begin{conj} Kneser-Poulsen. \\
Take $x_1, \cdots, x_k; y_1, \cdots, y_k \in \R^n$ and $\|y_i - y_j\| \leq \|x_i - x_j\|$ for all $i, j$. Then $\forall r_i \geq 0$
\[
\txtn{vol}\left(\bigcap_{i = 1}^k B(y_i, r_i) \right) \geq \txtn{vol}\left(\bigcap_{i = 1}^k B(x_i, r_i)\right)
\]
\end{conj}
This is known for $n = 2$, where volume is area using trigonometry and all kinds of ad-hoc arguments. It's also known for $k = n+2$. 
\end{rem}




%\bal
%\ve{Sh}_{\ell_\iy} &\le \int_{\rc 2B_X} \ve{H'(F(x))(h(x))}_{\ell_\iy} \,d\mu(x)\\
%&\le \int_{\rc 2B_X}\ub{\ve{H'(F(x))}_{Z\to \ell_\iy}}{=:D} \ve{h(x)}\,d\mu(x)\\
%%bounded norm between auxiliary spaces.
%&\le D \int_{\rc 2B_X} \ve{h(x)} \\
%&\le D\ve{h}_{L_1(\mu,Z)}\\
%\implies 
%\ve{S}_{L_1(\mu, Z)\to \ell_\iy}&\le D\\
%S\ub{Ty}h &=\int_{\rc 2B_X} H'(F(x)) ((Ty)(x))\,d\mu(x)\\
%&=\int_{\rc 2B_X} H'(F(x)) (F'(x)(y)) \,d\mu(x)\\
%&=\int_{\rc 2B_X} (H\circ F)'(x)(y)\,d\mu(x)&\text{chain rule}
%%smooth : use banach space, diffce quotient.
%%need finite dim for a million things.
%%htere is bounded operator
%\end{align*}

%\begin{proof}[Proof of Theorem~\ref{thm:bourgain-lp2}]
%There exists $f:\cal N_\de\to Y$ such that 
%\[
%\rc{D} \ve{x-y}_Y\le \ve{f(x)-f(y)}_Y \le \ve{x-y}_X
%\]
%for $x,y\in \cal N_\de$. By Bourgain's almost extension theorem, letting $Z=\spn(f(\cal N_\de))$, %there exists $F:X\to Z$ such that


\blu{3/30/16}

Here is an equivalent formulation of the Kirszbraun extension theorem.
\begin{thm}
Let $H_1,H_2$ be Hilbert spaces, $\{x_i\}_{i\in I}\subeq H_1$, $\{y_i\}_{i\in I}\subeq H_2$, $\{r_i\}_{i=1}^{\iy}\subeq (0,\iy)$. Suppose that $\ve{y_i-y_j}_{H_2}\le \ve{x_i-x_j}_{H_2}$ for all $i,j\in I$. Then $\bigcap_{i\in I} B_{H_1}(x_i,r_i) \ne \phi$ implies $\bigcap_{i\in I} B_{H_2} (y_i,r_i) \ne \phi$. 
\end{thm}
\begin{proof}
By weak compactness and orthogonal projection, it is enough to prove this when $I=\{1,\ldots, n\}$ and $H_1,H_2$ are both finite dimensional. Fix any $x\in \bigcap_{i=1}^n B_{H_1}(x_i,r_i)$. If $x=x_{i_0}$ for some $i_0\in \{1,\ldots, n\}$, $\ve{y_{i_0}-y_i}_{H_2}\le \ve{x_{i_0}-x_i}_{H_1}\le r_i$, and $y_{i_0}\in \bigcap_{i=1}^n B_{H_2}(y_i,r_i)$. Assume $x\nin \{x_1,\ldots, x_n\}$. 

Define $f:H\to \R$ by 
\[
f(y) = \max \bc{
\fc{\ve{y-y_1}_{H_2}}{\ve{x-x_1}_{H_1}}, \ldots, \fc{\ve{y-y_n}_{H_2}}{\ve{x-x_n}_{H_2}}
}.
\]
Note that $f$ is continuous and $\lim_{\ve{y}_{H_2}\to \iy} f(y)=\iy$.

So, the minimum of $f(y)$ over $\R^n$ is attained. Let 
\[
m=\min_{y\in \R^n} f(y).
\]
Fix $y\in \R^n$, $f(y)=m$.

Observe that we are done if $m\le 1$.
Suppose \bwoc{} $m>1$. Let $J$ be the indices where the ratio is the minimum. 
\[
J=\set{i\in \{1,\ldots, n\}}{\fc{\ve{y-y_i}_{H_2}}{\ve{x-x_i}_{H_1}}=m}.
\]
By definition, if $j\in J$, then $\ve{y-y_j}_{H_2}=m\ve{x-x_j}_{H_1}$, and for $j\nin J$, $\ve{y-y_j}_{H_2}<m\ve{x-x_j}_{H_1}$. 

\begin{clm}
$y\in \conv(\{y_j\}_{j\in J})$.
\end{clm}
\begin{proof}
If not, find a separating hyperplane. If we move $y$ a small enough distance towards $\conv(\{y_j\}_{j\in J})$ perpendicular to the separating hyperplane to $y'$, then for $j\in J$,
\[
\ve{y'-y_j}_{H_2}<\ve{y-y_j}_{H_2} = m\ve{x-x_j}_{H_1}.
\]
For $j\in J$, it is still true that $\ve{y'-y_j}_{H_2}<m\ve{x-x_j}_{H_1}$. Then $f(y')<m$, contradicting that $y$ is a minimizer.
\end{proof}

By the claim, there exists $\{\la_j\}_{j\in J}$, $\la_j\ge 0$, $\sum_{j\in J}\la_j=1$, $y=\sum_{j\in J}\la_jy_j$. 

We use this the coefficients of this convex combination to define a probability distribution. Define a random vector in $H_1$ by $X$, with
\[
\Pj(X=x_j) = \la_j,\quad j\in J.
\]
For all $j\in \{1,\ldots, n\}$, let $y_j=h(x_j)$. Let $\E[h(X)] = \sum_{j\in J} \la_jy_j=y$. Let $X'$ be an independent copy of $X$.

Using $\ve{h(X)-y}_{H_2} = m\ve{X-x}_{H_1}$, we have
\begin{align}%bwoc m>1
\E\ve{h(X)-\E h(X)}_{H_2}^2 & =\E \ve{h(X) - y}_{H_2}^2\\
&=m^2 \E\ve{X-x}_{H_1}^2\\
&> \E\ve{X-x}_{H_1}^2\\
&\ge \E\ve{X-EX}_{H_1}^2 \label{eq:kir1}
%&= \rc2 \E\ve{X-X'}_{H_2}^2.
\end{align}
%To see~\eqref{eq:kir1}, note
%\[
%E\ve{X-x}_{H_1}^2 \ge \E\ve{X-EX}_{H_1}^2
%\]
%expand
In Hilbert space, it is always true that 
\[
\E\ve{X-EX}_{H_1}^2 = \rc2 \E\ve{X-X'}_{H_2}^2.
\]
Using this on both sides of the above,
\[
\rc 2 \E\ve{h(X)-h(X')}_{H_2}^2 > \rc2\ve{X-X'}_{H_1}^2. 
\]

So far we haven't used the only assumption on the points. By the assumption $\ve{X-X'}_{H_1} \ge \ve{h(X)-h(X')}_{H_2}$. This is a contradiction.
\end{proof}
This is not true in $L^p$ spaces when $p\ne 2$, but there are other theorems one can formulate (ex. with different radii). We have to ask what happens to each of the inequalities in the proof.
As stated the theorem is a very ``Hilbertian" phenomenon.

\begin{df}
Let $(X,\ved_X)$ be a Banach space and $p\ge 1$. We say that $X$ has \ivocab{Rademacher type} $p$ if for every $n$, for every $x_1,\ldots, x_n\in X$, 
\[\pa{\EE_{\ep=(\ep_1,\ldots, \ep_n)\in \{\pm 1\}^n}\ba{\ve{\sumo in\ep_iX_i}^p_X}}^{\rc p}\le T\pa{\sumo in \ve{X_i}_X^p}^{\rc p}.
\] 
The smallest $T$ is denoted $T_p(X)$.
%2 is parallelogram identity. FOr other $p$ this is a generalization.
\end{df}
\begin{df*}[Definition~\ref{df:enflo}]
A metric space $(X,d_X)$ is said to have \ivocab{Enflo type} $p$ if for all $f:\{\pm 1\}^n\to X$,
\[
\pa{
\EE_{\ep} [d_X(f(\ep),f(-\ep))^p]
}^{\rc p}\lesssim_X
\pa{
\sumo jn \EE_{\ep} [d(f(\ep), f(\ep_1,\ldots,\ep_{j-1}, -\ep_j, \ep_{j+1},\cdots))^p]
}^{\rc p}.
\]
\end{df*}
If $X$ is also a Banach space of Enflo type $p$, then it is also of Rademacher type $p$. %linear functions.

\begin{qu}
Let $X$ be a Banach space of Rademacher type $p$. Does $X$ also have Enflo type $p$?
\end{qu}
We know special Banach spaces for which this is true, like $L^p$ spaces, but we don't know the anser in full generality.
\begin{thm}[Pisier]\llabel{thm:pisier}
Let $X$ be a Banach space of Rademacher type $p$. % Then $X$ has type $p-\ep$ for every $1>\ep>0$.
Then $X$ has Enflo type $q$ for every $1<q<p$.
\end{thm}
%set up to be trivially true when $p=1$ (by triangle ineq) All these things are 

We first need the following.
\begin{thm}[Kahane's Inequality]
For every $\iy>p,q\ge 1$, there exists $K_{p,q}$ such that for every Banach space, $(X,\ved_X)$ for every $x_1,\ldots, x_n\in X$, 
\[
\pa{\E\ve{\sumo in \ep_i x_i}_X^p}^{\rc p}\le K_{p,q}\pa{\E \ve{\sumo in \ep_i x_i}_X^q}^{\rc q}.
\]
%real: khintchine
\end{thm}
In the real case this is Khintchine's inequality.

By Kahane's inequality, $X$ has type $p$ iff
\[
\pa{\E\ve{\sumo in \ep_i x_i}_X^q}^{\rc q} \lesssim_{X,p,q} \pa{\sumo in \ve{x_i}_X^p}^{\rc p}.
\]

Define $T:\ell_p^n(X)\to L_p(\{\pm 1\}^n,X)$ by $T(x_1,\ldots, x_n)(\ep) = \sumo in \ep_iX_i$.
%vector valued 
%i will not discuss measurability

(Here $\ve{f}_{L_p(\{\pm 1\}^n,X)} = \pa{\rc{2^n} \sum_{\ep\in \{\pm 1\}^n} \ve{f(\ep)}_X^p}^{\rc p}$.)

%

Rademacher type $p$ mens
\[
\ve{T}_{\ell_p^n(X)\to L_q(\{\pm 1\}^n, X)}\lesssim_{X,p,q} 1.
\]
For an operator $T:Y\to Z$, the adjoint $T^*:Z^*\to Y^*$ satisfies 
\[
\ve{T}_{Y\to Z} = \ve{T^*}_{Z^*\to Y^*}. 
\]
%This is the usual duality we're used to.
Let $p^*=\fc{p}{p-1}$ be the dual of $p$ ($\rc{p}+\rc{p^*}=1$). Now 
\[
L_q(\{\pm 1\}^n, X)^*=L_{q^*} (\{\pm 1\}^n, X^*).
%finite dimensional
\]
Here, $g^*:\{\pm 1\}^n\to X^*$, $f:\{\pm 1\}^n\to X$, $g^*(f) = \E g^*(\ep) (f(\ep))$, $\ell_p^n(X) = \ell_{p^*}^n(X^*)$. 

Note 
\[\ve{T^*}_{L_{q^*}(\{\pm 1\}^n, X^*)\to \ell_{p^*}^n(X^*)}\lesssim 1,
\]
%formala for $T^*$. 
For $T:Y\to Z$, $T^*:Z^*\to Y^*$ is defined by $T^*(z^*)(y) = z^*(Ty)$. 

%natural $n$-tuple. Look at the Fourier transform. 
%field over 2 elements. 
For $g^*:\{\pm 1\}^n \to X^*$, $g^*  \sum_{A\subeq \{1,\ldots, n\}}\wh g^*(A)W_A$. %where $W_A(\ep)=\prod_{i\in A} \ep_i$.
%duality 5 times in every proof.
We claim
\[
T^*g^* = (\wh g^*(\{1\}),\ldots, \wh g(\{n\})). 
\]
We check
\bal
T^*g^* (x_1,\ldots, x_n)& = g^* (T(x_1,\ldots, x_n))\\
&\le
\E g^*(\ep) \pa{\sumo in \ep_iX_i}\\
&=\sumo in (\E g^*(\ep)\ep_i) (X_i)\\
&=\sumo in \pa{\sum_{A\subeq \{1,\ldots, n\}} \wh g^*(A) (\E W_A(\ep) \ep_i)}(x_i)\\
&=\sumo in \wh g^*(\{i\}) (x)\\
&= (\wh g^*(\{1\}),\ldots, \wh g^*(\{n\}))(x_1,\ldots, x_n).
\end{align*}
Rademacher type $p$ means for all $g_i^*:\{\pm 1\}^n \to X^*$, for all $q\in [1,\iy)$,
\[
\pa{\sumo in \ve{\wh g^*(\{i\})}_X^{p^*}} \lesssim \ve{g^*}_{L_q(\{\pm 1\}^n, X^*)}
\]
%Different way of writing Rademacher type.

\begin{thm}[Pisier]
If $X$ has Rademacher type $p$ and $1\le a<p$, then for every $b>1$, for all $f:\{\pm 1\}^n\to X$ with $\E f = 0$, 
\[
\ve{f}_b\precsim \ve{\pa{\sumo jn \ve{\pl_j f}_X^a}^{\rc a}}_b.
\]
\end{thm}
Here, for $f:\{\pm 1\}^n\to X$, $\pl_jf :\{\pm 1\}^n\to X$ is defined by 
\[
\pl_jf(\ep)  =\fc{f(\ep)- f(\ep_1,\ldots, \ep_{j-1},-\ep_j, \ep_{j+1},\ldots \ep_n)}2 = \sumr{A\subeq \{1,\ldots, n\}}{j\in A} \wh f(A) W_A.
\]
Note $\pl_j^2=\pl_j$. 

The Laplacian is $\De=\sumo jn \pl_j = \sumo jn \pl_j^2$. We've proved that
\[
\De f = \sum_{A\subeq \{1,\ldots,n\}} |A|\wh f(A) W_A.
\]
The heat semigroup on the cube\footnote{also known as the noise operator} is for $t>0$,
\[
e^{-t\De} f = \sum_{A\subeq \{1,\ldots, n\}} e^{-t|A|} \wh f(A)W_A.
\]
%heat flow proof of this, nice.
This is a function on the cube
\[
\ep \mapsto \pa{\sumo jn \ve{\pl_j f(\ep)}_X^a}^{\rc a}\in [0,\iy).
\]
For $b=a$, 
\[
\ve{f}_a\lesssim \pa{\sumo jn \ve{\pl_j f}_{L_a(\{\pm 1\}^n,X)}^q}^{\rc a}.
\]
%For enflo type
Let $f:\{\pm 1\}^n\to X$ be 
\begin{align*}
\ve{f(\ep)-f(-\ep)}_a
&\le \ve{f(\ep)-\E f}_a + \E\ve{\E f - \E f(-\ep)}_a\\%both are functions with mean 0.$ 
&\lesssim \pa{\sumo jn \ve{f(\ep) - f(\ep_1,\ldots, -\ep_j,\ldots, \ep_n)}_X^a}^{\rc a}.
\end{align*}

We need some facts about the heat semigroup.
\begin{fct}[Fact 1]
Heat flow is a contraction:
\[
\ve{e^{-t\De}f}_{L_q(\{\pm 1\}^n,X)}\le \ve{f}_{L_q(\{\pm 1\}^n,X)}.
\]
\end{fct}
\begin{proof}
We have
\[
e^{-t\De} f = \sum_{A\subeq \{1,\ldots, n\}}e^{-t|A|}\wh f(A) W_A = \sum R_t*f
%
\]
Here the convolution is defined as $\rc{2^n}\sum_{\de\in \{\pm 1\}^n} R_t(\ep\de)f(\de)$. %multiplicative boolean
This is a vector valued function. In the real case it's Parseval, multiplying the Fourier coefficients. 
%as long as not norms, identities of vectors. 
It's enough to prove for the real line. Identities on the level of vectors. 
Here
\[
R_t(\ep)=\sum_{A\subeq \{1,\ldots, n\}}  e^{-t|A|}W_A(\ep);
%no parseval ofr norms
\]
this is the Riesz product.

The function is 
\[
= \prod_{i=1}^n (1+e^{-t}\ep_i)=0,t>0.
\]
$\E R_t=1$ so $R_t$ is a probability measure and heat flow is an averaging operator.
%lots of Fourier analysis in next class.
\end{proof}

\blu{4/11: Continue Theorem~\ref{thm:pisier}. The dual to having Rademacher type $p$: for all $g^*:\{\pm 1\}^n\to X^*$, $\pa{\sumo in \ve{\wh g(\{i\})}_{X^*}^{p^*}}^{\rc{p^*}}\lesssim \ve{g^*}_{L^{r^*}(\{\pm 1\}^n, X^*)}$ for every $1<r<\iy$.

%if time prove Kahane. 
Note that for the purpose of proving Theorem~\ref{thm:pisier} we only need $r=p$ in Kahane's inequality. However, I recommend that you read up on the full inequality.
%, which is the original definition.
%Freedom to decouple the powers. If we just care about application, we don't need Kahane's inequality.
%kahane with r on LHS, \E over $\ep$, $p$ on RHS, $\lesssim_{X,r,p}$.

For $f:\{\pm 1\}^n\to X$, define $\pl_j f:\{\pm 1\}^n\to X$ by $\pl_j f = \fc{f(\ep)-f(\ep_1,\ldots, -\ep_j,\ldots, \ep_n)}{2}$. Then $\pl_j^2=\pl_j=\pl_j^*$ and $\pl_j W_A=\begin{cases}
W_A,&j\in A\\
0,&j\nin A.
\end{cases}$

The Laplacian is $\De=\sumo jn \pl_j=\sumo jn \pl_j^2$, 
\[
\De f = \sum_{A\subeq [n]}|A|\wh f(A) W_A.
\]
If $t>0$ then 
\[
e^{-t\De} = \sum_{A\subeq [n]} e^{-t|A|}W_A
\]
is the heat semigroup. It is a contraction. $e^{-t\De} = \pa{\prod_{i=1}^n (1+e^{-t}\ep_i)}*f$.
}

We have
\begin{align}
\ve{e^{-t\De} f}_{L_p(\{\pm 1\}^n, X)} &\ge e^{-nt} \ve{f}_{L^p(\{\pm 1\}^n, X)}\\
e^{-t\De} (V_{[n]} e^{-t\De} f) &=e^{-tn} W_{[n]}f
\end{align}
where $W_{[n]}(\ep)=\prod_{i=1}^n \ep_i$. For $f=\sum_{A\subeq [n]} \wh f(A) W_A$, we have
\begin{align}
W_{[n]}e^{-t\De}f &= \sum_{A\subeq [n]} e^{-t|A|}\wh f(A) W_{[n]\bs A}\\
%e^{-t\De} (W_{[n]}f)
e^{-t\De}(W_{[n]} e^{-t\De} f) &= \sum_{A\subeq [n]} e^{-t|A|} \wh f(A) e^{-t(n-|A|)}\\
&=e^{-tn} \sum_{A\subeq [n]} \wh f(A) W_{[n]\bs A} \\
&=e^{-tn} W_{[n]} f\\
e^{-tn} \ve{W_{[n]}f}_{L_p(\{\pm 1\}^n,X)}
&= \ve{e^{-t\De} (W_{[n]}(e^{-t\De} f))}_p\\
&\le \ve{\cancel{W_{[n]}} e^{-t\De}f}_p.
\end{align}
%$\rc{p^*}+\rc{p}=1$, $p^*=\fc{p}{p-1}$

The key claim is the following. 
\begin{clm}\llabel{clm:pis1}
Suppose $X$ has Rademacher type $p$. 
For all $1<r<\iy$, for all $t>0$, for all $g^*:\{\pm 1\}^n\to X^*$, 
\[
\ve{\pa{\ve{e^{-t\De} \pl_j g^*(\ep)}_{X^*}^{q^*}}^{\rc{ q^*}}}_{L_{r^*}(\{\pm 1\}^n)}\lesssim_{X,r,p} \fc{\ve{g^*}_{L_{r^*}(\{\pm 1\}^n, X^*)}}{(e^t-1)^{\fc{p^*}{q^*}}}
.
%integrable, singularity at 0.
\]
%Apply heat flow to it, do this in every direction. For every fixed $\ep$, compute the $L_{q^*}$ norm. Insider here is function of $\ep$.
\end{clm}
Assume the claim. We prove the following.
\begin{clm}\llabel{clm:pis2}
If $X$ has Rademacher type $p$, $1<q<p$, and $1<r<\iy$, then for every function $f:\{\pm 1\}^n \to X$ with $\E f = 0 = \wh f(\phi)$ we have 
\[
\ve{f}_{L_r(\{\pm 1\}^n, X)} \lesssim \rc{p-q} \ve{
\pa{\sumo jn \ve{\pl_j f(\ep)}_X^q
}^{\rc q}}_{L_r(\{\pm 1\}^n, X)}.
\]
\end{clm}
When $r=q$, 
\begin{align}
\ve{f-\E f}_q&\lesssim \prc{p-q}^q \EE_{\ep} \sumo jn \ve{\pl_j f(\ep)}_X^q\\
&= \prc{p-q}^q \rc{2^q} \sumo jn \ve{f(\ep)-f(\ep_1,\ldots, -\ep_j, \ep_n)}_X^q\\
\ve{f(\ep)-f(-\ep)}_q&\le %\ve{f(\ep)-}_q...
\\
&= 2\ve{f(\ep)-\E f}_q\\
&\lesssim \rc{p-q} \pa{\sumo jn \E\ve{f(\ep) - f(\ep_1,\ldots, -\ep_j, \ep_n)}^2}
%averge digonal at most constnt times... average edge.
\end{align}
by definition of Enflo type.
%$r=q$. Raise to the power $q$.

We show that the Key Claim~\ref{clm:pis1} implies Claim~\ref{clm:pis2}. 
\begin{proof}
Normalize so that
\[
\ve{\pa{\sumo jn \ve{\pl_j f(\ep)}_X^q}^{\rc q}}_{L_r(\{\pm 1\}^n, X)}=1.
\]
For every $s>0$, by Hahn-Banach, take $g_s^*:\{\pm 1\}^n\to X^*$ such that $\ve{g_s^*}_{L_{r^*}(\{\pm 1\}^n,X^*)}=1$ and
\[
\E\ba{g_s^*(\ep) (\De e^{-s\De} f(\ep))} = \ve{\De e^{-s\De} f}_{L_r(\{\pm 1\}^n, X)}.
\]
Write this as $\an{g_s^*, \De e^{-s\De}f}$. 

Recall that $L_r(\{\pm 1\}^n,X)^* = L_{r^*}(\{\pm 1\}^n, X^*)$. Taking $h\in L_r(\{\pm 1\}^n,X)^*$ and $g^*\in  L_{r^*}(\{\pm 1\}^n, X^*)$, we have 
\[
g^*(h)= \an{g^*,h} = \E[g^*(\ep)h(\ep)] = \E[\an{g^*,h}(\ep)].
\]

We have
\begin{align}
\De e^{-s\De} f &=\sum_{A\subeq [n]}|A|e^{-s|A|}\wh f(A)W_A\\
\ve{\De e^{-s\De}f}_{L^r(X)}&=\an{g_s^*, \sum \pl_j^2 e^{-s\De}f}\\
&=\sumo jn \an{g_s^*, \pl_j e^{-s\De}\pl_jf}\\
%use the fact that this is self-adjoint.
&=\sumo jn \an{e^{-s\De} \pl_jg_s^*, \pl_jf}\\
&=\ub{\pa{e^{-s\De} \pl_jg_s^*}_{j=1}^n}{\in L_{r^*}(\ell_{q^*}^n(X^*))} \ub{\pa{\pl_j f}_{j=1}^n}{\in L_r(\ell_q^n(X))}.
\end{align}
Note that $(L_r(\ell_q^n(X)))^* = L_{r^*}(\ell_{q^*}^n(X^*))$, so we have a pairing here.

\begin{align}
&\le \ve{ \pa{ \sumo jn \ve{e^{-s\De} \pl_j g_s^*(\ep)}_{X^*}^{q^*}}^{\rc{q}
%q^* %?
}}_{L^{r^*}(X^n)}
\ve{\pa{\sumo jn \ve{\pl_j f(\ep)}_X^q}^{\rc q}}_{L^r(X)}\\
%holder inequality, see as functionals.
&\le \fc{\ve{g_s^*}_{L_{r^*}(X^*)}}{(e^t-1)^{\fc{p^*}{q^*}}}
 &\text{by Key Claim~\ref{clm:pis1}}\\
&\lesssim \rc{(e^t-1)^{\fc{p^*}{q^*}}}.
%normalizing functional
\ve{\De e^{-s\De} f}_{L_r(X)} \lesssim_{X,p,r} \rc{(e^t-1)^{\fc{p^*}{q^*}}}.
\end{align}
Integrating,
\begin{align}
\int_0^{\iy} \De e^{-s\De} f &=\sum_{A\subeq [n]}\int_0^{\iy}|A|e^{-s|A|}\wh f(A)W_A.
\end{align}
Then
\begin{align}
\ve{f}_{L_r(X)} &= \ve{\int_0^{\iy} \De e^{-s\De} f\,ds}_{L_r(X)}\\
&\le \int_0^{\iy}\ve{\De e^{-s\De} f}_{L^r(X)}\,ds\\
&\le \iiy \fc{ds}{(e^t-1)^{\fc{p^*}{q^*}}}<\iy.
\end{align}
\end{proof}
We now prove the key claim~\ref{clm:pis1}.

\begin{proof}[Proof of Key Claim~\ref{clm:pis1}.]
This is an interpolation argument. We first introduce a natural operation.

Given $g^*:\{\pm 1\}^n\to X^*$, for every $t>0$ define a mapping $g_t^*:\{\pm 1\}^n\times \{\pm 1\}^n\to X^*$ as follows. For $\de\in \{\pm 1\}^n$, 
\begin{align}
g^*(\ep,\de) &= \sum_{A\subeq [n]} \wh g^*(A) \prod_{i\in A} (e^{-t}\ep_i + (1-e^{-t}) \de_i)\\
& = g^*(e^{-t} \ep + (1-e^{-t}) \de)
%not defined by cube, write F expansion, think of it as defined on $\R^n$. Then meaning
%hink of as real vector, replace by convex combination of
%linear interpolation between 2 cubes.
\end{align}
Think of $g^*(\ep) = \sum_{A\subeq [n]} \wh g^*(A) \prod_{i\in A}\ep_i$ (extended outside $\{\pm 1\}^n$ by interpolation).

Observe 
\begin{align}
g^*(\ep,\de)& = %\sum_{A\subeq [n]} \wh g^*(A) \sum_{B\subeq A} e^{-t|B|}(1-e^{-t})^{|A|-|B|} \prod_{i\in B} \ep_i \prod_{i\in A\bs B}\de_i\\
%&=\sum_{B\subeq [n]} e
\sum_{B\subeq [n]} e^{-t|B|} (1-e^{-t})^{n-|B|} g^*\pa{\sum_{i\in B} \ep_i e_i + \sum_{i\nin B} \de_i e_i}\\
g^*\pa{
\sum_{i\in B} \ep_ie_i + \sum_{i\nin B} \de_ie_i
}&= \sum_{A\subeq [n]} \wh{g^*}(A) W(A\cap B)(\ep)W_{A\bs B}(\de)\\
&\qquad
\sum_{B\subeq [n]} e^{-t|B|} (1-e^{-t})^{n-|B|} g^*(\sum_{i\in B} \ep_i e_i + \sum_{i\nin B} \de_i e_i)\\
&= \sum_{B\subeq [n]} e^{-t|B|}(1-e^{-t})^{n-|B|} \sum_{A\subeq [n]} \wh{g^*}(A) W_{A\cap B}(\ep)W_{A\bs B}(\de)\\
&=\sum_{A\subeq [n]}\wh{g^*}(A) \sum_{B\subeq [n]} e^{-t|B|}(1-e^{-t})^{n-|B|} W_{A\cap B}(\ep)W_{A\bs B}(\de).
\end{align}
Let $\ga= \begin{cases}
\ep_i,&\text{w.p. }e^{-t}\\
\de_i,&\text{w.p. }1-e^{-t}.
\end{cases}$.
Then
\begin{align}%\xiI
t\pa{\prod_{i\in A} \ga_i} & =\prod_{i\in A}(e^{-t}+ (1-e^{-t})\de_i\\
\sum_A \wh{g^*}(A) \E w_A(\ga) & = \E\pa{\sum_A \wh{g^*}(A)W_A(\ga)}\\
&=\sum_{B\subeq [n]} e^{-t|B|} (1-e^{-t})^{n-|B|}g^*(\ga)
%product number of terms 
\end{align}
Instead of a linear interpolation, we did a random interpolation.

What is $\ve{g_t^*}_{L_{r^*}(\{\pm 1\}^n\times \{\pm 1\}^n, X^*)} \le \sum_{B\subeq [n]} e^{-t|B|(1-e^{-t})^{n-|B|}}$? 
%$\ve{g_t^*}_{L_{r^*}(X^*)} \le \ve{g}_{r^*}$
%\ve{g}_{r^*}
\end{proof}

\blu{4/13: Continue Theorem~\ref{thm:pisier}}. 

We're proving a famous and nontrivial theorem, so there are more computations (this is the only proof that's known). 

We previously reduced everything to the following ``Key Claim'' \ref{clm:pis1}. 

Fix $t > 0$. Define $g_t^*: \{\pm 1\}^n \times \{\pm 1\}^n \to X^*$. Then 
\bal
g_t^*(\ep, \delta) &= g^*(e^{-t}\ep + (1 - e^{-t})\delta)
\\
&= \sum_{A \subseteq \left\{1, \cdots, n\right\}} \hat{g}^*(A) \prod_{i \in A} \left(e^{-t}\ep_i + (1 - e^{-t})\delta_i\right)
\end{align*}

Then it's a fact that $\|g_t^*\|_{L_{r^*}(\{\pm 1\}^n \times \{\pm 1\}^n, X^*) \leq \|g^*\|_{L_r(\{\pm 1\}^n, X^*)}}$ which is true for any Banach space from a Bernoulli interpretation. 

Let us examine the Fourier expansion of the original definition of $g^*$ in the variable $\delta_i$.
What is the linear part in $\de$? We have 
\bal
g_t^*(\ep, \de) &= \sum_{i = 1}^n \de_i \left(\sum_{A \subseteq \{1, \cdots, n\}, i \in A}(1 - e^{-t(|A| - 1)}\ep_j)\right) + \Phi(\ep, \de)
\end{align*}
where $\Phi$ is the remaining part. The way to write this is to write $\E\Phi(\ep, \de) \de_i = 0$, for all $i = 1, \cdots, n$ since we know the extra part is orthogonal to all the linear parts, since the Walsh function determines an orthogonal basis. 

So 
\bal
\sum_{A \subseteq \{1, \cdots, n\}, i \in A} (1 - e^{-t})e^{-t|A|} e^t\prod_{j \in A\setminus\{i\}}\hat{g}^*(A) &= (e^t - 1)\ep_i e^{-t\De}\partial_i g^*
\end{align*}

What does $\partial_i$ do to $g^*$? It only keeps the $i$s which belong to it, and it keeps it with the same coefficient. Then you hit it with $e^{-t\De}$, which corresponds to $e^{-t|A|}$. Then the last part is just the Walsh function. 

All in all, you get that $g_t^*(\ep, \de) = (e^t - 1)\sum_{i = 1}^n \de_i\ep_ie^{-t\De}\partial_ig^*(\ep) + \Phi(\ep, \de)$. 

Now fix $\ep \in \{\pm 1\}^n$. Recall the dual formulation of Rademacher type $p$: For every function $h^*: \{\pm 1\}^n \to X^*$, compute $\|(\sum_{i = 1}^n \|\hat{h}^*(i)\|_X^{p^*})^{1/p^*}\|_{L_{r^*}(\{\pm 1\}^n, X^*)} \leq \|h^*\|_{L_{r^*}(\{\pm 1\}^n, X^*)}$. 

Then, applying this inequality to our situation with $g_t^*(\ep, \de)$ where we have fixed $\ep$, we get
\bal
\left\|\left(\sum_{i = 1}^n \|e^{-t\De}\partial_ig^*(\ep)\|_{X^*}^{p^*}\right)^{1/p^*}\right\|_{L_{r^*}(\{\pm 1\}^n, X^*)} &\lesssim (e^t - 1)^{-1}\left(\E_{\de} \|g_t^*(\ep, \de)\|_{X^*}^{r^*}\right)
\end{align*}

Then, we get 
\bal
\left\|\left(\sum_{i = 1}^n \|e^{-t\De}\partial_ig^*\|_{X^*}^{p^*}\right)^{1/p^*}\right\|_{L_{r^*}(\{\pm 1\}^n, X^*)} &\lesssim (e^t - 1)^{-1}\|g_t^*\|_{L_{r^*}(\{\pm 1\}^n \times \{\pm 1\}^n, X^*)}
\\
&\lesssim (e^t - 1)^{-1}\|g\|_{L_{r^*}(\{\pm 1\}^n, X^*)}
\end{align*}

Now we want to take $t \to \infty$. Let us look at $\|e^{-t\De}\partial_i g^*\|_{L_{\infty}(\{\pm 1\}^n, X^*)}$. $e^{-t\De}$ is a contraction, and so does $\partial_i$. We proved the first term is an averaging operator, which does not decrease norms. We also have that $\partial_i$ is averaging a difference of two values, so it also does not decrease norm. Therefore, we can put a max on it and still get our inequality: 
\bal
\max_{1 \leq i \leq n} \|e^{-t\De}\partial_i g^*\|_{L_{\infty}(\{\pm 1\}^n, X^*)} \leq \|g^*\|_{L_{\infty}(\{\pm 1\}^n, X^*)}
\end{align*}

Then, 
\bal
\left\|\left(\sum_{i = 1}^n \|e^{-t\De}\partial_ig^*\|_{X^*}^{a}\right)^{1/a}\right\|_{L_b(\{\pm 1\}^n, X^*)} &= \|(\|e^{-t\De}\partial_i g^*\|_{X^*})_{i = 1}^n\|_{L_b(l_a^n(X^*))}
\end{align*}

Define $S: L_{r^*}(X^*) \to L_{r^*}(l_{p^*}^n(X^*))$, and it maps 
\bal
S(g^*) = (e^{-t\De}\partial_ig^*(\ep))_{i = 1}^n
\end{align*}
The first inequality is the same thing as saying the operator norm of $S$
\bal
\|S\|_{L_{r^*}(X^*) \to L_{r^*}(l_{p^*}^n(X^*))} \lesssim \fc{1}{e^t - 1}
\end{align*}

We also know that 
\bal
\|S\|_{L_{\infty}(X^*) \to L_{\infty}(l_{\infty}^n(X^*))} \leq 1
\end{align*}

We now want to interpolate these two statements ($r^*$ and $\infty$) to arrive at $q^*$, as in the last remaining portion of the proof which is left.

Define $\theta \in [0, 1]$ by $\fc{1}{q^*} = \fc{\theta}{p^*} + \fc{1 - \theta}{\infty} = \fc{\theta}{p^*}$, so $\theta = \fc{p^*}{q^*}$. 

By the vector-valued Riesz Interpolation Theorem (proof is identical to real-valued functions) (this is in textbooks), 

\bal 
\|S\|_{L_{a^*}(X^*) \to L_{a^*}(l_{q^*}^n(X^*))} \lesssim \fc{1}{(e^t - 1)^{\theta}}
\end{align*}
provided $\fc{1}{a^*} = \fc{\theta}{r^*} + \fc{1 - \theta}{\infty}$, or $a^* = \fc{r^*}{\theta} = \fc{r^*q^*}{p^*}$. $\fc{q^*}{p^*} > 1$, as $r$ ranges from $1 \to \infty$, so does $r^*$. In Pisier's paper, he says we can get any $a^*$ that we want. However, this seems to be false. But this does not affect us: If we choose $r = p$, then we get $a = q$, which is all we needed to finish the proof. 

So in \ref{clm:pis1}, we take $r^* > \fc{q^*}{p^*}$, and this completes the claim. 

Later I will either prove or give a reference for the interpolation portion of the argument. 

Remember what we proved here and the definition of Enflo type $p$. You assign $2^n$ points in Banach space, to every vertex of the hyper cube. We deduce the volume of parallelpiped for any number of points. The open question is do we need to pass to this smaller value, and we needed it to get the integral to converge. 

\chapter{Grothendieck's Inequality}

\section{Grothendieck's Inequality}
Next week is student presentations, let's give some facts about Grothendieck's inequality. Let's prove the big Grothendieck theorem (this has books and books of consequences). Applications of Grothendieck's inequality is a great topic for a separate course. 

The big Grothendieck Inequality is the following:

\begin{thm} Big Grothendeick. \\
There exists a universal constant $K_G$ (the Grothendieck constant) such that the following holds. Let $A = (a_{ij}) \in M_{m \times n}(\R)$, and $x_1, \ldots, x_m, y_1, \ldots, y_n$ be unit vectors in a Hilbert space $H$. Then there exist signs $\ep_1, \ldots, \ep_n, \de_1, \ldots, \de_n \in \{\pm 1\}^n$ such that if you look at the bilinear form 
\[
\sum_{i, j} a_{ij}\langle x_i, y_j \rangle \le  K_G \sum_{i,j} {a_{ij}\ep_i\de_j}
\]
\end{thm}
So whenever you have a matrix, and two sets of high-dimensional vectors in a Hilbert space, there is at most a universal constant (less than $2$ or $3$) times the same thing but with signs. 

Let's give a geometric interpretation. Consider the following convex bodies in $(\R^n)^2$. So $A$ is the convex hull of all matrices of the form $conv(\left\{(\ep_i\de_j): \ep_1, \cdots, \ep_m, \de_1, \cdots, \de_n \in \{\pm 1\} \right\}) \subseteq M_{m \times n}(\R) \cong \R^{mn}$. These matrices all have entries $\pm 1$. 
This is a polytope. 

$B = conv(\left\{\langle x_i, y_j \rangle: x_i, y_j \txtn{ unit vectors in a Hilbert space } \right\}) \subseteq M_{m \times n}(\R)$. It's obvious that $B$ contains $A$ since $A$ is a restriction to $\pm 1$ entries. 

Grothendieck says the latter half of the containment $A \subseteq B \subseteq K_G A$. If there is a point outside $K_G A$, you can find a separating hyperplane, which is a matrix. If $B$ is not in $K_G A$, then there exists $\langle x_i, y_j \rangle \not\in K_G A$ which means by separation there exists a matrix $a_{ij}$ such that $\sum a_{ij} \langle x_i, y_j > K_G\sum a_{ij}c_{ij}$ for all $c_{ij} \in A$, which is a contradiction by Grothendieck's theorem taking $c_{ij} = \ep_i\de_j$. 

We previously proved $l_{\infty} \to l_2$, the following is a harder theorem: 
\begin{lem} 
Every linear operator $T: l_{\infty}^n \to l_1^n$ satisfies for all $x_1, \cdots, x_m \in \ell_{\infty}^n$. 
\bal
\left(\sum_{i = 1}^n \|Tx_i\|_{l_1^n}^2 \right)^{1/2} &\leq K_G \cdot \|T\|_{l_{\infty}^n \to l_1^n} \max_{\|x\|_1 \leq 1, x_i \in l_1^n}\left( \sum_{i = 1}^n \langle x, x_i \rangle^2\right)^{1/2}
\end{align*}
\end{lem}

As an exercise, see how the lemma follows from Grothendieck, and how the little Grothendieck inequality follows from the big one in the case where $a_{ij}$ is positive definite (so it's just a special case; there the best constant is $\sqrt{\pi/2}$).

Note that $\max_{\|x\|_1 \leq 1} \langle x, x_i \rangle$ just gives $\|x_i\|_{\infty}$. But Grothendieck says for any number of $x_1, \cdots x_m$, we can for free improve our norm bound by a constant universal factor. 

In the little Grothendieck inequality, what we did was prove the above fact for an operator for $l_1 \to l_2$, and we deduced Pietch domination from that conclusion: There is a probability measure on the unit ball of $\ell_1^n$ such that $\|Tx\|_1^2 \leq K_G^2\|T\|_{\ell_{\infty}^n \to \ell_1^n} \int_{B_{\ell_1^n}} \langle y^*, x \rangle^2 d\mu(y^*)$. If you know there exists such a probability measure, you know the above lemma, since you just do this for each $x_i$, and this is at most the maximum so you can just multiply. 

This is an amazing fact though: Look at $T: \ell_{\infty}^n \to \ell_1^n$, and look at $L_2(\mu, )$. This is saying that if you think of an element in $\ell_{\infty}$ as a bounded function on the $L_2$ unit ball of its dual. You can think like this for any Banach space. Then, we have a diagram mapping from $\ell_2^n \to L_2(\mu)$ by identity into a Hilbert space subspace $H$ of $L_2(\mu)$, and $S: H \to \ell_1^n$. You got these operators to factor through a Hilbert space $H$, so we form a commutative diagram and the norm of $S$ is at most $\|S\| \leq K_G\|T\|$. 

This is how this is used. These theorems give you for free a Hilbert space out of nothing, and we use this a lot. After the presentations, from just this duality consequence, I'll prove one or two results in harmonic analysis and another result in geometry, and it really looks like magic. You can end up getting things like Parseval for free. We already saw the power of this in Restricted Invertibility. 

Let's begin a proof of Grothendieck's inequality. 
\begin{proof}
We will give a proof getting $K_G \leq \fc{\pi}{2\log(1 + \sqrt{2})} = 1.7\cdots$ (this is not from the original theorem). We don't actually know what $K_G$ is. People don't really care what $K_G$ is beyond the first digit. 
Grothendieck was interested in the worst configuration of points on the sphere, but it would not tell us what the configuration of points is. We want to know the actual number, or some description of the points. We know this is not the best bound since it doesn't give the worst configuration of points. 

You do the following: The input was points $x_1, \cdots, x_m, y_1, \cdots, y_n \in $ unit sphere in Hilbert space $H$. We will construct a new Hilbert space $H'$ and new unit vectors $x_i', y_j'$ such that if $z$ is uniform over the sphere of $H'$ according to surface area measure, then the expectation of $\E\txtn{sign}(\langle z, x_i' \rangle) *\txtn{sign}(\langle z, y_j'\rangle) = \fc{2\log(1 + \sqrt{2})}{\pi} \langle x_i, y_j \rangle$ for all $i, j$. 

So we have a sphere with points $x_i, y_j$, and there is a way to nonlinearly transform into a sphere in the same dimension with $x_i', y_j'$. On the new sphere, if you take a uniformly random direction on the sphere, and look at a hyperplane defined by $z$, then the sign just indicates whether $x_i', y_j'$ are on the same or opposite sides. Let's call $\ep_i$ the first ``random sign'', and $\de_j$ the second ``random sign''. Then, $\E \sum a_{ij} \ep_i\de_j = \fc{2\log (1 + \sqrt{2})}{\pi} \sum a_{ij} \langle x_i, y_j \rangle$. So there exists an instance (random construction) of $\ep_i, \de_j$ such that $\sum a_{ij} \langle x_i, y_j \rangle \leq \fc{\pi}{2\log (1 + \sqrt{2})} \sum a_{ij} \langle x_i, y_j \rangle$. If you succeed in making a bigger constant bound when bounding the expectation of the multiplication of the signs, you get a better Grothendieck constant. For the argument we will give, we will see that the exact number we get will be the best constant. 

\end{proof}



\blu{Undergrad Presentations}. 

\section{Grothendieck's Inequality for Graphs: Arka and Yuval}

We're going to talk about upper bounds for the Grothendieck constant for quadratic forms on graphs.

\begin{thm} Grothendieck's Inequality. \\
\bal
\sum_{i = 1}^n \sum_{i = 1}^m a_{ij} \langle x_i, y_i \rangle &\leq K_G \sum_{i = 1}^n \sum_{i = 1}^m a_{ij} \ep_i \de_j
\end{align*}
where $x_i, y_j$ are on the sphere. This is a bipartite graph where $x_i$s form one side and $y_j$s form the other side. Then you assign arbitrary weights $a_{ij}$. So you can consider arbitrary graphs on $n$ matrices. 
\end{thm}

For every $x_i, y_j$, there exist signs $\ep_i, \de_j$ such that
\bal
\sum_{i, j \in \{1, \cdots, n\} } a_{ij} \langle x_i, y_j \rangle &\leq C \log n \sum a_{ij} \ep_i \de_j
\end{align*}

We will prove an exact bound on the complete graph, where the Grothendieck constant is $C \log n$. 

For a general graph, we will prove that $K(G) = \mc{O}(\log \Theta(G))$, where we will define the $\Theta$ function of a graph later. 

\begin{df} $K(G)$ is the least constant $K$ s.t. for all matrices $A: V \times V \to \R$, 
\bal
\txtn{sup}_{f: V \to S^{n - 1}} \sum_{(u, v) \in E} A(u, v) \langle f(u), f(v) \rangle &\leq K \txtn{sup}_{\varphi: V \to E} \sum_{(u, v) \in E} A(u, v) \varphi(u)\varphi(v)
\end{align*}
\end{df}

\begin{df} The Gram representation constant of $G$. \\
Denote by $R(G)$ the infimum over constants $R$ s.t. for every 
$f: V \to S^{n - 1}$. There exists $F:V \to L_{\infty}^{[0, 1]}$
such that for every $v \in V$ we have $\|F(v)\|_{\infty} \leq R$ 
and $\langle f(u), f(v) \rangle = \langle F(u), F(v) \rangle = \int_0^1  F(u)(t)F(v)(t) dt$. 
For $(u, v)$ which is an edge, you find the constant $R$ so that you can embed functions $f$ into $F$ such that the $\ell_{\infty}$ norm is less than an explicit constant.
\end{df}

\begin{lem} Let $G$ be a loopless graph. Then $K(G) = R(G)^2$. 
\end{lem}
\begin{proof}
Fix $R > R(G)$ and $f: V \to S^{n - 1}$. Then there exists $F: V \to L_{\infty}[0, 1]$ such that
for every $v \in V$, we have $\|F(v)\|_{\infty} \leq R$ and for $(u, v) \in E$ $\langle f(u), f(v) \rangle \leq \langle F(u), F(v) \rangle$. 
Then, 
\bal 
\sum_{(u, v) \in E} A(u, v) \langle f(u), f(v) \rangle = \sum_{(u, v) \in E} A(u, v)\langle F(u), F(v) \rangle &= \int \sum_{(u, v) \in E} A(u, v)F(u)(t) F(v)(t) dt
\\
&\leq \int \txtn{sup}_{g: V \to [-R, R]} \sum A(u, v) g(u) g(v) dt
\end{align*}

We use definition and linearity to reverse the sum and integral. Now we use the loopless assumption to get to the definition of the right side of the Grothendieck inequality. We just need to fix the $[-R, R]$ to $[-1, 1]$. 
We then get the last term above is equivalent to $R^2 \txtn{sup}_{\varphi: V \to [-1, 1]} \sum_{(u, v) \in E} A(u, v)\varphi(u)\varphi(v)$, which completes the proof. 

Now, for each $f: V \to S^{n -1}$, consider $M(f) \subseteq \R^{|E|} = (\langle f(u), f(v))_{(u, v) \in E}$. For each $\varphi: V \to \{-1, 1\}$, define $M(\varphi) = (\varphi(u), \varphi(v))_{(u, v)} \in \R^{\varphi}$. Now we use the convex geometry interpretation of Grothendieck from the last class. Mimicking it, we write 
\bal 
\txtn{conv}\left\{ M(\varphi): \varphi: V \to \{-1, 1\}\right\} \subseteq \txtn{conv}\left\{M(f): f:V \to S^{n -1} \right\}
\end{align*}
The implication of Grothendieck's inequality gives us 
\bal
\txtn{conv}\left\{M(f): f: V \to S^{n - 1} \right\} \subseteq K(G) \cdot \txtn{conv}\left\{ M(\varphi): \varphi: V \to \{-1, 1\}\right\} 
\end{align*}

So there exist weights $\{\lambda_g: g: V \to \{-1, 1\}\}$ which satisfy $\sum_{q: V \to \{-1, 1\}} \lambda_g = 1$, $\lambda_g \geq 0$ and for all $(u, v) \in E$ $\langle f(u), f(v) \rangle = \sum_{q:v \to \{-1, 1\}} \lambda_g g(u)g(v) K(G)$. 

Now consider 
\[
F(u) = \sqrt{K(g)} \cdot g(u) [\lambda_1 + \cdots + \lambda_{g - 1}, \lambda_1 + \cdots + \lambda_g]
\]
Then, 
\bal
\langle F(u), F(v) \rangle = \sum_{g: \{V \to \{-1, 1\}\}} K(G) g(u)g(v) \lambda_g
\end{align*}
Then we consider our interval becomes $[-\sqrt{K(G)}, \sqrt{K(G)}]$, and thus $R(G) \leq \sqrt{K(G)}$ and $R(G)^2 \leq K(G)$. This proof can be modified for graphs with loops, but this is not quite true. 
\end{proof}

An obvious corollary is that if $H$ is a subgraph of $G$, then $R(H) \leq R(G)$, and $K(H) \leq K(G)$. This inequality is not obvious from the Grothendieck inequality directly, but is obvious going with our point of view. 

\begin{lem} Let $K_n^{\circ}$ denote the complete graph on $n$-vertices with loops. Then 
$R(K_n^{\circ}) = \mc{O}(\sqrt{\log n})$. 
\end{lem}
\begin{proof}
Let $\sigma$ be the normalized surface measure on $S^{n - 1}$. By computation, there exists $c$ s.t. $\sigma\left(\left\{ x \in S^{n - 1}: \|x\|_{\infty} \leq c\sqrt{\frac{\log n}{n}}\right\}\right) \geq 1 - \frac{1}{2n}$, which can be calculated through integration. 

When you get a function $f$ on the sphere to the $n - 1$ dimensional sphere, you want to find a rotation so that all these vectors have low coordinate vector value. We basically use the union bound. For each of the $n$ vectors, the probability you get a vector with low valued last coordinate is $1 - 1/(2n)$, do this for all the vectors and you get probability greater than $1/2$. Then you can magnify this to get almost surely.

For every $x \in S^{n - 1}$, the random variable on the orthogonal group $O(n)$ given by $U\to Ux$ is uniformly distributed on $S^{n - 1}$.  Thus for every $f: V \to S^{n - 1}$, there is a rotation $U \in O(n)$ such that $\forall v \in V$ $\|U(f(v))\|_{\infty} \leq c\sqrt{\frac{\log n}{n}}$. 

We want $F(v)$ to be equal to the $j^{th}$ coordinate of $Uf(v)$ on interval of length $1/n$. I.e., 
Let $F(u)(t) = (U(f(v))) \sqrt{n}$ on $\frac{j - 1}{n} \leq t \frac{j}{n}$. 

Thus $R \leq c \sqrt{\log n}$. 
\end{proof}

Now I will prove the thing mentioned at the beginning: 
\begin{thm} $K(G) \leq \log \chi(G)$, where $\chi(G)$ is the chromatic number. 
\end{thm}

This theorem generalizes since on bipartite graphs $\chi(G)$ is a constant. 

One thing we want to observe about the Grothendieck inequality is that it only cares about the Hilbert space structure. So we will just prove this in one specific (nice) Hilbert space and then we'll be done. Fix a probability space $(\Omega, P)$ such that $g_1, \cdots, g_n$ be i.i.d. standard Gaussians on $\Omega$ (for instance, $\Omega$ is the infinite product of $\R$). 

Now define the Gaussian Hilbert space $H = \left\{\sum_{i = 1}^{\infty} a_i g_i: \sum a_i^2 < \infty\right\} \subseteq L^2(\Omega)$. Every function in here will have mean zero since these are linear combinations of Gaussians. Note that the unit ball $B(H)$ consists of all Gaussian distributions with mean zero and variance at most $1$ (since the variance is norm squared). 

Let $\Gamma$ be the left hand side of the Grothendieck inequality: $\txtn{sup}_{f: V \to \{-1, 1\}} \sum_{(u, v) \in E} A(u, v) \langle f(u), f(v) \rangle$, and let $\Delta = \txtn{sup}_{f: V \to \{-1, 1\}} \sum_{(u, v) \in E} A(u, v)\varphi(u)\varphi(v)$. 

\begin{df} Truncation. \\
For all $M > 0$, $\psi \in L_2(\Omega)$, define the truncation of $\psi$ at $M$ to be
\[
\psi^M(x) = 
\begin{cases}
\psi(x) & |\psi(x)| \leq M \\
M & \psi(x) \geq M \\
-M & \psi(x) \leq -M
\end{cases}
\] 
\end{df}
We're just cutting things off at the interval $[-M, M]$. 
So fix $f$ maximizing $\Gamma$. 

\begin{lem} There exists Hilbert space $H$, $h: V \to H$, $M > 0$ with $\|h(v)\|_H^2 \leq 1/2$ for all vertices $v \in V$, $M \lesssim \sqrt{\log \chi(G)}$, and we can now write $\Gamma$ as a sum over all edges $\Gamma = \sum_{(u, v)} A(u, v) \langle f(u)^M, f(v)^M \rangle + \sum_{(u, v) \in E} A(u, v) \langle h(u), h(v) \rangle$, where the second term is an error term. 
\end{lem}
\begin{proof}
Let $k = \chi(G)$. A fact for all $s: V \to l_2$ s.t. $\langle s(u), s(v) \rangle = \frac{-1}{k - 1}$ for all $(u, v) \in E$, and $\|s(u)\| = 1$. 

Define another Hilbert space $U = l_2 \oplus \R$. Define $t, \hat{t}$, two functions from $V \to U$. They are defined as follows: 
\[
t(u) = (\sqrt{\frac{k - 1}{k} s(u)}) \oplus (\frac{1}{\sqrt{k}}e_1)
\]
\[
\hat{t}(u) = (-\sqrt{\frac{k - 1}{k}}s(u)) \oplus (\frac{1}{\sqrt{k}}e_1)
\]
Now, $t(u), \hat{t}(u)$ are unit vectors for all $(u, v) \in E$. Then $\langle t(u), t(v) \rangle = \langle \hat(t)(u), \hat{t}(v) \rangle = 0$. 
We also have $\langle t(u), \hat{t}(v) \rangle = \frac{2}{k}$. 

Our goal is to prove that such a function exists. Set $H' = U \otimes L_2(\Omega)$. We now write 
down a function 
\[
h(u) = \frac{1}{4}t(u) \otimes (f(u) + f(u)^M) + k\hat{t}(u) \otimes (f(u) - f(u)^M)
\]
It turns out that this function does everything we want. 
A couple of key facts: It's easy to check that the definition of $\Gamma$ holds, defined in terms of $h(u)$ (for the error term). Checking $\|h(v)\|_H^2 \leq 1/2$ holds is also possible. You can bound $\|h(u)\|^2 \leq (1/2 + k\|f(u) - f(u)^M\|)^2$ after evaluating inner products. Now we use th eproperty of the Hilbert space. $f(u)$ is in the ball of $H$, with mean zero and variance at most $1$. Then, $\|f(v) - f(v)^M\|^2$ is the probability that the Gaussian falls outside the interval $[-M, M]$. Therefore, from basic probability theory, 
\[
\|f(v) - f(v)^M\|^2 \leq \frac{\sqrt{2}{\pi}} \int_{M}^{\infty} x^2 e^{-x^2/2} dx \leq 2Me^{-M^2/2}
\]
where the last part follows by calculus. This is great since if we pick $M \sim \sqrt{\log k}$. So this will decay super quickly. Picking $M = 8\sqrt{\log k}$ gives that the entire thing will be $\leq 1/2$. That proves the lemma, so we're done.
\end{proof}

This lemma is actually all we need. Suppose we have proved this. Now we can say the following. We can bound the first term by taking expectations: 
$\sum_{(u, v) \in E} A(u, v)\langle f(u)^M, f(v)^M \rangle  = \E \sum_{(u, v) \in E} A(u, v)f(u)^Mf(v)^M \leq M^2 \Delta$ by the definition of $\Delta$. 

For the second term, we write $\sum_{(u, v) \in E} A(u, v)\langle h(u), h(v) \rangle \leq (\max_{v \in V} \|h(v)\|^2)\Gamma \leq \frac{1}{2}\Gamma$. 
by re-scaling. You need to pull out each of these maxima separately, using linearity each time. You can also multiply each thing by $\sqrt{2}$ makes it be in $B(H)$. You just have to multiply by $\sqrt{2}$ and divide by $\sqrt{2}$ and they're not dependent anymore. By the conditions of the lemma, our $h$ has norm at most $1/2$, so the 
This entire thing is Hilbert space independent, so the whole thing is $< \frac{\Gamma}{2}$. Thus $\Gamma \leq M^2\Delta + \frac{1}{2}\Gamma \implies \Gamma \leq 2M^2\Delta$. 

This implies $\Gamma \lesssim \log \chi(G)\Delta$, which is exactly what we wanted to say, since the left hand side of the Grothendieck inequality is the first term, and the right hand side of the Grothendieck inequality is the second term with Grothendieck constant $\log \chi(G)$. 

\section{Noncommutative Version of Grothendieck's Inequality - Thomas and Fred}

First we'll phrase the classical Grothendieck inequality in terms of an optimization problem. Given $A \in M_n(\R)$, we can consider 
\[
\max_{\ep_i, \de_j \in } \sum_{i, j = 1}^n A_{ij} \ep_i\de_j
\]
In general this is hard to solve, so we do a semidefinite relaxation
\[
\txtn{sup}_{d \txtn{ dimensions}} \txtn{sup}_{x, y \in (S^{d - 1})^n}  \sum_{i, j = 1}^n A_{ij}\langle x_i, y_j \rangle
\]
which implies that it's polynomially solveable, and Grothendieck inequality ensures you're only a constant factor off from the best. 

We can give a generalization to tensors. Given $M \in M_n(M_n(\R))$ consider 
\[
\txtn{sup}_{u, v \in O_n} \sum_{i, j, k, l = 1}^n M_{ijkl} U_{ij}V_{kl}
\]
Set $M_{iijj} = A_{ij}$, then we obtain
\[
\txtn{sup} \sum A_{ij} U_{ii}V_{jj} = \sup_{x, y \in \{-1, 1\}^n} \sum_{i, j = 1}^n A_{ij}x_iy_j = \max_{\ep, \de \in \{-1, 1\}^n} \sum A_{ij} \ep_i \de_j
\]

We relax to SDP over $\R$ of $M$: 
\[
\txtn{sup}_{d \in \N} \sup_{X, Y \in O_n(\R^d)} \sum_{i, j, k, l = 1}^n M_{ijkl} \langle X_{ij}, Y_{kl} \rangle
\]
Recall that $U \in O_n$ means that 
\[
\sum_{k = 1}^n U_{ik} U_{jk} = \sum_{k = 1}^n U_{ki} U_{kj} = \delta_{ij}
\]
If $X \in M_n(\R^d)$, let $XX^*, X^*X \in M_n(\R)$ defined by 
\[
(XX^*)_{ij} = \sum_{k = 1}^n \langle X_{ik}, Y_{jk} \rangle
\]
\[
(X^*X)_{ij} = \sum_{k = 1}^n \langle X_{ki}, Y_{kj} \rangle
\]
Then $O_n(\R^d) = \left\{X: M_n(\R^d): XX^* = X^*X = I\right\}$.

We want to say something about when we relax the search space, we get within a constant factor of the non-relaxed version of the program. We will prove this in the complex case.

We will write 
\[
\txtn{Opt}_{\C}(M) = \txtn{sup}_{U, V \in U_n} |\sum_{i, j, k, l = 1}^n M_{ijkl} U_{ij} \overline{V}_{kl}|
\]
The fact that Opt is less than the SDP, Pisier proved thirty years after Grothendieck conjectured it. 
What we will actually prove is that the SDP solution is at most twice the optimal: 
\[
\txtn{SDP}_{\C}(M) \leq 2 \txtn{Opt}_{\C}(M)
\]

\begin{thm} Fix $n, d \in \N$ and $\ep \in (0, 1)$. Suppose we have $M \in M_n(M_n(\C))$ and $X, Y \in U_n(\C^d)$ such that 
\[
|\sum_{i, j, k , l = 1}^n M_{ijkl} \langle X_{ij}, Y_{kl} \rangle | \geq (1 - \ep) \txtn{SDP}_{\C}(M)
\]
\end{thm}

So what we're saying is suppose some SDP algorithm gave a solution satisfying this from input $X, Y \in U_n(\C^d)$. Then we can give a rounding algorithm which will output $A, B \in U_n$ such that 
\[
\E |\sum_{i, j, k, l = 1}^n M_{ijkl} A_{ij}\overline{B}_{kl}| \geq (1/2 - \ep)\txtn{SDP}_{\C}(M)
\]

Now what's the algorithm? It's a slightly clever version of projection. We have $X, Y \in U_n(\C)$, and we want to get to actual unitary matrices. First sample $z \in \{1, -1, i, -i\}^d$ (complex unit cube). Then $x_{z} = \frac{1}{\sqrt{2}}\langle X, z \rangle$ (take inner products columnwise), similarly $Y_z = \frac{1}{\sqrt{2}}\langle Y, z \rangle$. Now we take the Polar Decomposition $A = U\Sigma V^*$ where $U, V$ are unitary. The polar decomposition is just $A = (UV^*)(V\Sigma V^*)$, and then the first guy is unitary, and the second guy is PSD (think of it as $e^{i\theta} * r$). 

Then we have $(A, B) = (U_z|X_z|^{it}, V_z|Y_z|^{-it})$ where $t$ is sampled from the hyperbolic secant distribution. It's very similar to a normal distribution. The PDF is more precisely
\[
\varphi(t) = \frac{1}{2} \txtn{sech}(\frac{\pi}{2}t) = \frac{1}{e^{\pi t/2} + e^{-\pi t/2}}
\]
When you have a positive definite matrix, you can raise it to imaginary powers, so we're rotating only the positive semidefinite part, and keeping the rotation side $(UV^*)$. 
Note that the $(A, B)$ are unitary. (You can also think of this as raising diagonals to $it$, these necessarily have eigenvalue of magnitude $1$).


\blu{4-20-16}

For $X,Y\in U_n(\C^d)$, do the following.
\begin{enumerate}
\item
Sample $z\in \{\pm 1,\pm i\}$ uniformly. Sample $t$ from the hyperbolic secant distribution.
\item
Let $X_z=\rc{\sqrt 2}\an{x,z}$, $Y_Z=\rc{\sqrt 2} \an{Y,z}$.
\item
Let $(A,B):=(U_Z|X_Z|^{it}, V_Z|Y_Z|^{-it})\in U_n\times U_n$ where $X_Z=U_Z|X_Z|, Y_Z=U_Z|Y_Z|$ (polar decomposition).
\end{enumerate}•
$M(X,Y)= \sumo{i,j,k,l}n M_{ijkl} \an{X_{ij},Y_{kl}}$. We want to show
\[
\E_t[M(A,B)] \ge \pa{\rc 2-\ep} M(X,Y) \ge (\rc2 - \ep)\text{SDP}_\C(M)
\]
We want to show that the rounded solution still has large norm.

It's possible that $|X_Z|$ has 0 eigenvalues. One solution is to add some Gaussian noise to the original $X,Y$ because the set of non-invertible matrices is an algebraic set of measure 0. Alternatively, replace the eigenvalues by $\ep\to 0$.

We have
\[
\E_z[M(X_z,Y_z)] = \rc 2 \E_z \ba{
\sumo{r,s}d \ol{z_r} z_s \sumo{i,j,k,l}n (X_{ij})r\ol{(Y_{kl})}_s
}=\rc 2 M(X,Y).
\]

Now we analyze step 3. The characteristic function of the hyperbolic secant distribution is, for $a>0$,
\[
\E[a^{it}] = \int a^{it}e(t)\,dt = \fc{2a}{Ha^2}
\]
by doing a contour integral.
%e^{itc}, up to scalar multiples the hyperbolic secand
%char function of gaussian is a gaussian.
Then 
\begin{align}
\EE_t[a^{it}] &=2a-\EE_t [a^{2+it}]\\
\EE_t[A^{it}]&=2A - \EE_t[A^{2+it}]\\
\EE_t[A\ot \ol B] &=\EE_t[(U_z|X_z|^{it}) \ot (\ol{V_z} |Y_z|^{it})]\\
&=(U_z\ot \ol{V_z}) \EE_t [(|X_z|\ot |Y_t|)^{it}]\\
&=2X_z\ot \ol{Y_z} - \EE_t[(U_z|X_z|^{2+it})\ot (\ol{V_z|Y_z|}^{2-it})]
\end{align}
We apply $M$. 
\[
\EE_{z,t} [M(A,B)] = M(X,Y) - \EE_{z,t}[M(U_z|X_z|^{2+it}, V_z|Y_z|^{2-it})]
\]
Because $A,B\in M_n(\C)$, we can write $M(A,B)$ in terms of the tensor product, 
\begin{align}
M(A,B) &= \sumo{i,j,k,l}n A_{ij} \ol{B_{kl}}\\
&= \sumo{i,j,k,l}n M_{ijkl} (A\ot \ol B)_{(ij),(kl)}
\end{align}
\begin{clm}[Key claim]
For all $t\in \R$,
\[
\ab{\EE_z \ba{M(U_z|X_z|^{2+it}, V_z|Y_z|^{2-it})}}\le \rc2\text{SDP}_\C(M).
\]
\end{clm}
\begin{proof}
We have $F(t),G(t)\in M_n(\C^{\{\pm 1,\pm i\}^d})$.  
where
\begin{align}
(F(t)_{jk})_z&= \rc{2^d} (U_z|X_z|^{2+it})_{jk}\\
(G(t)_{jk})_z&= \rc{2^d}(V_z|Y_z|^{2-it})_{jk}.
M(F(t),G(t)) &= \rc{4^d} \sum_{z\in \{\pm 1,\pm i\}^d} M(U_z|X_z|^{2+it}, V_z|Y_z|^{2-it})\\
&=\EE_z[M(U_z|X_z|^{2+it},V_z|Y_z|^{2-it})].
\end{align}•
%arb vec of arb dimension. We want to represent this error term. entries which are vectors in huge dimensional space
%exists solution in SDP...
%use any dimension we want

\begin{lem}
$F(t),G(t)$ satisfy
\[
\max\bc{
\ve{F(t)F(t)^*}, \ve{F(t)^*F(t)}, \ve{G(t)G(t)^*}, \ve{G(t)^*G(t)}
}
\]
\end{lem}
%We can find 2 matrices in the SDP space such that whe we plug them in,
\begin{lem}
Suppose $X,Y\in \cal M_n(\C^d)$ and $\max\{\ve{XX^*}, \ve{X^*X}, \ve{YY^*}, \ve{Y^*Y}\}$. Then there exist $R,S\in U_n(\C^{d+2n^2})$ such that $M(R,S) = M(X,Y)\le 1$ for every $M\in M_n(M_n(\C))$. 
\end{lem}

From the two lemmas, there exist $R(t), S(t)\in U_n(\C^{d+2n^2})$ such that $M(R(t),S(t))=M(\sqrt 2 F(t), \sqrt 2 G(t))$. Then
\begin{align}
|M(F(t),G(t))| &= \rc 2 |M(R(t), S(t))| \le \rc 2 2\text{SDP}_\C(M)\\
F(t)F(t)^* &=\rc{4^d} \sum_{z\in \{\pm 1,\pm i\}^d} U_z|X_z|^*U_z^* \\
&=\EE_z[U_z |X_z|^* U_2^*] = \EE_z[(X_zX_z^*)].
\end{align}

\begin{clm}
For $W\in M_n(\C^d)$, define for each $v\in [d]$, $(W_r)_{ij} = (W_{ij})_r$, $W_z= \an{W,z}$. Then
\[
\EE_z [(W_zW_z^*)^2] = (WW^*)^2 + \sumo rd W_r (W^*W-W_r^*W_r) W_r^* .
%every entry, $r$th coordinate, gives complex matrix.
\]
\end{clm}
Note $W_z = \sumo rd \Si_rW_r$ and
\begin{align}
WW^*&= \sumo rd  W_r W_r^*&
W^*W&= \sumo rd W_r^*W_r.
\end{align}
We compute
\begin{align}
\EE_z[(W_z W_z^*)^2] & = \EE_z \ba{
\sumo_{p,q,r,s}d \ol{z_p}z_q\ol{z_r}z_s W_pW_q^*W_rW_s^*
}\\
&=\sumo p d W_p W_p^* W_p W_p^* + \sumr{p,q=1}{p\ne q}^n (W_pW_q^*W_qW_q^* + W_pW_q^*W_qW_p^*)
\end{align}
\begin{multline}
\sumo{p,q}d W_pW_p^* W_qW_q^* + \sumo{p,q}d W_pW_q^*W_qW_p^* - \sumo pd W_pW_p^*W_pW_p^*
\\=\pa{\sumo pd W_pW_p^*}^2 + \sumo pd W_p\pa{\sumo qd W_q^*W_q}W_p^* - \sumo pd W_pW_p^*W_pW_p^*.
\end{multline}
Apply the claim with $W=\rc{\sqrt 2}X$. Recall that $XX^*=X^*X=I$. Then 
\[
F(t)F(t)^* + \rc 4\sumo rd X_rX_r^*X_rX_r^* = \rc 2 I = F(t)F(t)^* + \rc 4 \sumo rd X_r^*X_rX_r^*X_r
\]
and similarly for $G$.

\begin{proof}[Proof of Lemma 2]
%$\sqrt 2$.
Let $A=I-XX^*$, $B=I-X^*X$.  Note $A,B\succeq 0$, $\Tr(A) = \Tr(\ol B)$. We have
\begin{align}
A&=\sumo in \la_i (v_iv_i^*)\\
B&= \sumo jn \mu_j(v_jv_j^*)\\
\si &= \sumo in \la_j  =\sumo jn \mu_j\\
R&= X\opl \pa{
\bigopl_{i,j=1}^n \sfc{\la_i\mu_j}{\si}(u_iv_j^*)
} \opl O_{M_n(\C^{n^2})} \in M_n(\C^d\opl \C^{n^2}\opl \C^{n^2})\\
S&=Y\opl O_{M_n(\C^{n^2})} \opl \pa{\bigopl_{i,j=1}^n \sfc{\la_i\mu_j}{\si}(u_iv_j^*)}
\end{align}
Check $R\in U_n(\C^{d+2n^2})$, 
\begin{align}
RR^* &= XX^*+A\\
R^*R&= X^*X +B\\
M(R,S)&= M(X,Y).
\end{align}
\end{proof}
The $\si$ disappears because $\sum \mu_j=\si$.
%Then $S=Y\opl O$
\end{proof}
%amazing inequality.
The factor 2 in the noncommutative inequality is sharp. That the answer is 2 (rather than some strange constant) means there is something going on algebraically. The hyperbolic secant is the unique distribution that makes this work.

%H used geometric methods in a dual form. The hyperbolic secant fell out. The inequalities were a few months of trial and error. His proof had infinite tensor products. Believe it's going to be true. Without the hint that it works, it's magic. The hint is that it works.


%\chapter{Lipschitz extensions from finite sets}

\blu{4-27}


\section{Improving the Grothendieck constant}
\begin{thm}[Krivine (1977)]
\[
K_G \le \fc{\pi}{2\ln(1+\sqrt 2)}\le 1.7...
\]
\end{thm}
The strategy is \ivocab{preprocessing}.  

\ig{images/4-27-1}{.25}

There exist new vectors $x_1',\ldots, x_n',y_1',\ldots, y_n'\in \bS^{2n-1}$ such that if $z\in \bS^{2n-1}$ is chosen uniformly at random, then  for all $i,j$,
\[
\EE_z\ub{(\sign\an{z,x_i'})}{\ep_i} \ub{(\sign\an{z,y_j'})}{\de_j} = \ub{\fc{2\ln(1+\sqrt 2)}{\pi}}{c}\an{x_i,y_j}.
\]
Then
\[
\sum a_{ij} \an{x_i,y_j} = \E \ba{\fc{\pi}{2\ln (1+\sqrt 2)} \sum a_{ij}\ep_i\de_j}.
\]

We will take vectors, transform them in this way, take a random point on the sphere, take a hyperplane orthogonal, and then see which side the points fall on. 

\begin{thm}[Grothendieck's identity]
For $x,y\in \bS^k$ and $z$ uniformly random on $\bS^k$,
\[
\EE_z[\sgn(\an{x,z}) \sgn(\an{y,z})] = \fc 2\pi \sin^{-1}(\an{x,y}).
\]
%collect 1 if both on the same side . 1 if on both on the sameside.
\end{thm}
\begin{proof}
This is 2-D plane geometry. 

The expression is 1 if both $x,y$ are on the same side of the line, and $-1$ if the line cuts between the angle. 
\end{proof}
%used in GW min cut.

Once we have the idea to pre-process, the rest of the proof is natural. 
\begin{proof}
\begin{align}
\EE_z\ba{\sign(\an{z,x_i'})\sign(\an{z,y_j'})}
& = \fc 2\pi \sin^{-1} (\an{x_i',y_j'}) = c\an{x_i,y_j}\\
\an{x_i',y_j'} &=  \sin\Big(\ub{\fc{\pi c}2}u\an{x_i,y_j}\Big).
\end{align}
We write the Taylor series expansion for $\sin$. 
\begin{align}
\an{x_i',y_j'} &= \sin(u\an{x_i,y_j})\\
&=\sum_{k=0}^{\iy} \fc{(u\an{x_i,y_j})^{2k+1}}{(2k+1)!} (-1)^k\\
&=\sum_{k=0}^{\iy} \fc{(-1)^k u^{2k+1}}{(2k+1)!} \an{x_i^{\ot (2k+1)}, y_j^{\ot (2k+1)}}.
\end{align}
(For $a,b\in \ell_2$, $a\ot b = (a_ib_j)$, $a^{\ot 2}=(a_ia_j)$, $b^{\ot 2} = (b_ib_j)$, $\an{a^{\ot 2},b^{\ot 2}}=\sum a_ia_j b_ib_j = \an{a,b}^2$.)

Define an infinite direct sum corresponding to these coordinates.

Define 
\begin{align}
x_i' &= \bigopl_{k=0}^{\iy} \pa{ \fc{(-1)^k u^{\fc{2k+1}2}}{\sqrt{(2k+1)!}} x_i^{\ot 2k+1}}\\
y_j' &= \bigopl_{k=0}^{\iy} \pa{ \fc{u^{\fc{2k+1}2}}{\sqrt{(2k+1)!}} x_i^{\ot 2k+1}}\in \bigopl_{k=1}^{\iy} (\R^m)^{\ot 2k+1}\\
\an{x_i',y_j'} &=\sum_{k=0}^{\iy} (-1)^k \fc{u^{2k+1}}{(2k+1)!} \an{x_i^{\ot 2k+1}, y_j^{\ot 2k+1}}.
\end{align}
%\o\bigopl{}{}
The infinite series expansion of sin generated an infinite space for us. % space

%didn't check anything about u, have to be unit vectors. 
%Why do they have to be?
%identity not true otherwise. assume get unit vectors at the end.
We check
\begin{align}
\ve{x_i'}^2 &=\sum_{k=0}^{\iy} \fc{u^{(2k+1)}}{(2k+1)!} = \sinh(u) = \fc{e^u-e^{-u}}2=1.
\end{align}•
\end{proof}
We don't care about the construction, we care about the identity. All I want is to find $x_i',y_j'$ with given $\an{x_i',y_j'}$; this can be done by a SDP. We're using this infinite argument just to show existence.

%proj into 1 d, best you can do is sine.. Best you can do is sine
%Can do better in higher dim.

A different way to see this is given by the following picture. Take a uniformly random line and look at the orthogonal projection of points on the line. Is it positive or negative. A priori we have vectors in $\R^n$ that don't have an order. 

We want to choose orientations. A natural thing to do is to take a random projection onto $\R$ and use the order on $\R$. Krevine conjectured his bound was optimal.

What is so special about positive or negative? We can partition it into any 2 measurable sets. Any partition into 2 measurable sets produces a sign. It's a nontrivial fact that no partition beats this constant. This is a fact about measure theory.

The moment you choose a partition, %took product of signs
it forced the construction of the $x',y'$. %generalize to any 2 sets $A,B$, get another analytic funciton.
The whole idea is the partition; then the preprocessing is uniquely determined; it's how we reverse-engineered the theorem.

Here's another thing you can do. What's so special about a random line? The whole problem is about finding an orientation. 

Consider a floor (plane) chosen randomly, look at shadows. If you partition the plane into 2 half-planes, you get back the same constant. We can generalize to arbitrary partitions of the plane. This is equivalent to an isoperimetric problem. In many such problems the best thing is a half-space. We can look at higher-dimensional projections. It seems unnatural to do this---except that you gain!

In $\R^2$ there is a more clever partition that beats the halfspace! Moreover, as you take the increase the dimension, eventually you will converge to the Grothendieck constant. The partitions look like fractal objects!
%$\sqrt{n}$

%even though trying to find orientations, it pays off to do crazy partition.
%This is why we don't know the second digit...

\section{Application to Fourier series}

This is a classical theorem about Fourier series. %I'll state it for a general abelian group. 
Helgason generalized it to a general abelian group.

\begin{df}
Let $S=\R/\Z$. The space of continuous functions is $C(S')$. Given $m:\Z\to \R$ (the multiplier), 
define 
\begin{align}
\La_m:C(\bS') &\to \ell_\iy\\
\La_m(f) &= (m(n)\wh f(n))_{n\in \Z}.
\end{align}
\end{df}

For which multipliers $m$ is $\La_m(f)\in \ell_1$ for every $f\in C(\bS^1)$?

A more general question is, what are the possible Fourier coefficients of a continuous functions? For example, if $\log n$ works, then $\sum |\wh f(n)||m(n)|<\iy$. This is a classical topic.

An obvious sufficient condition is  that $m\in \ell_2$, by Cauchy-Schwarz and Parseval.
\begin{align}
\sum_{n\in \Z} |m(n)\wh f(n)|& \le \pa{\sum m(n)^2}^{\rc 2} \ub{\pa{\sum |\wh f(n)|^2}^{\rc 2}}{\ve{f}_2\le \ve{f}_{\iy}}\\
\ve{\La_m(f)}_{\ell_1}&\le \ve{m}_2\ve{f}_{\iy}.
\end{align}

This theorem says the converse.
\begin{thm}[Orlicz-Paley-Sidon] %, Helgason
%Let $G$ be a compact abelian group and $G'$ be the dual group.
If $\La_m(f)\in \ell_1$ for all $f\in C(\bS^1)$, then $m\in \ell_2$.
\end{thm}
We know the fine line of when Fourier coefficients of continuous functions converge. 

We can make this theorem quantitative. Observe that if you know $\La_m:C(\bS^1) \to \ell_1$, then $\La_m$ has a closed graph (exercise). The closed graph theorem (a linear operator between Banach spaces with closed graph is bounded) says that $\ve{\La_m}<\iy$. 
% $\set{(f,\La_m(f))}{f\in C()}$

\begin{cor}
$\sum|m(n)\wh f(n)|\le K\ve{f}_{\iy}$.
\end{cor}

We show $\ve{\La_m}\le \ve{m}_2\le K_G\ve{\La_m}$.

%deduce from Groth that you get this improved more than boundedness.
We will use the following consequence of Grothendieck's inequality (proof omitted).
\begin{cor}
Let $T:\ell_{\iy}^n\to \ell_1^m$. For all $x_1,\ldots, x_n\in \ell_\iy$, 
\[
\pa{\sum_i \ve{Tx_i}_1^2}^{\rc 2} \le K_G\ve{T}  \sup_{y\in \ell_1}{\sumo in \an{y,x_i}^2}.
\]

Equivalently, there exists a probability measure $\mu$ on $[n]$ such that $\ve{Tx}\le K_G\ve{T}\int  K_G\ve{T}\pa{\int x_j^2 \,d\mu(j)}^{\rc 2}$.
\end{cor}
\begin{proof}
Use duality. To get the equivalence, use the same proof as in Piesch Domination.
\end{proof}
%piesch domination theorem was another duality.

%start with operators between spaces and get Hilbert spaces. In our case.

\begin{proof}
Given $\La_m:C(\bS^1)\to \ell_1$, there exists $\mu$ on $\bS^1$ such that for every $f$, letting $f_\te(x)=f(e^{i\te}x)$,
%discretization
\begin{align}
\pa{\sum |m(n)\wh f_\te(n)|}^2&\le K_G^2 \ve{\La_m}^2\int_{S^1} (f_\te(x))^2 \,d\mu(x)\\
%has to be uniform measure.
\pa{\sum |m(n)\wh f(n)|}^2&\le K_G^2 \ve{\La_m}^2\int_{S^1} (f(x))^2 \,d\mu(x).
\end{align}
%f x n
Apply this to the following trigonometric sum
\[
f(x) = \sum_{n=-N}^N m(n) e^{-in\te},
%multiplier fight against itself.
\]
to get
\[
\pa{\sum_{n=-N}^N m(n)^2}^2 \le K_G^2 \ve{\La}_m^2 \sum_{n=-N}^N m(n)^2.
\]
%cancelling out, get!
\end{proof}

RIP also had this kind of magic.
%duality, auxiliary measure, select random point.

\section{? presentation}


\begin{thm}
Let $(X,d)$ be a metric space with $X=A\cup B$ such that 
\begin{itemize}
\item
$A$ embeds into $\ell_2^a$ with distortion $D_A$, and
\item
$B$ embeds into $\ell_2^b$ with distortion $D_B$.
\end{itemize}
Then $X$ embeds into $\ell_2^{a+b+1}$ with distortion at most $7D_AD_B+5(D_A+D_B)$. %If $D_A=D_B=1$, then $X$ embeds into $\ell_2^{a+b+1}$ with distortion $\le 12.07$.

Furthermore, given $\psi:A\to \ell_2^a$ and $\ph_B:B\to \ell_2^b$ with $\ve{\ph_A}_{\text{Lip}}\le D_A$ and $\ve{\ph_B}_{\text{Lip}}\le D_B$, there is an embedding $\Psi:X\hra \ell_2^{a+b+1}$ with distortion $7D_AD_B + 5(D_A+D_B)$ such that $\ve{\Psi(u) - \Psi(v)}\ge\ve{\ph_A(u)-\ph_A(v)}$ for all $u,v\in A$.
\end{thm}
For $a\in A$, let $R_a = d(a,B)$ and for $b\in B$, let $R_b=d(A,b)$. 

\begin{df}
%finite distortion, then also finite distortion.
%The most surprising thing about this theorem is that it's not trivial.

For $\al>0$, $A'\subeq A$ is an \vocab{$\al$-cover} for $A$ with respect to $B$ if
\begin{enumerate}
\item
For every $a\in A$, there exists $a'\in A'$ where $R_{a'} \leq R_a$ and $d(a,a') \le \al R_a$.
\item
For distinct $a_1', a_2'\in A'$, $d(a_1', a_2')\ge \al \min(R_{a_1'},R_{a_2'})$.
%building a net, but the $\ep$ of the net is a function of the distance to the set. %density is distance to net point. if close to set, strong requirement, weaker as farther away.
\end{enumerate}
This is like a net, but the $\ep$ of the net is a function of the distance to the set. %density is distance to net point. if close to set, strong requirement, 
The requirement is weaker for points that are farther away.

\end{df}

\begin{lem}
For all $\al>0$, there is an $\al$-cover $A'$ for $A$ with respect to $B$. 
\end{lem}

\begin{proof}
Induct. The base case is $\phi$, which is clear.

Assume the lemma holds for $|A|<k$; we'll show it holds for $|A|=k$. Let $u\in A$ be the point closest to $B$. Let $Z=A\bs B_{\al R_u}(u)=:B$; we have $|Z|<|A|$. Let $Z'$ be a cover of $Z$, and let $A'=Z'\cup \{u\}$.

We show that $A'$ is an $\al$-cover. We need to show the two properties.
\begin{enumerate}
\item
Divide into cases based on whether $a\in B$ or $a\nin B$.
\item
For $a_1',a_2'\in A'$, if both are in $z'$, we're done. 

Otherwise, without loss of generality, $a_1'=u$ and $a_2' \in Z'$. 
\end{enumerate}•
\end{proof}

%infinite-dimensional hilbet space, no compactness. There is abstract nonsense you can do, but it's not just compactness. It's more like compactness of logic. It's abstract, not geometry.

\begin{lem}
Define $f:A'\to B$ to send every point in the $\al$-cover to a closest point in $B$, $d(a',f(a'))=R_{a'}$. 

Then $\ve{f}_{\text{Lip}}\le 2\pa{1+\rc\al}$. 
By the triangle inequality,
\[
d(f(a_1'), f(a_2')) \le d(f(a_1'), a_1') + d(a_1',a_2') + d(a_2',f(a_2')).
\]
We have
\begin{align}
Ra_1' + Ra_2' + d(a_1',a_2') &= 2 \min (R_{a_1'}, R_{a_2'}) + |R_{a_1'} + R_{a_2'} + d(a_1',a_2')\\
& \le \rc{\al} d(a_1',a_2') + d(a_1',a_2') + d(a_1',a_2')\\
&=2\pa{1-\rc \al} d(\al_1' ,\al_2')
\end{align}
%no issue with countability. Look at subset largest that admits an $\al$-cover. This argument shows we can add a point to it.
\end{lem}

Let $\ph_B:B\hra \ell_2^b$ be an embedding with $\ve{\ph}_{\text{Lip}}\le D_B$; it is noncontracting.

\begin{lem}
There is $\psi:X\to \ell_2^b$ such that 
\begin{enumerate}
\item
For all $a_1,a_2\in A$, 
\[
\ve{\psi(a_1) - \psi(a_2)} \le 2\pa{1+\rc\al} D_AD_B d(\al_1,\al_2)
\]
\item
For all $b_1,b_2\in B$,
\[
d(b_1,b_2) \le \ve{\psi(b_1)-\psi(b_2)} = \ve{\ph_B(b_1)-\ph_B(b_2)} \le D_Bd(b_1,b_2).
\]
\item
For all $a\in A$, $b\in B$, $d(a,b)-(1+\al)(2D_AD_B+1)R_a\le \ve{\psi(a)-\psi(b)} \le 2(1+\al) (D_AD_B+(2+\al)D_B) d(a,b)$.
\end{enumerate}
\end{lem}
%K extension theorem

\begin{proof}
Let $g=\ph_B f\ph_A^{-1}$.

We have maps
\[
\xymatrix{
A\ha{r}^c \ar[d]^{\ph_A} & A' \ar[d]^{\ph_A} \ar[r]^f & B\ar[d]^{\ph_B}&\\
\ph(A) & \ph(A') \ar[l]_c \ar[r] & \ph(B)\ha{r}&\ell_2^b.
}
\]
Now
\[
\ve{g}_{\text{Lip}} \le \ve{\ph_B}_{\text{Lip}}\ve{f}_{\text{Lip}}\ve{\ph_A^{-1}}_{\text{Lip}} \le D_B2\pa{1+\rc \al}
\]
By the Kirszbraun extension theorem~\ref{thm:kirszbraun}, construct $\wt g:\ell_2^a\to \ell_2^b$ with 
\[
\ve{\wt g}_{\text{Lip}} \le D_B2\pa{1+\rc\al}.
\]
Define $\psi(x) = \begin{cases}
\wt g(\ph_A(x)),&\text{if }x\in A\\
\ph_B(x), & \text{if }x\in B.
\end{cases}$.

We show the three parts.
\begin{enumerate}
\item
\begin{align}
\ve{\psi(a_1)-\psi(a_2)}& = \ve{\wt g(\ph_A(a_1)) - \wt g(\ph_A(a_2))}\\
&\le \ve{\wt g}_{\text{Lip}} \ve{\ph_A}_{\text{Lip}} d(a_1,a_2).
\end{align}
\item
This is clear.
%$\ve{\psi(b_1)-\psi(b_2)}$
\item Let $b=f(a')$. Then $d(a',b')\le R_a$, $\psi(a')=\psi(b')$. 
We have
\begin{align}
\ve{\psi(a)-\psi(b)} &\le \ve{\psi(a)-\psi(a')} - \ve{\psi(a')-\psi(b')} - \ve{\psi(b) - \psi(b')}\\
%upper bound these gues
\ve{\psi(a)-\psi(b)} &\le 2\pa{1+\rc\al} D_AD_B d(a,a')  + D_B d(b,b')\\
d(a,a') &\le \al R_a \le \al d(a,b)\\
d(b,b') &\le d(b,a) + d(a,a') + d(a',b')\\
&\le (2+\al)d(a,b).
\end{align}
This shows the first half of (3). 

For the other inequality, use the triangle inequality against and get
\begin{align}
\ve{\psi(a)-\psi(b)}&\ge 
\ve{\psi(b) - \psi(b')} - \ve{\psi(a')-\psi(b')} - \ve{\psi(a') - \psi(a)}\\ %foldl (-) (zip2 (\(x,y) -> \ve{\psi(x) - \psi(y)}) [b,b',a'] ...) 
&\ge d(b,b') -2(1+\al) D_AD_BR_a.
\end{align}•
\end{enumerate}•
\end{proof}

Let 
\begin{align}
\psi_B&=\psi\\ %$\psi_A$. 
\be&=(1+\al)(2D_AD_B+1)\\
\ga &= \prc2\be\\
\psi_\De:X&\to \R\\
\psi_A(a)&=\ga R_a,&a\in A\\
\psi_A(b)&=-\ga R_b,&b\in B\\
\Psi:X&\to \ell^{a+b+1}\\
\Psi(x)&=\psi_A\opl \psi_B \opl \psi_\De \in \ell_2^{a+b+1}.
\end{align}
For $a_1,a_2\in A$,
\[
\ve{\Psi(a_1)-\Psi(a_2)} \ge \ve{\Psi_A(a_1)-\Psi_A(a_2)} \ge d(a_1,a_2).
\]
For $a\in A,b\in B$,
\[
\ve{\Psi(a)-\Psi(b)}^2 =\ve{\psi_A(a) - \psi_A(b)}^2 + \ve{\psi_B(a)-\psi_B(b)}^2 + \ve{\psi_A(a)-\psi_A(b)}^2.
\]
\begin{align}
\ve{\psi_A(a)-\psi_A(b)} &\ge d(a,b)-\be R_b\\
\ve{\psi_B(a)-\psi_B(b)} &\ge d(a,b)-\be R_a\\
\ve{\psi_A(a)-\psi_A(b)} &= \ga (R_a+R_b).
\end{align}
\begin{clm}
We have $\ve{\psi(a)-\psi(b)} \ge d(a,b)$. 
\end{clm}
\begin{proof}
\Wog{} $R_a\subeq R_b$. Consider 3 cases.
\begin{enumerate}
\item
$\be R_b \le d(a,b) $.
\item
$\be R_a \le d(a,b) \le \be R_b$.
\item
$d(a,b)\le \be R_a$.
\end{enumerate}
Consider case 2. The other cases are similar.
\begin{align}
\ve{\psi(a)-\psi(b)}^2 & \ge (d(a,b)-\be R_a)^2 + \be^2 (R_a+R_b)^2/2\\
&=d(a,b)^2 - 2\be d(a,b) R_a + \fc{\be^2}{2} (3R_a^2 + 2R_aR_b+R_b^2)\\
&\ge d(a,b)^2 - 2\be R_aR_b +\fc{\be^2}{2} (3R_a^2 + 2R_aR_b+R_b^2)\\
& = d(a,b)^2 + \be^2 ((\sqrt 3 R_a-R_b)^2 + 2(\sqrt 3 + R_aR_b))/2\\
&> d(a,b)
\end{align}

%for case 1, something to say.
\end{proof}

%_=_;

For $a_1,a_2\in X$, 
\begin{align}
\ve{\Psi(a_1)- \Psi(a_2)} &=\ve{\psi_A(a_1) + \psi_A(a_2)}^2 + \ve{\psi_B(a_1)-\psi_B(a_2)}^2+ \ve{\psi_B(a_1)-\psi_A(a_2)}\\
&\le \pa{D_A^2 + 4\pa{1+\rc\al}^2 D_A^2D_B^2}d(a_1,a_2)^2 + \ga^2(R_{a_1} - R_{a_2})\\
&\le \pa{D_A^2 + 4\pa{1+\rc\al}^2 D_A^2D_B^2+\ga^2}d(a_1,a_2)^2\\
|R_a-R_{a_2}|&\le d(a_1,a_2)\\
\ve{\Psi(b_1)-\Psi(b_2)}^2 &\le \pa{D_B^2 + 4\pa{1+\rc\al}^2 D_A^2D_B^2 + \ga^2} d(a,b)^2\\
|\Psi(a)-\Psi(b)|^2 & \le \ve{\psi_A(a)-\psi_B(b)}^2 + \ve{\psi_B(a)-\psi_B(b)}^2 + \ve{\psi_A(a)-\psi(b)}^2\\
&\le \cdots \ga^2(R_a+R_b)^2 \le\cdots
\end{align}
$R_a,R_b\le d(a,b)$.

%clarity relative to diff




%INSERT_HERE

\chapter{Lipschitz extension}

`

\section{Introduction and Problem Setup}

We give a presentation of the paper ``On Lipschitz Extension From Finite Subsets'', by Assaf Naor and Yuval Rabani, $(2015)$. For the convenience of the reader referencing the original paper, we have kept the numbering of the lemmas the same.

Consider the setup where we have a metric space $(X, d_X)$ and a Banach space $(Z, \|\cdot\|_Z)$. For a subset $S \subseteq X$,
consider a $1$-Lipschitz function $f: S \to Z$. Our goal is to extend $f$ to $F:X \to Z$ without experiencing too much growth
in the Lipschitz constant $\|F\|_{Lip}$ over $\|f\|_{Lip}$. 

\begin{df} $e(X, S, Z)$ and its variants. \\
Define $e(X, S, Z)$ to be the infimum over the sequence of $K$ satisfying
$\|F\|_{Lip} \leq K\|f\|_{Lip}$ (i.e., $e(X, S, Z)$ is the least upper bound for $\frac{\|F\|_{Lip}}{\|f\|_{Lip}}$ for a particular $S$, $X$, $Z$).

Then, define $e(X, Z)$ to be the supremum over all subsets $S$ for $e(X, S, Z)$: So of all subsets, what's the largest least upper bound for the ratio of Lipschitz constants? 

We may also want to consider supremums over $e(X, S, Z)$ for $S$ with a fixed size. We can formulate this in two ways.
$e_n(X, Z)$ is the supremum of $e(X, S, Z)$ over all $S$ such that $|S| = n$.
We can also describe $e_{\epsilon}(X, Z)$ as the supremum of $e(X, S, Z)$ over all $S$ which are $\epsilon$-discrete in the sense that $d_X(x, y) \geq \epsilon\cdot \txt{diam}(S)$ for distinct $x, y \in S$ and some $\epsilon \in [0, 1]$.
\end{df}

\begin{df} Absolute extendability. \\
We define $\textbf{ae}(n)$ to be the supremum of $e_n(X, Z)$ over all possible metric spaces $(X, d_X)$ and Banach spaces $(Z, \|\cdot\|_Z)$.
Identically, $\textbf{ae}(\epsilon)$ is the supremum of $e_{\epsilon}(X, Z)$ over all $(X, d_X)$ and $(Z, \|\cdot\|_Z)$. 
\end{df}

From now on, we will primarily discuss subsets $S \subseteq X$ with size $|S| = n$. 
Bounding the supremum, the absolute extendability $\textbf{ae}(n) < K$ allows us to make general claims about the extendability of maps from metric spaces into Banach spaces. \textbf{Any} Banach-space valued $1$-Lipschitz function defined on metric space $(M, d_M)$ can therefore be extended to \textbf{any} metric space $M'$ such that $M'$ contains $M$ (up to isometry; as long as you can embed $M$ in $M'$ with an injective distance preserving map) such that the Lipschitz constant of the extension is at most $K$. 

Therefore, for the last thirty years, it has been of interest to understand upper and lower bounds on $\textbf{ae}(n)$, as we want to understand the asymptotic behavior as $n \to \infty$. In the $1980$s, the following upper and lower bound were given by Johnson and Lindenstrauss and Schechtman:
\[
\sqrt{\frac{\log n}{\log \log n}} \lesssim \textbf{ae}(n) \lesssim \log n
\]
In $2005$, the upper bound was improved:
\[
\sqrt{\frac{\log n}{\log \log n}} \lesssim \textbf{ae}(n) \lesssim \frac{\log n}{\log \log n}
\]
In this talk, we improve the lower bound for the first time since $1984$ to
\[
\sqrt{\log n} \lesssim \textbf{ae}(n) \lesssim \frac{\log n}{\log \log n}
\]

\subsection{Why do we care?}

This improvement is of interest primarily not because of the removal of a $\sqrt{\log \log n}$ term in the denominator.
It is due to the fact that the approach taken to get the lower bound provided by Johnson-Lindenstrauss $1984$ has an inherent limitation. 
The approach of Johnson-Lindenstrauss to get the lower bound is to prove the nonexistance of linear projections of small norm. 
By considering a specific case for $f, X, S, Z$, we can get a lower bound on $\textbf{ae}(n)$.
Consider a Banach space $(W, \|\cdot\|_W)$ and let $Y \subseteq W$ be a $k$-dimensional linear subspace of $W$ with $N_{\epsilon}$ an $\epsilon$-net in the unit sphere of $Y$, and then define $S_{\epsilon} = N_{\epsilon} \cup \{0\}$. Fix $\epsilon \in (0, 1/2)$. 
We take $f: S_{\epsilon} \to Y$ to be the identity mapping, and wish to find an extension to $F:W \to Y$. Then, in our setup, we let $X = W$, $S = S_{\epsilon}$, $Z = Y$. We seek to bound the magnitude of the Lipschitz constant of $F$, call it $L$. Johnson-Lindenstrauss prove that for $\epsilon \lesssim \frac{1}{k^2}$, there \textbf{exists} a linear projection $P:W \to Y$ with $\|P\| \lesssim L$. We can now proceed to lower bound $L$ by lower-bounding $\|P\|$ for all $P$. The classical Kadec'-Snobar theorem says that there always exists a projection with $\|P\| \leq \sqrt{k}$. Therefore, the best (largest) possible lower bound we could get will be $L \gtrsim \sqrt{k}$ by Kadec'-Snobar. But this is bad:

Taking $n = |S_{\epsilon}|$, by bounds on $\epsilon$-nets we get $k \asymp \frac{\log n}{\log(1/\epsilon)}$ which implies
\[
L \gtrsim \sqrt{\frac{\log n}{\log (1/\epsilon)}}
\]
In order to get the lower bound on $\textbf{ae}(n)$ of $\sqrt{\log n}$, we must take $\epsilon$ to be a universal constant.
However, from a lemma by Benyamini (in our current setting), $L \lesssim e_{\epsilon}(X, Z) \lesssim 1/\epsilon = O(1)$, which means that any lower bound we get on $L$ will be too small (and won't even tend to $\infty$). Therefore, we must make use of nonlinear theory to get the $\sqrt{n}$ lower bound on $\textbf{ae}(n)$. 

\subsection{Outline of the Approach}

Let us formally state the theorem, and then give the approach to the proof. 
\begin{thm} Theorem $1$. \\
For every $n \in \N$ we have $\textbf{ae}(n) \gtrsim \sqrt{\log n}$. 
\end{thm}

We give a metric space $X$, a Banach space $Z$, a subset $S \subseteq X$, a function $f:S \to Z$ such that $f$ extends to $F: X \to Z$ where $\|F\|_{Lip} \leq K\|f\|_{Lip}$. 

Let $V_G$ be the vertices of a finite graph $G$ with distance metric the shortest path metric $d_G$ where edges all have length $1$. 

We define our metric space $X = (V_G, d_{G_r(S)})$ where $G_r(S)$ is the $r$-magnification of the shortest path metric on $V_G$. $S$ is an $n$-vertex subset $(S, d_{G_r(S)})$. 
Our Banach space $Z = (\R_0^{X}, \|\cdot\|_{W_1(X, d_{G_r(S)}})$ is equipped with the Wasserstein-$1$ norm induced by the $r$-magnification of the shortest path metric on the graph. Note that $\R_0^{X}$ is just weight distributions on the vertices of $X$ which sum to zero in the image. Our $f: S \to \R_0^{S} \subseteq Z$, and we extend to $F: X \to Z$. We will show how to choose $r$ and $|S|$ optimally to get the result. 

The rest of my section of the talk will give the requisite definitions and lemmas to understand the full proof. 

\section{$r$-Magnification}

\begin{df} $r$-magnification of a metric space. \\
Given metric space $(X, d_X)$ and $r > 0$, for every subset $S \subseteq X$ we define
$X_r(S)$ as a metric space on the points of $X$ equipped with the following metric:
\[
d_{X_r(S)}(x, y) = d_X(x, y) + r|\{x, y\} \cap S|
\]
and where $d_{X_r(S)}(x, x) = 0$. 
All this is saying is that when we have distinct points $x, y \in S$, we have the metric is
$2r + d_X(x, y)$, when one point is in $S$ and one point is outside, we have $r + d_X(x, y)$, and
when both $x, y$ are outside, the metric is unchanged.
\end{df}

The significance of this definition is as follows: It's easier for functions on $S$ to be Lipschitz (we enlarge the denominator) without affecting functions on $X \setminus S$. Thus, there are more potential $f$ we can draw from which satisfy $1$-Lipschitzness which can have potentially large Lipschitz extensions (i.e., large $K$) since we don't make it easier to be Lipschitz on $X \setminus S$ (which we must deal with in the extension space).

However, we can't make $r$ too large: the minimum distance between $x, y$ in $S$ becomes close to diam$(S)$ under $r$-magnification as $r$ increases. Let us assume the minimum distance between $x, y$ is $1$ (as it would be in an undirected graph with an edge between $x, y$ under the shortest path metric). Particularly, for distinct $x, y \in S$, since $\txt{diam}(S, d_{X_r(S)}) = 2r + \txt{diam}(S, d_X)$, 
\[
d_{X_r(S)}(x, y) \geq 2r + 1 = \frac{2r + 1}{2r + \txt{diam}(S, d_X)} \cdot \txt{diam}(S, d_{X_r(S)})
\]
Then recall that $e_{\epsilon}(X, Z)$ is the supremum over $S$ such that are $\epsilon$-discrete, where here, $\epsilon = \frac{2r + 1}{2r + \txt{diam}(S, d_X)}$. Earlier we saw a bound that 
\[
e_{\epsilon}(X, Z) \lesssim 1/\epsilon = \frac{2r + \txt{diam}(S, d_X)}{2r + 1} \leq 1 + \frac{\txt{diam}(S, d_X)}{r}
\]
Thus, if we make $r$ too large, we again are bounding $e_{\epsilon}(X, Z) \lesssim 1 = O(1)$, which means our choice of $X$ and $Z$ is not good to get a large lower bound (again, we're not even going to $\infty$). 

Thus we must balance our choice of $r$ appropriately.

\section{Wasserstein-$1$ norm}

Now we come to the second part of our choice of $Z$. Note that we will define $\R_0^X$ to be the set of functions on the points of $X$ such that for each $f \in \R_0^X$, $\sum_{x \in X} f(x) = 0$. We use $e_x$ to denote the indicator weight map with $1$ at point $x$ and $0$ everywhere else.

\begin{df} Wasserstein-$1$ Norm. \\
The Wasserstein-$1$ norm is the norm induced by the following origin-symmetric convex body in finite metric space $(X, d_X)$:
\[
K_{(X, d_X)} = \txt{conv}\left\{\frac{e_x - e_y}{d_X(x, y)}: x, y \in X, x \neq y \right\}
\]
This is a unit ball on $\R_0^X$. We denote the induced norm by $\|\cdot\|_{W_1(X, d_X)}$. 
\end{df}

We can give an equivalent (proven with the Kantorovich-Rubinstein duality theorem) definition of the Wasserstein-$1$ distance:
\begin{df} Wasserstein-$1$ distance and norm. \\
Let $\Pi(\mu, \nu)$ be all measures on $\pi$ on $X \times X$ such that 
\[
\sum_{y \in X} \pi(y, z) = \nu(z)
\]
for all $z \in X$
and 
\[
\sum_{z \in X} \pi(y, z) = \mu(y)
\]
for all $y \in X$. Then, the Wasserstein-$1$ distance (earthmover) is 
\[
W_1^{d_X}(\mu, \nu) = \inf_{\pi \in \Pi(\mu, \nu)} \sum_{x, y \in X} d_X(x, y) \pi(x, y)
\]
In the case that $\mu = \nu$, we automatically have $(\mu \times \nu)/\mu(X) \in \Pi(\mu, \nu)$ (normalizing by one measure trivially gives the other), so $\Pi$ is nonempty.
The norm induced by this metric for $f \in \R_0^X$ is
\[
\|f\|_{W_1(X, d_X)} = W_1^{d_X}(f^+, f^-)
\]
where $f^+ = \max(f, 0)$ and $f^- = \max(-f, 0)$ (since $\sum_{x \in X} f(x) = 0$, we need to make sure that $\mu, \nu$ are both nonnegative measure).
Futhermore, we have $\sum_{x \in X} f^+(x) = \sum_{x \in X} f^-(x)$ (same total mass), which means that $\Pi(f^+, f^-)$ is nonempty.
\end{df}

\begin{df} $\ell_1(X)$ norm. \\
Note that using standard notation, we can also define an $\ell_1$ norm on $f$ by 
\[
\|f\|_{\ell_1(X)} = \sum_{x \in X} |f(x)| = \sum_{x \in X} f^+(x) + f^-(x)
\]
Thus, for our restrictions on $f$, $\sum_{x \in X} f^+(x) = \sum_{x \in X} f^-(x) = \|f\|_{\ell_1(X)}/2$.
\end{df}

Now we give a simple lemma which gives bounds for the Wasserstein-$1$ norm induced by the $r$-magnification of a metric on $X$.
\begin{lem} Lemma $7$. \\
For $(X, d_X)$ a finite metric space, we have
\begin{enumerate}

\item $\|e_x - e_y\|_{W_1(X, d_X)} = d_X(x, y)$ for every $x, y \in X$. 

\item For all $f \in \R_0^X$, 
\[
\frac{1}{2}\min_{x, y \in X; x \neq y} d_X(x, y)\|f\|_{\ell_1(X)} \leq \|f\|_{W_1(X, d_X)} \leq \frac{1}{2}\txt{diam}(X, d_X)\|f\|_{\ell_1(X)}
\]

\item For every $r > 0$, $S \subseteq X$, for all $f \in \R_0^S$,
\[
r\|f\|_{\ell_1(S)} \leq \|f\|_{W_1(S, d_{X_r(S)})} \leq \left(r + \frac{\txt{diam}(X, d_X)}{2}\right)\|f\|_{\ell_1(S)}
\]
\end{enumerate}
\end{lem}
\begin{proof}

\begin{enumerate}

\item This follows directly from the unit ball interpretation of the Wasserstein-$1$ norm, since $\frac{e_x - e_y}{d_X(x, y)}$ is by the first definition on the unit ball.

\item Let $m = \min_{x, y} d_X(x, y) > 0$. For distinct $x, y \in X$, we have
\[
\max_{x, y \in X; x \neq y}\left\|\frac{e_x - e_y}{d_X(x, y)} \right\|_{\ell_1(X)} \leq \max_{x, y \in X; x \neq y} \frac{\|e_x - e_y\|_{\ell_1(X)}}{m} = \frac{2}{m}
\]
since $0 < m \leq d_X(x, y)$ and $1 + 1 = 2$. Therefore $\frac{e_x - e_y}{d_X(x, y)} \in \frac{2}{m}B_{\ell_1(X)}$. These elements span $K_{X, d_X}$, so we have $K_{X, d_X} \subseteq \frac{2}{m}B_{\ell_1(X)}$ and we get the first inequality. 
The second inequality follows from 
\[
\|f\|_{W_1(X, d_X)} = \inf_{\pi \in \Pi(f^+, f^-)} \sum_{x, y \in X} d_X(x, y) \pi(x, y) \leq \txt{diam}(X, d_X)\sum_{x \in X} f^{+}(x) = \txt{diam}(X, d_X)\|f\|_{\ell_1(X)}/2
\]

\item This inequality is a special case of the previous inequality. We have that for $X_r(S)$, $m \geq 2r$ (so $\frac{2}{m} \leq \frac{1}{r}$) and $\frac{1}{2}\txt{diam}(X, d_{X_r(S)}) \leq \frac{1}{2}\left(2r  + \txt{diam}(X, d_X)\right) = \left(r + \frac{\txt{diam}(X, d_X)}{2}\right)$. Plugging in these estimates give the inequality.

\end{enumerate}
\end{proof}

\section{Graph theoretic lemmas}

\subsection{Expanders}

We will need several properties of edge-expanders in our proof of the main theorem. 
For this section, we fix $n, d \geq 3$ and let $G$ be a connected $n$-vertex $d$-regular graph.
We can imagine $d = 3$ in this section, all that matters is that $d$ is fixed. 

First we record two basic average bounds on distance in the shortest-path metric, denoted $d_G$.
\begin{lem} Average shortest-path metric and $r$-magnified average shortest-path bounds. \\
\begin{enumerate}

\item $d_G$ lower bound: For nonempty $S \subseteq V_G$,
\[
\frac{1}{|S|^2}\sum_{x, y \in S}d_G(x, y) \geq \frac{\log |S|}{4\log d}
\]

\item $d_{G_r(S)}$ equality: For some $S \subseteq V_G$ and $r > 0$, 
\[
\frac{1}{|E_G|}\sum_{(x, y ) \in E_G} d_{G_r(x, y)} = 1 + \frac{2r|S|}{n}
\]

\end{enumerate}
\end{lem}
\begin{proof}

\begin{enumerate}
\item The smallest nonzero distance in $G$ is at least $1$. Thus, the average is bounded below by $\frac{1}{|S|^2}|S|(|S| - 1) = 1 - \frac{1}{|S|}$ since $G$ is connected (shortest case is complete graph on $n$ vertices). Then, $1 - 1/a \geq (\log a)/4 \log 3$ for $a \in [15]$ ($d = 3$ maximizes), so we proceed assuming $|S| \geq 16$. Let's bound the distance in the average. Since $G$ is $d$-regular, for every $x \in V_G$ the number of vertices $y$ such that $d_G(x, y) \leq k -1$ is at most $\sum_{i = 0}^{k - 1} d^i$. The rest of the vertices are farther away. Since $1 + \cdots + d^{k - 2} < d^{k -1}$ we have $\#\{y: d_G(x, y) \leq k - 1\} \leq 2d^{k - 1}$. Choosing $k = 1 + \lfloor \log_d(|S|/4) \rfloor$ gives that $2d^{k - 1} \leq \frac{|S|}{2}$. Therefore
\[
\frac{1}{|S|^2}\sum_{x, y \in S}d_G(x, y) \geq \frac{1}{|S|^2} * |S| * |S|/2 * (k - 1) = \frac{k -1}{2} = \frac{\log(|S|/4)}{2 \log d} \geq \frac{\log |S|}{4\log d}
\]
since $|S| \geq 16$.

\item Let $E_1$ be edges completely contained in $S$ and $E_2$ be edges partially contained in $S$. Because $G$ is $d$-regular, $2|E_1| + |E_2| = d|S|$ ($2$ vertices in $S$ for $E_1$, only $1$ vertex for $E_2$, then divide by $d$ for overcounting since each vertex in $S$ hits $d$ other vertices, and we count exactly the edges which have at least one vertex in $S$). Note that $|E_G| = dn/2$ by double-counting vertices. Then for each edge in $E_1$ we add $2r$, for each edge in $E_2$ we add $r$, and otherwise we add $0$ to the base distance of an edge, which is $1$. Therefore, 
\begin{align*}
\begin{split}
\frac{1}{|E_G|}\sum_{(x, y ) \in E_G} d_{G_r(x, y)} &= \frac{\left((0 + 1)|E_G \setminus (E_1 \cup E_2)| + (r + 1)|E_2| + (2r + 1)|E_1|\right)}{|E_G|} 
\\
&= 1 + \frac{r(2|E_1| + |E_2|)}{|E_G|} = 1 + \frac{rd|S|}{dn/2} = 1 + \frac{2r|S|}{n}
\end{split}
\end{align*}

\end{enumerate}
\end{proof}

Now we introduce the definition of edge expansion. 

\begin{df} Edge expansion $\phi(G)$. \\
$G$ is a connected $n$-vertex $d$-regular graph.
Consider $S, T \subseteq V_G$ disjoint subsets. Let $E_G(S, T) \subseteq E_G$ denote the set of edges which bridge $S$ and $T$.
Then the \textbf{edge-expansion} $\phi(G)$ is defined by
\[
\phi(G) = \sup \left\{\phi: |E_G(S, V_G\setminus S)| \geq \phi\frac{|S|(n - |S|)}{n^2}|E_G|, \forall S \subseteq V_G, \phi \in [0, \infty)\right\}
\]
\end{df}

We give an equivalent formulation of edge expansion via the cut-cone decomposition:
\begin{lem} Edge-Expansion: Cut-cone Decomposition of Subsets of $\ell_1$. \\
$\phi(G)$ is the largest $\phi$ such that for all $h:V_G \to \ell_1$,
\[
\frac{\phi}{n^2}\sum_{x, y \in V_G} \|h(x) - h(y)\|_1 \leq \frac{1}{|E_G|}\sum_{(x, y) \in E_G} \|h(x) - h(y)\|_1
\]
\end{lem}
\begin{proof}
We will assume this for this talk.
\end{proof}

Now we combine Lemma $7$ and the cut-cone decomposition to get Lemma $8$:
\begin{lem} Lemma $8$. \\
Fix $n \in \N$ and $\phi \in (0, 1]$. Suppose $G$ is an $n$-vertex graph with edge expansion $\phi(G) \geq \phi$. For all nonempty $S \subseteq V_G$ and $r > 0$, every $F:V_G \to \R_0^S$ satisfies
\[
\frac{1}{n^2}\sum_{x, y \in V_G} \|F(x) - F(y)\|_{W_1(S, d_{G_r(S)})} \leq \frac{2r + \txt{diam}(S, d_G)}{(2r + 1)\phi} \cdot \frac{1}{|E_G|} \sum_{(x, y) \in E_G}\|F(x) - F(y)\|_{W_1(S, d_{G_r(S)})}
\]
\end{lem}
Basically, whereas before we were bounding average norms (normal and $r$-magnified) in the domain space $V_G$, we are now bounding average norms in the image space using the Wasserstein-$1$ norm induced by the $r$-magnification of the shortest path metric on the graph.
\begin{proof}
First, plug in $h = F$ into the cut-cone decomposition, where the norm is now defined by $\ell_1(S)$. Then, we know that diam$(S, d_{G_r}(S)) = 2r + \txt{diam}(S, d_G)$ and the smallest positive distance in $(S, d_{G_r(S)})$ is $2r + 1$. Applying Lemma $7$, every $x, y \in V_G$ satisfy
\[
\frac{2r + 1}{2}\|F(x) - F(y)\|_{\ell_1(S)} \leq \|F(x) - F(y)\|_{W_1(S, d_{G_r(S)})} \leq \frac{2r + \txt{diam}(S, d_G)}{2}\|F(x) - F(y)\|_{\ell_1(S)}
\]
Plugging these estimates directly into the cut-cone decomposition
\[
\frac{1}{n^2}\sum_{x, y \in V_G} \|F(x) - F(y)\|_{\ell_1(S)} \leq \frac{1}{\phi|E_G|}\sum_{(x, y) \in E_G} \|F(x) - F(y)\|_{\ell_1(S)}
\]
gives the desired result.
\end{proof}

\subsection{An Application of Menger's Theorem}

Here, we want to bound below the number of edge-disjoint paths.

\begin{lem} Lemma $9$. \\
Let $G$ be an $n$-vertex graph and $A, B \subseteq V_G$ be disjoint. Fix $\phi \in (0, \infty)$ and suppose $\phi(G) \geq \phi$.
Then
\[
\#\left\{\txt{edge-disjoint paths joining }A \txt{ and } B\right\} \geq \frac{\phi |E_G|}{2n} \cdot \min\{|A|, |B|\}
\]
\end{lem}
\begin{proof}
Let $m$ be the maximal number of edge-disjoint paths joining $A$ and $B$. 
By Menger's theorem from classical graph theory, there exists a subset of edges $E^* \subseteq E_G$ with $|E^*| = m$ such that every path in $G$ joining $a \in A$ to $b \in B$ contains an edge from $E^*$.

Now consider the graph $G^* = (V_G, E_G \setminus E^*)$. In this graph, there are no paths between $A$ and $B$. Now, if we let $C \subseteq V_G$ be the union of all connected components of $G^*$ containing an element of $A$, then $A \subseteq C$ and $B \cap C = \emptyset$. 
Since we covered the maximal possible vertices reachable from $A$ with $C$, all edges between $C$ and $V_G \setminus C$ are in $E^*$.
Therefore, $|E_G(C, V_G \setminus C)| \leq |E^*| = m$.
By the definition of expansion,
\[
m \geq |E_G(C, V_G\setminus C)| \geq \phi\frac{\max\{|C|, n - |C|\} \cdot \min\{|C|, n - |C|\}}{n^2}|E_G| \geq \frac{\phi \min\{|A|, |B|\}\cdot |E_G|}{2n}
\]
since $\max\{|C|, n - |C|\} \geq n/2$ and since $A \subseteq C, B \subseteq V_G \setminus C$, we have $\min\{|C|, n - |C|\} \geq \min\{|A|, |B|\}$.

\end{proof}

%%%%%%%%%%%%%%%%%%%%%%%%%%%%%%%%%%%%%%%%%%%%%%%%%%%%%%%%%%%%%%%

\section{Main Proof}

We fix $d,n\in\mathbb{N},\phi\in(0,1)$ and let $G$ be a $d$-regular graph on $n$ vertices with $\phi(G)\ge\phi$. We also fix a nonempty subset $S\subset V_G$ and $r>0$ and define a mapping
$$f:(S,d_{G_r(S)})\mapsto\left(\mathbb{R}_0^S,\|\cdot\|_{W_1(S,d_{G_r(S)})}\right)\quad\textrm{s.t.}\quad f(x)=e_x-\frac{1}{|S|}\sum\limits_{z\in S}e_z\enskip\forall\enskip x\in S$$
Then suppose we have that some $F:V_G\mapsto\mathbb{R}_0^S$ extends $f$ and for some $L\in(0,\infty)$ we have
$$\|F(x)-F(y)\|_{W_1(S,d_{G_r(S)})}\le Ld_{G_r(S)}(x,y)$$
For $x\in V_G,s\in(0,\infty)$ define 
$$B_S(x)=\{y\in V_G:\|F(x)-F(y)\|_{W_1(S,d_{G_r(S)})}\le s\}$$
i.e.e $B_s$ is the inverse image of $F$ of the ball (in the Wasserstein 1-norm) of radius $s$ centered at $F(x)$. By the lemma consequence of Menger's Theorem, since $\phi\ge\phi(G)$ we have
$$m\ge\frac{\phi d}{4}\min\{|S\backslash B_s(x)|,|B_s(x)|\}\qquad\circled{1}$$
for $m$ edge-disjoint paths between $S\backslash B_s(x)$ and $B_s(x)$, i.e. we can find indices $k_1,\dots,k_m\in\mathbb{N}$ and vertex sets $\{z_{j,1},\dots,z_{j,k_j}\}_{j=1}^m\in V_G$ s.t. $\{z_{j,1}\}_{j=1}^m\subset S\backslash B_s(x)$, $\{z_{j,k_j}\}_{j=1}^m\subset B_s(x)$ (i.e. the beginnings and ends of paths are in different disjoint subsets) and such that $\{\{z_{j,1},z_{j,i+1}:j\in\{1,\dots,m\}\wedge i\in\{1,\dots,k_j-1\}\}$ are distinct edges in $E_G$ (i.e. edge-disjointedness). Now take an index subset $J\subset\{1,\dots,m\}$ s.t. $\{z_{j,1}\}_{j\in J}$ are distinct and $\{z_{j,1}\}_{j\in J}=\{z_{i,1}\}_{i=1}^m$. For $j\in J$ denote by $d_j$ the number of $i\in\{1,\dots,m\}$ for which $z_{j,1}=z_{i,1}$. Then since $G$ is $d$-regular and $\{\{z_{i,1},z_{i,2}\}\}_{i=1}^m$ are distinct, $\max\limits_{j\in J}d_j\le d$. Since $\sum\limits_{j\in J}d_j=m$, from \circled{1} we have that
$$|J|\ge\frac{m}{d}\ge\frac{\phi}{4}\min\{|S\backslash B_s(x)|,|B_s(x)|\}\qquad\circled{2}$$
\noindent The quantity $|J|$ can be upperbounded as follows:\\
\begin{lem} Lemma $10$:
$|J|\le\max\left\{d^{16(s-r)},\frac{16nLd\log d}{\log n}\left(1+\frac{2r|S|}{n}\right)\right\}$
\end{lem}
\begin{proof}
Assume $|J|\le d^{16(s-r)}$ (otherwise we are done). This is equivalent to 
$$s-r<\frac{\log|J|}{16\log d}$$
Now since $\{z_{j,1}\}_{j\in J}\subset S$ and $F(x)=f(x)$ $\forall$ $x\in S$ and is an isometry on $(S,d_{G_r(S)})$, by the definition of the $r$-magnified metric
$$\|F(z_{i,1})-F(z_{j,1})\|_{W_1(S,d_{G_r(S)})}=d_{G_r(S)}(z_{i,1},z_{j,1})=2r+d_G(z_{i,1},z_{j,1})$$
This gives us
\begin{align*}
\sum\limits_{j\in J}\|F(z_{j,1})-F(z_{j,k_j})\|_{W_1(S,d_{G_r(S)})}&=\frac{1}{2|J|}\sum\limits_{i,j\in J}\left(\|F(z_{i,1})-F(z_{i,k_i})\|_{W_1(S,d_{G_r(S)})}+\|F(z_{j,1})-F(z_{j,k_j})\|_{W_1(S,d_{G_r(S)})}\right)\\
&\ge\frac{1}{2|J|}\sum\limits_{i,j\in J}\left(\|F(z_{i,1})-F(z_{j,1})\|_{W_1(S,d_{G_r(S)})}+\|F(z_{z_i,k_i})-F(z_{j,k_j})\|_{W_1(S,d_{G_r(S)})}\right)
\end{align*}
Since $\{z_{j,k_j}\}_{j\in J}\subset B_s(x)$, by the definition of $B_s(x)$ we have 
$$\|F(z_{i,k_i})-F(z_{j,k_j})\|_{W_1(S,d_{G_r(S)})}\le\|F(z_{i,k_i})-F(x)\|_{W_1(S,d_{G_r(S)})}+\|F(x)-F(z_{j,k_j})\|_{W_1(S,d_{G_r(S)})}\le 2s\enskip\forall\enskip i,j\in J$$
so the previous inequality can be further continued as
$$\sum\limits_{j\in J}\|F(z_{j,1})-F(z_{j,k_j})\|_{W_1(S,d_{G_r(S)})}\ge\frac{1}{2|J|}\sum\limits_{i,j\in J}d_G(z_{i,1},z_{j,1})-(s-r)|J|\ge\frac{|J|\log|J|}{8\log d}-(s-r)|J|>\frac{|J|\log|J|}{16\log d}$$
where the second inequality above is a property of the expander graph. The same quantity can be bounded from above using the Lipschitz condition
$$\sum\limits_{j\in J}\|F(z_{j,1})-F(z_{j,k_j})\|_{W_1(S,d_{G_r(S)})}\le L\sum\limits_{j\in J}d_{G_r(S)}(z_{j,1},z_{j,k_j})\le L\sum\limits_{j\in J}\sum\limits_{i=1}^{k_j-1}d_{G_r(S)}(z_{j,1},z_{j,i+1})$$
By the edge-disjointedness of the paths (specifically since $\{\{z_{j,1},z_{j,i+1}\}:j\in J\wedge i\in\{1,\dots,k_j-1\}\}$) are distinct edges in $E_G$, we have that
$$\sum\limits_{j\in J}\sum\limits_{i=1}^{k_j-1}d_{G_r(S)}(z_{j,1},z_{j,i+1})\le\sum\limits_{\{u,v\}\in E_G}d_{G_r(S)}(u,v)=\frac{nd}{2}\left(1+\frac{2r|S|}{n}\right)$$
Everything together gives
$$\frac{Lnd}{2}\left(1+\frac{2r|S|}{n}\right)\ge\frac{|J|\log|J|}{16\log d}$$
By the simple fact that $a\log a\le b\implies a\le\frac{2b}{\log b}$ for $a\in[1,\infty),b\in(1,\infty)$, using $a=|J|$ and $b=8Lnd\log d\left(1+\frac{2r|S|}{n}\right)\ge n$ we have that
$$|J|\le\frac{16nLd\log d}{\log n}\left(1+\frac{2r|S|}{n}\right)$$
completing the proof.
\end{proof}

\noindent The lemma has two corollaries, both depending on the following condition:
$$d^{16(s-r)}\le\frac{\phi|S|}{8}\qquad L\le\frac{\phi|S|\log n}{128\left(1+\frac{2r|S|}{n}\right)nd\log d}\qquad\circled{3}$$
\begin{cor} Corollary $11$: $\max\limits_{x\in V_G}|B_s(x)|<\frac{|S|}{2}$
\end{cor}
\begin{proof}
Pick an $x\in V_G$. If $B_s(x)\cap S$ is nonempty, then again using the standard estimate on expander graphs we have that $\exists$ $y,z\in B_s(x)\cap S$ s.t.
$$d_G(y,z)\ge\frac{\log|B_s(x)\cap S|}{4\log d}$$
Since $y,z\in S$ and $F(x)=f(x)$ $\forall$ $x\in S$ and furthermore $F$ is an isometry on $(S,d_{G_r(S)})$, we have similarly to before that
$$d_G(y,z)+2r=d_{G_r(S)}(y,z)=\|F(y)-F(z)\|_{W_1(S,d_{G_r(S)})}\le\|F(y)-F(x)\|_{W_1(S,d_{G_r(S)})}+\|F(x)-F(z)\|_{W_1(S,d_{G_r(S)})}\le 2s$$
where we have used first the triangle inequality and second the fact that $y,z\in B_s(x)$. Then using the first inequality in the proof we have
$$|B_s(x)\cap S|\le d^{8(s-r)}\le\sqrt{\frac{\phi|S|}{8}}\le\frac{2|S|}{5}$$
where the second inequality comes from the first assumption. Now the above inequality implies that $|S\backslash B_s(x)|\ge\frac{3|S|}{5}$, which combined with Lemma 10 yields
$$\min\left\{\frac{3|S|}{5},|B_s(x)|\right\}<\max\left\{\frac{4d^{16(s-r)}}{\phi},\frac{64nLd\log d}{\phi\log n}\left(1+\frac{2r|S|}{n}\right)\right\}$$
However, the two original assumptions together imply that $\frac{3|S|}{5}$ is in fact greater than either value in the maximum, so
$$|B_s(x)|\le\max\left\{\frac{4d^{16(s-r)}}{\phi},\frac{64nLd\log d}{\phi\log n}\left(1+\frac{2r|S|}{n}\right)\right\}\le\frac{|S|}{2}$$
\end{proof}
\begin{cor} Corollary $12$: $L\ge\frac{\phi s}{2\left(1+\frac{\textrm{diam}(G,d_G)}{2r}\right)\left(1+\frac{2r|S|}{n}\right)}$
\end{cor}
\begin{proof}
By the definition of $B_s(x)$ for $x\in V_G$ and $y\in V_G\backslash B_s(x)$ we have $\|F(x)-F(y)\|_{W_1(S,d_{G_r(S)})}>s$, so
\begin{align*}
\frac{1}{n^2}\sum\limits_{x,y\in V_G}\|F(x)-F(y)\|_{W_1(S,d_{G_r(S)})}&\ge\frac{1}{n^2}\sum\limits_{x\in V_G}\sum\limits_{y\in V_G\backslash B_s(x)}\|F(x)-F(y)\|_{W_1(S,d_{G_r(S)})}\\
&\ge\frac{s}{n^2}\sum\limits_{x\in V_G}(n-|B_s(x)|)\ge s\left(1-\frac{\max\limits_{x\in V_G}|B_s(x)|}{n}\right)\ge\frac{s}{2}
\end{align*}
where the last inequality comes from Corollary 11. Then since Lemma 8 gives
\begin{align*}
\frac{1}{n^2}\sum\limits_{x,y\in V_G}\|F(x)-F(y)\|_{W_1(S,d_{G_r(S)})}&\le\frac{2r+\textrm{diam}(S,d_G)}{(2r+1)\phi|E_G|}\sum\limits_{x,y\in V_G}\|F(x)-F(y)\|_{W_1(S,d_{G_r(S)})}\\
&\le\frac{L\left(1+\frac{\textrm{diam}(G,d_G)}{2r}\right)}{\phi|E_G|}\sum\limits_{\{x,y\}\in E_G}d_{G_r(S)}(x,y)\\
&=\frac{L\left(1+\frac{\textrm{diam}(G,d_G)}{2r}\right)\left(1+\frac{2r|S|}{n}\right)}{\phi}
\end{align*}
where the ineqaulity on the second line is true since $F$ is an isometry on $(S,d_G)$ and from the Lipschitz constant of $f$ and the equality is average length of the $r$-magnification of the expander graph $G$. Combining these inequalities with the previous ones yields the corollary.
\end{proof}
\begin{thm} Theorem $13$: if $0<r\le\textrm{diam}(G,d_G)$ then 
$$L\ge_C\frac{\phi}{1+\frac{r|S|}{n}}\min\left\{\frac{|S|\log n}{nd\log d},\frac{16r^2\log d+r\log\left(\frac{\phi|S|}{8}\right)}{\textrm{diam}(G,d_G)\log d}\right\}$$
\end{thm}
\begin{proof}
Assume $16r\log d+\log\left(\frac{\phi|S|}{8}\right)>0$ (otherwise we are done) and choose $s=r+\frac{\log\left(\frac{\phi|S|}{8}\right)}{16\log d}$ s.t. $s>0$ and $d^{16(s-r)}=\frac{\phi|S|}{8}$. Then the first inequality in \circled{3} is satisfied, so either the second fails and $L$ thus has that expression as a lower bound, or both are satisfied and we have the lower bound in Corollary 12.
\end{proof}
\begin{thm} Theorem $1$: for every $n\in\mathbb{N}$ we have $\textbf{ae}(n)\ge_C\sqrt{\log n}$.
\end{thm}
\begin{proof}
Substituting $\phi\asymp 1$ and $\textrm{diam}(G,d_G)\asymp\frac{\log n}{\log d}$ (the $\asymp$ indicates asymptotically for large $n$), and using the assumption $0<r\le\textrm{diam}(G,d_G)$, the lower bound given in Theorem 13 becomes
$$L\ge_C\frac{1}{1+\frac{r|S|}{n}}\min\left\{\frac{|S|\log n}{nd\log d},\frac{r(r\log d+\log|S|)}{\log n}\right\}$$
Taking $S\subset V_G$ s.t. $|S|=\left\lfloor\frac{n\sqrt{d\log d}}{\sqrt{\log n}}\right\rfloor$ (we must have $n\\ge d^d$ to ensure $|S|\le n$) and $r\asymp\frac{\log d}{\sqrt{d\log d}}$, which gives a lower bound of $L\ge_C\frac{\sqrt{\log n}}{\sqrt{d\log d}}$. Therefore
$$e_{\left\lfloor\frac{n\sqrt{d\log d}}{\sqrt{\log n}}\right\rfloor}(G_r(S),W_1(S,d_{G_r(S)}))\ge_C\frac{\sqrt{\log n}}{\sqrt{d\log d}}$$
which completes the proof for fixed $d$.
\end{proof}


%\appendix
%\input{distribution_chapters/a.tex}

%\bibliographystyle{plain}
%\bibliography{\filepath/refs}

\printnomenclature
\printindex
\end{document}