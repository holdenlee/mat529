\blu{2-17}

Now we prove Step 2 (Lemma~\ref{lem:step2}). Note we can assume $\om=\{1,\ldots, m\}$ and that the rank is $m$.

First we need some lemmas.
\begin{lem}\llabel{lem:rip-step2-1}
Let $A:\R^m\to \R^n$ be such that $\{Ae_j\}_{j=1}^m$ are linearly independent, and $\si\subeq \{1,\ldots, m\}$, $t\in \N$. Then there exists $\tau\subeq \si$ with 
\[
|\tau|\ge \pa{1-\rc{2^t}}|\si|,
\]
such that denoting $\te = \tau\cup (\{1,\ldots, m\}\bs \si)$, $M=\max_{j \in \omega} \rc{{\|\txtn{Proj}_{F_j} Ae_j\|}}$, for all $\si\in \R^{\te}$,
\[
\sum_{i\in \tau}|a_i|\le 2^{\fc t2} M \sqrt{|\si|} \ve{\sum_{i\in \te} a_iAe_i}_2.
\]
%controlled by $L^2
%ganos pulus
\end{lem}

\ig{images/6-1}{.25}

This is proved by a nice inductive argument.
%This sets us up for the second iteration in the proof.
\begin{lem}\llabel{lem:rip-step2-2}
Let $m,n,t\in \N$ and $\be\subeq \{1,\ldots, m\}$. Let $A:\R^m\to \R^n$ be a linear operator such that $\{Ae_j\}_{j=1}^m$ are linearly independent. Then there exist two subsets
$
\si \subeq \tau \subeq \be
$
such that
$|\tau|\ge \pa{1-\rc{2^t}}|\be|$,  
$|\tau\bs \si| \le \fc{|\be|}{4}$,
and if we denote $\te=\tau\cup (\{1,\ldots, m\}\bs \be)$, $M=\max_{j \in \omega} \rc{{\|\txtn{Proj}_{F_j} Ae_j\|}}$, then 
%Restrict $A$ to the subset $\te$
\[
\ve{\Proj_{\R^{\si}}(AJ_{\te})^{-1}}_{S_{\iy}}\lesssim 2^{\fc t2}M.
\]
\end{lem}
%restrict with big set, compose with orthogonal projection to $\si$. Get operator norm bound

\ig{images/6-2}{.25}

\begin{proof}[Proof of Lemma~\ref{lem:rip-step2-2} from Lemma~\ref{lem:rip-step2-1}]
Apply Lemma~\ref{lem:rip-step2-1} with $\si=\be$. We find $\tau\subeq \be$ with $|\tau|\ge \pa{1-\rc{2^t}}|\be|$ such that 
\[
\sum_{i\in \tau} |a_i|\le 2^{\fc t2} M\sqrt{|\be|}\ve{\sum_{i\in \te} a_iAe_i}_2.
\]
Rewriting gives that %for all $a\in \R^{\si}$,
\bal
\forall a\in \R^{\si},\qquad
\ve{\Proj_{\R^{\tau}}a} &\le 2^{\fc t2} M \sqrt{|\be|} \ve{AJ_{\te}a}_2\\
\implies \ve{\Proj_{\R^{\tau}} (AJ_\te)^{-1}}_{\ell_2^{\te}\to \ell_1^{\tau}}&\lesssim 2^{\fc t2} M \sqrt{|\be|}.
\end{align*}
%projection times b... is exactly operator norm.
%extctly what came up in LGI in PDT. 
%applied to vector at most $L^2$ of vector
Denote $\ep=\fc{|\be|}{4|\tau|}$. By Lemma \fixme{which?}, there exists $\si\subeq \tau$, $|\si|\ge (1-\ep)|\tau|$ such that 
\bal
\ve{\Proj_{\R^{\si}} (AJ_{\te})^{-1}}_{S^{\iy}}=
\ve{\Proj_{\R^{\si}}\Proj_{\R^{\tau}} (AJ_\te)^{-1}}_{S^{\iy}}
&\lesssim \sfc{\pi}{2\ep |\tau|} 2^{\fc t2} M\sqrt{|\be|}\\
&\lesssim 2^{\fc t2}M.
\end{align*}
%sauer shelah, many good signs
%first lemma is $L^1$.
%pass to signs? get second lemma.
%Show lemma 2 and finish, or lemma 1.
\end{proof}

\begin{proof}[Proof of Lemma \ref{lem:step2} (Step 2)]
%volume, grothendieck
%only remaining is combinatorics (SS).
%inductive construction.
Fix an integer $r$ such that $\rc{2^{2r+1}}\le 1-\fc km\le \rc{2^{2r}}$. 
Proceed inductively as follows. First set
\bal
\tau_0&=\{1,\ldots, m\}\\
\si_0&=\phi.
\end{align*}
Suppose $u\in \{0,\ldots, r+1\}$ and we constructed $\si_k,\tau_k\subeq \{1,\ldots, m\}$ such that if we denote $\be_u =\tau_u \bs \si_u$, $\te_u = \tau_u \cup (\{1,\ldots, m\}\bs \be_{u-1})$, then
\begin{enumerate}
\item
$\si_u\subeq \tau_u \subeq \be_{u-1}$
\item
$|\tau_u|\ge \pa{1-\rc{2^{2r-u+4}}}|\be_{u-1}|$
\item
$|\be_u|\le \rc 4 |\be_{u-1}|$
\item
$\ve{\Proj_{\R^{\si_u}}(AJ_{\te_u})^{-1}}_{S_{\iy}} \lesssim 2^{r-\fc u2}M$.
\end{enumerate}

\ig{images/6-3}{.25}

Let $H=2r-u+4$. For instance, $|\tau_1|\ge \pa{1-\rc{2^{2r+3}}}|\be_0|$.
%the projection onto $\R^{\si}$. is $\fc p2$. constant don't care about.
What is the new $\be$? %$\tau_u0\si_u$.
%we can chop off a piece such that 
%projection onto $\R_{\si_2}$.

For the inductive step, apply Lemma~\ref{lem:rip-step2-2} on $\be_{u-1}$ with $t=2r-u+4$ to get $\si_u\subeq \tau_u\subeq \be_{u-1}$ such that $|\tau_u|\ge \pa{1-\rc{2^{2r-u+4}}}|\be_{u-1}|$, 
$
|\tau_u\bs \si_u|\le \fc{|\be_{u-1}|}{4}
$
%use lemma
%add omre and more pieces.
%set up so can get what want, ...
%\te decreasing
\beq{eq:rip-s2-1}
\ve{\Proj_{\R^{\si_u}}(AJ_{\te_u})^{-1}}\lesssim 
2^{r-u/2}M.
\eeq

We know $|\be_{u-1}|\le \fc{m}{4^{u-1}}$, 
\begin{align*}
\be_{u-1}&=\be_u \sqcup \si_u \sqcup (\be_{u-1}\bs \tau_u)\\
|\be_{u-1}|&=|\be_u| + |\si_u|  +(|\be_{u-1}| - |\tau_u|)\\
|\si_u|&=|\be_{u-1}|-|\be_u| -(|\be_{u-1}| - |\tau_u|)\\
&\ge |\be_{u-1}|-|\be_u|-\fc{|\be_{u-1}|}{2^{2r-u+4}}\\
&\ge |\be_{u-1}|-|\be_u| - \fc{m}{2^{2r+u+2}}.
\end{align*}
Our choice for the invertible subset is
\[
\si = \bigsqcup_{u=1}^{r+1}\si_u.
\]
Telescoping gives
\bal
|\si|&=\sumo u{r+1} |\si_u| \ge |\be_0|-|\be_{r+1}| - \fc{m}{2^{2r+2}}\sumo u{\iy} \rc{2^u} \\
&\ge m-\fc{m}{4^{r+1}}-\fc{m}{2^{2r+2}}\\
&=m\pa{1-\rc{2^{2r+1}}}\ge m\fc{k}m=k.
\end{align*}
%bookkeeping
%new universe in $\tau_u$. Next bit is inside
Observe that  $\si\subeq \bigcap_{u=1}^{r+1} \te_u$ and  for every $u$, 
%??? Where do we use this?
\bal
\si_u,\ldots, \si_{r+1}&\subeq \tau_u\\ 
\si_1,\ldots, \si_{u-1} &\subeq \{1,\ldots, m\} \bs \be_{u-1}
\end{align*}
%union is $\te_u$.
%more confusing on the board.

This allows us to use the conclusion for all the $\si_u$'s at once. 

For $a\in \R^{\si}$, $J_{\si}a\subeq J_{\te_u} \R^{\te_u}$, 
\[
\Proj_{\R^{\si_u}}(AJ_{\te_u})^{-1} (AJ_\si) a = \Proj_{\R^{\si_u}}J_\si a.
\] 
%a sucset of $J_\si$. Apply $A$ to it.
%R^\si$ inside $\R^{\si_u}$, so honest inverse. 
Then, breaking $J_\si a$ into orthogonal components,
\bal
\ve{J_\si a}_2^2 &=\sumo  u{r+1} \ve{\Proj_{\R^{\si_u}}J_{\si}a}_2^2\\
%blocks. l^2 norm is sum over blocks
&=\sum_{u=1}^{r+1}\ve{\Proj_{\R^{\si_u}}(AJ_{\si_u})^{-1}(AJ_\si)a}_2^2 \\
%??? Where did this line come from?
&\lesssim \sum_{u=1}^{r+1} 2^{2r-u} M^2 \ve{AJ_\si a}_2^2&\text{by~\eqref{eq:rip-s2-1}}\\
&\lesssim 2^{2r} M^2 \ve{AJ_\si a}_2^2\\
&\le \fc{M^2}{1-\fc km}\ve{AJ_\si a}_2^2\\
\ve{J_\si a}_2 &\le \sfc{m}{m-k} M \ve{AJ_\si a}_2
\end{align*}
Since this is true for all $a\in \R^{\si}$,
\bal
\implies \ve{(AJ_\si)^{-1}}_{S_{\iy}} &\lesssim \sfc{m}{m-k} M.
%use lemma as black box.
\end{align*}
%in the end it's combinatorics.
\end{proof}
It remains to prove Lemma~\ref{lem:rip-step2-1}.

%greedily instead of inductively?
%improve over original
%subset of proportional size, first bite gets the constrant fraction. To get to $1-\ep$. Each time, get half of what remains. it would be nice. get very close to rank. 
%aggressive
%initial proof they do it once.

%SS. first time go to half. 
%to get more, look at half of what remains. Hopefully we can continue. My hunch  a different way: skip over use of SS.

%not true that SS take half of what remains
%give bound from $L^1$ to $L^2$. Signs 
%grothendieck further subset
%correct wrong norms to operator norms.

%Another aspect: this bound $\lesssim \sfc{m}{m-k}$ is existential statement. 
%Hinted that Lemma 1 uses SS. 
%Cloud of points. Some subset of coordinates such that when you project it you see the full cube. I'm quite certain that it's NP-hard. (How is the set given to you?) Sets we care about have  simple description.
%Formulate correct algorithmic Sauer-Shelah lemma.
%subset of half the diension.
%formulate the question correctly. Some sort of description.
%formulate question.
%good chance for the particular sets we use, there is poly time algorithm.
%intersection of cube with ellipsoid. In this case make algorithmic.

How can we make this theorem algorithmic?

The way the Pietsch Domination Theorem~\ref{thm:pdt} worked was by duality. 
%pitch.
%I'm not worried. It worked by duality. 
We look at a certain explicitly defined convex set. We found a separating hyperplane which must be a probability measure. Then we had a probabilistic construction. This part is fine.

There was another place where we chose a subset, the subset that maximizes the volume out of all subsets of a given size (Lemma~\ref{lem:step1}). Specifically, we want the set of columns of a matrix that maximizes the volume of the convex hull.
Computing the volume of the convex hull requires some thought.
%subset max vol among all subsets of various size. look at one which maximizes volume. 
Also, there are $\binom nr$ subsets of size $r$; if $r=n/2$ there are exponentially many. We need a way to find subsets with maximum volume fast. There might be a replacement algorithm which approximately maximizes this volume.

The bottleneck for making this an algorithm (I do believe this will become an algorithm) is going through Sauer-Shelah, formulating the algorithmic question tailored to our sets (if the intersection of an ellipsoid (or other convex bodies) with the boolean cube has $>2^{n-1}$ points, find a large shattering set in polynomial time; this would have other applications). (There could be an algorithm out there, but I couldn't find one in the literature.)

The second part is finding a subset with maximal volume. 
%Bottleneck for making this algorithm. 
%tailored for our proof... intersection of cubes with ellipsoids, $>2^{n-1}$, polytime find subset. 
%cubes intersected with other convex bodies?
%I don't know the algorithmic literature.
%I searched and looked at the literature.= and asked 
%If you think about it, look at it.

%??? Where should figure 6-4 be?